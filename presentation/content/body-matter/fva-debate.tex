% !TEX root = sub-main.tex
\documentclass[main.tex]{subfiles}

\begin{document}
    \section{The FVA Debate}

    \begin{frame}
        \frametitle{Conclusions on the FVA Debate}

        \begin{enumerate}
            \item Funding costs became non-negligible after the financial crisis.
                \note[item]{
                    With the increase in borrowing rates after the Financial Crisis in 2008,
                    institutions saw their funding costs rise as well. 
                    This led them to start applying FVAs.
                }
            \item Dealers will necessarily incur funding costs.
                \note[item]{
                    The dealers in an institution 
                    will be charged the funding rate by the funding desk.
                    If they do not apply FVAs, they ignore the funding costs,
                    which could have influenced the trading decision.
                    We argue that this is a sensible argument,
                    and that institutions should in fact be using FVAs 
                    to determine the values of projects.
                }
            \item Theoreticians argue that a project's riskiness should determine the discount rate.
                \note[item]{
                    Theoreticians seem to disagree with the current market practice. 
                    The project's riskiness determines the discount rate; 
                    not the institutions funding rate.
                }
                \vspace{.5cm}
                \begin{center}
                    \resizebox{5cm}{!}{%
                    \begin{tikzpicture}
                        \import{\graphicsfolder/hw-example}{hw-example-simple}
                    \end{tikzpicture}        
                    }   
                \end{center}
                \note[item]{
                    HW's argument was shown by an example. 
                    A project with a PV of 100 is trading at 99.50.
                    This leaves a NPV of 0.50. 
                    HW argue that this project should be undertaken due to its positive NPV,
                    regardless of the funding rate of the institution.
                    Theoretically that seems right 
                    as the project looks like an arbitrage opportunity.
                    However, as was shown by examples, 
                    the NPV cannot simply be distributed to whoever finances the project.
                    There are frictions in institutions which creates a free rider problem. 
                }
        \end{enumerate}

    \end{frame}

\end{document}

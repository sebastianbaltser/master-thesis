% !TEX root = sub-main.tex
\documentclass[main.tex]{subfiles}

\begin{document}
    \section{Firm Funding Frictions}
    \begin{frame}
        \frametitle{Firm Funding Frictions}

        \begin{center}
            \resizebox{\textwidth}{!}{%
            \begin{tikzpicture}
                % !TEX root = ./test-graphics.tex

\pgfdeclarelayer{background}
\pgfsetlayers{background,main}

\tikzmath{
    \stateSeperator=5px;
    \firmSeperator=0.75cm;
    \valueMultiplier=2/3 px;
}

\tikzset{
    baseblock/.style n args = {1}{
        minimum width=0.55cm,
        inner sep=0,
        outer xsep=0,
        outer ysep=0,
        minimum height=#1*\valueMultiplier,
    },
    asset/.style n args = {1}{
        baseblock={#1},
        fill=wtf-red,
        anchor=south west,
    },
    new-asset/.style n args = {1}{
        asset={#1},
        anchor=south,
        postaction={pattern=north east lines, pattern color=white},
    },
    debt/.style n args = {1}{
        baseblock={#1},
        fill=wtf-blue,
        anchor=south west,
    },
    new-debt/.style n args = {1}{
        debt={#1},
        anchor=south,
        postaction={pattern=north east lines, pattern color=white},
    },
    lost-debt/.style n args = {1}{
        debt={#1},
        anchor=south,
        opacity=0.15,
    },
    equity/.style n args = {1}{
        baseblock={#1},
        fill=wtf-orange,
        anchor=south
    },
    new-equity/.style n args = {1}{
        equity={#1},
        anchor=south,
        postaction={pattern=north east lines, pattern color=white},
    },
    firm/.style n args = {1}{
        inner sep=0,
        label=below:\footnotesize #1,
    }
};

\tikzmath{
    \assetUp = 80;
    \assetDown = 40;
    \debtUp = 60;
    \debtDown = \assetDown;
    \equityUp = \assetUp - \debtUp;
    \equityDown = 0;
    \projectUp = 18;
    \projectDown = 10;
    \legacyDebtDownShare=0.5;
}

\coordinate (cursor) at (0, 0);

%%%% First block:
\coordinate (start) at (cursor);
\coordinate (block-south-west) at (cursor);
\node (asset) [asset={(\debtUp+\equityUp)}] at (cursor) {};
\coordinate (block-north-west) at (asset.north west);
\node (debt) [debt={\debtUp}] at (asset.south east) {};
\node (equity) [equity={\equityUp}] at (debt.north) {};
\coordinate (cursor) at ([xshift=\stateSeperator]debt.south east);

\node (label) at ([yshift=1em]asset.north east) {\scriptsize$\strut$Up};

\node (asset) [asset={(\debtDown)}] at (cursor) {};
\node (debt) [debt={\debtDown}] at (asset.south east) {};
\node [lost-debt={(\debtUp-\debtDown)}] at (debt.north) {};
\coordinate (block-south-east) at (debt.south east);

\node at (asset.north east |- label) {\scriptsize$\strut$Down};

\node[firm=Pre-project,
    fit=(block-south-west) (block-south-east) (block-north-west)] (firm-base) {};

%%%% Second block:

\coordinate (cursor) at ([xshift=\firmSeperator]firm-base.south east);
\coordinate (block-south-west) at (cursor);
\node (asset) [asset={\assetUp}] at (cursor) {};
\node (new-asset) [new-asset={\projectUp}] at (asset.north) {};
\coordinate (block-north-west) at (new-asset.north west);
\node (debt) [debt={\debtUp}] at (asset.south east) {};
\node (equity) [equity={(\equityUp-5)}] at (debt.north) {};
\node (new-debt) [new-debt={(\projectUp+5)}] at (equity.north) {};
\coordinate (cursor) at ([xshift=\stateSeperator]debt.south east);

\node (asset) [asset={\assetDown}] at (cursor) {};
\node (new-asset) [new-asset={\projectDown}] at (asset.north) {};
\node (debt) [debt={(\debtDown + \projectDown*\legacyDebtDownShare)}] at (asset.south east) {};
\coordinate (block-south-east) at (debt.south east);
\node (new-debt) [new-debt={(\projectDown*(1-\legacyDebtDownShare))}] at (debt.north) {};
\node [lost-debt={(\debtUp-\debtDown+\projectUp+5-\projectDown)}] at (new-debt.north) {};

\node[firm=Debt funding,
    fit=(block-south-west) (block-south-east) (block-north-west)] (firm-debt) {};


%%%% Third block:

\coordinate (cursor) at ([xshift=\firmSeperator]firm-debt.south east);
\coordinate (block-south-west) at (cursor);
\node (asset) [asset={\assetUp}] at (cursor) {};
\node (new-asset) [new-asset={\projectUp}] at (asset.north) {};
\coordinate (block-north-west) at (new-asset.north west);
\node (debt) [debt={\debtUp}] at (asset.south east) {};
\node (equity) [equity={(\equityUp-8)}] at (debt.north) {};
\node (new-equity) [new-equity={(\projectUp+8)}] at (equity.north) {};
\coordinate (cursor) at ([xshift=\stateSeperator]debt.south east);

\node (asset) [asset={\assetDown}] at (cursor) {};
\node (new-asset) [new-asset={\projectDown}] at (asset.north) {};
\node (debt) [debt={(\debtDown + \projectDown)}] at (asset.south east) {};
\node [lost-debt={(\debtUp-\debtDown-\projectDown)}] at (debt.north) {};
\coordinate (block-south-east) at (debt.south east);

\node[firm=Equity funding,
    fit=(block-south-west) (block-south-east) (block-north-west)] (firm-equity) {};


%%%% Fourth block:

\coordinate (cursor) at ([xshift=\firmSeperator]firm-equity.south east);
\coordinate (block-south-west) at (cursor);
\tikzmath{
    \projectPrice=(\projectUp+\projectDown)/2;
}
\node (asset) [asset={(\assetUp-\projectPrice)}] at (cursor) {};
\node (new-asset) [new-asset={\projectUp}] at (asset.north) {};
\coordinate (block-north-west) at (new-asset.north west);
\node (debt) [debt={\debtUp}] at (asset.south east) {};
\node (equity) [equity={(\equityUp-\projectPrice+\projectUp)}] at (debt.north) {};
\coordinate (cursor) at ([xshift=\stateSeperator]debt.south east);

\node (asset) [asset={(\assetDown-\projectPrice)}] at (cursor) {};
\node (new-asset) [new-asset={\projectDown}] at (asset.north) {};
\node (debt) [debt={(\debtDown - \projectPrice + \projectDown)}] at (asset.south east) {};
\node [lost-debt={(\debtUp-\debtDown + \projectPrice -\projectDown)}] at (debt.north) {};
\coordinate (block-south-east) at (debt.south east);

\node[firm=Cash funding,
    fit=(block-south-west) (block-south-east) (block-north-west)] (firm-cash) {};


\coordinate (origin) at ([xshift=-0.5cm]firm-base.south west);
\coordinate (x-limit) at ([xshift=0.5cm]firm-cash.south east);
\begin{pgfonlayer}{background}
    \draw [thick, dotted] 
        (origin) -- 
        (x-limit)
    ;
    \draw [thick, dotted]
        (origin) --
        ++(0, 2.6cm)
    ;
    \draw [dotted] 
        ([yshift=\assetUp*\valueMultiplier]origin) -- 
        ([yshift=\assetUp*\valueMultiplier]x-limit)
    ;
    \draw [dotted] 
        ([yshift=\assetDown*\valueMultiplier]origin) -- 
        ([yshift=\assetDown*\valueMultiplier]x-limit)
    ;
\end{pgfonlayer}

\node[fit=(firm-base) (firm-debt) (firm-equity) (firm-cash)] (main) {};

\tikzset{
    legend/.style n args = {1}{
        minimum width=0.4cm,
        minimum height=0.4cm,
        inner sep=0,
        outer xsep=0,
        outer ysep=0,
        label=right:\scriptsize#1,
        anchor=center,
    },
};

\node [
    matrix, 
    column sep = 1cm,
    row sep=0,
    inner sep=3px,
    nodes={minimum height=0.5cm},
] (labels) at ([yshift=-1.5cm]main.south) {
    \node[asset=0, legend=Legacy Assets] {}; &
    \node[debt=0, legend=Legacy Debt] {}; &
    \node[equity=0, legend=Legacy Equity] {}; \\
    \node[new-asset=0, legend=New Project] {}; &
    \node[new-debt=0, legend=New Debt] {}; &
    \node[new-equity=0, legend=New Equity] {}; \\
    & \node[lost-debt=0, legend=Lost Face Value] {};\\
};

            \end{tikzpicture}
            }
        \end{center}
        \note[item]<1>{
            We presented a simple structural model for capturing the effects of funding costs
            when a firm obtained projects.
            In the figure, a firm is operating in a single-period binomial tree, 
            and it considers obtaining a project yielding some payoff.
            This figure can to some extent explain most of the analysis we did.
        }
        \note[item]<1>{
            We considered three different funding methods:
            \begin{enumerate}
                \item \textit{Debt funding}
                    where the firm issues debt to new creditors.
                \item \textit{Equity funding} 
                    where the firm issues new shares.
                \item \textit{Cash funding} 
                    where the firm uses existing cash from its balance sheet.
            \end{enumerate}
        }
        \note<2>{
            The workings of the different funding methods:
            \begin{enumerate}
                \item Under debt funding, the new creditors charge a credit spread.
                      Shareholders pay this credit spread when the firm does not default,
                      and in the no-default state they receive a lower payoff.
                \item Under equity funding, the required return from the new shareholders
                      is even larger than that of the new creditors. 
                      New shareholders are not compensated in the default state;
                      hence they require a higher payoff in the no-default state.
                      Otherwise, this would not be a zero NPV investment for them.
                      The position of the legacy shareholders then deteriorates
                      compared to the the pre-project value.
                \item Under cash funding, there is no financing cost 
                      since the firm "borrows" from its balance sheet at the risk-free rate.
                      When the project's payoff is positively correlated with the no-default event,
                      the project essentially moves asset value 
                      from the default state to the no-default state. 
                      The shareholders end up with a higher payoff in the no-default state.
            \end{enumerate}
        }
        \note<3>{
            \begin{enumerate}
                \item \textit{The pecking order of financing preferences:}
                      Since borrowing cash from the balance sheet is the cheapest method of funding,
                      the shareholders will prefer this method.
                      As new shareholders gain no payoff in the default state
                      when issuing new shares,
                      they require a higher return than the new creditors do
                      when issuing new debt.
                      In other words, the legacy shareholders lose more wealth
                      to the funding entity by equity funding than debt funding.
                      Hence, legacy shareholders prefer debt funding over equity funding.
                \item \textit{The Free Rider Problem:}
                      Evidently, the shareholders are the ones financing the funding of the project
                      since they pay the return of the funding entity.
                      However, the shareholders do not reap the benefits of the project.
                      % cspell: ignore HW
                      This is why HW's example does not work in reality. 
                      If the project's price were to be reduced,
                      the blue and yellow striped bars would shrink,
                      but that would not necessarily make the shareholders content.
                \item \textit{The use of FVAs:}
                      A firm preserving its shareholders' valuation should not accept
                      buying the project at the price that leads to the payoffs in the graphic.
                      By applying FVAs to its valuation the firm can ensure that
                      the wealth of its shareholders is preserved,
                      as an FVA considers the funding costs that the shareholders incur.
                      Hence, FVAs are a way for the firm 
                      to align its incentives with its shareholders.
            \end{enumerate}
        }
        \note<4>{
            The size of the adjustment depends on a couple of factors:
            \begin{enumerate}
                \item The funding method due to the pecking order.
                \item How much the project's price is discounted in the first place. 
                \item How the projects payoff is structured. 
                      The correlation between the payoff and the default event 
                      has an impact on the shareholders' wealth loss.
                      The shareholders would rather not finance projects with cash flows
                      that are paid out when the firm defaults. 
            \end{enumerate}
        }
    \end{frame}

\end{document}

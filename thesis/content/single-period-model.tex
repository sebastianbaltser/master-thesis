% !TEX root = ../main.tex
\documentclass[../main.tex]{subfiles}

\begin{document}
    \section{Funding Costs in the Single Period Model}
        In order to develop the necessary understanding of funding costs and their implications to a firm's stakeholders, this section will explore them using the single period framework.
        The framework is a drastic simplification of reality but still it can provide useful results concerning funding value adjustments.

    \subsection{The single period model}
        The single period framework is modelling what will happen in an economy as it transitions from the present, time 0, to one time period ahead, time 1.
        At time 1 the economy can materialize in one of $N$ possible states, $\omega_{1}, \dots, \omega_{N}$, and each state occur with a strictly positive probability.
        A state is defined by the price of an Arrow-Debreu security paying one unit of numeraire if the state is achieved and zero otherwise.
        The price of this security is referred to as the state price and denoted $\psi_{i}$.
        The vector of state prices, $\psi \equiv \left(\psi_{1}, \dots, \psi_{N}\right)'$, can be used to price claims in the economy, since a state price defines the value of receiving a cash flow in a particular state.
        For simplicity, all investors are assumed to be in possession of the same market information.

        To characterize the market valuation of financial derivatives represented on a firm's balance sheet, the finite set, $\mathcal{L}$, of payoffs at time 1 is assigned a fair value at time 0 by some "fair market value"-function $V\, \colon \mathcal{L} \rightarrow \mathbb{R}$.
        Two required assumptions are imposed on the market valuation assignment: 
        (i) $V(\cdot)$ is linear, meaning that the value of a portfolio holding different cash flows is the sum of the values of the elements of the portfolio, and 
        (ii) $V(\cdot)$ is increasing in payoffs, meaning that if payoff $X$ is greater than or equal to payoff $Y$ in all $N$ states, and if $X>Y$ in some states, then $V(X)>V(Y)$.
        
        The price of a payoff $d=\left(d_{1}, \dots, d_{N}\right)'$ is given as:
            \begin{equation}
                \pi(d) = d'\psi = \sum_{i=1}^{N} d_{i}\psi_{i}
            \end{equation}
        Especially a zero coupon bond surely paying one unit of numeraire in every state has the price:
            \begin{equation}
                d_{0} = \sum_{i=1}^{N} \psi_{i}
            \end{equation}
        from which the risk free rate can be defined as:
            \begin{equation}
                r_{f} = \frac{1}{d_{0}} - 1
            \end{equation}
        The product between the gross risk free rate and the state price vector produces the risk neutral probability distribution, $\mathbb{Q}$, i.e. $q_{i} = (1 + r_{f})\psi_{i}$.

        In an OTC market, the market valuations of a derivative $i$ need not coincide with the price, $U_{i}(q)$, at which dealers are trading this derivative.
        At almost the same point in time, the same asset can be traded at several different prices reflecting distinct bids and asks of different dealers.
        The associated deviation in prices is partly explained by search costs, differences in dealer-client relationships, and differences in the dealers' capital structure.

    \subsection{Firm capital structure}
        Consider now a firm operating in a single period economy with $S$ states and state prices.
        The firm can invest in assets with risk but needs financing for supporting it's investments.
        Assume that the firm has no cash, such that financing can only be obtained at two different sources, namely equity funding through distribution of the firm's own stocks or issuance of debt.
        The sum of the amount of equity funding obtained, $S_{0}$, and the amount of debt issued, $D_{0}$, decides the value of risky assets the firm can invest in, $A_{0}$.
        The value of the firm at time $1$ is given by the random variable $A$.
        At time 1 the creditors are promised to receive the face value of the debt $D_{FV}$ earning an interest rate on their investment.
        In the event that the value of the firm $A$ is not sufficient to pay the face value in it's entirety, the firm defaults and the creditors takes over the remaining estate. 
        The firms' default event will be denoted by $\mathcal{D}$.
        The payoff to the creditors is thus a random variable denoted by $D$.
        In a default event the firm can suffer distress costs due to for example liquidation of its assets.
        The remaining estate after a default depends on the recovery parameter $\kappa \in (0;1]$, such that the remaining estate is $\kappa A$.
        Shareholders receives the remainder of the firm value after the debt has been paid and in particular receives nothing when the firm defaults.
        The shareholders' random payoff is denoted by $S$.
        Hence, the payoff of creditors and shareholders is given respectively by:
            \begin{gather}
                D = \min\left(
                        D_{FV},\; A
                    \right)
                    = D_{FV} - \max\left(
                        D_{FV} - A, 0
                    \right) \label{eqn:legacy-creditor-payoff}\\
                S
                    = \max\left(
                        A - D_{FV},\; 0
                    \right)
            \end{gather}
        It is worth noting that the creditors security, i.e. the debt claim, can be replicated by buying a riskless bond with face value $D_{FV}$ and selling a put option on the firm with strike $D_{FV}$.
        The shareholders claim is equivalent to buying a call option on the firm with strike $D_{FV}$.

        After deciding it's capital structure, the firm has an opportunity to engage in a new project, e.g. buying or selling a security.
        The project has a, possibly random, payoff denoted by $Y$, a value denoted by $Y_{0}$, and a price denoted by $\pi(Y)$.
        If the project is fairly priced the price equals the value, i.e. $\pi(Y) = Y_{0}$, but this is not necessarily the case in the following examples.
        In order to invest in the project the firm must pay the upfront price, $\pi(Y)$, or receive it if it's negative.
        If the upfront is a payable, the firm must be able to finance it or if it is a receivable, the firm can use it to retire debt.
        The issues to examine is the funding costs or benefits of entering into the project.
        One issue is how the funding costs or benefits affect the value of the project as experienced by the firm and whether a funding value adjustment should be made to the price of the projects.
        Another issue is how the how the project and it's associated funding impacts the firm value and the value of its stakeholders claims.

        These issues will be assessed by the means of examples in the following sections.

        \subfile{shareholders-financing-costs}

    \subsection{Funding costs of risk free project}
        Consider the economy defined by a single period model with $5$ states, and the following properties:
            \begin{table}[H]
                \centering
                \begin{tabular}{l|rrrrr}
                    $i$ & 1 & 2 & 3 & 4 & 5 \\
                    \hline
                    $\psi_{i}$ & $0.06$ & $0.24$ & $0.29$ & $0.28$ & $0.12$ \\
                \end{tabular}
            \end{table}
        implying a discount factor of $d_{0} = 0.99$ and a risk free interest rate of $r_{f} = 1.0101\%$.
        A firm operates in this economy and invests in risky assets which are funded by equity and debt deposits and returns payoff specified shortly.
        The firm has been funded by debt such that the face value of debt is $D_{FV} = 80$, which gives rise to the following payoff structure:
        \begin{table}[H]
            \centering
            \begin{tabular}{l|rrrrr}
                $i$ & 1 & 2 & 3 & 4 & 5 \\
                \hline
                $A(\omega_{i})$ & 120 & 110 & 100 & 95 & 60 \\
                $D(\omega_{i})$ & 80 & 80 & 80 & 80 & 60 \\
                $S(\omega_{i})$ & 40 & 30 & 20 & 15 & 0
            \end{tabular}
        \end{table}
        The associated present values of the payoffs, $A$, $D$ and $S$, are discounted expected value of the payoff with respect to the risk neutral probability measure:
            \begin{gather}
                \pi(A) = \frac{1}{1+r_{f}} \mathbb{E}^{\mathbb{Q}}\left[A\right] = 96.40 \\
                \pi(D) = \frac{1}{1+r_{f}} \mathbb{E}^{\mathbb{Q}}\left[D\right] = 76.80
                \qquad \pi(S) = \frac{1}{1+r_{f}} \mathbb{E}^{\mathbb{Q}}\left[S\right] = 19.60
            \end{gather}

        Assume that the firm faces a new risk free project in which it can invest.
        The project is risk free in the sense that the payoff is known with certainty at time 0, so:
            \begin{table}[H]
                \centering
                \begin{tabular}{l|rrrrr}
                    $i$ & 1 & 2 & 3 & 4 & 5 \\
                    \hline
                    $Y(\omega_{i})$ & $10$ & $10$ & $10$ & $10$ & $10$
                \end{tabular}
            \end{table}
        which implies a present value of $Y_{0} = 15d_{0} = 9.90$, since the project is merely an investment in the risk free asset.
        Obtaining the payoff requires an upfront cost equal to the fair value of the project, $\pi(Y) = Y_{0}$.
        The firm can either obtain the funding required by issuing debt or by selling equity.
        The former case is examined in the following section and the latter in the section after that.

    \subsubsection{Funding by debt issuance}
        In order to finance the upfront of the project, $\pi(Y)$, the firm issues debt to new creditors, specifically the price of the debt, denoted $\pi(\tilde{D})$, should equal the price of the project.
        Assume that the new debt ranks pari passu with the legacy debt, such that all creditors experience the same loss rate in states where the firm defaults.
        A claim with face value $\tilde{D}_{FV}$ which ranks pari passu to another claim with face value $D_{FV}$ has the random recovery rate given as:
            \begin{align}
                \tilde{\rho} = \min\left(
                    1,\;
                    \frac{A}{\tilde{D}_{FV} + D_{FV}}
                \right)
            \end{align}
        If the asset value is larger than the total face value to be repaid the firm does not default and the recovery rate is $1$.
        If the firm defaults the recovery rate is equal to the share of the total face value remaining as asset value.
        The random payoff of the pari passu debt is denoted by $\tilde{D}$ and given as:
            \begin{align}
                \tilde{D}
                    = \tilde{\rho}\tilde{D}_{FV}
            \end{align}

        Due to the credit risk of the firm, the new creditors will require a credit spread on the debt in addition to the risk free return.
        This additional interest has implications for the firm's perceived value of the project, which raises the problem for debate about whether the price should be reduced to accommodate the firm's assessment of the value.

        In order for the firm to attract new creditors they must offer a large enough interest rate on the debt, such that buying the debt is a zero net present value investment.
        Therefore the face value must be chosen to solve the following equation:
            \begin{align}
                \pi(Y) &= \pi(\tilde{D}) \\
                \Leftrightarrow  \qquad
                9.90 &= \mathbb{E}^{\mathbb{Q}}\left[\tilde{\rho}\right] \tilde{D}_{FV} \\
                \Leftrightarrow  \qquad
                \tilde{D}_{FV} &= 10.28
            \end{align}
        This face value implies an interest rate of $10.28 / 9.90 - 1 = 3.84\%$
        and therefore a credit spread of $\tilde{s} = 3.84\% - r_{f} = 2.83\%$.
        Entering into the project and issuing new debt with face value $10.28$ alters the payoffs associated with the firm in the following way:
        \begin{table}[H]
            \centering
            \begin{tabular}{l|rrrrr||r}
                $i$ & 1 & 2 & 3 & 4 & 5 & Present value \\
                \hline
                $A(\omega_{i})$ & $130$ & $120$ & $110$ & $105$ & $70$ & $106.30$ \\
                $D(\omega_{i})$ & $80$ & $80$ & $80$ & $80$ & $62.03$ & $77.04$ \\
                $S(\omega_{i})$ & $19.72$ & $29.72$ & $14.72$ & $39.72$ & $0$ & $19.36$ \\
                $\tilde{D}(\omega_{i})$ & $10.28$ & $10.28$ & $10.28$ & $10.28$ & $7.97$ & $9.90$ \\
            \end{tabular}
        \end{table}

        Investing in the project increases the present value of the firm's assets by $106.30 - 96.40 = 9.90$, which, not surprisingly, is the value of the the project.
        More interesting is the impact to the shareholder's claim that decreases by an amount $19.36 - 19.60 = -0.24$, why investing in the project is of negative value to the shareholders.
        When the firm defaults the shareholders still receive nothing, but in all other states where the firm does not default, the shareholders pay the present value of the interest owed to the new creditors due to the credit risk in the firm.
        The observed decrease in shareholder value aligns well with 
        \cref{eqn:marginal-shareholder-value-debt-financing}. 
        The risk neutral default probability is 
        $\mathbb{P}^{\mathbb{Q}}\left(\mathcal{D}\right) 
            = \mathbb{P}^{\mathbb{Q}}\left(\omega_{5}\}\right) 
            = 12.12\%$,
        why $p^{\mathbb{Q}} = 87.88\%$.
        The promised marginal profit of the new project is:
        $\pi = 0.99 * 10 - 9.90 = 0$.
        The payoff is constant, so the default event and the payoff have zero covariance. 
        The expected default loss to creditors is 
        $\mathbb{E}^{\mathbb{Q}}\left[\phi\right] = 2.72\%$,
        and the limiting spread is:
        \begin{equation}
            S 
            = \frac{
                2.72\% * (1 + \rfrate)
            }{
                1 - 2.72\%
            } 
            = 2.83\%
        \end{equation}
        The marginal valuation to the legacy creditors can then be calculated as:
        \begin{equation}
            \Phi 
            = 87.88\% * \frac{1}{1+\rfrate}  * 9.90 * 2.83\% 
            = 0.2435
        \end{equation}

        The loss of the shareholders can then be calculated as: 
            \begin{align}
                G = 87.88\%*0 - \frac{1}{1+\rfrate} * 0 - 0.2435 = -0.2435
            \end{align}
        This is also the fair price, with opposite sign, 
        of a security that pays the promised return of the new debt in every no default state and zero otherwise. 
        While the project is a negative net present value investment for the shareholders,
        the investment is a positive net present value investment for the legacy creditors.
        When the firm does not default they still receive their promised payoff corresponding to the face value of their debt.
        The new project has increased the asset base of the firm, and when the firm defaults, 
        the legacy creditors will share part of the payoff from the option with the new creditors.
        Hence, the legacy creditors receive a larger payoff when the firm defaults. 

        The Modigliani-Miller invariance proposition assures that making a zero net present value investment 
        does not increase the value of the firm.
        Therefore, the wealth lost by the shareholders must be entirely transferred to the legacy creditors 
        increasing the present value of their debt claim by $-G = 0.2435$. 
        This is also the observed effect when comparing the new present value with the old 
        $77.04 - 76.80 = 0.24$.

        The wealth transfer from shareholders to legacy creditors due to the new project 
        suggests defining and quantifying the \FVA/ of the project. 
        According to \cite{ADS2019} there are multiple ways of calculating \FVA/ used in practice and theory,
        and a couple of viable definitions will be explained in the following.

        Since the firm borrows funds for the project at a credit spread in excess of the risk free rate, 
        the shareholders pays an additional rate when the firm does not default. 
        This is a form of funding costs and argues in favor of defining the \FVA/ as
        the present value of the funding costs in excess of the risk free rate paid by the shareholders: 
        \begin{align}
            \FVA/ 
            &= \discountfactor \left(
                U (1 + \rfrate + \tilde{s})
                - U (1 + \rfrate)
            \right)
            = \discountfactor U \tilde{s} \\
            &= 0.99 * 9.90 * 2.83\%
            = 0.2771
        \end{align}
        Since the entire credit spread is compensation for credit risk ,
        the \FVA/ defined here is the compensation provided to the creditors,
        for the possibility that the firm might default. 
        If the firm defaults, this amount is not to be paid and the \FVA/ is therefore, in addition, 
        equal to the expected benefit to the firm from defaulting, also known as the DVA. 
        That the \FVA/ equals the DVA when credit spreads are merely compensation for credit risk, is shown by \cite{HullWhiteFVA}.

        The FVA could also be defined by the quantity $\Phi$ in \cref{eqn:marginal-shareholder-value-debt-financing}, 
        i.e. the marginal valuation of the new project to the firm's legacy creditors. 
        This value is also equal to the wealth transfer from the shareholders to the creditors. 
        Compared to the previous definition of FVA, 
        $\Phi$ captures the expected funding cost as opposed to the excess funding cost.

        Using this definition, the FVA is equal to:
            \begin{equation}
                \FVA/ = \Phi = 0.2435
            \end{equation}
        The two quantities differ by the no default probability, such that 
        $0.2771 * (1-\mathbb{P}\left(\mathcal{D}\right)) = 0.2435$.

    \subsubsection{Funding by share issuance}
        Assume that the premium paid by the firm to the counterparty is funded with new equity instead of taking on debt from a new creditor. 
        Doing so will create a dilution, and the value of the already existing shares, owned by the legacy shareholders, will be reduced. 
        The counterparty of the new project which the firm enters is still supposed to be risk free, meaning that the payoff at time 1, $Y$, is known with certainty. 
        What differences from funding a project with new creditors does this take have?

        The new shareholders will subscribe to the newly issued shares only to the extent that the investment has a net present value of zero; paying an amount equal to the premium of the derivative, $\pi(Y)$:
        \begin{equation}\label{eqn:derivative-zero-npv}
            \pi(Y) = \pi(\tilde{S})
        \end{equation}
        Intuitively, the total asset value will then increase by the fair value of the derivative, and, all things being equal, the project will result in more value left for the creditors in the default state. 
        The creditors have not had anything changed in their deal with the firm, and the increment in asset base due to the receivables of the derivative will be added to their payoff if the firm defaults. 
        This reduces their loss rate as well as the credit spread.
        The creditors are then better off, since the market value of the debt increases when the project is funded by new equity.\\
        The legacy shareholders, on the other hand, are worse off by an amount equal to the increment of the debt value, and the only change of structure is the transfer of wealth from the legacy shareholders to the legacy creditors. 
        The loss of value in the firm owned by the legacy shareholders comes from the dilution, where their share of the payoff has fallen without any compensation in form of capital. 
        The wealth transferred is significantly higher than in the case of debt funding because the legacy creditors are not obliged to share the payoff of the derivative with other new creditors in the case of default. 
        The derivative contract has no impact on the total asset value of the firm, and the funding cost are borne by the legacy shareholders.

        Assume a firm enters a risk free project with a known payoff at time 1 of $Y=10$. 
        The present value of the derivative is then $10/(1+0.10)=9.90$. 
        The payoff of the derivative is added to the firm's total asset value. 
        While the payoff of the legacy creditors, with a face value of 80, only change in the default state, the payoff of the shareholders is computed slightly differently than in the case of debt issuance. 
        It's now essential to know the share of the payoff, $\alpha$, which the new equity owners are entitled to, remembering that the payoff obtained by the legacy creditors is given as in \cref{eqn:legacy-creditor-payoff}. 
        The difference between the total asset value of the firm, and the payoff of the creditor is what's left to be distributed to the shareholders, who are now separated into two bases. 
        Compute the expected discounted value of this and divide it by the net present value of the investment made by the new shareholders:
        \begin{equation}
            \alpha = \frac{\sum_i (A_{i}-D{i})\psi_{i}}{\pi(Y)}
        \end{equation}
        The new shareholders then receive their share of the remaining payoff in each state, and the payoff of the legacy shareholders is still the residual with a share of $(1-\alpha)$ of the remainder of the distributed payoff that is left. 
        The results of the example are shown in \cref{tbl:equity-funding-payoff}, where the share of the new equity owners is computed as $\alpha = 34.982\%$.

        \begin{table}[h]
            \centering\begin{tabular}{l|rrrrr||r}
                $i$ & 1 & 2 & 3 & 4 & 5 & Present value \\
                \hline
                $A_{i}$ & 130 & 120 & 110 & 105 & 70 & 106.30 \\
                $D_{i}$ & 80 & 80 & 80 & 80 & 70 & 78 \\
                $S_{i}$ & 32.51 & 26.01 & 19.51 & 16.25 & 0 & 18.40 \\
                $\tilde{S}_{i}$ & 17.49 & 13.99 & 10.49 & 8.75 & 0 & 9.90 \\
            \end{tabular}
            \label{tbl:equity-funding-payoff}
        \end{table}

        After the transaction, the firm's balance sheet has increased with an amount equal to the premium of the derivative. 
        The firm has a new asset on the asset side, the derivative receivable worth 9.90, and that is funded by 9.90 worth of new equity.

        The project is a negative net present value investment for the legacy shareholders. The decrease in the equity value is computed as:
        \begin{equation}
            \Delta \pi(S) = 18.40 - 19.60 = -1.20
        \end{equation}
        This amount is also what is referred to as the transfer of wealth, $\Delta W$, when no excess yield/donation is received on the derivative. 
        The wealth transfer is higher than in the case of debt funding, hence share holders will always prefer debt funding over equity funding. 
        This result is also justified by the \textit{pecking order theory} as described by \cite{ADS2019}: "Proposition A1: Suppose that the firm's probability of default is not zero and that the marginal investment cost is strictly positive. The marginal value to the firm's existing shareholders of financing the investment with existing cash is strictly higher than the marginal value under debt financing, which in turn is strictly higher than the marginal value under equity financing."

        The project is a positive net present value investment for the creditors. 
        Their increase in wealth is computed as:
        \begin{equation}\label{eqn:wealth-transfer-equality}
            \Delta \pi(D) = -\Delta \pi(S) = 78.00 - 76.80 = 1.20
        \end{equation}
        The value of the debt increases, since the creditors get the entire derivative's receivables in the default state.\\
        A possible donation by the counterparty would also have to be significantly higher than in the case of debt funding for the legacy shareholders' equity value to be left unaffected by the investment. 

        The marginal impact on the firm's balance sheet of entering the project is given by:
        \begin{equation}
            \pi(A) + \pi(Y) = (\pi(S) + \Delta \pi(S) + \pi(\tilde{S})) + (\pi(D) + \Delta \pi(D))
        \end{equation}
        where $\pi(\tilde{S})$ is the market value of the newly issued equity. 
        \Cref{eqn:wealth-transfer-equality} implies that:
        \begin{equation}
            \pi(A) + \pi(Y) = (\pi(S) + \pi(\tilde{S})) + \pi(D)
        \end{equation}
        As mentioned in \cref{eqn:derivative-zero-npv}, the firm assumes to raise an amount of new equity equal to the premium of the derivative. 
        This means that the Modigliani-Miller invariance proposition holds, and also shows, that regardless of the strategy of funding a financial project, the funding costs has no impact on the valuation of the derivative nor the asset base of the firm itself.

        When trading decisions are made, the firm's preferences are assumed to be determined by the shareholders.
        However, in the case of funding by share issuance, it's clear that the legacy shareholders' position deteriorates, as they bear the funding costs.\\
        An adjustment to the value of the derivative would require a donation by the counterparty, such that the value of derivatives' receivables would be worth more than the premium.
        This should only go to an extent so that the legacy shareholders become indifferent of whether the the firm enters the project or not.
        But whether the counterparty has incentives to do so requires more information about their capital structure, and will remain indeterminate.
        The project should not be adjusted for funding costs, and the value of the firm is not affected by the funding costs required by derivative receivables.\\
        These conclusions hold for both types of funding, but the amount of wealth transfer does depend on the instrument used. Shareholders are worse off with share issuance than with debt issuance.
        The availability of cash in the firm's deposits would solve the issue faced by the shareholders:
        Suppose the firm uses their own cash to purchase the derivative from the counterparty at a market fair value.
        This is a zero net present value investment for the firm, where, in addition, no wealth is transferred from the shareholders, and neither does the creditors become better off. 

\end{document}
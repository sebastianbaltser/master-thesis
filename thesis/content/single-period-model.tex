% !TEX root = ../main.tex
\documentclass[../main.tex]{subfiles}

\begin{document}
    \section{Funding Costs in the Single Period Model}

    \section{Equity funding}
        We now assume that the premium paid by the bank to the counterparty is funded with new equity instead of taking on debt from a new creditor. Obviously this will create a dilution, and the value of the already existing shares owned by the legacy shareholders will reduce. We still suppose that the counterparty of the new project, that the bank enters is risk free meaning that the payoff at time 1, $Y$ is known with certainty. What differences from funding a project with new creditors does this take have? 

        The new shareholders will subscribe to the newly issued shares only to the extent that the investment has a net present value of zero, paying an amount equal to the premium of the derivative, $\pi(Y)$:
        \begin{equation}
            \pi(Y) = \pi(\tilde{S})
        \end{equation}
        Intuitively, the total asset value will then increase by the fair value of the derivative, and all things being equal the project will result in more value left for the creditors in the default state. The creditors have not had anything changed in their deal with the bank, and the increment in asset base due to the receivables of the derivative will seem like extra "free cash" for them to collect if the bank defaults. This reduces their loss rate as well as the credit spread. The creditors are then better off, since the market value of the debt increases when the project is funded by new equity.\\
        The legacy shareholders are on the hand worse off by an amount equal to the increment of the debt value, and the only change of structure is the transfer of wealth from the legacy shareholders to the legacy creditors. The loss of value in the bank owned by the legacy shareholders comes from the dilution, where their share of the payoff has fallen without any compensation in form of capital. The wealth transferred is significantly higher than in the case of debt funding because the legacy creditors are not obliged to share the payoff of the derivative with other new creditors in the case of default. The derivative contract has no impact on the total asset value of the bank, and the funding cost are borne by the legacy shareholders.

        We continue our example assuming the bank enters a risk free project with a known payoff of $Y=10$. The present value of the derivative is then $10/(1+0.10)=9.90$. To compute the share of the payoff, which the new equity owners are entitled to, 

\end{document}
        
% !TEX root = front-matter.tex
% cSpell:disable
\documentclass[main.tex]{subfiles}

\begin{document}
    \begin{otherlanguage}{danish}
    \thispagestyle{empty}
    \begin{center}
    {\LARGE Resumé}
    \end{center}

    Efter finansielle institutioners vanskeligheder under finanskrisen 
    er opfattelsen af deres fallitrisiko steget, og således er deres låneudgifter øget kraftigt.
    Som følge af dette har finansielle aktiver med finansieringsbehov oplevet et øget fokus, 
    da udgående pengestrømme til deres vedligeholdelse potentielt kan koste dyrt i låneudgifter.
    For at tage højde for dette har finansielle institutioner taget 
    \textit{Funding Value Adjustments}, \FVA/, i anvendelse.

    Sådanne justeringer har medført meget debat. 
    På den ene side har teoretikere kaldt dem brud på traditionel prisfastsættelsesteori.
    Modsat har udøverne af justeringerne set sig nødsaget til at foretage dem,
    da de nødvendigvis må betale låneudgifterne 
    og derfor tage hensyn til dem, når de bestemmer værdien på et aktiv. 

    Formålet med denne afhandling er at opsummere debatten
    samt at forstå, hvordan behovet for \FVA/ opstår, 
    og hvilke konsekvenser deres anvendelse har for en finansiel institution. 

    Afhandlingen konkluderer, at behovet for \FVA/ opstår, 
    når et finansielt aktiv har udgå\-ende pengestrømme, som kræver finansiering fra institutionen.
    Typisk vil aktivers markedsrisiko være afdækket,
    hvilket i sig selv kan reducere finansieringsbehovet, grundet de modsatrettede pengestrømme.
    Dog kan ufuldstændig risikoafdækning eller forskelle i kravene til sikkerhedsstillelse
    skabe uregelmæssigheder i disse pengestrømme, hvorfor institutionen må finansiere forskellen.
    
    Gennem matematiske udledninger vises finansieringsomkostningerne til et finansielt aktiv
    at blive betalt af institutionens aktionærer,
    der, grundet dette, samlet set ender med tab.
    \FVA/ medregner aktionærernes tab i værdifastsættelsen af aktivet
    og er dermed en måde, hvorpå institutionen, der tager investeringsbeslutningen, 
    kan strømline deres incitamenter med sine aktionærer. 
    Resultatet for institutioner med sine aktionærers interesser forrest
    er investeringsbeslutninger, som bedre varetager deres mål og intentioner.
    
    \end{otherlanguage}
\end{document}
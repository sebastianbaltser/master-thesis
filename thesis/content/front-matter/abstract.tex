% !TEX root = front-matter.tex
% cSpell:disable
\documentclass[main.tex]{subfiles}

\begin{document}
    \begin{otherlanguage}{danish}
    \thispagestyle{empty}
    \begin{center}
    {\LARGE Resumé}
    \end{center}

    Efter finansielle institutioners vanskeligheder under finanskrisen 
    er opfattelsen af deres fallitrisiko steget, og således er deres låneudgifter skudt i vejret.
    Som følge af dette har finansielle aktiver med finansieringsbehov oplevet et øget fokus, 
    da udgående pengestrømme til deres vedligeholdelse potentielt koster dyrt i låneudgifter.
    For at tage højde for dette
    har finansielle institutioner tyet til såkaldte finansieringsværdijusteringer.

    Sådanne justeringer har oplevet meget debat. 
    På den ene side har teoretikere kaldt dem brud på traditionel prisfastsættelsesteori.
    Modsat har udøverne af justeringerne set sig nødsaget til at foretage dem,
    da de nødvendigvis må betale låneomkostningerne 
    og derfor tage hensyn til dem, når de bestemmer værdien på et aktiv. 

    Formålet med denne afhandling er at opsummere debatten
    samt at forstå, hvordan behovet for finansieringsværdijusteringer opstår, 
    og hvilke konsekvenser deres anvendelse har for en finansiel institution. 

    Afhandlingen finder at behovet for finansieringsværdijusteringer opstår 
    når et finansielt aktiv har udgående pengestrømme, som kræver finansiering fra institutionen.
    Typisk vil aktivers markedsrisiko være afdækket,
    hvilket i sig selv kan reducere finansieringsbehovet, grundet de modsatrettede pengestrømme.
    Dog kan ufuldstændig risikoafdækning eller forskelle i kravene til sikkerhedsstillelse
    skabe uregelmæssigheder i disse pengestrømme, hvorfor institutionen må finansiere forskellen.
    
    Finansieringsomkostningerne til projektet vises at blive betalt af institutionens aktionærer,
    der, grundet dette, samlet set ender med tab.
    Finansieringsværdijusteringer medregner aktionærernes tab i værdifastsættelsen af projektet
    og er dermed en måde, hvorpå institutionen, der tager investeringsbeslutningen, 
    kan strømline deres incitamenter med sine aktionærer. 
    Resultatet for institutioner med sine aktionærers interesser forrest
    er investeringsbeslutninger, som bedre varetager deres mål og intentioner.
    
    \end{otherlanguage}
\end{document}
% !TEX root = front-matter.tex
% cSpell:disable
\documentclass[main.tex]{subfiles}

\begin{document}
    \begin{otherlanguage}{danish}
    \thispagestyle{empty}
    \begin{adjustwidth}{-1cm}{-1cm}
    \begin{center}
    {\LARGE Resumé}
    \end{center}

    Efter finansielle institutioners vanskeligheder under finanskrisen 
    er opfattelsen af deres fallitrisiko steget og således er deres låneudgifter skudt i vejret.
    Som følge af dette, har finansielle aktiver med finansieringsbehov oplevet et øget fokus, 
    da udgående pengestrømme til deres vedligeholdelse potentielt koster dyrt i låneudgifter.
    For at tage højde for omkostningerne ved at dække disse finansieringsbehov,
    har finansielle institutioner tyet til såkaldte finansieringsværdijusteringer.

    Sådanne justeringer har oplevet meget debat. 
    På den ene side har teoretikere kaldt dem brud på tradionel prisfastsættelsesteori.
    Modsat har udøverne af justeringerne set sig nødsaget til at foretage dem,
    da de nødvendigvis må betale låneomkostningerne 
    og derfor tage hensyn til dem når de bestemmer værdien på et aktiv. 

    Formålet med denne afhandling er at opsummere debatten
    samt at forstå hvordan behovet for finansieringsværdijusteringer opstår 
    og hvilke konsekvenser deres anvendelse har for en virksomhed.
    Sidstnævnte undersøges ved at opstille simple eksempler,
    hvor en virksomhed i diskret tid kan påtage sig et projekt
    hvilket har implikationer for virksomhedernes gældsejere og aktionærer.

    Afhandlingen finder at institutionernes behov for finansieringsværdijusteringer opstår 
    når et finansielt aktiv har udgående pengestrømme som kræver finansiering fra institutionen.
    Typisk vil finansielle institutioner afdække aktivers markedsrisiko,
    hvilket i sig selv kan reducere finansieringsbehov, grundet de modsatrettede pengestrømme.
    Dog kan ufuldstændig risikoafdækning eller forskelle i kravene til sikkerhedsstillelse
    skabe uregelmæssigheder i disse pengestrømme. 
    Når dette sker må institutionen dække forskellen 
    og betale de resulterende finansieringsomkostninger.
    
    Finansieringsomkostningerne til projektet vises at blive betalt af institutionens aktionærer.
    På trods af dette oplever disse ikke værdiforøgelsen af projektet
    og står derfor samlet set til et tab.
    Ved at foretage finansieringsværdijusteringer tages der højde for aktionærernes tab,
    sådan at institutionen ikke optager projekter der forringer aktionærernes værdi.
    Finansieringsværdijusteringer er altså en måde hvorpå institutionen, 
    der tager investeringsbeslutninger, kan strømline deres incitamenter med sine aktionærer. 
    Resultatet for institutioner, med sine aktionærers interesser forrest,
    er investeringsbeslutninger, som bedre varetager deres mål og intentioner.
    
    \end{adjustwidth}
    \end{otherlanguage}
\end{document}
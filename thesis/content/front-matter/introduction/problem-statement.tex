% !TEX root = sub-main.tex
\documentclass[main.tex]{subfiles}

\begin{document}
    \subsection{Research Questions}
        
    Evidently, \FVA/ is a disputed topic 
    but one that has been cemented as an important valuation adjustment by practitioners.
    However, as the debate about its validity has never been settled,
    it is a difficult topic to study with many sources pointing in different directions.
    Even if one decides to accept \FVA/, it is a complex matter,
    whose exact calculation, and perhaps even definitions, is still under discussion.

    This paper will attempt to collect the pieces of the \FVA/ topic
    and present them in a way that can properly explain what \FVA/ is and what it is accounting for.
    This necessarily involves describing the debate about the relevancy of the adjustment
    by presenting arguments from both sides of the controversy 
    and understanding the standpoint of each side.
    To draw conclusions on the debate, it will also be essential to properly define the concept.
    However, to really grasp the topic, the thesis will seek to define \FVA/ in such simple terms 
    that the essence is not lost in practicalities and implementation details.

    These ideas, and the research they involve, can be expressed by the following research questions:
    \begin{enumerate}
        \item \researchQuestionFundingCosts
        \item \researchQuestionFvaDebate
        \item \researchQuestionFvaImplications
    \end{enumerate}

\end{document}
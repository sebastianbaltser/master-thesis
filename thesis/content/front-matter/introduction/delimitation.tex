% !TEX root = sub-main.tex
\documentclass[main.tex]{subfiles}

\begin{document}
    \subsection{Delimitation}

    To answer the research questions, the paper will generally aim for simple models,
    that give acquaintance with- and insight into funding costs and \FVA/s.
    This familiarity with the topic could then be brought into more complex modelling
    in further research.

    The first research question will be answered in a general, conceptual, and non-tech\-nical way.
    There are potentially many mechanisms that lead to funding costs,
    but it is not the intention to provide a comprehensive list 
    of all possible collateralization schemes, regulations, etc.

    The answer to the second research question will merely be a description 
    of, what is considered, the primary arguments of the debate about \FVA/s.
    The primary arguments will mainly be from 
    \textcite{HullWhite2012FVA} and \textcite{Castagna2012FVA};
    some others will be included and cited when used.
    The paper will not attempt to settle the debate or provide a definite solution,
    but the arguments will be challenged when presented.

    For the third research question, the paper will use a corporate finance approach
    by considering a debt and equity financed institution making investment decisions.
    The concern will primarily be the impact on the shareholders and creditors 
    of this single financial institution depending on the financing and investment decision it makes.

    Prices and payoffs will be modelled using a discrete time approach 
    with discrete random variables.
    An approach that would be more in line with reality would be to use stochastic processes;
    however, in favour of simplicity, the discrete framework is chosen 
    since it conveys the basic ideas of \FVA/s better.

    With the purpose of quantifying the funding implications that the firm might face from financial projects,
    some numerical computation will be presented.
    The values of the firm's assets, and the payoff of the investments it obtains,
    do not aim to reflect a fully realistic image of a firm's capital structure in the real world.
    Rather, they are chosen in a way that ease the interpretation of the funding implications.

    With the above choice of procedure, it must be acknowledged that the paper will not be able to
    present a full-fledged valuation framework for use in financial institutions.
    The models do not capture the complexity of the real world,
    but their straightforwardness is an advantage in understanding this convoluted topic.

\end{document}
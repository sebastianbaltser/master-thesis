% !TEX root = sub-main.tex
\documentclass[main.tex]{subfiles}

\begin{document}
    \section{Delimitation}

    To answer the research questions, the thesis will generally aim for simple models,
    that give acquaintance with- and insight into funding costs and value adjustments.
    This familiarity with the topic could then be brought into more complex modelling.

    The thesis will use a corporate finance approach to investigate funding value adjustments,
    by considering a debt and equity financed institution making investment decisions.
    The thesis will primarily investigate the impact on the shareholders and creditors 
    of this single financial institution depending on the financing and investment decision it makes.

    To model market movements the thesis will use a discrete time approach 
    with discrete random variables to model prices and payoffs.
    An approach that would be more in line with reality would be to use stochastic processes,
    but, in favour of simplicity, the discrete framework is chosen 
    since it conveys the basic ideas of funding value adjustments better.

    With the above choice of procedure, it must be acknowledged that the thesis will not be able to
    present a full-fledged valuation framework for use in financial institutions.
    The models do not capture the complexity of the real world,
    but their straightforwardness is an advantage in understanding this convoluted topic.

\end{document}
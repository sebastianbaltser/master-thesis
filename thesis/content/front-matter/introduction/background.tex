% !TEX root = sub-main.tex
\documentclass[main.tex]{subfiles}

\begin{document}
    \subsection{Background}
        
    Prior to the financial crisis, banks could obtain funding from the interbank market
    at very low lending rates, as they were considered almost free of credit risk.
    With the default of large financial institutions, 
    it became clear that counterparties in the interbank market 
    were not as safe as previously thought.
    Interbank lending would have to account for a higher default risk of the counterparty,
    which entailed an increase in the borrowing rates 
    and therefore also an increase in the funding costs of banks.
    \\
    As borrowing rates were low before the financial crisis, 
    the funding costs were of negligible size 
    and they were largely ignored in the derivatives pricing process.
    Soon after the crisis it became apparent that funding costs had to be considered,
    much like institutions would consider their counterparty's default risk by 
    credit value adjustments.

    Valuation adjustments that accounts for the funding implications 
    have been aptly named \textit{Funding Value Adjustments}, \FVA/.
    These adjustments have been a source of great controversy in the financial field,
    as theoreticians strongly oppose them while practitioners deem them crucial for their operations.
    The most intense part of debate was sparked by \textcite{HullWhite2012FVA}
    when they answered the question "Is FVA a cost for derivatives desks?" 
    with an unmistakable "no".
    They argued that \FVA/s go against traditional derivatives pricing frameworks
    and will create arbitrage opportunities if applied.
    The difference between a "yes" and a "no" could translate to a difference 
    in the proximity of hundreds of millions of dollars for the largest financial institutions.
    \\
    As practitioners turned out to very much disagree with Hull \& White,
    the rejection of \FVA/ as a valid adjustment lured out a vast number of counterarguments.
    In essence, practitioners argue that their derivatives valuation 
    has to account for the actual costs of managing the derivative.
    They believe that funding is an important cost and therefore that \FVA/s are essential.

    Regardless of how the debate about \FVA/ ended, banks started universally accepting the inclusion of \FVA/
    in their derivatives pricing.
    To get a glimpse of the magnitude; when JP Morgan Chase first implemented an \FVA/ framework
    for its over-the-counter derivatives it recorded a whopping 1.5 billion dollar loss
    for a one-time adjustment to its portfolio, according to \textcite{JPMorganEarnings}.
    Since then, the debate has been less focused on the appropriateness of \FVA/s
    and more focused on the way it should be accounted for.
    Still, there is really no universally accepted definition of \FVA/,
    and market practice differs significantly between dealers.

\end{document}
% !TEX root = sub-main.tex
\documentclass[main.tex]{subfiles}

\begin{document}
    \section{Summary of the FVA Debate}
        Since the financial crisis, when banks started applying them,
        theoreticians and practitioners have been debating the use of \FVA/s
        when determining the prices of financial assets.
        The focal point of the discussion is specifically 
        whether it is appropriate to apply \FVA/s in the valuation of financial assets.
        This section will provide a summary of the debate, 
        by presenting arguments from what could be considered the three main parties;
        namely dealers, accountants, and theoreticians.

        \subsection{\FVA/ According to Dealers}
            The dealers in the derivatives desk selling or buying derivatives 
            will incur the average funding rate of the bank, as charged by their funding desk.
            For derivatives that require financing the funding desk will charge the dealers.
            If the dealers do not take into account the funding cost when pricing the derivatives,
            a loss will be shown for trades that require funding.
            To see this, consider the following example.

            \begin{example}
            Assume a dealer buys a simple option at some upfront price.
            The option has a theoretical price or value that might have been calculated by assuming
            that financing could be obtained at risk-free rates. 
            Should the dealer buy the option at this theoretical price, 
            she will be making a negative net present value investment, 
            since the trade will generate funding costs. 
            If she is trading outside of a CSA agreement, there will be no collateral posted
            to offset the upfront cost and the payment will be subject to the
            funding rate, which is \textit{above} the risk-free rate. 
            
            Alternatively, she will be trading under a CSA agreement,
            and the collateral will offset the upfront cost.
            Still, she might want to hedge the position and, as the option and the hedge
            will not constitute a self-financing portfolio, 
            unlike the assumptions in traditional pricing frameworks,
            the hedge might require leverage, which will cost the funding rate.
            \end{example}

            In conclusion, the funding desk will charge the actual funding rate; not the risk-free rate.
            Rather than accepting the theoretical price, the dealer should assess the funding cost of the trade 
            and buy the option at the theoretical price adjusted by the funding cost,
            in order to break even on the value of the trade.
            In other words, the dealer should make an \FVA/, 
            since the funding rate is a very real cost, which the dealer will be subject to.
            
            "The dealers" in this argument might as well be other stakeholders that make decisions about new investments.
            In a less complex setup "the funding desk" could simply be the creditors and "the dealers" the company itself.
            The company might be making decisions in a way that optimises shareholder value,
            and therefore aligns its interests with them.
            Shareholders experience the same issue as dealers of being charged funding costs on investments 
            and will therefore be just as eager to evaluate investments in light of their funding costs.
            This simple setup of creditors versus shareholders, 
            will later provide a medium through which the impact of funding costs can be explored.

        \subsection{\FVA/ According to Accountants}
            As financial statement auditors, accountants are concerned with providing objective valuations of derivatives,
            such that the valuation of the firm itself is fair and accurate.
            Therefore, accountants seek to value derivatives at their exit price, i.e. the price that clears the market,
            which is dependent on how other market participants price the transaction.
            To determine the value of derivatives, accountants use the notion of \textit{fair value}.
            \textcite{IFRS13} defines the fair value of an asset as: 
            \textit{"the price that would be received from the sale of an asset or paid to transfer a liability
            in an orderly transaction between market participants at the measurement date"}.
            In addition, the fair value is clearly described as being market based and not entity based.

            This description of fair value is used by \textcite{HullWhiteFVA} as an argument to
            rule out using the funding costs when evaluating prices.
            They claim that, since the funding rate, and therefore the funding costs, is entity specific,
            the fair value should be independent of these.
            While this might be true in isolation, 
            according to \textcite{KPMGFVA}, the current market practice is for dealers, 
            especially OTC derivatives dealers, to include \FVA/ in their valuations.  
            When evaluating the fair value, the dealers should consider the pricing practices 
            that would be used by market participants if the derivative was being sold;
            if the market practice is to apply adjustments for funding costs,
            \FVA/s may very well be included in the fair value.
            The dealer should however consider the funding costs that market participants would consider,
            which is not necessarily equal to the dealer's own funding costs.

            The original argument of \textcite{HullWhiteFVA} was that banks using their own funding costs to evaluate fair value,
            would lead to different banks pricing the same derivative differently,
            since the funding costs were entity specific.
            This is true, however, as mentioned in the previous paragraph,
            to evaluate fair value the bank should not be using its own funding costs, 
            but rather the funding costs of another hypothetical market participant, 
            to which the bank can sell its derivative.
            With that said, it is generally very difficult to determine the generic level of bank funding spread
            representing the funding rate for the hypothetical counterparty in a hypothetical transaction.
            \textcite[Proposition 4]{KPMGFVA} states that, for this reason,
            some banks have in fact been applying their own funding spreads in order to determine \FVA/ under IFRS. 
            Banks do this not for the sake of using their own funding spreads,
            but for it to act as a proxy for the market funding cost.
            If the bank can be assumed to have comparable funding costs to other market participants,
            KPMG believes that this practice is supportable.

            In conclusion, accountants should accept adjusting fair value for funding costs,
            as long as it is market practice to make these adjustment,
            and as long as banks use the funding costs of other market participants, 
            or proxies of those, as reference.

        \subsection{FVA According to Theoreticians}
            Again in the context of derivatives valuations, 
            the arguments of theoreticians, like Hull and White themselves, 
            are rooted in the assumptions of the valuation frameworks commonly used in derivatives pricing.
            The price of a derivative can be obtained by replicating the cash flows of the derivative 
            with a self-financing trading strategy. 
            In the absence of arbitrage, 
            the value of the replicating portfolio and the derivative in question must be equal.
            The risk-neutral valuation principles require discounting at the risk-free rate,
            which is why theoreticians oppose using a different discount rate,
            claiming that there is no theoretical basis for making an \FVA/.
            The assumptions in this theoretical setup is however 
            that the derivative exists in a very specific economic environment. 
            In this economy, the replication of the derivative
            can be funded by borrowing and lending endlessly at the risk-free rate.
            As described in previous sections,
            this seems at odds with how the actual real markets behave and how much financing costs.
            Different entities will obtain funding at different costs above the risk-free rate due to size,
            default risk and multiple other market frictions,
            and interest rate spreads between collateralised and unsecured funding is significant.

            Extending their arguments to other assets than derivatives, 
            theoreticians argue that finance theory requires the discount rate for a project 
            to be determined by the risk of the project.
            Using an \FVA/ in derivatives pricing corresponds to replacing the risk-free rate by the higher funding cost
            when discounting cash flows.
            Hence, the value of the project will seemingly depend on the riskiness of the firm that undertakes the project.
            \textcite{HullWhite2012FVA} use the following example to shed some light on this argument.

            \begin{example}
            Consider a dealer with an average excess funding rate of $200\basispoint$. 
            The dealer assesses the opportunity to buy a nearly risk-free bond with a promised excess yield of $30\basispoint$,
            such that the theoretical price of the bond is the expected cash flow 
            discounted by the risk-free rate plus $30\basispoint$.
            However, for unspecified reasons, the bond is trading at a discount compared to the theoretical price.
            At the price offered it promises an excess yield of $80\basispoint$.
            
            To put some numbers to this example,
            consider a dealer in a firm,
            financed by equity and debt, 
            with a riskiness such that the firm's weighted average cost of capital is $200\basispoint$.
            Assume a risk-free rate of $0$ for simplicity.
            The dealer considers a new project with a present value of $100$,
            and a discount rate of $30\basispoint$.
            However, the project is actually trading at a price of $99.50$,
            and therefore has an excess yield of approximately $80\basispoint$,
            since $100*(1+30\basispoint) / 99.50 - 1 \approx 80\basispoint$.
            This situation is depicted in \cref{fig:hw-example}.

            Based on the information provided, \textcite{HullWhite2012FVA} argue that the bond should be acquired,
            since it trades at a discount and therefore has a positive net present value.
            The dealer only needs to get financing either from equity- or debt issuance,
            and the provider of capital can receive a share of the net present value of the project.
            Since the riskiness of the project only corresponds to a yield of $30\basispoint$,
            the dealer should be able to obtain financing at a lower rate 
            than the actual return of $80\basispoint$.
            Furthermore, as the dealer's average excess funding rate is higher than the excess yield of the bond, 
            the average riskiness of the dealer's existing projects is higher than the riskiness of the new bond.
            Acquiring the bond would therefore reduce the dealer's riskiness,
            which should reduce the credit spread on new debt and therefore also reduce her subsequent funding rate.
            \end{example}

            \begin{figure}
                \centering
                \resizebox{10cm}{!}{%
                \begin{tikzpicture}
                    \import{\graphicsfolder/hw-example}{hw-example.tex}
                \end{tikzpicture}        
                }   
                \caption{
                    Depiction of Hull \& White's example where a dealer in a debt and equity financed firm
                    considers investing in a project with a positive net present value.
                }
                \label{fig:hw-example}
            \end{figure}
            
            From a theoretical perspective the conclusion of \textcite{HullWhite2012FVA} does seem plausible. 
            Obtaining the bond is a positive net present value investment, 
            so surely the dealer should be able to find some distribution of the value to the relevant stakeholders, 
            such that obtaining the bond is a positive net present value investment for all entities.
            The bond counterparty is essentially providing a donation, 
            and some providers of funding should be willing to finance the bond for a share of the donation.
            However, the problem with this conclusion becomes apparent,
            when considering how the example would play out in reality. 
            The bond might very well have such a low risk that an excess yield of $30\basispoint$ is justified;
            but, according to \textcite{Castagna2012FVA}, 
            the financing cost of a firm will usually only gradually update to reflect change in riskiness.
            In reality the dealer will also be quite restrained 
            in the way it can distribute the value obtained from the bond.
            The dealer can experience a free rider problem, 
            where the profits from the bond might not be earned by the ones financing it,
            since the bond's payoffs will be a part of the asset pool of the firm,
            which different stakeholders have different claims on.

            Further investigation into the mechanisms of this apparent firm friction,
            will be postponed to later sections where a framework has been developed,
            and a more concrete discussion can be ensured.  
            For the time being, the argument will be ended with the result that, 
            while the theoretical conclusion is simply that a project trading at a discount should be obtained,
            the practical conclusion is less clear-cut.
            Even when the project counterparty is making a donation through the discount,
            the firm might not be able to properly distribute the donation to entities,
            in a way that makes the project attractive.

            \textcite{HullWhite2012FVA} extend the previous argument by referencing 
            an apparent inconsistency in financial institutions' application of \FVA/s.
            It is common for dealers to invest in government bonds, such as T-Bills,
            and generally other low-yielding assets.
            However, credit risky institutions, such as derivatives dealers, are clearly going to have
            a higher average funding cost than the yield of a T-Bill backed by the U.S. Government.
            Therefore, it would seem that the dealers are not applying \FVA/s to these assets.
            \\
            The implication of the argument is that there is an inconsistency in dealer practice,
            as the low-yielding instruments would be unprofitable if an \FVA/ was made.
            Hence, dealers must be recognising that it is 
            the risk of the project that decides if a trade should be made; 
            not the dealers' funding rate.

            It is true that a simple comparison of a dealer's \textit{average} funding rate
            to the yield of a T-Bill would result in the bond being deemed a losing trade.
            The problem with this argument is that dealers are not going to finance T-Bills
            using unsecured borrowing and paying their unsecured borrowing rate.
            For Treasury instruments and the like,
            very developed repo markets exist such that 
            the purchase of these instruments can be financed with secured funding.
            Consequently, the funding rate of buying these instruments is much lower 
            than the dealers' average funding rate,
            which explains why dealers willingly buy them.
            In previous sections, it was even explained that
            the lack of a repo market for derivatives is exactly the reason 
            why derivatives require unsecured funding in the first place.
            On the contrary, the existence of a repo market for bonds is exactly the reason
            why dealers can buy bonds without suffering a loss.

            In fact, \textcite{Green2015XVA}
            uses the Semi-Replication model by \textcite{BurgardKjaer2013Funding},
            to show that the existence of a derivatives repo market
            would eliminate the need for \FVA/s on derivatives. 
            However, it is unlikely that a repo market for derivatives will be developed.
            Bonds are standardised and have high liquidity which make them easy to repo,
            but that is not the case for derivatives.

        In conclusion, the summary of the debate can be distilled into two parts.
        Theoreticians are against \FVA/s, 
        with the argument that they break the assumptions of traditional valuation frameworks.
        Finance theory states that the risk of a project 
        should determine the discount rate for the cash flows, 
        and that the funding rates are irrelevant.
        Dealers, however, are subject to the funding rates of their funding desk;
        if they do not account for funding costs, they will lose on trades that requires financing.

\end{document}
        
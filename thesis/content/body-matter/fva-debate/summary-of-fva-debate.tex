% !TEX root = ../../../main.tex
\documentclass[../../../main.tex]{subfiles}

\begin{document}
    \subsection{Summary of the \FVA/ Debate}
        Since the financial crisis, when the idea of risk-free banks where disproved,
        theoreticians and practitioners have been debating the use of Funding Value Adjustments (FVA)
        when determining the prices of derivatives.
        As mentioned earlier, the existence of funding costs have led some entities 
        to apply adjustments to their prices when trading derivatives, 
        in order to account for the funding costs they experience.
        The focal point of the discussion is specifically 
        whether it is appropriate to apply \FVA/s to the prices of derivatives.
        The paper of \cite{HullWhiteFVA} argues \FVA/ should generally not be made,
        but provides a summary of the debate that serves as a good starting point for further discussion of the topic.
        This summary is paraphrased in the the following paragraphs where the arguments of different parties are presented:

        \begin{description}
            \item[Traders] The traders in the derivatives desk selling or buying derivatives 
            will incur the average funding cost of the bank as charged by the funding desk.
            For derivatives that require funding the funding desk will charge funding costs from the traders.
            If the traders does not take into account the funding cost when pricing the derivatives,
            a loss will be shown for trades that require funding.
            Imagine, for example, a trader buying a simple option at some upfront price.
            The option have some theoretical price or value that might have been calculated by assuming
            that financing could be obtained at risk-free rates. 
            Should the trader buy the option at this theoretical price, 
            she will be making a negative net present value investment, 
            because the actual funding rate on the upfront is \textit{above} the risk-free rate. 
            The funding desk will charge the actual funding rate; not the risk-free rate.
            Rather, the trader should assess the funding cost of the trade 
            and buy the option at the theoretical price adjusted by the funding cost,
            in order to break even on the value of the trade.
            In other words, the trader should make an \FVA/, 
            since the funding rate is a very real cost, which the trader will be subject to.
            
            "The traders" in this argument might as well be other stakeholders that makes decisions about new investments.
            In a less complex setup "the funding desk" could simply be the creditors and "the traders" the company itself.
            The company might be making decision in a way that optimizes shareholder value
            and therefore align its interests with them.
            Shareholders experience the same issue as traders of being charged funding costs on investments, 
            and will therefore be just as eager to evaluating investments in light of the funding costs that they generate.
            This simple setup of creditors versus shareholders, 
            will later provide a medium through which the impact of funding costs can be explored.

            \item[Accountants] The accountants seek to value derivatives at their exit price, i.e. the price that clears the market,
            which is dependent on how other market participants price the transaction.
            To determine the value of derivatives accountants uses the notion of \textit{fair value}.
            The fair value of a derivative is defined as \textit{the price that would be received from the sale of an asset or paid to transfer a liability in an orderly transaction between market participants at the measurement date}\footcite{IFRS13}.
            In addition, the fair value is clearly described in \cite{IFRS13} as being market based and not entity based,
            which rules out using the entity specific funding cost when evaluating prices.
            From the accountants point of view making an \FVA/ can lead to multiple fair values for the same transaction,
            why they oppose this practice.
            \item[Theoreticians] Theoreticians argue that finance theory requires that the discount rate for a project should be determined by the risk of the project.
            This conflicts with the practice of using an \FVA/ in derivatives pricing,
            which corresponds to replacing the risk-free rate by the higher funding cost of the entity doing the derivatives pricing.
            Thus theoreticians claims that there is no theoretical basis for making an \FVA/.
        \end{description}

        The arguments of theoreticians like Hull and White are rooted in the assumptions of the valuation frameworks commonly used in derivatives pricing.
        The price of a derivative can be obtained by replicating the cash flows of the derivative with a self-financing trading strategy, 
        since, in the absence of arbitrage, the value of the trading strategy and the derivative in question must be equal.
        The assumptions in this theoretical setup is however 
        that the replication can be funded by borrowing and lending endlessly at the risk-free rate.
        This seems at odds with how the actual real markets behaves and how much funding costs.
        Different entities will obtain funding at different costs due to size,
        default risk and multiple other market frictions,
        and interest rate spreads between collateralized and unsecured funding is significant.

        Arguments supporting this apparent shortcoming of the traditional derivative valuation frameworks are presented in the following section by discussing how funding costs can arise from the aforementioned market frictions.
    
    \subsection{Funding costs causing asymmetric valuations}
        When a corporation decides to enter a deal with the bank on a derivative transaction,
        it's because the fair value of the transaction to them is greater than the price.
        On the other hand, the banks valuation of the transaction is different,
        since the two counterparties are in two completely different situations.
        This difference is partially caused by the funding costs,
        which for the corporate is derived from the market costs of borrowing funds,
        as well as the costs from raising funds by the normal business activities.
        The funding costs of the bank are in contrary derived from treasury activities such as wholesale funding.
        This difference in the valuation doesn't have to mean that the transaction will never occur 
        as long as they can agree on a price, that doesn't neglect either of the fair values.

        The asymmetry between the valuation of a transaction between two counterparties with two different positions in the market can be explained by other components as well.
        The portfolio management is different in the sense,
        that in the bank the derivative is a part of a large derivative portfolio with many counterparties,
        whereas in the the corporation most likely uses the derivative to hedge balance sheet exposures,
        and provide better cash flow management.
        Also, the corporation does not have to place any regulatory capital against the position,
        but the bank must in contrary have the appropriate regulatory capital in place.

        (Even the transaction made between two banks can have asymmetric valuations.)

\end{document}
        
% !TEX root = ../../../main.tex
\documentclass[../../../main.tex]{subfiles}

\begin{document}
    \subsection{The origin of Funding Value Adjustments}
        Before the financial crisis of 2008, the spreads on CDS contracts referencing banks were very low. 
        As banks were considered almost risk-free, 
        they could all fund themselves at the interbank lending rates implied by LIBOR, Euribor, etc.
        Banks were seemingly similar to each other in this respect, 
        and the notion of a financial product's fair value was very much driving the pricing of derivatives issued by banks,
        since mispricing would lead to arbitrage opportunities.
        If one bank offered a financial contract at a lower price than a financial contract with identical terms
        offered by another bank, arbitrageurs would exploit this apparent mispricing. 
        The illusion of risk-free banks was shattered in 2008 with the failure of some large banks, 
        especially the bankruptcy of the investment bank Lehmann Brothers.
        These defaults made it clear that banks were in fact not risk-free; 
        firms financing banks needed to do so only when receiving a credit spread,
        i.e. a lending rate exceeding the interest rate that was considered the risk-free rate. 
        
        As a consequence, banks are now forced to pay additional interest on their funding, 
        which depends on their perceived default intensity.
        Due to the varying investment- and risk profiles of banks, 
        the perceived default intensity differs between them, 
        why each bank might face completely different funding schemes and costs than its competitors.
        Differing funding costs have massively challenged the concept of a derivative's fair value,
        since the price that different banks are willing to bid or offer might vary significantly. 
        This has lead many dealers to systematically applying adjustments to their prices
        in order to accommodate their funding costs; 
        adjustments that are aptly named Funding Value Adjustments. 
        This practice is condemned by some theoreticians, 
        which has ensued heavy debate.
        
        Before diving into the implications of differing funding costs, or the lack thereof,
        and the discussion of whether applying \FVA/s are appropriate,
        the circumstances that fosters funding costs will be explored in the following section.

\end{document}
        
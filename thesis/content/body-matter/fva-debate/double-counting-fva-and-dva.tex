% !TEX root = ./sub-main.tex
\documentclass[main.tex]{subfiles}

\begin{document}

    \subsection{\FVA/, \DVA/, and double counting}
    \label{sec:fva-dva-double-counting}

    When discussing the validity of \FVA/s it has been argued
    that there exists an overlap between \FVA/ and \DVA/.
    This is for example yet another argument from \textcite{HullWhite2012FVA} against \FVA/s.
    If there is such overlaps, accounting for both \FVA/ and \DVA/
    might lead to so called \textit{double counting}
    which refers to the action of performing, at least a part of, a valuation adjustment twice. 
    Clearly, this can lead to both over- or underpricing of assets
    and therefore needs to be investigated.
    This section examines the truthfulness of the argument and possible implications 
    if there is an overlap between \FVA/ and \DVA/.

    For a fruitful discussion, 
    consider first the following example motivating the aforementioned argument,
    by shedding light on a common dependency of the two valuation adjustments.

    The CDS credit spread reflects how much the seller of the CDS must receive from the buyer
    to compensate for the probability that the seller must pay the notional 
    if the underlying firm defaults.
    The funding spread reflects how much the funding institution must receive from the firm 
    to compensate for the probability that the firm does not repay its debt, 
    i.e. the funding institution loses the notional.
    Usually the funding spread and the CDS credit spread 
    are different due to a credit liquidity premium,
    but imagine an ideal world with no friction, i.e. infinite information, infinite liquidity, 
    instantaneous price updates that reflect all available information, etc.
    From the previous definitions, it can be seen that in this ideal world
    the CDS credit spread of each firm is going to be the same 
    as the spread the firm pays on unsecured financing, i.e. its funding spread.
    The same result would also be implied by the simple assumption 
    that the entire funding spread is compensation for the firm's default risk.

    The CDS credit spread is an input in \DVA/ as calculated by the firm.
    But, if the CDS credit spread and the funding spread is identical, 
    \DVA/ can be interpreted differently;
    namely, the cost to the firm of ensuring that it does not default, 
    as evident from the following example.
    If the firm wants to make sure it does not default,
    it can borrow today its future expected liabilities and put them aside.
    Whenever a liability matures, the firm uses funds from this pool of cash to cover the debt.
    For borrowing the pool of cash, the firm pays the interest of unsecured loans, 
    i.e. its cost of funding.
    However, the firm should always be able to pay its liabilities,
    and, by setting aside these funds, the firm has essentially hedged its own default. 
    So, the cost of hedging its own default is the cost of funding, 
    which should show that \DVA/ is related to \FVA/ through the funding costs.
    
    If \DVA/ and \FVA/ are related, it would seem like there might be an overlap between the two,
    such that accounting for both in the valuation of a derivative leads to a double-counting.
    However, following \textcite{Ruiz2015XVA}, 
    this would be a misunderstanding of how \DVA/ and \FVA/ occurs in reality.
    Suppose a firm with default risk managing multiple derivatives with multiple counterparties.
    Without changing the conclusion of the following, 
    the liquidity premium can be assumed to be zero, 
    such that the firm's CDS credit spread is identical to its funding spread.
    As was covered when defining \DVA/, the firm's \DVA/ is the aggregated cost
    paid by counterparties for hedging the firm's credit risk.
    In the previous paragraph, it was derived that if the firm managed its own default risk 
    by borrowing today all future liabilities,
    the \DVA/ can also be seen as the firm's funding costs.
    However, in reality that would not be an actual cost that the firm face,
    since it would not manage its own default risk in that manner.
    To actually account for its own credit risk in the valuation process,
    the firm should consider what it itself actually does to manage it;
    not what counterparties do or what the firm would have done in an ideal world.
    There is a difference in the value and interpretation 
    of the counterparties' cost of hedging the firm's credit risk, \DVA/,
    and the firm's cost of hedging its own default, \FVA/.
    In the following paragraph it will be argued why.

    First, \DVA/ does not account for any funding benefits due to inter-netting sets.
    Rather, it considers each netting set separately, by representing the cost 
    of borrowing funds today to meet the future liability of each netting set.
    In reality, a firm can gain many benefits by inter-netting sets,
    which can only be realized by considering the entirety of the firm's portfolio.
    In addition, firms manage their own credit risk,
    not by borrowing unsecured funds for every future liability,
    but by cleverly covering liabilities
    with hedging and collateralization.
    The firms attempt to align future cash flows to offset each other
    and potential mismatches are covered by posting initial margins. 
    Hedging and collateralization have a cost; 
    the present value of that cost is the \FVA/.

    Hence, it would be wrong to refer to \DVA/ as the firm's cost of funding.
    It is the cost to the counterparties of managing the firm's credit risk,
    which they can hedge by buying CDS contracts,
    but the cost to the firm of managing its own risk is very different;
    the firm manages its own credit risk by hedging and collateralization.
    \FVA/ considers the cost of this strategy, and therefore takes a more holistic view, 
    reflecting netting benefits and the costs of operations that the firm will actually perform.

    When a firm needs to value a derivative, 
    it should evaluate the actual costs that it will be facing.
    Calculating the minimum price at which the firm would want to enter a project,
    should involve the true cost of actually executing the project.
    It must consider netting benefits, funding surpluses due to mismatches in cash flows,
    as well as costs due to initial margin.

    In conclusion, \FVA/ and \DVA/ should not introduce any double counting if used properly.
    They are related in their calculation but each are used for different purposes.
    \DVA/ is used for calculating the market price of a derivative or a netting set
    while the \FVA/ is used for calculating the cost of obtaining the derivative,
    i.e. including it in a portfolio with possible offsetting cash flow.
    Hence, \FVA/ is necessary for a firm to take the right decision when it is evaluating
    how much a derivative is worth to the firm.
    The firm cannot take such decisions only by considering the market price of the derivative.
    
\end{document}
        
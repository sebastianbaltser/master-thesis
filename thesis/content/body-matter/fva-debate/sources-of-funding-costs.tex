% !TEX root = ../../../main.tex
\documentclass[../../../main.tex]{subfiles}

\begin{document}
    \subsection{Sources of funding costs}
        In the context of an OTC derivatives dealer, 
        \cite{Ruiz2013FVA} mentions some of the most likely sources of funding costs as
        either asymmetry in collateral agreements or the payment liabilities due to the contract itself.
        Both of these sources will be discussed in the following sections.
        As will be concluded funding costs can be viewed as a component of the cost of trading derivatives
        that are not subject to a perfect CSA agreement and whose market risk cannot be perfectly replicated.

    \subsubsection{Funding costs from asymmetrical collateral agreements}
        If a perfect market risk hedge exists, funding costs can be viewed 
        as a component of the cost of trading derivatives that are not subject to a perfect CSA agreement.
        A perfect CSA agreement would require collateral to be posted continuously with no thresholds on the margin amounts,
        such that counterparty exposure is effectively eliminated.
        An often needed constituent of the derivatives dealer's portfolio
        is trades that have the sole purpose of hedging market risk,
        which is brought in by the dealer's financial contracts with its counterparties.
        When the collateralization scheme with a derivative counterparty is imperfect, 
        the collateral postings will most likely not match the collateral postings stemming from the market risk hedges.
        This is often the case as the dealer will likely hedge market risk in an exchange requiring full collateralization.
        Imbalance between the collateral schemes of the OTC deals and the market risk hedges
        is exactly the facility of funding costs, with which this section is concerned.
        This source of funding cost is best observed in the most extreme case, where the dealer trades unsecured derivatives. 

        Suppose that the derivatives dealer is selling an unsecured OTC derivative to a counterparty.
        In order to avoid the market risk of the investment, 
        the dealer will perform the opposite (perhaps synthetical) trade with an exchange,
        i.e. the dealer hedges its market risk to the counterparty.
        The exchange requires full collateralization.
        Any collateral that needs to be posted to the exchange must be borrowed from the dealer's funding institution.
        The loan is unsecured why the funding institution charges a spread,
        more specifically the dealer pays the rate $\OIS/ + \text{funding spread}$.
        The collateral posted at the exchange is secured by the actual investment demanding the collateral,
        and therefore earns the \OIS/ rate.
        The difference between the interest paid on the loan and the interest earned on the collateral is the funding cost;
        suffered by the derivatives dealer for entering the trade with the counterparty. 

        The funding cost can be reduced if the dealer has in place a CSA agreement allowing rehypothication with the counterparty.
        The posting of collateral by the counterparty reduces the credit risk of the dealer,
        but the dealer still faces the market risk from the derivative, 
        so again it creates a market risk hedge at the exchange.
        Yet, since rehypothication is allowed,
        the collateral posted by the counterparty can be passed onto the exchange as collateral demanded by the hedge.
        This reduces the dealer's need for borrowing unsecured funds from the funding institution 
        and ultimately reduces the funding costs of the dealer.
        However, funding costs are still present, 
        at least to the extent that the collateral agreement with the counterparty is asymmetric to that with the exchange.
        The dealer still need to borrow the difference from the funding institution.
        In case the two collateral agreements are identical, the funding cost will be eliminated, 
        since the collateral received from the counterparty will completely suffice as collateral posted to the exchange.

        These examples should display how a derivatives dealer might face funding costs,
        when the lack of perfect symmetry in collateral needs drives the obtainment of unsecured funding.
        It might be argued that the funding costs could be eliminated by trading derivatives,
        whose replication can be operated by repurchase agreements. 
        Seemingly, buying the hedging strategy at repo should eliminate the need for unsecured financing
        and thus the funding costs. 
        However, even when repo market exists, hedging might require unsecured funding,
        which can be shown with an example by \cite{Castagna2012FVA}.
        Consider a dealer buying a european call option, with the intention to hedge the market risk.
        To create the replication strategy the dealer must short an amount of the underlying asset corresponding to the option Delta of the call option.
        If the underlying asset can be sold at repo the dealer will receive the repo rate.
        However, to pay the premium for the call option the dealer must borrow unsecured funds from its funding institution,
        paying again the funding spread. 

        Funding costs, such as this, has left financial institutions justifying an adjustment to the price of the derivatives
        corresponding to the funding cost they will suffer from the trade.
        This justification is however refused by \cite{HullWhite2012FVA}.
        They state that, since a hedge consists of buying and selling assets for their market prices,
        performing a hedge is an investment with zero net present value.
        Exchanging money for assets of identical value,
        is simply moving value around and these operations should not influence valuations. 

        In any case,
        the funding spread could very well be part of the hedging strategy which makes it undeniable for dealers when performing valuation of the initial investments.

        Another argument proposed by \cite{HullWhite2012FVA} is the fact that dealers buys Treasury instruments and other low-yielding assets returning less than the dealers' average funding cost.
        They do this without applying an FVA and the implication of the argument is that there is an inconsistency in dealer practice,
        as the low-yielding instruments would be unprofitable if an FVA was made.
        The previous discussion however nests an explanation for this dealer behavior.
        For Treasury instruments and the like,
        very developed repo markets exists such that the purchase of these instruments can be financed with secured funding.
        Consequently the funding cost of buying these instruments is likely much lower than the dealers' average funding cost,
        which partly explains why dealers willingly buy them.
        As shown by the example of the OTC derivatives dealer, 
        the lack of a repo market for derivatives is exactly the reason why funding costs appears in the first place.
        In fact \cite[Section~9.4.1]{Green2015XVA}, 
        using the \footcite{BurgardKjaer2013Funding} Semi-Replication model,
        shows that the existence of a repo market would eliminate the need for an \FVA/ on derivatives. 
        
    \subsubsection{Funding costs from derivatives cash flows}
        The fact that the dealer does not have access to risk-free funding
        can have further contributions to the funding costs, besides that from collateral financing.
        First, note that cash flows taking place in a derivative must also be funded at a non-risk-free rate
        and that these cash flows can be used for netting other cash flows in the dealers portfolio.
        The second point implies that with the existence of a perfect replication,
        the cash flows from a derivative can be completely offset by the hedging strategy,
        such that no additional funding is required and no funding costs are generated.

        In reality, replications of OTC derivatives are most often imperfect,
        such that there is no one-to-one correspondence in cash flows between the derivative and the hedge.
        Of course, the dealer might not even try to hedge a derivative directly,
        but rather let a netting set of cash flows offset each other. 
        The principle is the same whether the derivative cash flows are offset by an imperfect hedge
        or another different derivative;
        both are again sources of funding cost, which can be illustrated by the following examples,
        due to \cite[Section~12.3]{Ruiz2015XVA}.

        One source of funding costs, where the dealer does not hedge, occurs when trading simple options.
        Selling an option entitles the dealer to the option premium,
        which decreases the funding needs elsewhere in the organization,
        while buying an option has the opposite effect. 
        Another source of funding costs materializes in the next example when the dealer applies an imperfect hedge.
        The dealer have entered into a 10 year OTC swap with semi-annual payments, 
        but for the replication the dealer is only able to obtain a 10 year swap with quarterly payments.
        As a result, the dealer will have a cash flow from the hedge every three months
        that does not correspond to the cash flow from the OTC derivative. 
        Both of these are examples of funding costs essentially stemming from the lack of perfect hedging.

        Having established some possible sources of funding costs, 
        the discussion can move on to the debate about whether accounting for these is appropriate for derivatives pricing.

\end{document}
        
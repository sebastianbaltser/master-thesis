% !TEX root = ./sub-main.tex
\documentclass[main.tex]{subfiles}

\begin{document}
    \subsection{The difference between Price and Value}

    It would seem like the debaters could not disagree more 
    about whether \FVA/s are appropriate or not.
    One side argues against adjustments due to funding costs, while the other argues in favour.
    The reason for this debate is simply due to theoreticians and practitioner
    are considering an adjustment to two different quantities.
    Theoreticians are arguing that the derivative \textit{prices} should not be adjusted,
    while practitioners are arguing that derivative \textit{values} should be adjusted.
    This section will try to explain the differences between a derivative's price and value
    as well as presenting the arguments of the previous sections in light of these definitions.

    When a derivative dealer considers entering a trade it is going to calculate
    two different numbers, namely the trade's price and the trade's value to the dealer.
    The trade's value to the dealer is given by the difference in the trade's price
    and the cost to the dealer of obtaining and holding the trade:
        \begin{equation}
            \text{Value to Dealer} 
            =
            \text{Risk-Neutral Price}
            -
            \text{Cost}
        \end{equation}
    The trade's risk-neutral price is the price that theoreticians considers as the reference price.
    If the market satisfied the assumptions of the traditional derivative pricing frameworks
    this would be the price at which the trade would be bought and sold. 
    The risk-neutral price must reflect the cost of hedging the derivative in the risk-neutral world
    and can therefore be decomposed into three components. 
    The first being the price of hedging when all parties are risk-free, 
    the second being the cost of hedging credit risk from the counterparty, $\CVA/$,
    and the third being the cost of hedging credit risk from the dealer itself, $\DVA/$:
        \begin{equation}
            P = 
            P_{rf} 
            -
            \CVA/
            +
            \DVA/
        \end{equation}
    The trade's value to the dealer is how much the trade is worth to the dealer.
    As already mentioned, this corresponds to the trade's price less the cost to the dealer
    of "creating" or "producing" the trade by buying it.
    In the context of this thesis the discussion concerns funding costs,
    which is one of the costs that entering into a trade will entail.
    Therefore, the trade's value to the dealer is given by:
        \begin{equation}
            V = P - \FVA/
        \end{equation}
    The value of the trade is going to decide whether it will be economical 
    for the dealer to enter the trade. 
    If the value is negative, the dealer will lose money on the trade;
    if the value is positive, the dealer will make money.
    
\end{document}
        
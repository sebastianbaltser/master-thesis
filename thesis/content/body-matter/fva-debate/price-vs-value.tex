% !TEX root = ./sub-main.tex
\documentclass[main.tex]{subfiles}

\begin{document}
    \section{The Difference Between Price and Value}
    \label{sec:price-versus-value}

    It would seem that, if theoreticians are correct in their assessment of \FVA/s,
    the paper could be concluded at this point.
    However, the final judgement on this debate has yet to be reached,
    and, even at that point, the conclusion will likely be nuanced.
    In fact, it will soon be apparent that the two sides might be debating
    about applying \FVA/s to multiple different concepts.
    This section will come to a conclusion about whether \FVA/s should be used
    on, respectively, a derivative's market price and a derivative's value.
    These two quantities will be defined shortly. 
    Still, this will, of course, not settle the entire debate,
    as there are many other aspects in which \FVA/'s use are being discussed.
    This paper will not try to reach a conclusion on all of these other aspects.
    The following arguments are inspired by \textcite{Ruiz2015XVA},
    and start off by defining the market price and value of a derivative.

    When a dealer sells a derivative, she is going to receive some price at the inception date of the derivative.
    This price is the market price.
    Since she is selling at that price, the dealer will want it to be as high as possible.
    \\
    "Creating" or "manufacturing" the derivative has an associated cost to the dealer,
    which is constituted by multiple different terms.
    Assume, after selling the derivative, the dealer is going to hedge the market risk
    by trading derivatives on an exchange.
    This market is highly collateralized and has a minimum of credit risk;
    for simplicity, assume that the dealer is able to hedge the market risk by paying
    the price of the derivative without credit risk. 
    This price will be referred to as the risk-free price.
    The dealer is also going to hedge the counterparty's default risk which costs \CVA/.
    Additionally, the dealer will be paying or receiving her funding rate on unsecured borrowing
    for the derivative's cash flows, which costs \FVA/.
    There might be additional costs, 
    but, for the purposes of this argument, only the ones mentioned will be considered.
    
    In total, the cost of creating the derivative is:
        \begin{equation*}
            \text{Cost} = \text{Risk Free Price} + \CVA/ + \FVA/
        \end{equation*}
    The trade's value to the dealer is given by the difference in the trade's price
    and the cost to the dealer of manufacturing it:
        \begin{align*}
            \text{Value to Dealer} 
            &=
            \text{Market Price}
            -
            \text{Cost} 
            \\
            &= \text{Market Price} - \text{Risk Free Price} - \CVA/ - \FVA/
        \end{align*}
    The value of the trade is going to decide whether it will be economical 
    for the dealer to enter the trade. 
    If the value is negative, the dealer will lose money on the trade;
    if the value is positive, the dealer will make money.
    Setting the value to zero in the above equation and solving for the market price, 
    would yield the breakeven price:
        \begin{align*}
            \text{Breakeven Price}
            =
            \text{Risk Free Price} + \CVA/ + \FVA/
        \end{align*}
    The question is now how the market price and the \FVA/ of the firm are related.
    Remember that the market price of a trade is set by supply and demand forces.
    It will therefore not necessarily adhere to some rigorous pricing scheme,
    but rather be influenced by many different factors in the market.
    One of these factors is the cost of funding the trade.
    If a trade requires a lot of funding, which will generate a lot of funding costs,
    the price of the trade will generally be relatively high.
    Since the trade would be expensive to operate, 
    dealers will generally need to charge a higher price, 
    in order for the value to stay above a level which they can accept.
    But, the price of the trade does not have to be high 
    just because the cost of operating it is high.
    The price is set by the market by the invisible hand of supply and demand,
    and will therefore be driven by many other factors than the operating cost of the trade.
    
    Therefore, there is no strict and rigorous \FVA/ in the market price that always ensures the same price, 
    but there is certainly an \FVA/ in the value to the dealer.

    A confusing case occurs if the dealer's institution has a very dominant position in the market.
    In that case, the dealer is able to influence the market price.
    Of course, the dealer is going to increase the price to a level above her breakeven price
    for as long as there is a counterparty willing to buy the derivative at that price.
    Since the breakeven price is a lower bound for the price that the dealer is happy with,
    it might be said that the \FVA/ is setting the price.
    However, that is inaccurate. 
    If the dealer charges a price $\hat{P}$ that ensures the derivative's value is positive,
    she is only able to do so because the demand side accepts the price.
    \\
    If, for some reason, the dealer's \FVA/ drops to zero,
    she is clearly not going to change its price from $\hat{P}$,
    simply because the demand side accepts the current price 
    and the dealer will always charge as high a price as possible.

    That \FVA/ should not be included in the market price 
    is part of the conclusion that \textcite{HullWhite2012FVA} argue for: 
    \textit{%
        "FVA should not be considered when determining the value of the 
        derivatives portfolio, and it should not be considered when determining the prices 
        the dealer should charge when buying or selling derivatives."
    }%
    One reasoning from practitioners for using \FVA/s is 
    that funding costs are very real costs to a dealer's operation.
    This is an argument for applying \FVA/s to the value of the trade,
    which is a very sensible thing to do.
    An \FVA/ is necessary to arrive at the dealer's breakeven price;
    therefore, the dealer will not be able to make sound trading decisions without an \FVA/.
    
    As such, theoreticians and practitioners are not disagreeing entirely on the subject.
    However, the debate is not limited to the difference between price and value
    that was accounted for here. 
    There exists yet another price to consider, which is what the quote above 
    refers to as \textit{the value of the derivatives portfolio}.

    Assume that the dealer is instead the buyer of a derivative.
    The dealer has calculated the value of the trade and decides to buy it.
    After obtaining the trade, at the price decided by the market, another price forms. 
    The dealer is now interested in the price she would receive 
    if she sold the trade again to a third party; namely, the exit price of the trade.
    This is the price that is marked-to-market for balance sheet accounting.
    The exit price is problematic because it is not observable in the market; 
    it can only be estimated.
    If the dealer could, she would ask all potential buyers to reveal the price 
    at which they would buy the trade, and use that information to calculate the actual exit price.
    That is not possible, hence, the dealer must instead try to estimate the price
    while conforming to the accounting rules.
    Accounting rules say that the dealer should account for her own credit risk 
    when marking to market, which could prompt the dealer to making a \DVA/.
    However, any buyer of the trade will calculate the value of the trade
    based on their own funding rate to determine the price at which the trade is beneficial.
    The dealer is therefore left to choose between making adjustments to the exit price
    using her own funding rate or someone else's funding rate.
    \\
    If the dealer decides on a funding rate, the problem expands further
    because each institution will have different funding rates, but no one will reveal their own.
    It is therefore impossible to really estimate the correct adjustment to make,
    which is why theoreticians like \textcite{HullWhite2012FVA} oppose \FVA/s to exit prices.
    \\
    Coming to a conclusion about whether to use \FVA/s when marking-to-market a trade
    is an interesting but vast topic.
    Asserting this could include research on the current market practice and
    the opinion of accounting institutions.
    However, this topic is too broad to fit within the scope of this paper,
    and it will therefore be left as a possible extension.

    Still, with this discussion, 
    a partial conclusion on the appropriateness of \FVA/ can be reached.
    By the previous definitions, an \FVA/ should be used to calculate the value of a trade.
    If the dealing desk does not make an \FVA/, it will not be accounting for the costs of the trade,
    and it will therefore not be able to make the correct decision about obtaining the project based on the full information.
    This conclusion will be the primer for the discussions following in the rest of the paper.
    
    In the following sections the discussion will move beyond the validity of \FVA/,
    into the topic of how it should be defined and applied.
    
\end{document}
        
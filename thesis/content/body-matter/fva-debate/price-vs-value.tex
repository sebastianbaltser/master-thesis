% !TEX root = ./sub-main.tex
\documentclass[main.tex]{subfiles}

\begin{document}
    \subsection{The difference between Price and Value}

    It would seem like the debaters could not disagree more 
    about whether \FVA/s are appropriate or not.
    One side argues against adjustments due to funding costs, while the other argues in favour.
    However, following \cite{Ruiz2015XVA}, 
    it appears that much of the disagreement in the debate is due to theoreticians and practitioners
    considering an adjustment to two different quantities.
    Theoreticians are arguing that the derivative \textit{prices} should not be adjusted,
    while practitioners are arguing that derivative \textit{values} should be adjusted.
    This section will try to explain the differences between a derivative's price and value
    as well as presenting some arguments of the previous sections in light of these definitions.

    When a derivatives dealer considers entering a trade, 
    it is going to take into account two different numbers; 
    namely, the trade's price and the trade's value to the dealer.
    The trade's value to the dealer is given by the difference in the trade's price
    and the cost to the dealer of obtaining and holding the trade:
        \begin{equation}
            \text{Value to Dealer} 
            =
            \text{Price}
            -
            \text{Cost}
        \end{equation}
    The trade's price is the price that the dealer could buy or sell the trade for in the market.
    If the market satisfied the assumptions of the traditional derivative pricing frameworks,
    this would be the risk-neutral price and therefore the price that 
    theoreticians would consider as the "true" price.
    The price must reflect the cost of hedging the derivative
    and can thus be decomposed into three components. 
    The first being the price of hedging when all parties are free of credit risk, 
    the second being the cost of hedging credit risk from the counterparty, $\CVA/$,
    and the third being the cost of hedging credit risk from the dealer itself, $\DVA/$:
        \begin{equation}
            P = 
            P_{rf} 
            -
            \CVA/
            +
            \DVA/
        \end{equation}
    The trade's value to the dealer represents how much the trade is worth to the dealer.
    As already mentioned, this corresponds to the trade's price less the cost to the dealer
    of "creating" or "producing" the trade by buying it.
    In the context of this thesis the discussion concerns funding costs,
    which is one of the costs that entering into a trade will entail.
    Therefore, the trade's value to the dealer is given by:
        \begin{equation}
            V = P - \FVA/
        \end{equation}
    The value of the trade is going to decide whether it will be economical 
    for the dealer to enter the trade. 
    If the value is negative, the dealer will lose money on the trade;
    if the value is positive, the dealer will make money.
    
    According to these definitions, 
    \FVA/ is a component in the trade's value, but not in the trade's price. 
    Therefore, it would seem as if the market price of the trade 
    is completely unconnected to the funding costs of the trade;
    but, there is in fact a relation between the two. 
    To see this, remember that the market price of a trade 
    is actually set by supply and demand forces.
    It will therefore not necessarily adhere to some rigorous pricing scheme,
    but rather be influenced by many different factors in the market.
    One of these factors is the cost of funding the trade.
    If a trade requires a lot of funding that will generate a lot of funding costs,
    the price of the trade will generally be relatively high.
    Since the trade would be expensive to operate, 
    dealers will generally need to charge a higher price, 
    in order for the value to stay non-negative.
    But the price of the trade is not necessarily high because the cost of operating it is high.
    The price is set by the market by the invisible hand of supply and demand,
    and will therefore be driven by many other factors than the operating cost of the trade.

    Assume that the dealer have calculated the value of the trade, and decided to buy it.
    After obtaining the trade at the price decided by the market, another price forms. 
    The dealer is now interested in the price it would receive 
    if it sold the trade again to a third party; namely, the exit price of the trade.
    This is the price that is marked-to-market for balance sheet accounting.
    The problem with this price is that it is not observable in the market; it can only be estimated.
    If the dealer could, it would ask all potential buyers to reveal the price 
    at which they would buy the trade, and use that information to calculate the actual exit price.
    That is not possible, hence, the dealer must instead try to estimate the price
    while conforming to the accounting rules.
    Accounting rules say that the dealer should account for its own credit risk 
    when marking to market, which would prompt the dealer to making a \DVA/.
    However, the buyer of the trade, whoever that is, will calculate the value of the trade
    based on their own funding rate to determine the price at which the trade is beneficial.
    The dealer is therefore left to choose between making adjustments to the exit price
    using their own funding curve or someone else's funding curve.
    If the dealer decides on a funding curve, the problem expands further
    because each institution will have different funding curves but no one will reveal their own.
    It is therefore impossible to really estimate the correct adjustment to make,
    which is why theoreticians like Hull and White oppose these adjustments and \FVA/.
    Coming to a conclusion about whether to use \FVA/s when marking-to-market a trade
    is an interesting but vast topic.
    Asserting this could include research on the current market practice and
    the opinion of accounting institutions.
    However, this topic is to broad to fit within the scope of this paper,
    and it will therefore be left as a possible extension.

    With this discussion a partial conclusion on the appropriateness of \FVA/ can be reached.
    By the previous definitions, an \FVA/ should be used to calculate the value of a trade.
    If the dealing desk does not make an \FVA/, it will not be accounting for the costs of the trade
    and it will therefore not be able to make the correct decision about obtaining the project.
    This conclusion will be the primer for the discussions following in the rest of the thesis.
    
\end{document}
        
% !TEX root = ../../../main.tex
\documentclass[../../../main.tex]{subfiles}

\begin{document}
    \subsection{Credit Valuation Adjustments}
        Credit Valuation Adjustment (CVA) is the market price of the credit risk attached to a financial instrument.
        This price can be derived by taking the difference between the price of the instrument including credit risk
        and the price of the instrument excluding credit risk, 
        i.e. where both counterparties to the instrument is considered credit risk free.

        In a transaction with two counterparties there is also two sources of credit risk.
        The side from which \CVA/ is considered is not unimportant, 
        why "the firm" will be used to refer to the reporting institution 
        and "the counterparty" to refer to the counterparty of the firm.
        \CVA/ might either consider only the credit risk of the counterparty 
        or consider both the credit risk of the counterparty and the firm itself,
        which qualifies it as respectively unilateral or bilateral \CVA/.
        The difference between the results of two models, arising from the credit risk of the firm,
        is a term coined as Debit Valuation Adjustment (DVA).
        When the credit worthiness of the firm declines, i.e. its default risk increases,
        the firm can book an accounting gain on its derivative portfolio,
        reflecting the fact that should the firm default, it will not fully repay all its obligations.

\end{document}
        
% !TEX root = ./sub-main.tex
\documentclass[main.tex]{subfiles}

\begin{document}
    \FVA/ is closely related to \DVA/ in the sense that the quantity they represents have potential overlaps.
    If these overlaps are not carefully considered, they might lead to so called \textit{double counting}
    which refers to the action of performing, at least a part of, a valuation adjustment twice. 
    Clearly, this can lead to both over- or underpricing of assets.
    This section covers the overlap between the \FVA/ and parts of the \DVA/.

    For this purpose it is useful to separate the factors that contributes to the \DVA/,
    meaning the sources of \DVA/. 
    A firm defaults if it is not able to meet its liabilities;
    the possibility of defaulting on each liability contribute to the \DVA/.
    Let $\DVA/_{\text{derivatives}}$ refer to the value arising
    because the firm might default on its derivative obligations,
    and let $\DVA/_{\text{funding}}$ refer to the value arising 
    because it might default on the funding required for the derivative portfolio.

    Assume that the entire credit spread is compensation for default risk.
    Then the \FBA/ is exactly equal to $\DVA/_{\text{funding}}$. 
    
    \DVA/ can be seen as the cost of ensuring the firm does not default.
    If the firm wants to make sure it does not default 
    it can borrow its future expected liabilities and put them aside.
    For that the firm pays the interest of unsecured loans, i.e. its cost of funding.
    By doing this the firm is essentially hedging its own default. 

\end{document}
        
% !TEX root = ./sub-main.tex
\documentclass[main.tex]{subfiles}

\begin{document}
    \subsection{Credit Value Adjustments}
        Credit Value Adjustment (CVA) is the market price of the credit risk attached to a financial instrument.
        This price can be derived by taking the difference between the price of the instrument including credit risk
        and the price of the instrument excluding credit risk, 
        i.e. where both counterparties to the instrument is considered credit risk free.

        In a transaction with two counterparties there are also two sources of credit risk.
        The side from which \CVA/ is considered is not unimportant, 
        why \textit{the firm} will be used to refer to the reporting institution 
        and \textit{the counterparty} to refer to the counterparty of the firm.
        \CVA/ might either consider only the credit risk of the counterparty 
        or consider both the credit risk of the counterparty and the firm itself,
        which qualifies it as respectively unilateral or bilateral \CVA/.
        The difference between the results of two models, arising from the credit risk of the firm,
        is a term coined as \DVA/.
        When the credit worthiness of the firm declines, i.e. its default risk increases,
        the firm can book an accounting gain on its derivative portfolio,
        reflecting the fact that should the firm default, it will not fully repay all its obligations.

        In a bilateral setting, both counterparties can, in theory, agree on the necessary credit valuation adjustment,
        since each party is also calculating the counterparty's \CVA/ and \DVA/, however with the labels switched.
        If the firm is calculating $\CVA/_{F}$ and $\DVA/_{F}$, 
        and the counterparty is calculating $\CVA/_{C}$ and $\DVA/_{C}$,
        the following symmetric relations hold:
            \begin{align*}
                \CVA/_{F} &= \DVA/_{C} \\
                \DVA/_{F} &= \CVA/_{C}
            \end{align*}
        The price of the counterparty credit risk experienced by the firm, $\CVA/_{F}$,
        is equal to the price of the counterparty credit as experienced by the counterparty itself, $\DVA/_{C}$,
        and vice versa.
        To this extent, the counterparties will reach symmetric valuations of the financial instrument. 
        This result is however merely theoretical; 
        in practice the counterparties will apply different \CVA/ models
        and be operating under different accounting rules, 
        which will inevitably lead to the parties reaching different adjustments. 
        
        Eliminating \CVA/ and \DVA/ is possible by zeroing out the credit exposure due to the instrument,
        which is possible with a collateralization scheme. 
        Perfect collateralization would ensure no credit exposure,
        but even in strong CSA agreements collateral calls are no more frequent than daily.
        Still CSA agreements will reduce the credit exposure and hence \CVA/ and \DVA/.

\end{document}
        
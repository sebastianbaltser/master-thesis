% !TEX root = ./sub-main.tex
\documentclass[main.tex]{subfiles}

\begin{document}
    \subsection{Funding Value Adjustments}
        Funding Value Adjustment (FVA) is a quantity meant for accounting for the funding costs 
        experienced by a firm due to a financial instrument.
        Funding risks and costs arises wherever there is a cash flow in the firm. 
        Outgoing cash flows will need to be financed while incoming cash flows
        can be used to retire debt elsewhere in the firm, thus reducing the need for funding.

        \cite[Proposition 6]{KPMGFVA} states that FVA conceptually can be thought of 
        as consisting of three elements, according to the following decomposition:
        \begin{equation}
            \FVA/ = \FCA/ + \FBA/ + \FVA/_{\text{buffer}}
        \end{equation}
        \FCA/, also elsewhere referred to as Funding Cost Adjustment (FCA), 
        is a negative contribution of funding costs to the price of the financial instrument.
        It captures the effect of outgoing cash flows, which will require funding and generate costs.
        On the other hand, \FBA/, also known as Funding Benefit Adjustment (FBA), 
        captures the positive impact on the instrument price, from the incoming cash flows.
        Without going too much into details of their calculation, which will be covered later,
        \FCA/ is calculated based on the expected positive exposure, 
        while \FBA/ is calculated using the expected negative exposure. 

        $\FVA/_{\text{buffer}}$ refers to an adjustment for the funding costs
        due to the maintenance of liquidity buffers. 
        These buffers are in place for dealers to have access to funding, in the event
        that funding markets fail to function when the dealer faces unexpected funding requirements.
        This adjustment will be ignored throughout the rest of this paper, 
        which is possible by simply not assuming any liquidity buffers. 
        
        Further investigation of Funding Value Adjustments will be focused on the two other components,
        \FCA/ and \FBA/.

\end{document}
        
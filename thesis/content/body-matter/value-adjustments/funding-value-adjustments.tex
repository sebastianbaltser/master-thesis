% !TEX root = ./sub-main.tex
\documentclass[main.tex]{subfiles}

\begin{document}
    \subsection{Funding Value Adjustments}
        Funding Value Adjustment (FVA) is a quantity meant for accounting for the funding costs 
        experienced by a firm due to their obtainment of a financial instrument.
        Funding risks and costs arise wherever there is a cash flow in the firm. 
        Outgoing cash flows will need to be financed while incoming cash flows
        can be used to retire debt elsewhere in the firm, thus reducing the need for funding.

        Again, funding costs, and the valuation adjustments that accounts for them,
        are best introduced by an example.

        Assume a firm buys a bond from a financial institution.
        The upfront price of the bond is the discounted payoff at maturity,
        if the financial institution is considered credit risk-free.
        This amount will be paid by the firm with financing, where the different types of financing will be elaborated later in this paper.
        The costs of funding this derivative will cause the firm to value the bond differently.
        Essentially, this means that the firm will demand a valuation adjustment 
        in order for the bond not to be a negative net present value investment. 
        If, however, the firm is assumed to have credit risk, obtaining debt will become more expensive. 
        This is because of a credit spread which the funding institution charges in addition to the usual interest rate.
        
        This valuation adjustment can also work in the opposite direction, 
        where the cash flow at time 0 is positive for the firm.
        In this case, the firm allegedly benefits from the transaction 
        in the sense that it has the possibility to buy off existing liabilities.
        This issue will be analyzed further in \cref{sec:example-funding-benefit}.
        
        From the example, it is clear that FVA conceptually can be thought of 
        as consisting of two elements, according to the following decomposition:
        \begin{equation}
            \FVA/ = \FCA/ + \FBA/
        \end{equation}
        \FCA/, also elsewhere referred to as Funding Cost Adjustment (FCA), 
        is a negative contribution of funding costs to the price of the financial instrument.
        It captures the effect of outgoing cash flows, which will require funding and generate costs.
        On the other hand, \FBA/, also known as Funding Benefit Adjustment (FBA), 
        captures the positive impact on the instrument price, from the incoming cash flows.
        Without going too much into details of their calculation, which will be covered later,
        \FCA/ is calculated based on the expected positive exposure, 
        while \FBA/ is calculated using the expected negative exposure. 

        \textcite{KPMGFVA} suggests introducing another term in the decomposition, namely
        $\FVA/_{\text{buffer}}$.
        This term refers to an adjustment for the funding costs
        due to the maintenance of liquidity buffers. 
        These buffers are in place for dealers to have access to funding, in the event
        that funding markets fail to function when the dealer faces unexpected funding requirements.
        This adjustment will be ignored throughout the rest of this paper, 
        which is possible by simply not assuming any liquidity buffers. 
        
        Further investigation of Funding Value Adjustments will be focused on the two other components,
        \FCA/ and \FBA/.

        \subsubsection*{Margin requirements}
            Suppose an OTC derivative transaction between the firm and a counterparty requires the firm to post collateral.
            The firm may necessarily be obliged to fund the margin requirements on top of the price of the derivative.
            To understand this setup,
            consider a single interest rate swap between a firm and a counterparty that have in place a CSA agreement.
            The CSA agreement ensures that collateral is posted or received,
            as the market value of the swap changes over time.
            If the value of the firm's leg is lower than the counterparty's,
            the firm is required to post collateral, 
            which it must necessarily fund by some method of financing.
            This type of funding is a different category within the \FVA/ framework, 
            but it must still be perceived as a funding cost to the firm.
            The funding occurs due to the posting of margin requirements as consequence of the swap contract,
            and not directly from the price of an OTC derivative transaction,
            as described in the earlier example.
            Later in this article the issue is discussed further, 
            where in \cref{sec:example-secured-derivative} a numerical example is analyzed.
            
            Assume that the firm chooses to fund the collateral by obtaining debt
            that is borrowed at the firm's unsecured borrowing rate, which depends on the firm's credit rating.
            When posting the funds as collateral, the firm earns the OIS rate.
            The asymmetry between the cost of borrowing and posting collateral
            adds additional costs to the swap transaction; 
            these are an example of funding costs and an \FVA/ is a price alteration made to account for these.

            On the contrary, if the firm must receive collateral that can be rehypothecated,
            they pay the OIS rate and can receive a higher rate
            by reducing funding elsewhere in the organization.
            This asymmetry can be referred to as funding benefit rather than a cost.

\end{document}
        
% !TEX root = ./sub-main.tex
\documentclass[main.tex]{subfiles}

\begin{document}
    \subsection{Funding Value Adjustments}
        Funding Value Adjustment (FVA) is a quantity meant for accounting for the funding costs 
        experienced by a firm due to their obtainment of a financial instrument.
        Funding risks and costs arises wherever there is a cash flow in the firm. 
        Outgoing cash flows will need to be financed while incoming cash flows
        can be used to retire debt elsewhere in the firm, thus reducing the need for funding.

        Again, funding costs, and the valuation adjustments that accounts for them,
        are best introduced by an example.
        Consider a single swap between a firm and a counterparty that have in place a CSA agreement.
        The CSA agreement ensures that collateral is posted or received,
        as the market value of the swap changes.
        If the value of the firm's leg is lower than the counterparty's,
        the firm must post collateral, which it must necessarily fund by some method of financing.
        Assume that the firm chooses to fund the collateral by obtaining debt
        that is borrowed at the firm's unsecured borrowing rate, which depends on the firm's credit rating.
        When posted as collateral, the funds earn the OIS rate.
        The asymmetri between the cost of borrowing and posting collateral
        adds additional costs to the swap transaction; 
        these are an example of funding costs and an \FVA/ is a price alteration made to account for these.

        On the contrary, if the firm must receive collateral that can be rehypothecated,
        they pay the OIS rate and can receive a higher rate
        by reducing funding elsewhere in the organization.
        This asymmetri can be referred to as funding benefit rather than a cost.
        
        From the example, it is clear that FVA conceptually can be thought of 
        as consisting of two elements, according to the following decomposition:
        \begin{equation}
            \FVA/ = \FCA/ + \FBA/
        \end{equation}
        \FCA/, also elsewhere referred to as Funding Cost Adjustment (FCA), 
        is a negative contribution of funding costs to the price of the financial instrument.
        It captures the effect of outgoing cash flows, which will require funding and generate costs.
        On the other hand, \FBA/, also known as Funding Benefit Adjustment (FBA), 
        captures the positive impact on the instrument price, from the incoming cash flows.
        Without going too much into details of their calculation, which will be covered later,
        \FCA/ is calculated based on the expected positive exposure, 
        while \FBA/ is calculated using the expected negative exposure. 

        Following \textcite{KPMGFVA} could introduce another term in the decomposition, namely
        $\FVA/_{\text{buffer}}$.
        This term refers to an adjustment for the funding costs
        due to the maintenance of liquidity buffers. 
        These buffers are in place for dealers to have access to funding, in the event
        that funding markets fail to function when the dealer faces unexpected funding requirements.
        This adjustment will be ignored throughout the rest of this paper, 
        which is possible by simply not assuming any liquidity buffers. 
        
        Further investigation of Funding Value Adjustments will be focused on the two other components,
        \FCA/ and \FBA/.

\end{document}
        
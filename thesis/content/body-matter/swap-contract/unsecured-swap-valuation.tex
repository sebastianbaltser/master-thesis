% !TEX root = ./sub-main.tex
\documentclass[main.tex]{subfiles}

\begin{document}
    \subsection{Shareholders' Financing Costs of Swaps}
    \label{sec:unsecured-swap-valuation}
        This section will derive the marginal shareholder valuation of the firm
        buying or selling an interest rate swap using debt funding.    

        For simplicity and for the purpose of marginal valuation,
        if a swap party defaults at the interim date, the coupon payments will have been paid immediately prior.
        This assumption allows the contingent liabilities from the swap to be paid along with the rest of the liabilities at the given time of default.
        And, as all liabilities rank pari passu, the computational part becomes more interpretive.

        The default time of the swap counterparty is denoted by $\tau_{C}$.
        At this point in time, the firm recovers a fraction of the contractual amount denoted by $\beta_{t}$, where $\beta_{t} = 1$ if $\tau_{C}>t$.

        Regardless of the method of funding, the apparent marginal market value of the swap is defined by:
        \begin{align}
            V &= V_{rf} + \DVA/ - \CVA/
            \nonumber
            \\
            &=
            \mathbb{E}_{0}^{\rnmeasure}\left[
                \delta_2 \left(\mathcal{C}_{1} R_{1} + \mathcal{C}_{2}\right)
                - u
            \right]
            \nonumber
            \\
            &\quad +
            \mathbb{E}_{0}^{\rnmeasure}\left[
                \sum_{t=1}^{2} \delta_{t} \mathbbm{1}_{\{\tau_{F} = t\}}\mathbbm{1}_{\{\tau_{C}>t-1\}} \phi_t Y_{t}^{-} 
            \right]
            \nonumber
            \\
            &\quad -
            \mathbb{E}_{0}^{\rnmeasure}\left[
                \sum_{t=1}^{2} \delta_{t} \mathbbm{1}_{\{\tau_{C} = t\}}\mathbbm{1}_{\{\tau_{F}>t-1\}} (1-\beta_t) Y_{t}^{+} 
            \right]
            \label{eqn:swap-value-multi-period}
        \end{align}
        where $Y_{1} = \mathbb{E}_{1}^{\rnmeasure}\left[\mathcal{C}_{2}\right]/R_{2}$, $Y_{2} = \mathcal{C}_{2}$, and $V_{rf}$ is the credit risk free swap value.
        The swap value takes into account the credit risks of both the firm and the counterparty.
        Therefore, two adjustments are performed on the credit risk-free value, $V_{rf}$.
        From the firm's point of view,
        the \DVA/ is the second term in \cref{eqn:swap-value-multi-period} describing the lost amount of the outstanding liability related to the swap in case the firm defaults.
        Additionally, the \CVA/ is the third term describing the recovered outstanding positive cash flow in case the counterparty defaults.
        \\
        As a result of this formulation, the total swap value for the swap counterparty is $-V$.

        In this section it is assumed that the firm always uses debt funding for financing financial instruments.
        Many established companies have a high credit score,
        hence they have no issues obtaining debt and become leveraged.
        It is not unusual for large firms to have multiple creditors.
        This common use of taking on loans constitutes the decision of the above assumption.
        By financing the upfront with new short-term debt, the marginal valuation of the firm's shareholders of entering the swap contract is now analysed.
        The net positive cash flows stemming from the swap are perceived as funding benefits as they will be used to retire some of the firm's existing short-term debt.
        
        To finance the upfront price, the firm issues new debt.
        The face value of the new debt is the upfront price including a spread, 
        such that the loan is a zero net present value investment for the new creditors.
        With the face value of the new debt now defined as $D_{2}(q) = U(q)(R_{0} + s_{0}(q))(R_{1} + s_{1}(q))$,
        the marginal increase in the value of the firm's equity per unit investment is:
        \begin{align}
            G_{\text{swap}} &=
            \left.\frac{\partial}{\partial q}
            \mathbb{E}_{0}^{\rnmeasure}\left[
                \delta_{2} \mathbbm{1}_{\{\tau_{F}(q)>2\}} \mathbbm{1}_{\{\tau_{C}(q)>1\}}
                \left(
                    A - W
                    + q\left(\mathcal{C}_{1} R_{1} + \mathcal{C}_{2}\right)
                    - L_{2}
                    - D_{2}(q)
                \right)
            \right]\right\rvert_{q=0}
            \nonumber
            \\
            &\quad+
            \left.\frac{\partial}{\partial q}
            \mathbb{E}_{0}^{\rnmeasure}\left[
                \delta_2 \mathbbm{1}_{\{\tau_{F}(q)>2\}} \mathbbm{1}_{\{\tau_{C}(q)>1\}}
                \left(q\mathcal{C}_{1} s_1(q) + \theta_{1}R_{1}\right)
            \right]\right\rvert_{q=0}
            \nonumber
            \\
            &\quad-
            \left.\frac{\partial}{\partial q}
            \mathbb{E}_{0}^{\rnmeasure}\left[
                \mathbbm{1}_{\{\tau_{F}(q)>2\}}
                \left(
                    \sum_{t=1}^{2} \delta_{t} \mathbbm{1}_{\{\tau_{C}(q)=t\}}q\left(1-\beta_t\right)Y_{t}^{+}
                \right)
            \right]\right\rvert_{q=0}
            \label{eqn:G-debt-multi-step-1}
        \end{align}
        The first term describes the residual of the total asset value at time 2 if the firm does not default in either of the time periods, and the counterparty does not default at time 1.
        The residual of the total asset value is the asset value including the swap cash flow less the value of the debt paid at time 1 and the debt to pay at time 2.
        \\
        At the interim date, the firm is obliged to either pay of receive the contractual coupon payment.
        If the cash flow is negative for the firm, it issues new debt to finance the payment;
        if the cash flow is positive for the firm, a portion of the firm's existing debt is retired.
        \\
        The second term expresses the cost (benefit) of having (retiring) debt from time 1 to time 2.
        Evidently, if the cash flow at time 1 is positive, the term is a funding benefit;
        if the cash flow at time 1 is negative, the term is a funding cost.
        Furthermore, the second term includes the random dividend that the shareholders receive at the interim date.
        \\
        The third term reflects the cases of the counterparty defaulting.
        If the counterparty defaults, the contractual outstanding amount, 
        i.e. the potential positive cash flow at time 2, will not be fully paid to the firm.
        
        By expressing the derivative in \cref{eqn:G-debt-multi-step-1} 
        as a difference quotient, using $U(0)=0$,
        and by the linearity of the expectation operator,
        the following limit is obtained:
        \begin{align}
            G_{\text{swap}} 
            &=
            \lim_{q \rightarrow 0}
            \frac{
                \mathbb{E}_{0}^{\rnmeasure}\left[
                    \delta_{2} \mathbbm{1}_{\{\tau_{F}(q)>2\}} \mathbbm{1}_{\{\tau_{C}(q)>1\}}
                    \left(
                        q\left(\mathcal{C}_{1} R_{1} + \mathcal{C}_{2}\right)
                        - D_{2}(q)
                    \right)
                \right]
            }{
                q
            }
            \nonumber
            \\
            &\quad+
            \lim_{q \rightarrow 0}
            \frac{
                \mathbb{E}_{0}^{\rnmeasure}\left[
                    \delta_2 \mathbbm{1}_{\{\tau_{F}(q)>2\}} \mathbbm{1}_{\{\tau_{C}(q)>1\}}
                    q\mathcal{C}_{1} s_1(q)
                \right]
            }{
                q
            }
            \nonumber
            \\
            &\quad-
            \lim_{q \rightarrow 0}
            \frac{
                \mathbb{E}_{0}^{\rnmeasure}\left[
                    \mathbbm{1}_{\{\tau_{F}(q)>2\}}
                    \left(
                        \sum_{t=1}^{2} \delta_{t} \mathbbm{1}_{\{\tau_{C}(q)=t\}}q\left(1-\beta_t\right)Y_{t}^{+}
                    \right)
                \right]
            }{
                q
            }
            \nonumber
            \\
            &\quad-
            \lim_{q \rightarrow 0}
            \frac{
                \mathbb{E}_{0}^{\rnmeasure}\left[
                    \delta_{2} \left(
                    \mathbbm{1}_{\{\tau_{F}>2\}} \mathbbm{1}_{\{\tau_{C}>1\}} - \mathbbm{1}_{\{\tau_{F}(q)>2\}} \mathbbm{1}_{\{\tau_{C}(q)>1\}}\right)
                    \left(
                        A - W
                        - L_{2}
                        + \theta_{1}R_{1}
                    \right)
                \right]
            }{
                q
            }
            \label{eqn:G-debt-multi-step-2}
        \end{align}
        It is assumed that $A$, $W$, $L_{2}$, $\mathcal{C}_1$, $\mathcal{C}_2$, and $\theta_1$ all have finite expectations,
        which allows for interchanging the limit and the expectation.
        \\
        In the single-period model, the probability of the assets being exactly equal to the liabilities is 0.
        The same argument is assumed in the multi-period model for both the firm and the counterparty.
        Then, the derivative of the difference between 
        the post-project default event for an infinitesimal investment
        and the pre-project default event is well defined.
        Hence, similarly, 
        L'H\^{o}pital's % cspell: disable-line
        rule can be applied to show that the fourth term equals 0.
        
        Expanding the limiting face value, $D_{2}$, yields:
        \begin{equation*}
            D_2 = u(R_0 + S_0)(R_1 + S_1) = u R_0 R_1 + u R_0 S_1 + u R_1 S_0 + u S_0 S_1
        \end{equation*}
        Focussing on the addend $u S_0 S_1$, recall that the value of each credit spread depends on the loss rate from the respective time period.
        By substituting the expressions for the credit spread, it can be seen that the term
        $\mathbb{E}_{0}^{\rnmeasure}\left[\phi_1\right]\mathbb{E}_{1}^{\rnmeasure}\left[\phi_2\right]$
        appears in $S_0 S_1$.
        Using the linearity of expectations and the tower property, the following derivation can then be used:
        \begin{equation*}
            \mathbb{E}_{0}^{\rnmeasure}\left[\phi_1\right]
            * \mathbb{E}_{1}^{\rnmeasure}\left[\phi_2\right]
            = \mathbb{E}_{1}^{\rnmeasure}\left[
                \phi_2 \mathbb{E}_{0}^{\rnmeasure}\left[\phi_1\right]
            \right] 
            = \mathbb{E}_{1}^{\rnmeasure}\left[
                \mathbb{E}_{0}^{\rnmeasure}\left[
                    \phi_2
                    \mathbb{E}_{0}^{\rnmeasure}\left[
                        \phi_1
                    \right]
                \right]
            \right]
            = \mathbb{E}_{1}^{\rnmeasure}\left[
                \mathbb{E}_{0}^{\rnmeasure}\left[
                    \phi_2 \phi_1
                \right]
            \right]
            = 0
        \end{equation*}
        where the last equality uses the fact that the loss rate must be zero in at least one of the two periods.
        This implies $u S_0 S_1 = 0$, which can be substituted into \cref{eqn:G-debt-multi-step-2}.
        
        By dividing $q$ into the expectation of the first limit in \cref{eqn:G-debt-multi-step-2}
        it can be recognised that $U(q)/q$ is the marginal investment cost, $u$.
        Rearranging and taking the limits yields:
        \begin{align}
            G_{\text{swap}} &= 
            \mathbb{E}_{0}^{\rnmeasure}\left[
                \mathbbm{1}_{\{\tau_{F}>2\}}
                \left(
                    \delta_{2} \left(\mathcal{C}_{1} R_{1} + \mathcal{C}_{2}\right) - u
                \right)
            \right]
            \nonumber
            \\
            &\quad-
            \mathbb{E}_{0}^{\rnmeasure}\left[
                \mathbbm{1}_{\{\tau_{F}>2\}}
                \left(
                    \sum_{t=1}^{2} \delta_{t} \mathbbm{1}_{\{\tau_{C}=t\}}\left(1-\beta_t\right)Y_{t}^{+}
                \right)
            \right]
            \nonumber
            \\
            &\quad-
            \mathbb{E}_{0}^{\rnmeasure}\left[
                \delta_2 \mathbbm{1}_{\{\tau_{F}>2\}} \mathbbm{1}_{\{\tau_{C}>1\}} u \left(
                    R_{1} S_{0} + R_{0} S_{1}
                \right)
            \right]
            \nonumber
            \\
            &\quad+
            \mathbb{E}_{0}^{\rnmeasure}\left[
                \delta_2 \mathbbm{1}_{\{\tau_{F}>2\}} \mathbbm{1}_{\{\tau_{C}>1\}}
                \mathcal{C}_{1} S_1
            \right]
            \label{eqn:shareholders-value-multi-period-swap}
        \end{align}
        where the third and the fourth terms together constitute the second interpretation of \FVA/ according to \cref{sec:defining-fva}.
        They correspond to the quantity $\Phi$, which is the marginal valuation of the swap contract to the firm's legacy creditors.
        \\
        Since the cash flow at time 2 is 0 if the counterparty defaults at the interim date,
        and the marginal investment cost is constant regardless of the default times of either counterparties,
        the indicator of $\{\tau_{C}>1\}$ is redundant and has been removed from the first term.

        To find the value of the swap contract that makes the shareholders indifferent to entering the project,
        the marginal shareholder valuation in 
        \cref{eqn:shareholders-value-multi-period-swap}
        can be set equal to 0 and solved for $u$:
        \begin{equation*}
            G_{\text{debt}} = 0
        \end{equation*}
        %
        \indent
        This valuation considers several scenarios, as different counterparties can default at different dates.
        As a consequence, both the firm and the counterparty have a possibility of losing an otherwise promised income.

        This concludes the derivation of the marginal shareholder valuation of a swap contract.
        These results will be applied in the following section where a quantitative example,
        will be presented.
        Just as the marginal shareholder valuations in the single-period model
        could be extended to include collateralisation and hedges, so can this equation.
        This will not be pursued in this paper 
        but the section will finish with a few remarks on the topic.

        When a firm trades an unsecured swap with a counterparty,
        it is likely that it combines the position with an appropriate hedge.
        Practically, it is often the case that the firm would use two hedges;
        one hedge position to mitigate the counterparty credit risk, e.g. a CDS,
        and the other hedge position to account for market risk exposure of the floating payments.

        For the shareholders to value such a portfolio,
        an extension to the marginal valuation in 
        \cref{eqn:shareholders-value-multi-period-swap} would have to be applied.
        The cash flows would in this expanded setup, to some extent, offset each other,
        and most of the funding implications would stem from the collateralisation.
        As described earlier in this paper, collateralisation agreements can take on many faces.
        A strict CSA agreement could require the firm to mark-to-market at each time period,
        whereas partially collateralised projects could also lead to funding issues.
        Additionally, an independent amount are likely to be established,
        which would potentially leave the firm with more financial obligations.

        \textcite{ADS2018} derives the shareholders' valuation of entering a project 
        including the same swap as above as well as a fully collateralised hedge position.
        For the purpose of quantifying \FVA/ in a multi-period framework,
        this paper sticks to the numerical computations of only an unsecured swap,
        and then draw comparisons to the idea of the hedged position.
        This is the motive of the following section.
        
\end{document}
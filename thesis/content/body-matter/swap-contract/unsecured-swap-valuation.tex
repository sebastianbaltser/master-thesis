% !TEX root = ./sub-main.tex
\documentclass[main.tex]{subfiles}

\begin{document}
    \subsection{Valuing unsecured swaps}
        In this section the client swap is assumed to be fully unsecured.
        This means that no collateralization is performed at the transaction date nor the interim date between the dealer and the swap client.
        For simplicity and the purpose of marginal valuation,
        all counterparties involved in a swap contract defaulting at the interim date default after the coupon payments.
        This assumption allows the contingent liabilities from the swap to be paid along with the rest of the liabilities at the given time of default.
        And, as all liabilities rank pari passu, the computational part becomes more interpretive.

        The swap client's time of default is denoted by $\tau_{C}$.
        At this point in time, the dealer recovers a fraction of the contractual amount denoted by $\beta_{t}$, where $\beta_{t} = 1$ if $\tau_{C}>t$.
        \\
        \textcolor{red}{Initially, it is also assumed that there are no legacy positions between the dealer and the swap client.
        This assumption will be relaxed in SECTION ???, where the effect of netting the swap cash flows against those of the legacy positions with the same client is analysed.}

        Regardless of the method of funding, the apparent marginal market value of the swap is defined by:
        \begin{align}
            V &=
            \mathbb{E}_{0}^{\rnmeasure}\left[
                \sum_{t=1}^{2} \delta_{t} \mathcal{C}_{t} - u
            \right]
            \nonumber
            \\
            &\quad +
            \mathbb{E}_{0}^{\rnmeasure}\left[
                \sum_{t=1}^{2} \delta_{t} \mathbbm{1}_{\{\tau_{F} = t,\tau_{C}>t-1\}} \phi_t Y_{t}^{-} 
            \right]
            \nonumber
            \\
            &\quad -
            \mathbb{E}_{0}^{\rnmeasure}\left[
                \sum_{t=1}^{2} \delta_{t} \mathbbm{1}_{\{\tau_{F} = t,\tau_{F}>t-1\}} (1-\beta_t) Y_{t}^{+} 
            \right]
            \label{eqn:swap-value-multi-period}
        \end{align}
        where $Y_{1} = \mathbb{E}_{1}^{\rnmeasure}\left[\mathcal{C}_{2}\right]/R_{2}$ and $Y_{2} = \mathcal{C}_{2}$.

        The swap value takes into account the credit risks of both the firm and the counterparty.
        Therefore, two adjustments are performed on the the credit risk-free value, $V_0$.
        From the dealer's point of view,
        the \DVA/ is the second term in \cref{eqn:swap-value-multi-period} adding to the credit risk-free swap value,
        whereas the \CVA/ is the third term subtracting from the value.
        As a result of that, the total swap value for the swap client is $-V$.

        Supposing the dealer finances the upfront with new short-term debt, the marginal valuation of the dealer's shareholders of entering the swap contract is now analysed.
        The net positive cash flows stemming from the swap are perceived as funding benefits as they will be used to retire some of the dealer's short-term debt.
        
        As defined in \cref{sec:marginal-valuation-debt-issuance},
        the face value of the new debt is the upfront,
        which is the borrowed amount, including the interest rates paid.
        With the face value of the new debt now defined as $D_{2}(q) = U(q)(R_{0} + s_{0}(q))(R_{1} + s_{1}(q))$,
        the marginal increase in the value of the firm's equity per unit investment is:
        \begin{align}
            G_{\text{debt}} &=
            \left.\frac{\partial}{\partial q}
            \mathbb{E}_{0}^{\rnmeasure}\left[
                \delta_{2} \mathbbm{1}_{\{\tau_{F}(q)>2,\tau_{C}(q)>1\}}
                \left(
                    A - W
                    + q\left(\mathcal{C}_{1} R_{1} + \mathcal{C}_{2}\right)
                    - L_{2}
                    - D_{2}(q)
                \right)
            \right]\right\rvert_{q=0}
            \nonumber
            \\
            &\quad+
            \left.\frac{\partial}{\partial q}
            \mathbb{E}_{0}^{\rnmeasure}\left[
                \delta_2 \mathbbm{1}_{\{\tau_{F}(q)>2,\tau_{C}(q)>1\}}
                \left(q\mathcal{C}_{1} s_1(q) + \theta_{1}R_{1}\right)
            \right]\right\rvert_{q=0}
            \nonumber
            \\
            &\quad-
            \left.\frac{\partial}{\partial q}
            \mathbb{E}_{0}^{\rnmeasure}\left[
                \mathbbm{1}_{\{\tau_{F}(q)>2\}}
                \left(
                    \sum_{t=1}^{2} \delta_{t} \mathbbm{1}_{\{\tau_{C}(q)=t\}}q\left(1-\beta_t\right)Y_{t}^{+}
                \right)
            \right]\right\rvert_{q=0}
        \end{align}
        The first term describes the residual of the total asset value at time 2 if the firm does not default in either of the time periods, and the counterparty does not default at time 1.
        The second term is spread of the debt equalling the amount that the firm must pay at time 1 plus the dividend that the shareholders receive.
        The funding costs are as always paid by the shareholders.
        However, if the cash flow at time 1 is positive, the term has a positive impact of the shareholders' value as they save the potential credit spread from time 1 to time 2 by instead retiring some existing debt.
        The third term reflects the case of the counterparty defaulting.
        The contractual outstanding amount, i.e. the potential positive cash flow at time 2, will not be fully paid to the firm.
        The fraction of the amount recovered is indicated by $\beta_{t}$.
        \\
        By expressing this derivative as a difference quotient, using $U(0)=0$,
        and by the linearity of the expectation operator,
        the following limit is obtained:
        \begin{align*}
            &=
            \lim_{q \rightarrow 0}
            \frac{
                \mathbb{E}_{0}^{\rnmeasure}\left[
                    \delta_{2} \mathbbm{1}_{\{\tau_{F}(q)>2,\tau_{C}(q)>1\}}
                    \left(
                        q\left(\mathcal{C}_{1} R_{1} + \mathcal{C}_{2}\right)
                        - D_{2}(q)
                    \right)
                \right]
            }{
                q
            }
            \\
            &\quad+
            \lim_{q \rightarrow 0}
            \frac{
                \mathbb{E}_{0}^{\rnmeasure}\left[
                    \delta_2 \mathbbm{1}_{\{\tau_{F}(q)>2,\tau_{C}(q)>1\}}
                    q\mathcal{C}_{1} s_1(q)
                \right]
            }{
                q
            }
            \\
            &\quad-
            \lim_{q \rightarrow 0}
            \frac{
                \mathbb{E}_{0}^{\rnmeasure}\left[
                    \mathbbm{1}_{\{\tau_{F}(q)>2\}}
                    \left(
                        \sum_{t=1}^{2} \delta_{t} \mathbbm{1}_{\{\tau_{C}(q)=t\}}q\left(1-\beta_t\right)Y_{t}^{+}
                    \right)
                \right]
            }{
                q
            }
            \\
            &\quad-
            \lim_{q \rightarrow 0}
            \frac{
                \mathbb{E}_{0}^{\rnmeasure}\left[
                    \delta_{2} 
                    \left(\mathbbm{1}_{\{\tau_{F}(q)>2,\tau_{C}(q)>1\}} - \mathbbm{1}_{\{\tau_{F}>2,\tau_{C}>1\}}\right)
                    \left(
                        A - W
                        - L_{2}
                        + \theta_{1}R_{1}
                    \right)
                \right]
            }{
                q
            }
        \end{align*}
        It is assumed that $A$, $W$, $L_{2}$, $\mathcal{C}_1$, $\mathcal{C}_2$, and $\theta_1$ all have finite expectations,
        which allows for interchanging the limit and the expectation.
        $\mathbbm{1}_{\{\tau_{F}(q)>2,\tau_{C}(q)>1\}} - \mathbbm{1}_{\{\tau_{F}>2,\tau_{C}>1\}}$ equals 0 in the limit,
        which entails that the fourth terms equals 0.
        
        By dividing $q$ into the expectation of the first term, it can be recognized that $U(q)/q$ is the marginal investment cost, $u$.
        Rearranging and taking the limit yields:
        \begin{align}
            G_{\text{debt}} &= 
            \mathbb{E}_{0}^{\rnmeasure}\left[
                \mathbbm{1}_{\{\tau_{F}>2\}}
                \left(
                    \delta_{2} \left(\mathcal{C}_{1} R_{1} + \mathcal{C}_{2}\right) - u
                \right)
            \right]
            \nonumber
            \\
            &\quad-
            \mathbb{E}_{0}^{\rnmeasure}\left[
                \mathbbm{1}_{\{\tau_{F}>2\}}
                \left(
                    \sum_{t=1}^{2} \delta_{t} \mathbbm{1}_{\{\tau_{C}=t\}}\left(1-\beta_t\right)Y_{t}^{+}
                \right)
            \right]
            \nonumber
            \\
            &\quad-
            \mathbb{E}_{0}^{\rnmeasure}\left[
                \delta_2 \mathbbm{1}_{\{\tau_{F}>2,\tau_{C}>1\}} u \left(
                    R_{1} S_{0} + R_{0} S_{1} + S_{0} S_{1}
                \right)
            \right]
            \nonumber
            \\
            &\quad+
            \mathbb{E}_{0}^{\rnmeasure}\left[
                \delta_2 \mathbbm{1}_{\{\tau_{F}>2,\tau_{C}>1\}}
                \mathcal{C}_{1} S_1
            \right]
        \end{align}
        where the third and the fourth terms together constitutes the second interpretation of \FVA/ according to \cref{sec:defining-fva}.
        % This result differs a tiny bit from the one ADS presents. The term: $u S_0 S_1$ is somehow gone from ADS' equation but I don't know how. It appears in the third expectation here for now though.
        \\
        Since the cash flow at time 2 becomes 0 if the counterparty defaults at the interim date,
        and the marginal investment cost is constant regardless of the default times of either counterparties,
        the indicator of $\{\tau_{C}>1\}$ can be removed from the first term.

        To find the value of the swap contract that make the shareholders indifferent of entering the project, $u^{*}$, the following equation must be solved for $u$:
        \begin{equation}
            0 = G_{\text{debt}}
        \end{equation}

        This valuation considers several scenarios, as different counterparties can default at different dates.
        As a consequence, the both the dealer and the counterparty has a possibility to lose an otherwise promised income.
        In the next section the dealer hedges the unsecured swap with an interdealer swap that requires both initial margin and variation margin.
\end{document}
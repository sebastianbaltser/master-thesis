% !TEX root = sub-main.tex
\documentclass[main.tex]{subfiles}

\begin{document}
    \subsection{Valuing unsecured swaps}
        In this section the client swap is assumed to be fully unsecured.
        This means that no collateralization is performed at the transaction date nor the interim date between the dealer and the swap client.
        For simplicity and the purpose of marginal valuation,
        all counterparties involved in a swap contract defaulting at the interim date default after the coupon payments.
        This assumption allows the contingent liabilities from the swap to be paid along with the rest of the liabilities at the given time of default.
        And, as all liabilities rank pari passu, the computational part becomes more interpretive.

        The swap client's time of default is denoted by $\tau_{C}$.
        At this point in time, the dealer recovers a fraction of the contractual amount denoted by $\beta_{t}$, where $\beta_{t} = 1$ if $\tau_{C}>t$.
        \\
        \textcolor{red}{Initially, it is also assumed that there are no legacy positions between the dealer and the swap client.
        This assumption will be relaxed in SECTION ???, where the effect of netting the swap cash flows against those of the legacy positions with the same client is analysed.}

        Regardless of the method of funding, the apparent marginal market value of the swap is defined by:
        \begin{align}
            V &=
            \mathbb{E}_{0}^{\rnmeasure}\left[
                \sum_{t=1}^{2} \delta_{t} \mathcal{C}_{t} - u
            \right]
            \nonumber
            \\
            &\quad +
            \mathbb{E}_{0}^{\rnmeasure}\left[
                \sum_{t=1}^{2} \delta_{t} \mathbbm{1}_{\{\tau_{F} = t,\tau_{C}>t-1\}} \phi_t Y_{t}^{-} 
            \right]
            \nonumber
            \\
            &\quad -
            \mathbb{E}_{0}^{\rnmeasure}\left[
                \sum_{t=1}^{2} \delta_{t} \mathbbm{1}_{\{\tau_{F} = t,\tau_{F}>t-1\}} (1-\beta_t) Y_{t}^{+} 
            \right]
            \label{eqn:swap-value-multi-period}
        \end{align}
        where $Y_{1} = \mathbb{E}_{1}^{\rnmeasure}\left[\mathcal{C}_{2}\right]/R_{2}$ and $Y_{2} = \mathcal{C}_{2}$.

        The swap value takes into account the credit risks of both the firm and the counterparty.
        Therefore, two adjustments are performed on the the credit risk-free value, $V_0$.
        From the dealer's point of view,
        the \DVA/ is the second term in \cref{eqn:swap-value-multi-period} adding to the total swap value,
        whereas the \CVA/ is the third term subtracting from the total swap value.
        As a result of that, the total swap value for the swap client is $-V$.

        Supposing the dealer finances the upfront with new short-term debt, the marginal valuation of dealer's shareholders of entering the swap contract is now analysed.
        The net positive cash flows stemming from the swap are perceived as funding benefits as they will be used to retire some of the dealer's short-term debt.
        \\
        As defined in \cref{sec:marginal-valuation-debt-issuance},
        the face value of the new debt is the upfront.
        
\end{document}
% !TEX root = ./sub-main.tex
\documentclass[main.tex]{subfiles}

\begin{document}
    \subsection{Valuing swaps with a fully collateralized hedged position}
        When a dealer trades an unsecured swap with a counterparty,
        it is likely that she combines the position with an appropriate hedge.
        Practically it is often the case that the dealer would use two hedges; one hedge position will mitigate the counterparty credit risk, i.e. a CDS, and the hedge position other will account for market risk exposure of the floating payments, $X_t$.
        
        The cash flow at time $t$, stemming from the swap contract between the dealer and the counterparty,
        is still denoted by $\mathcal{C}_t$.
        The hedge swap contract then takes the form of an offsetting position paying the dealer $-\mathcal{C}_t$.
        As a convenient simplification, this swap position covers both the counterparty credit risk as well as the underlying market risk.
        \\
        Suppose the hedge is executed with another dealer referred to as the \textit{hedge dealer}.
        Similar to the secured derivative in the single-period model analysed in \cref{sec:example-secured-derivative},
        and as is common practice,
        the dealer is required to post both initial margin, $I_{t}$, and variation margin, $M_{t}$.
        Both postings of margins happen at time 0 and time 1, and the variation margins are defined as:
        \begin{align}
            M_0 &= \mathbb{E}_{0}^{\rnmeasure}\left[\delta_t (X_t - K_t)\right]
            \\
            M_1 &= \frac{1}{R_1}\mathbb{E}_{1}^{\rnmeasure}\left[X_2 - K_2\right] 
        \end{align}

\end{document}
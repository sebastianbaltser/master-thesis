% !TEX root = ./sub-main.tex
\documentclass[main.tex]{subfiles}

\begin{document}
    \subsection{Valuing swaps with a fully collateralized hedged position}
        When a dealer trades an unsecured swap with a counterparty,
        it is likely that it combines the position with an appropriate hedge.
        Practically, it is often the case that the dealer would use two hedges;
        one hedge position will mitigate the counterparty credit risk, i.e. a CDS,
        and the other hedge position will account for market risk exposure of the floating payments, $X_t$.
        
        The cash flow at time $t$, stemming from the swap contract between the dealer and the counterparty,
        is still denoted by $\mathcal{C}_t$.
        The hedge swap contract then takes the form of an offsetting position paying the dealer $-\mathcal{C}_t$.
        As a convenient simplification, this swap position covers both the counterparty credit risk as well as the underlying market risk.
        \\
        Suppose the hedge is executed with another dealer referred to as the \textit{hedge dealer}.
        Similar to the secured derivative in the single-period model analysed in \cref{sec:example-secured-derivative},
        and as is common practice,
        the dealer is required to post both initial margin, $I_{t}$, and variation margin, $M_{t}$.
        Both postings of margins happen at time 0 and time 1,
        and, assuming the dealer is still the payer swap,
        the variation margins are defined as:
        \begin{align}
            M_0 &= \mathbb{E}_{0}^{\rnmeasure}\left[\sum_{t=1}^{2}\delta_t (X_t - K_t)\right]
            \\
            M_1 &= \frac{1}{R_1}\mathbb{E}_{1}^{\rnmeasure}\left[X_2 - K_2\right] 
        \end{align}
        i.e. the market value of the hedging swap at the given point in time.
        The hedging swap is assumed to have an upfront payment of $\tilde{u}$, such that $M_0 = \tilde{u}$.
        In addition to providing default protection for both the dealer and the hedge dealer,
        the variation margin mechanism provides an instinctive source of cash funding of the hedged position. 

        The valuation of the secured swap contract does not differ too much from the previous section analysing an unsecured swap.
        However, a natural extension to the expression in \cref{eqn:shareholders-value-multi-period-swap}
        must take place in order to account for the margin requirements.
        The dealer is assumed to finance margin postings by obtaining new short-term debt
        as well as use incoming margin postings to retire existing debt.
        The extension to the shareholder marginal valuation expression
        as a result of the fully collateralized hedged position is denoted by $\Gamma$.
        This quantity is then defined by:
        \begin{align}
            \Gamma &=
            \left.\frac{\partial}{\partial q}
            \mathbb{E}_{0}^{\rnmeasure}\left[
                \delta_2 \mathbbm{1}_{\{\tau_{F}(q)>2\}}
                \mathbbm{1}_{\{\tau_{C}(q)>1\}}
                \left(
                    - q\left(\mathcal{C}_1 R_1 + \mathcal{C}_2
                    + M_0 R_0 s_1(q)
                    + M_1 R_1\right)
                \right) 
            \right]\right\rvert_{q=0}
            \nonumber
            \\
            &\quad+
            \left.\frac{\partial}{\partial q}
            \mathbb{E}_{0}^{\rnmeasure}\left[
                \delta_2 \mathbbm{1}_{\{\tau_{F}(q)>2\}}
                \mathbbm{1}_{\{\tau_{C}(q)>1\}}
                \left(
                    I_0 R_0 s_1(q)
                    + I_1 R_1
                \right) 
            \right]\right\rvert_{q=0}
            \nonumber
            \\
            &\quad-
            \left.\frac{\partial}{\partial q}
            \mathbb{E}_{0}^{\rnmeasure}\left[
                \delta_2 \mathbbm{1}_{\{\tau_{F}(q)>2\}}
                \mathbbm{1}_{\{\tau_{C}(q)>1\}}
                \left(
                    q\left(\mathcal{C}_{1} s_{1}(q)
                    + M_1 s_{1}(q)\right)
                    + I_1 s_{1}(q)
                \right)
            \right]\right\rvert_{q=0}
            \nonumber
            \\
            &\quad-
            \left.\frac{\partial}{\partial q}
            \mathbb{E}_{0}^{\rnmeasure}\left[
                \delta_2 \mathbbm{1}_{\{\tau_{F}(q)>1\}}
                \left(
                    I_0 s_{0}(q) R_{1}
                    + qM_0 s_{0}(q) R_{1}
                \right) 
            \right]\right\rvert_{q=0}
            \label{eqn:sharholders-value-hedged-swap-step-1}
        \end{align}
        At the interim date the dealer receives back both the variation margin and the initial margin posted at time 0.
        Both are including the risk-free rate as interest.
        The dealer then immediately use the amounts to retire existing debt.
        Similarly, at the maturity date the dealer receives back the variation margin and the initial margin posted at time 1.
        These inflows are reflected in the first and the second term of \cref{eqn:sharholders-value-hedged-swap-step-1},
        where also the cash flows stemming from the hedged swap contract are reflected in the first term.
        Notice that only the amount of the variation margin posted,
        and not the initial margin,
        is dependent on the size of the investment, $q$.
        \\
        Term 3 describes the margin requirements that the dealer must finance by obtaining short-term debt at the interim date.
        Also the cash flow from the hedged swap contract is either financed by- or retiring some debt depending on the floating payment.
        This quantity will however be offset with the amount stemming from the original swap contract between the dealer and the counterparty.
        \\
        Finally, term 4 is the amount funded to cover the margin requirements at time 0.

        To deviate from the almost identical long mathematical derivations,
        the shareholders' marginal valuation of a swap contract with a fully collateralized hedged position is, according to \textcite{ADS2018}, given by:
        \begin{align}
            G_{\text{swap,hedged}} &= G_{\text{swap}} + \Gamma
            \\
            &=
            \mathbb{E}_{0}^{\rnmeasure}\left[
                \mathbbm{1}_{\{\tau_{F}>2\}}
                \left(
                    M_0 - u
                \right)
            \right]
            \nonumber
            \\
            &\quad-
            \mathbb{E}_{0}^{\rnmeasure}\left[
                \mathbbm{1}_{\{\tau_{F}>2\}}
                \left(
                    \sum_{t=1}^{2} \delta_{t} \mathbbm{1}_{\{\tau_{C}=t\}}\left(1-\beta_t\right)Y_{t}^{+}
                \right)
            \right]
            \nonumber
            \\
            &\quad-
            \mathbb{E}_{0}^{\rnmeasure}\left[
                \delta_2 \mathbbm{1}_{\{\tau_{F}>2\}} \mathbbm{1}_{\{\tau_{C}>1\}}
                \left(
                    u R_{1} S_{0} + M_{1} S_{1} + (u - M_0)
                \right)
            \right]
            \nonumber
            \\
            &\quad-
            \mathbb{E}_{0}^{\rnmeasure}\left[
                \delta_1 \mathbbm{1}_{\{\tau_{F}>1\}}
                I_0 S_0 
            \right] 
            +
            \mathbb{E}_{0}^{\rnmeasure}\left[
                \delta_2 \mathbbm{1}_{\{\tau_{F}>2\}} \mathbbm{1}_{\{\tau_{C}>1\}}
                I_1 S_1 
            \right] 
        \end{align}

        % By expressing \cref{eqn:sharholders-value-hedged-swap-step-1} as a difference quotient, and by linearity of expectations, the following limit is obtained:
        % \begin{align}
        %     &=
        %     \lim_{q \rightarrow 0}
        %     \frac{
        %         \mathbb{E}_{0}^{\rnmeasure}\left[
        %             \delta_2 \mathbbm{1}_{\{\tau_{F}(q)>2\}}
        %             \mathbbm{1}_{\{\tau_{C}(q)>1\}}
        %             \left(
        %                 - q\left(\mathcal{C}_1 R_1 + \mathcal{C}_2
        %                 + M_0 R_0 s_1(q)
        %                 + M_1 R_1\right)
        %             \right)
        %         \right]
        %     }{
        %         q
        %     }
        %     \nonumber
        %     \\
        %     % &\quad+
        %     % \lim_{q \rightarrow 0}
        %     % \frac{
        %     %     \mathbb{E}_{0}^{\rnmeasure}\left[
        %     %         \delta_2 \mathbbm{1}_{\{\tau_{F}(q)>2\}}
        %     %         \mathbbm{1}_{\{\tau_{C}(q)>1\}}
        %     %         \left(
        %     %             I_0 R_0 R_1
        %     %             % I_0 R_0 s_1(q)
        %     %             % The dealer is assumed to only retire debt with only the received variation margin, and not initial margin.
        %     %             + I_1 R_1
        %     %         \right) 
        %     %     \right]
        %     % }{
        %     %     q
        %     % }
        %     % \nonumber
        %     % \\
        %     &\quad-
        %     \lim_{q \rightarrow 0}
        %     \frac{
        %         \mathbb{E}_{0}^{\rnmeasure}\left[
        %             \delta_2 \mathbbm{1}_{\{\tau_{F}(q)>2\}}
        %             \mathbbm{1}_{\{\tau_{C}(q)>1\}}
        %             \left(
        %                 q\left(\mathcal{C}_{1} s_{1}(q)
        %                 + M_1 s_{1}(q)\right)
        %                 + I_1 s_{1}(q)
        %             \right)
        %         \right]
        %     }{
        %         q
        %     }
        %     \nonumber
        %     \\
        %     &\quad-
        %     \lim_{q \rightarrow 0}
        %     \frac{
        %         \mathbb{E}_{0}^{\rnmeasure}\left[
        %             \delta_2 \mathbbm{1}_{\{\tau_{F}(q)>1\}}
        %             \left(
        %                 I_0 s_{0}(q) R_{1}
        %                 + qM_0 s_{0}(q) R_{1}
        %             \right) 
        %         \right]
        %     }{
        %         q
        %     }
        %     \nonumber
        %     % \\
        %     % &\quad-
        %     % \lim_{q \rightarrow 0}
        %     % \frac{
        %     %     \mathbb{E}_{0}^{\rnmeasure}\left[
        %     %         \delta_2
        %     %         \mathbbm{1}_{\{\tau_{F}>2\}}
        %     %         \mathbbm{1}_{\{\tau_{C}>1\}}
        %     %         \left(
        %     %             I_0 R_0 S_1
        %     %             + I_1 R_1
        %     %             - I_1 S_1
        %     %         \right)
        %     %         -
        %     %         \delta_2
        %     %         \mathbbm{1}_{\{\tau_{F}>1\}}
        %     %         I_0 S_0 R_1
        %     %     \right]
        %     % }{
        %     %     q
        %     % }
        %     \\
        %     &\quad-
        %     \lim_{q \rightarrow 0}
        %     \frac{
        %         \mathbb{E}_{0}^{\rnmeasure}\left[
        %             \delta_2
        %             \left(
        %                 \mathbbm{1}_{\{\tau_{F}>2\}}
        %                 \mathbbm{1}_{\{\tau_{C}>1\}}
        %                 -
        %                 \mathbbm{1}_{\{\tau_{F}(q)>2\}}
        %                 \mathbbm{1}_{\{\tau_{C}(q)>1\}}
        %             \right)
        %             \left(
        %                 I_0 R_0 R_1
        %                 + I_1 R_1
        %             \right)
        %         \right]
        %     }{
        %         q
        %     }
        %     \label{eqn:sharholders-value-hedged-swap-step-2}
        % \end{align}
        % Following the exact same proof as in the previous section $\mathbbm{1}_{\{\tau_{F}(q)>2\}}
        % \mathbbm{1}_{\{\tau_{C}(q)>1\}}
        % -
        % \mathbbm{1}_{\{\tau_{F}>2\}}
        % \mathbbm{1}_{\{\tau_{C}>1\}}
        % = 0$,
        % and the fourth term becomes 0.

        % \begin{align}
        %     \Gamma &=
        %         \mathbb{E}_{0}^{\rnmeasure}\left[
        %             \delta_2 \mathbbm{1}_{\{\tau_{F}>2\}}
        %             \mathbbm{1}_{\{\tau_{C}>1\}}
        %             \left(
        %                 -\left(\mathcal{C}_1 R_1 + \mathcal{C}_2
        %                 + M_0 R_0 S_1
        %                 + M_1 R_1\right)
        %             \right)
        %         \right]
        %     \nonumber
        %     \\
        %     &\quad-
        %     \mathbb{E}_{0}^{\rnmeasure}\left[
        %         \delta_2 \mathbbm{1}_{\{\tau_{F}>2\}}
        %         \mathbbm{1}_{\{\tau_{C}>1\}}
        %         \left(
        %             \mathcal{C}_{1} S_1
        %             + M_1 S_1
        %             + I_1 S_1
        %         \right)
        %     \right]
        %     \nonumber
        %     \\
        %     &\quad-
        %     \mathbb{E}_{0}^{\rnmeasure}\left[
        %         \delta_2 \mathbbm{1}_{\{\tau_{F}>1\}}
        %         \left(
        %             I_0 S_0 R_{1}
        %             + M_0 S_0 R_{1}
        %         \right) 
        %     \right]
        % \end{align}

        This value defines the transfer of wealth for the dealer's shareholders.
        If it assumed that the project is a zero net present value investment for the counterparty, the new creditors, and the hedge dealer,
        then the legacy creditors will end up with a wealth transfer of $-G_{\text{swap,hedged}}$.

        Fixing the price of the swap contract between the dealer and the counterparty,
        the value of the swap contract that makes the shareholders indifferent of entering the project, $\tilde{u}^{\ast}$,
        is found by solving the following equation for $\tilde{u}$:
        \begin{equation}
            G_{\text{swap,hedged}} = 0
            \label{eqn:shareholder-breakeven-swap-hedged}
        \end{equation}
        The \FVA/ is then the donation of the hedge dealer, i.e. the difference between the first price of the swap and the regulated price:
        \begin{equation}
            \FVA/ = \tilde{u}^{\ast} - \tilde{u}
        \end{equation}

        Alternatively, the value that makes the project a zero net present value investment for the shareholders is found by instead fixing the price of the hedged swap contract,
        and then solve \cref{eqn:shareholder-breakeven-swap-hedged}
        for $u$.
        The two approaches will provide the same \FVA/ amount.

        Having obtained a result for both an unsecured swap contract as well as a swap contract with a fully collateralized hedged position,
        the next section will apply the results in a simply constructed multi-period framework.
        The examples will have multiple states in each of the two time periods, where also the interest will change over time.

\end{document}
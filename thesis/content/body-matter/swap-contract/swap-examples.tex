% !TEX root = ./sub-main.tex
\documentclass[main.tex]{subfiles}

\begin{document}
    \subsection{Numerical computations of a multi-period swap contract}
    \label{sec:swap-examples}
        This section aims to illustrate
        how the funding costs and benefits of a multi-period swap contract
        can be described by a simple economical setup.
        The examples will be based on the two results derived in the previous sections,
        namely the shareholders' marginal valuation of both an unsecured swap as well as a secured hedging swap.
        As opposed to the single-period model analysed in \cref{sec:single-period-model},
        different fundamental values in this \namecref{sec:swap-examples} will have a probability of changing over time.
        The values in reference are specifically
        the asset base of the dealer,
        the risk-free rate,
        the swap market value,
        and of course the swap floating rate.
        
        Maintaining the purposes of the examples,
        the count of possible states in each time period is narrowed down from 5 to 2.
        These will be referred to as the \textit{up-state} and the \textit{down-state},
        as the asset base will either increase in value or decrease in value.
        At the interim date, the two possible states are denoted by $\omega_u$ and $\omega_d$.
        Correspondingly, at the maturity date the two possible states are either
        $\omega_{u,u}$ and $\omega_{u,d}$,
        or $\omega_{d,u}$ and $\omega_{d,d}$,
        depending on the asset value going up or down in the first time period respectively.
        \\
        At the state $\omega_{d,d}$, the liabilities are assumed to exceed the asset value,
        and the dealer defaults.
        In this case the shareholders receive a payoff of 0.
        \\
        The asset value structure is defined using a binomial tree
        and illustrated in \cref{fig:example-asset-value-multi-period-dealer}.
        Each period the asset values are assumed to either increase with a factor $\mathscr{u} = 1.1$
        or decrease with a factor $\mathscr{d} = 0.85$.
        The asset values are determined under the risk-neutral measure,
        such that at each point in time the value is equal to the expected discounted value of the next period.
        By that definition, and by a short-term risk-free rate assumed to be $r_{0,1}=2\%$ at the inception date,
        the probability of being in the up-state at the interim date is calculated as:
        \begin{equation*}
            \mathbb{P}^{\rnmeasure}\left(A_1 = 110\right) = 
            p_{\mathscr{u}} =
            \frac{
                1 + 2\% - 0.85
            }{
                1.1 - 0.85
            }
            = \num{0.68}
        \end{equation*}
        And consequently, $p_{\mathscr{d}} = 1 - p_{\mathscr{u}} = \num{0.32}$.

        The long-term risk-free discount rate from time 0 to time 2 is assumed to be $r_{0,2}=2.2\%$.
        To determine the expected short-term rate at the interim date, the forward rate is calculated:
        \begin{equation*}
            r_{1,2} = \left(
                \frac{
                    \left(1 + 2.2\%\right)^{2}
                }{
                    \left(1 + 2\%\right)^{1}
                }
            \right)^{1/(2-1)} - 1
            = 2.4\%
        \end{equation*}
        
        The risk-neutral probabilities of moving to the three different states at the maturity date are calculated as:
        \begin{align*}
            \mathbb{P}^{\rnmeasure}\left(A_2 = 121\right) = 
            p_{\mathscr{u},\mathscr{u}}
            &=
            % p_{\mathscr{u}}
            % \frac{
            %     1 + 2.4\% - 0.85
            % }{
            %     1.1 - 0.85
            % }
            % =
            \num{0.473291}
            \\
            \mathbb{P}^{\rnmeasure}\left(A_2 = 93.5\right) = 
            p_{\mathscr{u},\mathscr{d}} +
            p_{\mathscr{d},\mathscr{u}}
            &=
            % p_{\mathscr{d}}
            % \frac{
            %     1 + 2.4\% - 0.85
            % }{
            %     1.1 - 0.85
            % }
            % +
            % p_{\mathscr{u}}
            % \frac{
            %     1.1 - (1 + 2.4\%)
            % }{
            %     1.1 - 0.85
            % }
            % =
            \num{0.429434}
            \\
            \mathbb{P}^{\rnmeasure}\left(A_2 = 72.25\right) = 
            p_{\mathscr{d},\mathscr{d}}
            &=
            % p_{\mathscr{d}}
            % \frac{
            %     1.1 - (1 + 2.4\%)
            % }{
            %     1.1 - 0.85
            % }
            % =
            \num{0.097275}
        \end{align*}

        At the interim date, the dealer's creditors are assumed to have a pre-project known claim of 10,
        i.e. the short-term liabilities,
        meaning $W=10(1+r_{1,2})=\num{10.240039}$.
        Meanwhile, the long-term liabilities, which are also known with certainty, are $L_{2} = 70$.
        The dividend paid to the shareholders at the interim date is assumed to be $\theta_1 = 0$, and there are no distress costs.
        As mentioned above, the asset value does not cover the total liabilities at state $\omega_{\mathscr{d},\mathscr{d}}$ as $72.25 < L_{2} + W$,
        hence the dealer defaults.
        \\
        The short-term liabilities are, on the other hand, not large enough to trigger a default at the interim date
        regardless of the state outcome.

        \begin{figure}[H]
            \centering
            \begin{tikzpicture}[>=stealth,sloped]
                \matrix (tree) [
                  matrix of nodes,
                  minimum size=1cm,
                  column sep=3.5cm,
                  row sep=0.85cm,
                ]
                {
                &               & $\mathscr{u}^{2}A_{0}=121$ \\
                & $\mathscr{u}A_{0} = 110$ & \\
    $A_0 = 100$ &               & $\mathscr{u}\mathscr{d}A_{0}=93.5$ \\
                & $\mathscr{d}A_{0} = 85$  & \\
                &               & $\mathscr{d}^{2}A_{0}=72.25$ \\
                };
                \draw[->] (tree-3-1) -- (tree-2-2) node [midway,above] {\tiny
                $p_{\mathscr{u}}=\num{0.68}$};
                \draw[->] (tree-3-1) -- (tree-4-2) node [midway,below] {\tiny
                $p_{\mathscr{d}}=\num{0.32}$};
                \draw[->] (tree-2-2) -- (tree-1-3) node [midway,above] {\tiny
                $p_{\mathscr{u},\mathscr{u}}=\num{0.473291}$};
                \draw[->] (tree-2-2) -- (tree-3-3) node [midway,below] {\tiny
                $p_{\mathscr{u},\mathscr{d}}=\num{0.222725}$};
                \draw[->] (tree-4-2) -- (tree-3-3) node [midway,above] {\tiny
                $p_{\mathscr{d},\mathscr{u}}=\num{0.206709}$};
                \draw[->] (tree-4-2) -- (tree-5-3) node [midway,below] {\tiny
                $p_{\mathscr{d},\mathscr{d}}=\num{0.097275}$};
            \end{tikzpicture}
            \caption{This figure shows the asset value of the dealer in each time period as a binomial tree.}
            \label{fig:example-asset-value-multi-period-dealer}
        \end{figure}

        Having defined the setup of the dealer's capital structure,
        a swap contract is now introduced between the dealer and a counterparty.
        For simplicity, the counterparty is assumed to have no credit-risk.
        The swap will have a relatively small notional payoff to the extend
        that it does not affect the default- and no-default states of the dealer.

        The first example will apply the theory from \cref{sec:unsecured-swap-valuation}
        to find the shareholders' valuation of the project being entered.
        In the second example the same swap will be agreed between the dealer and the counterparty.
        Additionally, the dealer takes a fully collateralized hedged position at a hedge dealer.
        This allows for the theory discussed in \cref{sec:secured-swap-valuation} to be applied.

        \subsubsection{Example of unsecured swap}
            Suppose the dealer enters a swap contract with the credit risk-free counterparty.
            The dealer pays a constant fixed leg, $K_t$, to the counterparty at each time period whose amount is $\num{1}$.
            Correspondingly, at each time period,
            the dealer receives a random floating leg, $X_t$, from the counterparty.

            As a consequence of the counterparty being credit-risk free,
            the dealer can always expect to receive the full contractual amount of the floating leg.
            Therefore, when valuing the swap contract,
            the dealer will not make a \CVA/; 
            the only difference between the risk-free value and the credit risk adjusted price is the \DVA/.

            The floating legs can take on any amount within reason,
            such that the cash flows at each time period are random.
            However, the market neutral expected cash flows are assumed to be known
            by both the dealer and the counterparty.
            If the swap is of positive value for the dealer,
            the dealer will have to pay an upfront price to the counterparty,
            and vice versa.

            The time 0 expected payments of the swap contract are summarized in \cref{tbl:swap-legs}.
            Notice, that at the interim date the dealer expects a positive cash flow of
            $\mathcal{C}_1 = X_1 - K_1 = \num{0.2}$,
            which will be used to retire existing debt.
            At the maturity date the dealer is assumed to liquidate all of its assets,
            which it uses to pay the remaining outstanding amounts.
            \\
            Since the expected discounted market value of the floating payments is largest,
            the dealer seems to be obliged to pay the counterparty an upfront price of.
            However, it is assumed that the swap is traded as a zero net present value investment
            for both the dealer and the counterparty,
            where no wealth is destroyed as per the Modigliani-Miller proposition.
            For this to be applicable, the price of the swap must also account for the dealer's own default risk,
            i.e. make the aforementioned \DVA/.
            
            \begin{table}[H]
                \centering
                \begin{tabular}{l|rr||r}
                    $t$ & 1 & 2 & Present value \\
                    \hline
                    \rule{0pt}{1.1em}
                    $\mathbb{E}_{0}^{\rnmeasure}\left[K_t\right]$ & $\num{1}$ & $\num{1}$ & $\num{1.9378}$ \\
                    \rule{0pt}{1.1em}
                    $\mathbb{E}_{0}^{\rnmeasure}\left[X_t\right]$ & $\num{1.2}$ & $\num{0.95}$ & $\num{2.08601}$ \\
                \end{tabular}
                \caption{}
                \label{tbl:swap-legs}
            \end{table}

            When defaulting at time 2
            the dealer's total liabilities being paid to the creditors are calculated as:
            \begin{align*}
                \mathbb{E}_{t}\left[
                    (1-\phi)\left(L_{2} - 
                    \mathcal{C}_{1}R_{1}
                    - \mathcal{C}_2\right)
                \right]
                &=
                (1-\phi)\left(\num{70} - \num{0.204801} + \num{0.05}\right)\\
                &=
                (1-\phi)\num{69.8452}
            \end{align*}
            And by definition, this amount has equal the total remaining asset value.
            Remember that the dealer has at time 2 already paid the liabilities due at the interim date.
            To find the price of the project
            as well as the loss rate after obtaining the project,
            the following equation system must be solved must be solved for $u$ and $\phi$:
            \begin{align}
                \num{72.25} - W
                =
                (1-\phi)\num{69.8452}
                \\
                u =
                (\num{2.08601} - \num{1.9378})(1-\phi)\left(1-p_{\mathscr{d},\mathscr{d}}\right)
                +
                (\num{2.08601} - \num{1.9378})p_{\mathscr{d},\mathscr{d}}
            \end{align}
            Solving this equation system yields a market neutral swap price of $u=\num{0.115334}$,
            and a loss rate of $\phi=\pct{0.11218}$.
            The dealer's creditors of the long-term debt then receives a payoff of $70(1-\pct{0.11218})=\num{62.1474}$ in case of default.
            This suggests a post-project actual credit spread of:
            \begin{equation}
                S = \frac{
                    \num{70}
                }{  
                    \delta_{2}\left(
                        \num{70}(1-p_{\mathscr{d},\mathscr{d}})
                        +
                        \num{62.03197}p_{\mathscr{d},\mathscr{d}}
                    \right)
                } - (1+r_{0,2})^{2}
                =
                \pct{0.0115265}
            \end{equation}
            Since the dealer survives at time 1 with 100\% certainty,
            the upfront price can be funded with short-term debt paying 0 credit spread.
            If the firm were to pay a spread on the short-term debt starting at time 0,
            this would be subtracted from the liquidated asset value. 

            By using the risk-neutral marginal market valuation of the swap contract
            defined in \cref{eqn:swap-value-multi-period},
            the \DVA/ can then be computed as:
            \begin{align}
                \DVA/ &= V_{rf} - u
                \\
                &= \sum_{t=1}^{2}\delta_t \mathcal{C}_t - u
                = 0.0129949
            \end{align}

            Turning to the shareholder valuation of the project.
            As cash flows from swap contract are either funded by obtaining debt if negative or retiring some existing debt if positive,
            an impact on the shareholders' value is may have occurred.
            Using the market neutral expected discounted value to determine the price, the project is a zero net present value investment for the counterparty.
            If a wealth transfer has happened, it is bound to be between the shareholders and the long-term creditors.

            \textcolor{red}{Unfinished}

        \subsubsection{Example of secured hedging swap}
\end{document}
% !TEX root = ./sub-main.tex
\documentclass[main.tex]{subfiles}

\begin{document}
    \section{Numerical computations of a multi-period swap contract}
    \label{sec:swap-examples}
        This section aims to illustrate
        how the funding costs and benefits of a multi-period swap contract
        can be described by a simple economical setup.
        The examples will be based on the two results derived in the previous sections,
        namely the shareholders' marginal valuation of both an unsecured swap as well as a secured hedging swap.
        As opposed to the single-period model analysed in \cref{sec:single-period-model},
        different fundamental values in this \namecref{sec:swap-examples} will have a probability of changing over time.
        The values in reference are specifically
        the risk-free rate,
        the swap market value,
        and of course the swap floating rate.
        
        Maintaining the purposes of the examples,
        the count of possible states in each time period is narrowed down from 5 to 2.
        These will be referred to as the \textit{up-state} and the \textit{down-state},
        as the asset base will either increase in value or decrease in value.
        At the interim date, the two possible states are denoted by $\omega_u$ and $\omega_d$.
        Correspondingly, at the maturity date the two possible states are either
        $\omega_{u,u}$ and $\omega_{u,d}$,
        or $\omega_{d,u}$ and $\omega_{d,d}$,
        depending on the asset value going up or down in the first time period respectively.
        \\
        In the state $\omega_{d,d}$, the liabilities exceed the asset value,
        and the dealer defaults.
        In this case the shareholders receive a payoff of 0.
        \\
        The asset value structure is defined using a binomial tree
        and illustrated in \cref{fig:example-asset-value-multi-period-dealer}.
        Each period the asset values are assumed to either increase with a factor $\mathscr{u} = 1.1$
        or decrease with a factor $\mathscr{d} = 0.85$.
        The asset values are determined under the risk-neutral measure,
        such that at each point in time the value is equal to the expected discounted value of the next period.
        By that definition, and by a short-term risk-free rate assumed to be $r_{0,1}=2\%$ at the inception date,
        the probability of being in the up-state at the interim date is calculated as:
        \begin{equation*}
            \mathbb{P}^{\rnmeasure}\left(A_1 = 110\right) = 
            p_{\mathscr{u}}^{0} =
            \frac{
                1 + 2\% - 0.85
            }{
                1.1 - 0.85
            }
            = \num{0.68}
        \end{equation*}
        And consequently, $p_{\mathscr{d}}^{0} = 1 - p_{\mathscr{u}}^{0} = \num{0.32}$.

        The long-term risk-free discount rate from time 0 to time 2 is assumed to be $r_{0,2}=2.2\%$.
        To determine the expected short-term rate at the interim date, the forward rate is calculated:
        \begin{equation*}
            r_{1,2} = \left(
                \frac{
                    \left(1 + 2.2\%\right)^{2}
                }{
                    \left(1 + 2\%\right)^{1}
                }
            \right)^{1/(2-1)} - 1
            = 2.4\%
        \end{equation*}
        
        The risk-neutral probabilities of moving to the three different states at the maturity date are calculated as:
        \begin{align*}
            \mathbb{P}^{\rnmeasure}\left(A_2 = 121\right) = 
            p_{\mathscr{u},\mathscr{u}}
            &=
            % p_{\mathscr{u}}
            % \frac{
            %     1 + 2.4\% - 0.85
            % }{
            %     1.1 - 0.85
            % }
            % =
            \num{0.473291}
            \\
            \mathbb{P}^{\rnmeasure}\left(A_2 = 93.5\right) = 
            p_{\mathscr{u},\mathscr{d}} +
            p_{\mathscr{d},\mathscr{u}}
            &=
            % p_{\mathscr{d}}
            % \frac{
            %     1 + 2.4\% - 0.85
            % }{
            %     1.1 - 0.85
            % }
            % +
            % p_{\mathscr{u}}
            % \frac{
            %     1.1 - (1 + 2.4\%)
            % }{
            %     1.1 - 0.85
            % }
            % =
            \num{0.429434}
            \\
            \mathbb{P}^{\rnmeasure}\left(A_2 = 72.25\right) = 
            p_{\mathscr{d},\mathscr{d}}
            &=
            % p_{\mathscr{d}}
            % \frac{
            %     1.1 - (1 + 2.4\%)
            % }{
            %     1.1 - 0.85
            % }
            % =
            \num{0.097275}
        \end{align*}
        since the risk neutral probability of moving to an up-state in the second time period is $(1+r_{1,2}-0.85)/(1.1-0.85) = \num{0.696016}$.

        At the interim date, the dealer's creditors have a known claim of 10,
        i.e. the short-term liabilities,
        meaning $W=10(1+r_{1,2})=\num{10.240039}$.
        Meanwhile, the long-term liabilities, which are also known with certainty, are $L_{2} = 70$.
        The dividend paid to the shareholders at the interim date is $\theta_1 = 0$, and there are no distress costs.
        As mentioned above, the asset value does not cover the total liabilities at state $\omega_{\mathscr{d},\mathscr{d}}$ as $72.25 < L_{2} + W$,
        hence the dealer defaults.
        \\
        The short-term liabilities are, on the other hand, not large enough to trigger a default at the interim date
        regardless of the state outcome.

        \begin{figure}[H]
            \centering
            \begin{tikzpicture}[>=stealth,sloped]
                \matrix (tree) [
                  matrix of nodes,
                  minimum size=1cm,
                  column sep=3.5cm,
                  row sep=0.85cm,
                ]
                {
                &               & $\mathscr{u}^{2}A_{0}=121$ \\
                & $\mathscr{u}A_{0} = 110$ & \\
    $A_0 = 100$ &               & $\mathscr{u}\mathscr{d}A_{0}=93.5$ \\
                & $\mathscr{d}A_{0} = 85$  & \\
                &               & $\mathscr{d}^{2}A_{0}=72.25$ \\
                };
                \draw[->] (tree-3-1) -- (tree-2-2) node [midway,above] {\tiny
                $p_{\mathscr{u}}=\num{0.68}$};
                \draw[->] (tree-3-1) -- (tree-4-2) node [midway,below] {\tiny
                $p_{\mathscr{d}}=\num{0.32}$};
                \draw[->] (tree-2-2) -- (tree-1-3) node [midway,above] {\tiny
                $p_{\mathscr{u},\mathscr{u}}=\num{0.473291}$};
                \draw[->] (tree-2-2) -- (tree-3-3) node [midway,below] {\tiny
                $p_{\mathscr{u},\mathscr{d}}=\num{0.206709}$};
                \draw[->] (tree-4-2) -- (tree-3-3) node [midway,above] {\tiny
                $p_{\mathscr{d},\mathscr{u}}=\num{0.222725}$};
                \draw[->] (tree-4-2) -- (tree-5-3) node [midway,below] {\tiny
                $p_{\mathscr{d},\mathscr{d}}=\num{0.097275}$};
            \end{tikzpicture}
            \caption{This figure shows the asset value of the dealer in each time period as a binomial tree.}
            \label{fig:example-asset-value-multi-period-dealer}
        \end{figure}

        Having defined the setup of the dealer's capital structure,
        a swap contract is now introduced between the dealer and a counterparty.
        For simplicity, the counterparty is assumed to have no credit-risk.
        The swap will have a relatively small notional payoff to the extend
        that it does not affect the default- and no-default states of the dealer.

        The first example will apply the theory from \cref{sec:unsecured-swap-valuation}
        to find the shareholders' valuation of the project being entered.
       In the second example, the dealer additionally hedges the swap's market risk 
       by trading the opposite position in an exchange.
       The original trade will be unsecured 
       while the hedge will be traded under a collateralization agreement,
       which allows for the theory discussed in \cref{sec:secured-swap-valuation} to be applied.

        \subsection{Example of unsecured swap}
            Suppose the dealer enters a swap contract with the credit risk-free counterparty.
            She pays a constant fixed leg, $K$, to the counterparty at each time period whose amount is $\num{1}$.
            Correspondingly, at each time period,
            the dealer receives a random floating leg, $X_t$, from the counterparty.

            Since the counterparty is credit-risk free,
            the dealer can always expect to receive the full contractual amount of the floating leg.
            Therefore, when valuing the swap contract,
            she will not make a \CVA/; 
            the difference between the risk-free value and the credit risk adjusted price is the \DVA/.

            The floating legs can take on any amount within reason,
            such that the cash flows at each time period are random.
            However, the market neutral expected cash flows are known
            by both the dealer and the counterparty.
            If the swap is of positive value for the dealer,
            she will have to pay an upfront price to the counterparty,
            and vice versa.

            The time 0 expected payments of the swap contract are summarized in \cref{tbl:swap-legs}.
            Notice, that at the interim date the dealer expects a positive cash flow of
            $\mathcal{C}_1 = X_1 - K = \num{0.2}$,
            which will be used to retire existing debt.
            The expected cash flow at time 2 does not change over time.

            At the maturity date the dealer liquidates all of her assets,
            which it uses to pay the remaining outstanding amounts.
            Additionally, the existing creditors' liabilities are assumed to rank pari passu with the counterparty's contingent liabilities.
            \\
            Since the expected discounted market value of the floating payments is largest,
            the dealer seems to be obliged to pay the counterparty an upfront price of.
            However, it is assumed that the swap is traded as a zero net present value investment
            for both the dealer and the counterparty,
            where no wealth is destroyed as per the Modigliani-Miller proposition.
            For this to be applicable, the price of the swap must also account for the dealer's own default risk,
            i.e. make the aforementioned \DVA/.
            
            \begin{table}[H]
                \centering
                \begin{tabular}{l|rr||r}
                    $t$ & 1 & 2 & Present value \\
                    \hline
                    \rule{0pt}{1.1em}
                    $\mathbb{E}_{0}^{\rnmeasure}\left[K\right]$ & $\num{1}$ & $\num{1}$ & $\num{1.937802706}$ \\
                    \rule{0pt}{1.1em}
                    $\mathbb{E}_{0}^{\rnmeasure}\left[X_t\right]$ & $\num{1.2}$ & $\num{0.95}$ & $\num{2.08601}$ \\
                    \rule{0pt}{1.1em}
                    $\mathbb{E}_{0}^{\rnmeasure}\left[\mathcal{C}_t\right]$ & $\num{0.2}$ & $\num{-0.05}$ & $\num{0.1482079039}$ \\
                \end{tabular}
                \caption{}
                \label{tbl:swap-legs}
            \end{table}

            When defaulting at time 2
            the dealer's total liabilities are calculated as:
            \begin{align*}
                \mathbb{E}_{0}^{\rnmeasure}\left[
                    (1-\phi)\left(
                        L_{2} - 
                        \mathcal{C}_{1}R_{1}
                        - \mathcal{C}_2
                    \right)
                \right]
                &=
                (1-\phi)\left(\num{70} - \num{0.204801} + \num{0.05}\right)\\
                &=
                (1-\phi)\num{69.8452}
            \end{align*}
            And by definition, this amount is equal to the total remaining asset value.
            Remember that at time 2 the dealer has already paid the liabilities due at the interim date.
            To find the price of the project
            as well as the loss rate given default after obtaining the project,
            the following equation system must be solved for $u$ and $\phi$:
            \begin{align}
                &\left\{\begin{array}{@{}l@{}}
                    \mathscr{d}^{2}A_{0} - W=
                    \mathbb{E}_{0}^{\rnmeasure}\left[
                        (1-\phi)\left(L_{2} - 
                        \mathcal{C}_{1}R_{1}
                        - \mathcal{C}_2\right)
                    \right]
                    \\
                    u =
                    \mathbb{E}_{0}^{\rnmeasure}\left[
                        \sum_{t=1}^{2}
                        \delta_{t}\mathcal{C}_{t}
                    \right]
                \end{array}\right.
                \label{eqn:swap-example-equation-system}
                \\
                \Leftrightarrow
                \qquad
                &\left\{\begin{array}{@{}l@{}}
                    \displaystyle
                    \num{72.25} - \num{10.24}
                    =
                    (1-\phi)\num{69.8452}
                    \\
                    \displaystyle
                    u =
                    \frac{
                        \num{0.1482079039}(1-\phi)\left(1-p_{\mathscr{d},\mathscr{d}}\right)
                        +
                        \num{0.1482079039}p_{\mathscr{d},\mathscr{d}}
                    }{
                        (1+r_{0,2})^{2}
                    }
                \end{array}\right.
                \nonumber
            \end{align}
            The first equation in \cref{eqn:swap-example-equation-system} assures that the creditors receive the entire asset base if the firm defaults.
            The second equation sets the price of the swap as the discounted expected payoff.
            Solving this equation system yields a market neutral swap price of $u=\num{0.140347}$,
            and a loss rate in the default state of 
            $\phi(\omega_{\mathscr{d},\mathscr{d}})=\pct{0.11218}$.
            Using the loss rate, the value of the legacy creditors' long term claim can be calculated.
            At time 2 the payoff in the default state is 
            $L_{2}*(1-\phi(\omega_{\mathscr{d},\mathscr{d}})) = \num{62.147436}$.
            At time 1 the expected value of the claim is therefore \num{67.613161},
            and at time 0 the value is \num{66.287413}.

            The credit spread in each period is the debt's no-default value one period ahead
            divided by its current value less the risk-free rate.
            Hence the post-project actual credit spreads for time 0 and time 1 is:
            \begin{align}
                S_{0} 
                &=
                \frac{
                    \num{67.613161}
                }{
                    \num{66.287413}
                }
                -
                R_{0}
                =
                \pct{0}
                \\
                S_{1} 
                &= 
                \frac{
                    \num{70}
                }{
                    \num{67.613161}
                } 
                -
                R_{1}
                =
                \pct{0.011298}
            \end{align}
            Since the dealer survives at time 1 with 100\% certainty,
            the upfront price can be funded with short-term debt paying 0 credit spread.
            If the firm were to pay a spread on the short-term debt starting at time 0,
            this would be subtracted from the liquidated asset value. 

            By using the risk-neutral marginal market valuation of the swap contract
            defined in \cref{eqn:swap-value-multi-period},
            the \DVA/ can then be computed as:
            \begin{align}
                \DVA/ &= V_{rf} - u
                \nonumber
                \\
                &= \mathbb{E}_{0}^{\rnmeasure}\left[\sum_{t=1}^{2}\delta_t \mathcal{C}_t\right] - u
                = \num{0.00154881}
            \end{align}

            Turning to the shareholder valuation of the project.
            As cash flows from swap contract are either funded by obtaining debt if negative or retiring some existing debt if positive,
            an impact on the shareholders' value is may have occurred.
            Using the market neutral expected discounted value to determine the price, the project is a zero net present value investment for the counterparty.
            If a wealth transfer has happened, it is bound to be between the shareholders and the long-term creditors.

            The funding costs of financing the project with a short-term loan is paid by the shareholders.
            However, since the dealer cannot default at the interim date,
            the shareholders do not pay any credit spread from obtaining that debt.
            The dealer's cash flow at the interim date is on, the other hand, positive,
            and the debt expiring at time 2 is reduced with the exact cash flow amount.
            This entails a funding benefit, as the credit spread paid from time 1 to time 2 has decreased,
            and the shareholders' payoff increases.
            
            With $\mathbb{E}_{t}\left[\mathcal{C}_1\right]=\num{0.2}$,
            $\mathbb{E}_{t}\left[\mathcal{C}_2\right]=\num{-0.05}$,
            $u=\num{0.140347}$,
            and the assumption that the counterparty is credit risk-free,
            the total impact on the shareholders' benefit is calculated using \cref{eqn:shareholders-value-multi-period-swap}:
            \begin{equation}
                G_{swap} = \num{0.0076609}
            \end{equation}
            This value suggests a wealth transfer of $\num{0.0076609}$
            between the shareholders and the creditors of the long-term debt,
            such that the project is a negative net present value investment for the creditors.

            The price that would make this project a zero net present value investment for the shareholders,
            is found by fixing $G_{swap}$ to 0,
            and solving for $u^{\ast}$.
            The difference between the market neutral credit risk adjusted price and $u^{\ast}$ is then the \FVA/:
            \begin{equation}
                G_{swap} = 0
            \end{equation}
            yielding a funding adjusted price of $u^{\ast} = \num{0.148739}$.
            \\
            The donation from the counterparty required is then:
            \begin{equation}
                \FVA/ = u^{\ast} - u = \num{0.008392}
            \end{equation}

            The project's positive impact on the shareholders' wealth is an of expression of the limited funding costs
            that the dealer faces in this setup.
            In the next section the example is extended to include a fully collateralized hedged position.
            The margin requirements will then potentially imply an exposure to funding costs or funding benefits.

        \subsection{Example of secured hedging swap}
            Suppose that the dealer enters the same swap contract with the same counterparty as described above.
            Additionally, she hedges the swap contract by entering the reversed position with a hedging dealer
            by paying the floating leg.
            The hedging dealer requires the dealer to post margin requirements,
            i.e. both initial margin and variation margin.
            The initial margin is,
            similar to the single-period model,
            assumed to be 50\% of the market neutral credit risk adjusted value of the swap,
            and it is posted at both the inception date and the interim date.
            The margin requirements have higher priority than any other existing liabilities and outstanding contractual amounts due to the swap contract,
            hence the collateralization must be paid first when the dealers's assets are liquidated at time 2.
            Corresponding to the dealer, the hedging dealer is assumed credit risk-free at the interim date.

            The swap contract is still assumed to be priced as the market neutral credit risk adjusted value.
            Since the dealer enters two offsetting positions of the swap,
            there is no amount received or funded at time 0 due to the swap price.
            Furthermore, the cash flows at time 1 and time 2 from the two positions also offset each other.
            The dealer must however fund the margin requirements by obtaining new short-term debt,
            and similarly, she uses the received margin postings to retire existing debt.
            \\
            The specific required margin postings are summarized in \cref{tbl:swap-margin-postings}.

            \begin{table}[H]
                \centering
                \begin{tabular}{l|rr}
                    $t$ & 0 & 1 \\
                    \hline
                    \rule{0pt}{1.3em}
                    $\mathbb{E}_{0}^{\rnmeasure}\left[M_t\right]$ & $u$ & $\frac{1}{1+r_{1,2}}\mathbb{E}_{1}^{\rnmeasure}\left[\mathcal{C}_{2}\right]$ \\
                    \rule{0pt}{1.3em}
                    $\mathbb{E}_{0}^{\rnmeasure}\left[I_t\right]$ & $\frac{1}{2}u$ & $\frac{1}{2}\frac{1}{1+r_{1,2}}\mathbb{E}_{1}^{\rnmeasure}\left[\mathcal{C}_{2}\right]$ \\
                \end{tabular}
                \caption{}
                \label{tbl:swap-margin-postings}
            \end{table}

            As in the previous section the dealer cannot default at time 1,
            hence no credit spread is paid for the short-term loans at time 0.
            Posting margin requirements increases the asset value,
            and funding them increases the short-term liabilities.
            When the dealer receives back the forward discounted collateral,
            she then uses it to pay the debt with the exact amount at the interim date.
            \\
            The dealer will not have any contingent liabilities due to the swap payments at the maturity date.
            The margin requirements received (posted) at the interim date will be paid (received) back at the maturity date.
            As a result of that,
            the liabilities at time 2 has both increased and decreased with the margin requirements at time 1.
            If defaulting, the dealer's creditors will have a total payoff calculated as:
            \begin{align}
                \mathbb{E}_{0}^{\rnmeasure}\left[
                    (1-\phi)\left(
                        L_{2}
                        -(M_1+I_1)R_{1}
                        +(M_1+I_1)R_{1}
                    \right)
                \right] 
                &= (1-\phi)\num{70}
            \end{align}
            To find the market neutral credit risk adjusted price of the project, $u$, as well as the loss rate, $\phi$,
            the following equation system must be solved.
            \begin{align}
                \mathscr{d}^{2} A - W
                &=
                (1-\phi)\num{70}
                \\
                u &=
                \mathbb{E}_{0}^{\rnmeasure}\left[
                    \mathbbm{1}_{\{\tau_{F}>2\}} 
                    \sum_{t=1}^{2}
                    \delta_{t}\mathcal{C}_{t}
                \right]
            \end{align}
            As in the example with the unsecured swap the first equation makes sure
            that the entire asset value amount is distributed to the creditors.
            The second equation again sets the price as the discounted expected cash flows.
            \\
            By solving, a swap price of $u=\num{0.126904}$ is obtained.
            The price is slightly smaller than in the previous example,
            however it is only visible with five decimals,
            and not too relevant for this analysis.
            The new loss rate is computed as $\phi = \pct{0.114143}$.

            With the long-term debt creditors' expected payoff at time 1 calculated as $\num{67.600104}$,
            the credit spread at time 1 is given by:
            \begin{equation}
                S_1 =
                \frac{
                    \num{70}
                }{
                    \num{67.600104}
                }
                - R_{1}
                \nonumber
                = \pct{0.01149743885}
            \end{equation}
            where the credit spread at time 0 is 0,
            as the dealer cannot default at the interim date.
            
            Finally, the shareholders' valuation of the project is analysed.
            As the project is priced to be a zero net present value investment for the counterparty,
            a potential change of wealth must again be between the dealer's creditors and shareholders.
            The shareholders' change of wealth is computed the derived expression in \cref{eqn:shareholder-valuation-swap-hedged}.
            This derivation includes the extension of the fully collateralized hedged position.
            \\
            With the above computed credit spreads, and $u=\num{0.126904}$
            the shareholders' marginal valuation of entering the swap contract is calculated as:
            \begin{equation}
                G_{\text{swap,hedged}} = \num{-0.000727784}
            \end{equation}
            suggesting a negative net present value investment for the shareholders. 
            The price that makes the shareholders indifferent of entering the project, $u^{\ast}$, is then defined by fixing $G_{\text{swap,hedged}}$ to 0.
            This implies:
            \begin{equation}
                u^{\ast} = \num{0.12649211399483182}
            \end{equation}
            and an \FVA/ defined as:
            \begin{equation}
                \FVA/ = u^{\ast} - u = \num{-0.000411886}
            \end{equation}
            Hence the donation required of the counterparty would be a price reduction of \num{-0.000411886}.
            
\end{document}
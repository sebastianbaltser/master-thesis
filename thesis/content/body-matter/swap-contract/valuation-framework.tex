% !TEX root = sub-main.tex
\documentclass[main.tex]{subfiles}

\begin{document}
    \subsection{Marginal valuation framework}
        The multi-period framework extends the assumptions made in \cref{sec:single-period-model} to include multiple time periods.
        An additional date $2$ is introduced as the new time of maturity.
        At the interim time 1 the market economy is defined by a collection of random variables,
        and at maturity, all the uncertainty is settled. 
        The single-period gross risk-free returns at time 0 and time 1 are $R_0$ and $R_1$ respectively, where $R_1$ is random at time 0.
        This means that the risk-free rate is not required to be constant,
        and the market value of cash flows is defined as,
        \begin{equation}
            \pi(\mathcal{C}_{t}) =
            \sum_{t=1}^{2}
            \delta_t \mathbb{E}_{0}^{\rnmeasure}\left[\mathcal{C}_{t}\right]
        \end{equation}
        where $\delta_1 = \frac{1}{R_0}$,
        and $\delta_2 = \frac{1}{R_{0}R_{1}}$.
        The notation $\mathbb{E}_{0}^{\rnmeasure}$ is always assumed to describe the time 0 risk-neutral expectation, unless otherwise clarified.

        This setup considers a dealer whose assets at time 2 have a payoff given by some random variable $A$.
        Additionally, the firm has short-term liabilities, $L_1$, that expires at time 1 as well as long-term liabilities, $L_2$, that expires at time 2.
        \\
        As always, the firm defaults the first time it fails to meet its financial obligations.
        At time 1 the firm liquidates a portion of its assets to cover the short-term liabilities.
        The time 2 value of this portion is denoted as $W = L_1 R_1$.
        Also, at time 1 the firm pays out a dividend, $\theta_1 \geq 0$, to its shareholders,
        %Not sure about \theta being the right notation for the dividend
        which is considered a random variable.
        The default time of the firm is denoted by $\tau_{F}$,
        where $\tau_{F} = \infty$ if the firm survives at time 2.
        In case of no default, the firm is liquidated, and the remaining cash is distributed to the shareholders after the creditors are paid back.
        \\
        Finally, it is assumed that all liabilities rank pari passu,
        and the recovery rates at time 1 and time 2 are denoted by $\kappa_1$ and $\kappa_2$ respectively,
        i.e. the portion of the total asset value after distress costs that the creditors receive in case of default.

        To describe the total value of the firm's equity, it is essential to realise that, as always, the shareholders receive a payoff of 0, if the firm defaults.
        Looking at the potential payoff of the shareholders in case of no default, the total value is defined as following:
        \begin{equation}
            \pi(S)=
            \mathbb{E}_{0}^{\rnmeasure}\left[
                \delta_1\mathbbm{1}_{\{\tau_{F}>1\}} \theta_1
            \right]
            +
            \mathbb{E}_{0}^{\rnmeasure}\left[
                \delta_{2}\mathbbm{1}_{\{\tau_{F}>2\}} (A-W-L_{2})
            \right]
            \label{eqn:shareholder-value-multi-period}
        \end{equation}
        The first term in \cref{eqn:shareholder-value-multi-period} defines the discounted expected value of the dividend payout at time 1 in case of no default before time 2.
        If the dividend is non-zero, it benefits the shareholders,
        in the sense that it guarantees the shareholders to receive some payoff before a possible default event at time 2.
        The second term defines the discounted expected residual of the liquidated assets at time 2.
        If the firm does not have enough assets to cover the long-term liabilities, the creditors receive the remaining assets, and the firm defaults.
        This leaves a payoff of 0 for the shareholders.

        On the contrary, the creditors receive a payoff in every scenario.
        The size of the payoff is however fully dependent on if the firm default or not.
        The total value of the creditor's payoff is defined as following:
        \begin{align}
            \pi(D)&=
            \mathbb{E}_{0}^{\rnmeasure}\left[
                \delta_1
                \left(\mathbbm{1}_{\{\tau_{F}>1\}} L_1 + \mathbbm{1}_{\{\tau_{F}=1\}} 
                \frac{
                    \kappa_1 \mathbb{E}_{1}^{\rnmeasure}\left[A\right]
                }{
                    R_1
                }
                \right)
            \right]
            \nonumber
            \\
            &\quad+
            \mathbb{E}_{0}^{\rnmeasure}\left[
                \delta_{2}
                \left(\mathbbm{1}_{\{\tau_{F}>2\}}L_{2}
                +
                \mathbbm{1}_{\{\tau_{F}=2\}}\kappa_{2}(A-W)
                \right)
            \right]
            \label{eqn:creditor-value-multi-period}
        \end{align}
        In \cref{eqn:creditor-value-multi-period} the first and the second term basically defines the same thing but in the two different time periods.
        The first term in each expectation is the expiring liability payoff at the respective point in time given no default.
        The second term in each expectation is the total asset value in case of default.
        Note that for $\tau_{F}>1$ the firm survives at time 1.
        This means that at time 2 for both the shareholders and the creditors, the liquidated value at time 1, $W$,
        that was used to cover the short-term liabilities, is subtracted from the asset value.

        To ensure that the present value of the risk-neutral expected value of the new creditors' payoff equals the amount needed to fund the swap contract, $U(q)$,
        the idea is fundamentally the same as in the single-period model.
        To unravel the explicit credit spread in a two-period model,
        \cref{eqn:new-creditors-breakeven} is now extended.
        \\
        As the setup has an additional time period, the default indicator is separated into the two possible default scenarios:
        $\tau_F = 1$ and $\tau_F = 2$.
        The firm is assumed to receive a payoff from its secured position, $Y_{1,t}$, as well as its unsecured position, $Y_{2,t}$ at both time periods.
        Furthermore, the liabilities are, as mentioned, separated into short-term and long-term.
        Since the liabilities have two different terms, 
        it will be to also define two different credit spreads;
        one for the excess yield from period 0 to period 1,
        and another for the excess yield from period 1 to period 2.
        The former will be denoted $s_{0}(q)$ and the latter $s_{1}(q)$.
        The credit spreads must then solve the following equation,
        where the right hand side is the value of buying the bond from the firm,
        and the left hand side is the firm's funding needs:
        \begin{align}
            U(q)
            &=
                \mathbb{E}_{0}^{\rnmeasure}\left[
                    \delta_2
                    \mathbbm{1}_{\{\tau_{F}(q)>2\}}
                    U(q)
                    (R_{0} + s_{0}(q))(R_{1} + s_{1}(q)))
                \right]
            \nonumber
            \\
            &\quad+ \mathbb{E}_{0}^{\rnmeasure}
            \left[
                \delta_1\mathbbm{1}_{\{\tau_{F}(q)=1\}} \Pi_1 U(q)(R_{0} + s_{0}(q))(R_{1} + s_{1}(q)))
            \right]
            \nonumber
            \\
            &\quad+ \mathbb{E}_{0}^{\rnmeasure}
            \left[
                \delta_2\mathbbm{1}_{\{\tau_{F}(q)=2\}} \Pi_2 U(q)(R_{0} + s_{0}(q))(R_{1} + s_{1}(q)))
            \right]
            \label{eqn:new-creditors-breakeven-two-period-model}
        \end{align}
        where
        \begin{align}
            \Pi_1
            &=
            \frac{
                \kappa_1
                \mathbb{E}_{1}^{\rnmeasure}\left[ 
                    A + q Y_{1,1} + q Y_{2,1}^{+}
                \right]
            }{
                L_{1}
                +
                \frac{\mathbb{E}_{1}^{\rnmeasure}\left[L_{2}\right]}{R_{1}}
                +
                U(q)
                (R_{0} + s_{0}(q))(R_{1} + s_{1}(q)))
                +
                \frac{\mathbb{E}_{1}^{\rnmeasure}\left[q Y_{2,1}^{-}\right]}{R_{1}}
            }
            \nonumber
            \intertext{and}
            \Pi_2
            &=
            \frac{
                \kappa_2 \left(
                A - W + q Y_{1,2} + q Y_{2,1}^{+}
                \right)
            }{
                L_{2}
                +
                U(q)
                (R_{0} + s_{0}(q))(R_{1} + s_{1}(q)))
                +
                q Y_{2,2}^{-} 
            }
            \nonumber
        \end{align}
        The complement of defaulting in neither of time periods is to default at either time 1 or time 2. Using this fact, \cref{eqn:new-creditors-breakeven-two-period-model} can be rewritten as:
        \begin{equation}
            s(q) =
            \mathbb{E}_{0}^{\rnmeasure}\left[
                \delta_2\left(
                1
                + \mathbbm{1}_{\{\tau_{F}(q) = 1\}}(\Pi_1 - 1)
                + \mathbbm{1}_{\{\tau_{F}(q) = 2\}}(\Pi_2 - 1)
                \right)
            \right]^{-1} - \mathbb{E}_{0}^{\rnmeasure}\left[R_0 R_1\right]
        \end{equation}
        The credit spread is based on the likelihood of the firm defaulting and can therefore change over time.
        This means that the credit spread is calculated for each time period, making it interpretable as multiple single-periods.
        \\
        Corresponding to the single-period model,
        let $S_0 + S_1$ denote the limiting spread which is defined as the credit spread on debt charged for obtaining an infinitesimal quantity of the project,
        i.e. $S_0 + S_1 \equiv \lim_{q \downarrow 0} s(q)$.
        Applying this to the equation:
        \begin{equation}
            S_0 + S_1 =
            \mathbb{E}_{0}^{\rnmeasure}\left[
                \delta_2\left(
                1
                + \mathbbm{1}_{\{\tau_{F} = 1\}}(\Pi_1 - 1)
                + \mathbbm{1}_{\{\tau_{F} = 2\}}(\Pi_2 - 1)
                \right)
            \right]^{-1} - \mathbb{E}_{0}^{\rnmeasure}\left[R_0 R_1\right]
            \label{eqn:aggregate-credit-spread-multi-period}
        \end{equation}
        With the objective to present a credit spread for each time interval, the equation is now divided into two.
        This will define an expression for both $S_0$ and $S_1$.
        \\
        The gross risk-free rate has no effect on the credit spread of a different time period than the applicable.
        Using this, every instance of the expectation of $R_1$ is removed from the computation of $S_0$.
        Similarly, the observed gross risk-free rate of time 0 is removed from the computation of $S_1$.
        Inserting the expression for $\Pi_1$, the credit spread of the first time interval is then defined as following: 
        \begin{align}
            S_0 &=
            \mathbb{E}_{0}^{\rnmeasure}\left[
                \delta_2\left(
                    1
                    + \mathbbm{1}_{\{\tau_{F} = 1\}}\left(
                        \frac{\kappa_1 \mathbb{E}_{1}^{\rnmeasure}\left[A\right]}{L_1 + \frac{\mathbb{E}_{1}^{\rnmeasure}\left[L_2\right]}{R_1}} - 1
                    \right)
                \right)
            \right]^{-1} - \mathbb{E}_{0}^{\rnmeasure}\left[R_0 R_1\right]
            \nonumber
            \\
            &=
            \mathbb{E}_{0}^{\rnmeasure}\left[
                \delta_2 \left(
                    1
                    + \mathbbm{1}_{\{\tau_{F} = 1\}}
                    \frac{
                        \kappa_1 \mathbb{E}_{1}^{\rnmeasure}\left[A\right] - \left(L_1 + \frac{\mathbb{E}_{1}^{\rnmeasure}\left[L_2\right]}{R_1}\right)
                    }{
                        L_1 + \frac{\mathbb{E}_{1}^{\rnmeasure}\left[L_2\right]}{R_1}
                    }
                \right)
            \right]^{-1} - \mathbb{E}_{0}^{\rnmeasure}\left[R_0 R_1\right]
            \nonumber
            \\
            &=
            \mathbb{E}_{0}^{\rnmeasure}\left[
                \frac{1}{R_0} \left(
                    1
                    + \mathbbm{1}_{\{\tau_{F} = 1\}}
                    \frac{
                        \kappa_1 \mathbb{E}_{1}^{\rnmeasure}\left[A\right] - \left(L_1 + \frac{\mathbb{E}_{1}^{\rnmeasure}\left[L_2\right]}{R_1}\right)
                    }{
                        L_1 + \frac{\mathbb{E}_{1}^{\rnmeasure}\left[L_2\right]}{R_1}
                    }
                \right)
            \right]^{-1} - R_0
            \nonumber
        \end{align}
        So, similar as in \cref{sec:single-period-model}, the firm's marginal credit spread for short-term debt at time 0 is computed as following:
        \begin{equation}
            \Leftrightarrow \qquad
            S_0 =
            \frac{
                \mathbb{E}_{0}^{\rnmeasure}\left[\phi_1\right]R_0
            }{
                1 - \mathbb{E}_{0}^{\rnmeasure}\left[\phi_1\right] 
            }
        \end{equation}
        where
        \begin{equation}
            \phi_1 =
            \mathbbm{1}_{\{\tau_{F}=1\}}
            \frac{
                L_{1}+\mathbb{E}_{1}^{\rnmeasure}\left[L_{2}\right]/R_1 - \kappa_{1}\mathbb{E}_{1}^{\rnmeasure}\left[A\right]
            }{
                L_{1} + \mathbb{E}_{1}^{\rnmeasure}\left[L_{2}\right]/R_1
            }
        \end{equation}
        is the creditors' loss rate if the firm defaults at time 1.
        
        Likewise, if the firm survives at time 1, the marginal credit spread for short-term debt starting at the interim date can be separated from \cref{eqn:aggregate-credit-spread-multi-period}.
        Here the state of the market filtration for time 1 is known, hence the expectation is now taken from this point in time.
        The credit spread of the second time interval is computed as following:
        \begin{align}
            S_1 &=
            \mathbb{E}_{1}^{\rnmeasure}\left[
                \frac{1}{R_1} \left(
                    1
                    + \mathbbm{1}_{\{\tau_{F} = 2\}}\left(
                        \frac{\kappa_2 (A-W)}{L_2} - 1
                    \right)
                \right)
            \right]^{-1} - R_1
            \nonumber
            \\
            &=
            \mathbb{E}_{1}^{\rnmeasure}\left[
                \frac{1}{R_1} \left(
                    1
                    + \mathbbm{1}_{\{\tau_{F} = 2\}}
                    \frac{
                        \kappa_2 \left(A - W\right) - L_2
                    }{
                        L_2
                    }
                \right)
            \right]^{-1} - R_1
            \nonumber
        \end{align}
        And finally:
        \begin{equation}
            \Leftrightarrow \qquad
            S_1 =
            \frac{
                \mathbb{E}_{1}^{\rnmeasure}\left[\phi_2\right]R_1
            }{
                1 - \mathbb{E}_{1}^{\rnmeasure}\left[\phi_2\right] 
            }
        \end{equation}
        where
        \begin{equation}
            \phi_2 =
            \mathbbm{1}_{\{\tau_{F}=2\}}
            \frac{
                L_{2}-\kappa_{2}(A-W)
            }{
                L_{2}
            }
        \end{equation}
        is the creditors' loss rate if the firm defaults at time 2.

        A swap contract is an OTC derivative.
        In this two-period setup, before considering counterparty credit risk, a swap promises a floating payment, $X_t$, in exchange for a fixed payment, $K_t$, for $t = 1,2$, where the fixed payments are not necessarily constant.
        Hence, for the payer swap, the cash flows stemming from the swap contract are defined as $\mathcal{C}_t = X_t - K_t$.
        Focussing on the payer swap, the positive cash flows will indicate an asset to the firm, whereas the negative cash flows will be a contingent liability.
        \\
        A position in this swap contract of size $q$ requires the firm to make an upfront payment of $U(q)$ where, as in \cref{sec:single-period-model}, $u = \lim_{q \downarrow 0} U(q)/q$ is assumed to exist.
\end{document}
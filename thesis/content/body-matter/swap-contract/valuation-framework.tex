% !TEX root = sub-main.tex
\documentclass[main.tex]{subfiles}

\begin{document}
    \subsection{Marginal valuation framework}
        The multi-period framework extends the assumptions made in \cref{sec:spm} to include multiple time periods.
        An additional date $2$ is introduced as the new time of maturity.
        At the interim time 1 the market economy is defined by a collection of random variables,
        and at maturity, all the uncertainty is settled. 
        The single-period gross risk-free returns at time 0 and time 1 are $R_0$ and $R_1$ respectively, where $R_1$ is random at time 0.
        This means that the risk-free rate is not required to be constant,
        and the market value of cash flows,
        $\pi(\mathcal{C}_{t})$, is defined as
        $\delta_t \sum_{t=1}^{2}\mathbb{E}^{\rnmeasure}\left[\mathcal{C}_{t}\right]$,
        where $\delta_1 = \discountfactorone$,
        and $\delta_2 = \discountfactortwo$.
        The notation $\mathbb{E}^{\rnmeasure}$ is always assumed to describe the time 0 risk-neutral expectation, unless otherwise clarified.

        This setup considers a dealer whose assets at time 2 have a payoff given by some random variable $A$.
        Additionally, the firm has short-term liabilities, $L_1$, that expires at time 1 as well as long-term liabilities, $L_2$, that expires at time 2.
        \\
        As always, the firm defaults if it does not meet its financial obligations at a certain time period.
        At time 1 the firm liquidates a portion of its legacy assets to cover the short-term liabilities.
        The time 2 value of this portion is denoted as $W$.
        Also at time 1 the firm pays out a dividend, $\theta_1 \geq 0$, to its shareholders,
        %Not sure about \theta being the right notation for the dividend
        which is considered a random variable.
        The time of the event of default is denoted as $\tau_{\mathcal{D}}$,
        where $\tau_{\mathcal{D}} = \infty$ if the firm survives at time 2.
        In case of no default, the firm is liquidated, and the remaining cash is distributed to the shareholders after the creditors are paid back.
        \\
        Finally, it is assumed that all liabilities rank pari passu,
        and the recovery rates at time 1 and time 2 are denoted $\kappa_1$ and $\kappa_2$ respectively.

        The total value of the firm's equity and liabilities are then computed as following:
        \begin{align}
            \pi(S)&=
            \delta_1 
            \mathbb{E}^{\rnmeasure}\left[
                \mathbbm{1}_{\{\tau_{\mathcal{D}}>1\}} \theta_1\right]
                +
                \delta_2 \mathbb{E}^{\rnmeasure}\left[\mathbbm{1}_{\{\tau_{\mathcal{D}}>2\}} (A-W-L_{2})
            \right]
            \label{eqn:shareholder-value-multi-period}
            \\
            \nonumber
            \pi(D)&=
            \delta_1
            \mathbb{E}^{\rnmeasure}\left[
                \mathbbm{1}_{\{\tau_{\mathcal{D}}>1\}} L_1 + \mathbbm{1}_{\{\tau_{\mathcal{D}}=1\}} 
                \frac{
                    \kappa_1 \mathbb{E}_{t=1}^{\rnmeasure}\left[A\right]
                }{
                    R_1
                }
            \right]
            \\
            &\quad+
            \delta_2
            \mathbb{E}^{\rnmeasure}\left[
                \mathbbm{1}_{\{\tau_{\mathcal{D}}>2\}}L_{2}
                +
                \mathbbm{1}_{\{\tau_{\mathcal{D}}=2\}}\kappa_{2}(A-W)
            \right]
            \label{eqn:creditor-value-multi-period}
        \end{align}
        The first term in \cref{eqn:shareholder-value-multi-period} defines the discounted expected value of the dividend payout at time 1 in case of no default before time 2.
        If the dividend is non-zero, it benefits the shareholders,
        in the sense that it guarantees the shareholders to receive some payoff before a possible default event at time 2.
        The second term defines the discounted expected residual of the liquidated assets at time 2.
        If the firm does not have enough assets to cover the long-term liabilities, the creditors receive the remaining assets, and the firm defaults.
        This leaves a payoff of 0 for the shareholders.
        \\
        In \cref{eqn:creditor-value-multi-period} the first and the second term basically defines the same thing but in the two different time periods.
        The first term in each expectation is the expiring liability payoff at the respective point in time given no default. The second term in each expectation is the total asset value in case of default. Note that at time 2 for both the shareholders and the creditors, the liquidated value at time 1, $W$, that was used to cover the short-term liabilities is subtracted from the asset value. This is obviously the case as $\tau_{\mathcal{D}}=2$ means the company survived at time 1.

        As in \cref{sec:spm}, the firm's marginal credit spread for short-term debt is:
        \begin{equation}
            S_0 =
            \frac{
                \mathbb{E}^{\rnmeasure}\left[\phi_1\right]R_0
            }{
                1 - \mathbb{E}^{\rnmeasure}\left[\phi_1\right] 
            }
        \end{equation}
        where
        $\phi_1 = \mathbbm{1}_{\{\tau_{\mathcal{D}}=1\}} (L_{1}+\mathbb{E}_{t=1}^{\rnmeasure}\left[L_{2}\right]/R_1 - \kappa_{1}\mathbb{E}_{t=1}^{\rnmeasure}\left[A\right]) / (L_{1} + \mathbb{E}_{t=1}^{\rnmeasure}\left[L_{2}\right]/R_1)$ is the creditors' loss rate if the firm defaults at time 1. If the firm survives at time 1, the marginal credit spread for short-term debt starting at the interim date is:
        \begin{equation}
            S_1 =
            \frac{
                \mathbb{E}_{t=1}^{\rnmeasure}\left[\phi_2\right]R_1
            }{
                1 - \mathbb{E}_{t=1}^{\rnmeasure}\left[\phi_2\right] 
            }
        \end{equation}
        where $\phi_2 = \mathbbm{1}_{\{\tau_{\mathcal{D}}=2\}} (L_{2}-\kappa_{2}(A-W))/L_{2}$ is the creditors' loss rate if the firm defaults at time 2.

        A swap contract is an OTC derivative.
        In this two-period setup, before considering counterparty credit risk, a swap promises a floating payment, $X_t$, in exchange for a fixed payment, $K_t$, for $t = 1,2$.
        Hence, the cash flows stemming from the swap contract are defined as $\mathcal{C}_t = X_t - K_t$.
        Focussing on the payer swap, the positive cash flows will indicate an asset to the firm, whereas the negative cash flows will be a contingent liability.
        \\
        A position in this swap contract of size $q$ requires the firm to make an upfront payment of $U(q)$ where, as in \cref{sec:spm}, $u = \lim_{q \downarrow 0} U(q)/q$ is assumed to exist.
\end{document}
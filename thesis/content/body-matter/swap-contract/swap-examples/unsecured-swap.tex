% !TEX root = ../sub-main.tex
\documentclass[main.tex]{subfiles}

\begin{document}
    \subsection{Obtaining a Swap Contract}
        Suppose the firm enters a swap contract with a credit risk-free counterparty.
        The firm pays a constant fixed leg $K=\num{1}$ to the counterparty at each time period.
        Correspondingly, at each time period,
        the firm receives a random floating leg, $X_t$, from the counterparty.

        Since the counterparty is credit-risk free,
        the firm can always expect to receive the full contractual amount of the floating leg.
        Therefore, there is no \CVA/ and
        the difference between the risk-free value and the credit risk adjusted price is the \DVA/.

        The floating legs are considered random.
        Their probability distribution will not be specified,
        but the expected cash flows are known
        by both the firm and the counterparty.
        If the firm's swap leg has a higher value than the counterparty's,
        it will have to pay an upfront price to the counterparty,
        and vice versa.

        The time 0 expected payments of the swap are summarized in \cref{tbl:swap-legs}.
        Notice that at the interim date the firm expects a positive cash flow of
        $\mathbb{E}_{0}^{\rnmeasure}\left[\mathcal{C}_1\right] = \mathbb{E}_{0}^{\rnmeasure}\left[X_1 - K\right] = \num{0.2}$,
        which will be used for retiring existing debt.
        The expected cash flow at time 2 does not change over time.

        \begin{table}[H]
            \centering
            \begin{tabular}{l|rr||r}
                $t$ & 1 & 2 & Present value \\
                \hline
                \rule{0pt}{1.1em}
                $\mathbb{E}_{0}^{\rnmeasure}\left[K\right]$ & $\num{1}$ & $\num{1}$ & $\num{1.937802706}$ \\
                \rule{0pt}{1.1em}
                $\mathbb{E}_{0}^{\rnmeasure}\left[X_t\right]$ & $\num{1.2}$ & $\num{0.95}$ & $\num{2.08601}$ \\
                \rule{0pt}{1.1em}
                $\mathbb{E}_{0}^{\rnmeasure}\left[\mathcal{C}_t\right]$ & $\num{0.2}$ & $\num{-0.05}$ & $\num{0.1482079039}$ \\
            \end{tabular}
            \caption{}
            \label{tbl:swap-legs}
        \end{table}

        At the maturity date the firm liquidates all of its assets,
        which it uses to pay the remaining outstanding amounts.
        Additionally, the existing creditors' liabilities are assumed to rank pari passu with the counterparty's contingent liabilities.
        \\
        Since the expected discounted market value of the floating payments is largest,
        the firm may be obliged to pay the counterparty an upfront price.
        As per the Modigliani-Miller proposition, 
        a zero net present value investment leaves the aggregate wealth unchanged.
        For this to be applicable, the price of the swap must also account for the firm's own default risk,
        i.e. include the aforementioned \DVA/.
        Otherwise, it cannot be a zero net present value investment for both the firm and the counterparty.

        If the firm defaults at time 2, i.e. if the state $\omega_{\mathscr{d},\mathscr{d}}$ is realized,
        the present value of the total payoff to its creditors is calculated as:
        \begin{align*}
            \mathbb{E}_{0}^{\rnmeasure}\left[
                (1-\phi(\omega_{\mathscr{d},\mathscr{d}}))\left(
                    L_{2} - 
                    \mathcal{C}_{1}R_{1}
                    - \mathcal{C}_2
                \right)
            \right]
            &=
            (1-\phi(\omega_{\mathscr{d},\mathscr{d}}))\left(\num{70} - \num{0.2}R_{1} + \num{0.05}\right)\\
            &=
            (1-\phi(\omega_{\mathscr{d},\mathscr{d}}))*\num{69.8452}
        \end{align*}
        And by definition, this amount is equal to the total remaining asset value.
        Remember that at time 2 the firm has already paid the liabilities due at the interim date.
        To find the price of the project
        as well as the loss rate given default after obtaining the project,
        the following system of equations must be solved for $u$ and $\phi(\omega_{\mathscr{d},\mathscr{d}})$:
        \begin{align}
            &\left\{\begin{array}{@{}l@{}}
                \mathscr{d}^{2}A_{0} - W
                =
                \mathbb{E}_{0}^{\rnmeasure}\left[
                    (1-\phi(\omega_{\mathscr{d},\mathscr{d}}))\left(L_{2} - 
                    \mathcal{C}_{1}R_{1}
                    - \mathcal{C}_2\right)
                \right]
                \\
                u =
                \mathbb{E}_{0}^{\rnmeasure}\left[
                    \delta_{1}(X_1-K)+(\delta_{2}(X_2-K(1-\phi)))
                \right]
            \end{array}\right.
            \label{eqn:swap-example-equation-system}
            \\[1.1em]
            \Leftrightarrow
            \qquad
            &\left\{\begin{array}{@{}l@{}}
                \displaystyle
                \num{72.25} - \num{10.24}
                =
                (1-\phi(\omega_{\mathscr{d},\mathscr{d}}))*\num{69.8452}
                \\
                \displaystyle
                u =
                \delta_1 * \num{0.2}
                +
                \delta_2
                \left(
                    \left(\num{-0.05}(1-\phi(\omega_{\mathscr{d},\mathscr{d}}))\right)p_{\mathscr{d},\mathscr{d}}-\num{0.05}(1-\mathscr{d})
                \right)
            \end{array}\right.
            \nonumber
        \end{align}
        The first equation in \cref{eqn:swap-example-equation-system} ensures that the creditors receive the entire asset base if the firm defaults.
        The second equation sets the price of the swap as the discounted expected payoff.
        Solving this system of equations yields a swap price of $u=\num{0.158655}$,
        and a loss rate in the default state of 
        $\phi(\omega_{\mathscr{d},\mathscr{d}})=\pct{0.112180}$.
        Using the loss rate, the value of the legacy creditors' long term claim can be calculated.
        At time 2 the payoff in the default state is 
        $L_{2}*(1-\phi(\omega_{\mathscr{d},\mathscr{d}})) = \num{62.14739601}$.
        At time 1 the expected value of the claim is therefore \num{67.61315718},
        and at time 0 the value is \num{66.287413}.

        The credit spread in each period is the debt's no-default value one period ahead
        divided by its current value less the risk-free rate.
        Hence the post-project actual credit spreads for time 0 and time 1 are respectively:
        \begin{align*}
            \frac{
                \num{67.61791744}
            }{
                \num{66.29207592}
            }
            -
            R_{0}
            =
            \pct{0}
            \\
            \frac{
                \num{70}
            }{
                \num{67.61791744}
            } 
            -
            R_{1}
            =
            \pct{0.0112246}
        \end{align*}
        Since the firm cannot default at time 1,
        the upfront price can be funded with short-term debt at the risk-free interest rate.
        If the firm were to pay a spread on the short-term debt starting at time 0,
        this would be subtracted from the liquidated asset value. 

        By using the equation for the shareholder's marginal market valuation of the swap contract
        defined in \cref{eqn:swap-value-multi-period},
        the \DVA/ can then be computed as:
        \begin{align*}
            \DVA/ &= u - V_{rf}
            \nonumber
            \\
            &= u - \mathbb{E}_{0}^{\rnmeasure}\left[\sum_{t=1}^{2}\delta_t \mathcal{C}_t\right]
            = \num{0.0104476}
        \end{align*}

        Turning to the shareholder valuation of the project.
        Cash flows from the swap contract may have an impact on the shareholders' valuation of the project.
        Since the price of the swap is its risk-neutral value adjusted for credit risk,
        the project is a zero net present value investment for the counterparty.
        If a wealth transfer has happened, it is bound to be between the shareholders and the long-term creditors.

        With $\mathbb{E}_{0}^{\rnmeasure}\left[\mathcal{C}_1\right]=\num{0.2}$,
        $\mathbb{E}_{0}^{\rnmeasure}\left[\mathcal{C}_2\right]=\num{-0.05}$,
        $u=\num{0.158655}$,
        and the assumption that the counterparty is credit risk-free,
        the total impact on the shareholders' benefit is calculated using \cref{eqn:shareholders-value-multi-period-swap}:
        \begin{equation*}
            G_{\text{swap}} = \num{-0.00906097}
        \end{equation*}
        This value suggests a wealth transfer of $\num{-0.00906097}$
        between the shareholders and the creditors of the long-term debt,
        such that the project is a negative net present value investment for the shareholders.

        The price that would make this project a zero net present value investment for the shareholders,
        is found by fixing $G_{\text{swap}}$ to 0,
        and solving for $u^{\ast}_{\text{swap}}$.
        The difference between the credit risk adjusted price and $u^{\ast}_{\text{swap}}$ is then the \FVA/:
        \begin{equation*}
            G_{\text{swap}} = 0
        \end{equation*}
        yielding a funding adjusted price of $u^{\ast}_{\text{swap}} = \num{0.148727}$.
        The donation from the counterparty required is then:
        \begin{equation*}
            \FVA/ = u^{\ast}_{\text{swap}} - u = \num{-0.00992852}
        \end{equation*}

        This example has shown a situation where some of the derivative's cash flows have
        been a source of funding benefits,
        but where the overall effect were still a loss to the shareholders.
        The primary effect is clearly from the upfront price,
        that is the largest outgoing cash flow of the swap.
        Still, there are smaller effects which are interesting to study.

        The payment time of a cash flow has an influence on its funding implications.
        As the upfront price should be paid at time 0, 
        the firm can obtain funding in the first period at a credit spread of \pct{0}.
        However, the intermediate incoming cash flow can retire debt
        that would otherwise have appreciated 
        by the firm's credit spread at time 1 of \pct{0.011298}.
        Therefore, the time 1 cash flow provides a funding benefit,
        but the time 0 cash flow has no funding cost.
        Of course, the setup considered here is incredible simplistic,
        which exaggerates the actual effects.
        In reality, creditors would clearly not offer a credit spread of zero on, say,
        1 month tenors because they feel certain that the firm can only default in 2 months.
        Rather, the firm has some default intensity, which fluctuates over time,
        but not by jumping up from zero when a debt's maturity gets closer.

        The general lesson is useful anyway.
        If institutions have high credit spreads in certain periods,
        it will be more costly for them to obtain projects 
        that have outgoing cash flows in these periods.
        This is the case at play in the example.
        The firm has a large face value maturing at time 2, 
        so its default probability, and therefore credit spread,
        is high around this time period.
        Hence funding requirements at time 1 have a high impact.

        As has been covered previously, trading unsecured derivatives without hedging market risk
        has the most extreme funding implications;
        any outgoing cash flow has to be funded, 
        as there are no offsetting cash flows from hedges.

        If the firm was to extend its portfolio by hedging the swap
        with a fully collateralized position in the opposite trade,
        the funding implications would mainly be due to the margin requirements.
        An example of this situation was depicted in \cref{fig:funding-costs-asymmetrical-csa},
        where differences in the collateral call frequency of the swap and its hedge
        led to funding shortfalls and surpluses.
        \\
        Including a collateralized hedge in the current example, would convey some of the same messages.
        The interest payments from the swap would more or less 
        offset the interest payments from the hedge, 
        which would reduce the amount of funding costs or benefits from the market risk cash flows.
        The extent of reduction would depend on the accuracy of the hedge;
        a perfect hedge would have the exact opposite interest payments as the original,
        but mismatches would provide some funding friction.
        \\
        Some funding costs would occur as the upfront price paid for the swap 
        would be higher than the upfront price received on the hedge.
        The hedge would be secured and probably be priced close to its risk-free value, 
        while the swap counterparty would require compensation for taking on the credit risk.
        Hence, the firm would likely have to provide funding 
        for the differences in upfront payments.
        Of course, practically speaking, the firm would actually have to provide funding
        for the entire upfront payment on the original swap;
        the upfront payment from the hedge would have to be immediately posted as collateral.
        \\
        Whether to attribute this funding cost to the upfront payment or to the collateralization
        is up to interpretation, but is only a matter of definition.

        The asymmetry in the collateral agreements would also be an important contributor,
        to funding costs or benefits.
        Since the original swap is assumed completely unsecured, 
        any collateral calls from the hedge counterparty would require funding.
        Likewise, any collateral postings would yield funding benefits.
        Again, trading unsecured derivatives provides the most extreme examples of funding costs.

        Extending the portfolio in this example with a hedge,
        would confirm many of the conclusions already made throughout this paper.
        The market risk would be reduced and so would the funding implications 
        from the derivative's cash flows,
        but an \FVA/ could still be necessary due to the differences in collateral agreements.

\end{document}
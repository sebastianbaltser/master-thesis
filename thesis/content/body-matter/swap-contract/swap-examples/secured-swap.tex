% !TEX root = ../sub-main.tex
\documentclass[main.tex]{subfiles}

\begin{document}
    \subsection{Example of secured hedging swap}
        Suppose that the dealer enters the same swap contract with the same counterparty as described above.
        Additionally, she hedges the swap contract by entering the opposite position with a hedging dealer
        where she pays the floating leg.
        The dealer and the hedging dealer have in place a CSA agreement 
        requiring full collateralization, i.e. with no thresholds or minimum transfer amounts.
        Besides the variation margin,
        the CSA requires an initial margin of 50\% of the swap's price.
        The initial margin is posted both at the inception date and the interim date.
        Corresponding to the dealer, the hedging dealer is assumed credit risk-free at the interim date.

        Assume that the swap contract's price equals its credit risk adjusted value.
        With the hedge transaction included,
        the dealer's portfolio is free of market risk,
        and all outgoing cash flows from the swap's interest payments
        are perfectly offset by an incoming cash flow.
        The dealer must however fund the margin requirements by obtaining new short-term debt,
        and similarly, she uses the received margin postings to retire existing debt.
        \\
        The specific required margin postings are summarized in \cref{tbl:swap-margin-postings}.

        \begin{table}[H]
            \centering
            \begin{tabular}{l|rr}
                $t$ & 0 & 1 \\
                \hline
                \rule{0pt}{1.3em}
                $\mathbb{E}_{0}^{\rnmeasure}\left[M_t\right]$ & $\tilde{u}$ & $\frac{1}{1+r_{1,2}}\mathbb{E}_{1}^{\rnmeasure}\left[\mathcal{C}_{2}\right]$ \\
                \rule{0pt}{1.3em}
                $\mathbb{E}_{0}^{\rnmeasure}\left[I_t\right]$ & $\frac{1}{2}\tilde{u}$ & $\frac{1}{2}\frac{1}{1+r_{1,2}}\mathbb{E}_{1}^{\rnmeasure}\left[\mathcal{C}_{2}\right]$ \\
            \end{tabular}
            \caption{
                Expected collateral payments. 
                Positive amounts will be collateral calls from the hedge dealer,
                negative amounts from the dealer herself.
            }
            \label{tbl:swap-margin-postings}
        \end{table}

        As in the previous section, the dealer cannot default at time 1,
        hence no credit spread is paid for the short-term loans at time 0.
        Posting collateral increases the asset value,
        and financing it increases the short-term liabilities.
        When the dealer receives back the forward discounted collateral,
        she then uses it to pay the debt with the exact amount at the interim date.
        \\
        The dealer will not have any contingent liabilities due to the swap payments at the maturity date.
        The collateral received (posted) at the interim date will be paid (received) back at the maturity date.
        As a result of that,
        the liabilities at time 2 has both increased and decreased with the margin requirements at time 1.
        If defaulting, the dealer's creditors will have a total payoff calculated as:
        \begin{align}
            \mathbb{E}_{0}^{\rnmeasure}\left[
                (1-\phi(\omega_{\mathscr{d},\mathscr{d}}))\left(
                    L_{2}
                    -(M_1+I_1)R_{1}
                    +(M_1+I_1)R_{1}
                \right)
            \right] 
            &= (1-\phi(\omega_{\mathscr{d},\mathscr{d}}))\num{70}
        \end{align}
        The short-term debt is paid at the interim date with liquidated assets.
        If there is a difference in price between the two swap contracts,
        this will either increase or reduce the amount due,
        depending on which swap is more expensive.
        The total asset base at the default event will then have to account for covering this amount. 

        To find the market neutral credit risk adjusted price of the project, $u$, 
        as well as the loss rate given default, $\phi(\omega_{\mathscr{d},\mathscr{d}})$,
        the following equation system must be solved.
        \begin{equation}
            \left\{\begin{array}{@{}l@{}}
                \mathscr{d}^{2} A - W - (\tilde{u}-u)R_{0}R_{1}
                =
                (1-\phi(\omega_{\mathscr{d},\mathscr{d}}))\num{70}
                \\
                u =
                \mathbb{E}_{0}^{\rnmeasure}\left[
                    \delta_{1}(X_1-K_1)+(\delta_{2}(X_2-K_2(1-\phi)))
                \right]
            \end{array}\right.
        \end{equation}
        As in the example with the unsecured swap the first equation ensures
        that the entire asset value amount is distributed to the creditors in a default.
        The price of the secured swap, $\tilde{u}$, is perceived as the credit-risk free expected discounted cash-flows.
        The second equation again sets the price as the discounted expected cash flows.
        \\
        By solving, a new loss rate of $\phi = \pct{0.11423357857841265}$ is obtained.
        The new swap price is computed as $u=\num{0.15425039456453227}$,
        which is slightly smaller than in the previous example,
        since the counterparty now trades with a marginally more secured dealer.

        With the long-term debt creditors' expected payoff at time 1 calculated as $\num{67.60088401}$,
        the credit spread at time 1 is given by:
        \begin{equation}
            S_1 =
            \frac{
                \num{70}
            }{
                \num{67.59950202}
            }
            - R_{1}
            \nonumber
            = \pct{0.01150666521}
        \end{equation}
        The credit spread at time 0 is 0,
        as the dealer cannot default at the interim date.

        Finally, the shareholders' valuation of the project is analysed.
        As the project is priced to be a zero net present value investment for the counterparty,
        a potential change of wealth must again be between the dealer's creditors and shareholders.
        The shareholders' change of wealth is computed the derived expression in \cref{eqn:shareholder-valuation-swap-hedged}.
        This derivation includes the extension of the fully collateralized hedged position.
        \\
        With the above computed credit spreads, and $u=\num{0.15425039456453227}$
        the shareholders' marginal valuation of entering the swap contract is calculated as:
        \begin{equation}
            G_{\text{swap,hedged}} = \num{-0.00994871}
        \end{equation}
        suggesting a negative net present value investment for the shareholders. 
        The price that makes the shareholders indifferent of entering the project, $u^{\ast}$, is then defined by fixing $G_{\text{swap,hedged}}$ to 0.
        This implies:
        \begin{equation}
            u^{\ast} = \num{0.1486201204475151}
        \end{equation}
        and an \FVA/ defined as:
        \begin{equation}
            \FVA/ = u^{\ast} - u = \num{-0.00563027}
        \end{equation}
        Hence the donation required of the counterparty is \num{-0.00563027},
        meaning the dealer would pay less than the original swap price.
        Seemingly, this would result in a negative net present value investment for the counterparty,
        and whether they have incentives to enter the funding value adjusted project remains indeterminate.

\end{document}
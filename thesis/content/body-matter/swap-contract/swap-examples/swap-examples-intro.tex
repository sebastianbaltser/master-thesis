% !TEX root = ../sub-main.tex
\documentclass[main.tex]{subfiles}

\begin{document}
    \section{Quantifying Funding Costs}

        This section aims to illustrate
        how the funding costs and benefits of a multi-period swap contract
        can be described by a simple economical setup.
        The example will be based on the result derived in the previous sections,
        namely the shareholders' marginal valuation of an unsecured swap.
        As opposed to the single-period model analysed in \cref{sec:single-period-model},
        different fundamental values in this \namecref{sec:swap-examples} will have a probability of changing over time.
        The values in reference are specifically
        the risk-free rate,
        the swap market value,
        and of course the swap floating rate.

    \subsection{A Dealer in a Multi-Period Economy}
    \label{sec:swap-examples}
        
        Maintaining the purpose of analysing funding implications,
        the possible states are now represented by a binomial tree.
        At the interim date, the two possible states are denoted by $\omega_\mathscr{u}$ and $\omega_\mathscr{d}$.
        Correspondingly, at the maturity date the two possible states are either
        $\omega_{\mathscr{u},\mathscr{u}}$ and $\omega_{\mathscr{u},\mathscr{d}}$,
        or $\omega_{\mathscr{d},\mathscr{u}}$ and $\omega_{\mathscr{d},\mathscr{d}}$,
        depending on the asset value going up or down in the first time period respectively.
        \\
        In the state $\omega_{\mathscr{d},\mathscr{d}}$, the liabilities exceed the asset value,
        and the dealer defaults.
        In this case the shareholders receive a payoff of 0.
        \\
        The asset value structure is illustrated in 
        \cref{fig:example-asset-value-multi-period-dealer}.
        In each period the asset values either increase with a factor $\mathscr{u} = \num{1.1}$
        or decrease with a factor $\mathscr{d} = \num{0.85}$.
        The asset values are determined under the risk-neutral measure,
        such that at each point in time the value is equal to the expected discounted value of the next period.
        By that definition, and by a short-term risk-free rate assumed to be $r_{0,1}=\pct{0.02}$ at the inception date,
        the risk-neutral probability of being in the up-state at the interim date is calculated as:
        \begin{equation*}
            p_{\mathscr{u}}^{0} =
            \frac{
                \num{1} + \pct{0.02} - \num{0.85}
            }{
                \num{1.1} - \num{0.85}
            }
            = \num{0.68}
        \end{equation*}
        And consequently, $p_{\mathscr{d}}^{0} = 1 - p_{\mathscr{u}}^{0} = \num{0.32}$.
        Moving to the up-state from time 1 has a probability of
        $p^1_\mathscr{u}=(R_1-\num{0.85})/(\num{1.1}-\num{0.85}) = \num{0.696016}$.

        The long-term risk-free discount rate from time 0 to time 2 is assumed to be $r_{0,2}=\pct{0.022}$.
        To determine the expected short-term rate at the interim date, the forward rate is calculated:
        \begin{equation*}
            r_{1,2} =
                \frac{
                    \left(\num{1} + r_{0,2}\right)^{2}
                }{
                    \num{1} + r_{1,2}
                }
            - \num{1}
            = \pct{0.024}
        \end{equation*}
        
        The risk-neutral probabilities of moving to the three different states at the maturity date are calculated as:
        \begin{align*}
            p_{\mathscr{u},\mathscr{u}}
            &=
            % p_{\mathscr{u}}
            % \frac{
            %     1 + 2.4\% - 0.85
            % }{
            %     1.1 - 0.85
            % }
            % =
            \num{0.473291}
            \\
            p_{\mathscr{u},\mathscr{d}} +
            p_{\mathscr{d},\mathscr{u}}
            &=
            % p_{\mathscr{d}}
            % \frac{
            %     1 + 2.4\% - 0.85
            % }{
            %     1.1 - 0.85
            % }
            % +
            % p_{\mathscr{u}}
            % \frac{
            %     1.1 - (1 + 2.4\%)
            % }{
            %     1.1 - 0.85
            % }
            % =
            \num{0.429434}
            \\
            p_{\mathscr{d},\mathscr{d}}
            &=
            % p_{\mathscr{d}}
            % \frac{
            %     1.1 - (1 + 2.4\%)
            % }{
            %     1.1 - 0.85
            % }
            % =
            \num{0.097275}
        \end{align*}

        At the interim date, the dealer's creditors have a known claim of $L_{1} = \num{10}$,
        meaning $W=\num{10}*R_1=\num{10.240039}$.
        Meanwhile, the long-term liabilities, which are also known with certainty, are $L_{2} = \num{70}$.
        For simplicity, the dividend paid to the shareholders at the interim date is assumed to be $\theta_1 = 0$, and there are no distress costs.
        As mentioned above, the asset value does not cover the total liabilities at state $\omega_{\mathscr{d},\mathscr{d}}$ as $\num{72.25} < L_{2} + W$,
        hence the dealer defaults.
        \\
        The short-term liabilities are, on the other hand, not large enough to trigger a default at the interim date
        regardless of the state outcome.

        Before entering any projects, the dealers loss rate given default is
        $\phi(\omega_{\mathscr{d},\mathscr{d}})=(L_{2}-(\mathscr{d}^{2}A_{0}-W))/L_{2}=\pct{0.114143}$.
        The credit spread at the inception date is \pct{0}, as the dealer cannot default at time 1,
        and the credit spread at the interim date is
        $(\phi(\omega_{\mathscr{d},\mathscr{d}})*p_{\mathscr{d},\mathscr{d}}R_{1})/(1-\phi(\omega_{\mathscr{d},\mathscr{d}}) p_{\mathscr{d},\mathscr{d}}) = \pct{0.0114975}$.

        \begin{figure}[H]
            \centering
            \begin{tikzpicture}[>=stealth,sloped]
                \matrix (tree) [
                  matrix of nodes,
                  minimum size=1cm,
                  column sep=3.5cm,
                  row sep=0.85cm,
                ]
                {
                &               & $\mathscr{u}^{2}A_{0}=121$ \\
                & $\mathscr{u}A_{0} = 110$ & \\
    $A_0 = 100$ &               & $\mathscr{u}\mathscr{d}A_{0}=93.5$ \\
                & $\mathscr{d}A_{0} = 85$  & \\
                &               & $\mathscr{d}^{2}A_{0}=72.25$ \\
                };
                \draw[->] (tree-3-1) -- (tree-2-2) node [midway,above] {\tiny
                $p_{\mathscr{u}}=\num{0.68}$};
                \draw[->] (tree-3-1) -- (tree-4-2) node [midway,below] {\tiny
                $p_{\mathscr{d}}=\num{0.32}$};
                \draw[->] (tree-2-2) -- (tree-1-3) node [midway,above] {\tiny
                $p_{\mathscr{u},\mathscr{u}}=\num{0.473291}$};
                \draw[->] (tree-2-2) -- (tree-3-3) node [midway,below] {\tiny
                $p_{\mathscr{u},\mathscr{d}}=\num{0.206709}$};
                \draw[->] (tree-4-2) -- (tree-3-3) node [midway,above] {\tiny
                $p_{\mathscr{d},\mathscr{u}}=\num{0.222725}$};
                \draw[->] (tree-4-2) -- (tree-5-3) node [midway,below] {\tiny
                $p_{\mathscr{d},\mathscr{d}}=\num{0.097275}$};
            \end{tikzpicture}
            \caption{The asset value of the dealer in each time period.}
            \label{fig:example-asset-value-multi-period-dealer}
        \end{figure}

        Having defined the setup of the dealer's capital structure,
        a swap contract is now introduced between the dealer and a counterparty.
        For simplicity, the counterparty is assumed to have no credit-risk.
        The swap will have a relatively small notional payoff to the extent
        that it does not affect the default- and no-default states of the dealer.

        The numerical example will apply the theory from \cref{sec:unsecured-swap-valuation}
        to find the shareholders' valuation of the project being entered.
        The purpose of the example is to examine how financing needs
        in different time periods influence the shareholders' valuation,
        and consequently how they affect the \FVA/.

\end{document}
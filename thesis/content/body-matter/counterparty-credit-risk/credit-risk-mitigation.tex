% !TEX root = ./sub-main.tex
\documentclass[main.tex]{subfiles}

\begin{document}
    \section{Credit risk mitigation}
    
    Credit risk mitigation is actions taken by firms 
    with the purpose of reducing their own credit risk,
    by protecting against the negative impact of a counterparty defaulting.
    Reducing counterparty credit risk has important implications.
    From a risk management point of view,
    the amount of counterparty risk a firm is willing to take on 
    is given by the default risk of the counterparty as well as the firm's risk appetite.
    When this limit is set, the firm will be able to obtain some amount of trades,
    before the risk limit is met.
    If the firm is to cleverly apply credit risk mitigations,
    it might be able to obtain an even larger amount of trades,
    increasing its capacity to do business.
    If the firm applies credit risk mitigations but does not increase its holdings, 
    it will have reduced its counterparty credit risk,
    which could lead to its capital requirements being reduced.
    Having motivated the use of credit risk mitigations,
    this section will discuss three methods of mitigating credit risk;
    namely, netting sets, collateralization, and trading through a central clearing house.

    \subsection{Netting}
        A netting set is a group of trades whose value can be \textit{netted off}, 
        when a counterparty defaults.
        When netting off a collection of trades their values are aggregated,
        offsetting liabilities with payables, such that, in the liquidation of a counterparty,
        only one single amount is owed.
        Netting sets are important to credit exposure, 
        as counterparty risk is settled at each set;
        therefore, netting is central to mitigate credit risk.
        To see this consider the following example.

        \begin{example}    
        A derivatives dealer has two different netting sets, 
        referred to as netting set X and netting set Y.
        The dealer sells a derivative with netting set X and, 
        not wanting the exposure, buys the same derivative with netting set Y.
        The market risk is eliminated, so the dealer is neutral from this point of view.
        However, the credit risk exposure will include the sum of the credit risk exposures
        for both netting set.
        If instead the derivative is bought with netting set X,
        both the market risk and the credit risk will be offset.
        \end{example}

        This should show that when evaluating a new trade,
        netting sets can be taken into consideration to give a better view 
        of the credit risk implications of obtaining the new trade.
        By evaluating the incremental credit risk measures from the trade,
        potential netting benefits can be accounted for. 
        The credit risk of the stand alone trade could potentially 
        be different from the credit risk of the trade in a netting set.

        By strategically considering netting benefits the firm can reduce its credit risk. 
        Additionally, the firm can further reduce its exposure by in advance
        taking custody of some of its payables from the counterparty,
        such that they are protected from other creditors if the counterparty defaults.

    \subsection{Collateralization and Securing Positions}
        Credit risk can be reduced significantly, and even eliminated completely,
        if the counterparty provides some form of security, which the dealer has priority to.
        Should the counterparty default, the dealer can use the security to cover any
        outstanding receivables from the counterparty. 
        Likewise, the dealer can provide security to a counterparty and 
        therefore receive better prices.
        The type of security used, and the circumstances 
        under which they are available to the other entity, can vary significantly.
        This section will describe two common ways of providing security, 
        namely \textit{repo transactions} and \textit{CSA agreements}.

        \subsubsection{Repurchase Agreements}
        A repo, short for repurchase agreement, is a trade in which a dealer 
        sells bonds to a counterparty,
        with the promise of buying them back at a fixed price on a fixed date.
        The counterparty will be referred to as the lender.
        The dealer receives cash from the lender and the lender receives the bonds,
        which it uses as security for the money lent to the dealer. 
        If the dealer defaults and fails to honour her obligation to buy back the bond,
        the lender can keep it and possibly liquidate it to recover the cash lent. 
        Repos can be used as a form of short-term secured borrowing for the seller
        and an opportunity for the lender to invest funds for a customized period of time. 
        Traders can use repos to apply leverage by buying long positions in bonds 
        which they then post as the collateral in the repo. 

        Purchasing bonds through repos is beneficial to the dealer,
        as she can fund her purchase at the secured borrowing rate,
        which is going to be lower than the unsecured borrowing rate.

        \subsubsection{CSA agreements}
        Derivative transactions can be secured in a similar way 
        if the counterparties agree upon and adhere to a Credit Support Annex, CSA, agreement.
        A CSA defines terms and conditions for collateral postings between counterparties, 
        such that transaction between them are secured.
        Most CSA agreements require both entities in the transaction to post collateral 
        if they have negative exposure to their counterparty. 
        This is important for derivatives transactions that can have both 
        positive and negative mark-to-market values, 
        such that an entity can be both positively and negatively exposed to the counterparty
        during the lifetime of the derivative.
        When collateral are posted through a CSA, it provides a couple of benefits for the receiver.
        Most obvious, the receiver can use the collateral to cover losses incurred 
        due to the default of the other party.
        However, another important feature of CSA agreements is the ability of firms to
        \textit{rehypothecate} collateral.
        To rehypothecate collateral refers to passing it on to other parties that
        require collateral from the firm.
        By doing this, the firm can reduce, or even eliminate, the need for providing its own funds
        for supporting other CSA agreements. 
        This mechanism will prove to be very important in the context of funding value adjustments.

        Whether collateral should be posted, and how much should be posted, 
        depends on a number of parameters defined in the CSA agreement
        as well as external values due to the general market conditions.
        The parameters deemed most important to this thesis will be described in the following.
        \begin{description}
            \item[Threshold:] 
            the level of positive or negative exposure above which collateral will be posted. 
            Allowing a higher threshold will also yield a higher exposure;
            however, only to the extent of the threshold size. 
            Hence, the threshold is the amount of exposure the firm can accept being uncollateralized.
            Typically, CSA agreements between banks will have zero thresholds, 
            such that derivative transactions are fully collateralized.
            
            \item[Minimum transfer amount:] 
            how much positive or negative exposure the two firms can have before collateral is posted;
            hence, it is the smallest amount of collateral that can be exchanged between counterparties.
            It is often used to avoid the operational overhead from posting collateral for negligible amounts,
            but, like the exposure threshold, it comes at the cost of increased exposure.

            This parameter is also a threshold and the difference between this
            and the actual threshold-parameter might seem subtle.
            The threshold refers to the exposure directly due to the mark-to-market value,
            however the minimum transfer amount considers the exposure after taking into 
            account the mark-to-market, the current collateral posted, and the threshold.

            \item[Independent amount:]
            an amount of collateral posted regardless of the derivative's mark-to-market value. 
            Its purpose is to account for potential unexpected market moves
            that could lead to under-collateralization.
            A party posting an independent amount decreases the exposure for the other party
            since it works as an initial buffer for credit risk mitigation.
            This quantity can also be referred to as \textit{the initial margin}, 
            however, strictly speaking, 
            that term tends to be used when clearing through a central counterparty,
            which will be described in the following section.
            Conceptually, the two quantities are the same, and throughout this paper
            the two terms will be used interchangeably as the difference is subtle
            and irrelevant for the purposes of the analysis. 

            \item[Collateral call frequency:]
            how often a party can submit calls for collateral.
            Typically, CSA agreements between banks will have daily call frequency;
            other organizations will have a lower frequency simply due to the operational burden
            of posting collateral as often as daily. 
            A lower call frequency means larger exposure to the counterparty,
            as the mark-to-market will have longer to diverge from the collateral posted 
            at the last valuation date.
        \end{description}

        Besides the parameters mentioned, CSA agreements will have specifications of the type of collateral allowed,
        i.e. which currencies can be posted or if government bonds can be used as collateral,
        as well as a procedure to follow if one of the parties is downgraded to a specified credit rating.
        It should be emphasized that the parameters might not be identical for each party,
        if one party has a different credit quality than the other.
        If the credit qualities differ a lot, the CSA agreement might even be one-directional,
        such that only one party posts collateral.

        For the purpose of benchmarking collateralization schemes, 
        it will be helpful to define the theoretical concept of a \textit{perfect} CSA agreement.
        A perfect CSA will have zero threshold and zero minimum transfer amount.
        In addition, it will have a, so to speak, infinitely high collateral call frequency,
        meaning that collateral calls are made continuously and instantly 
        whenever there is the slightest change in the mark-to-market value. 
        If the counterparty instantly meets collateral calls and have no cure period,
        a perfect CSA agreement would eliminate the exposure to the counterparty
        and therefore the credit risk.
        Of course, this is merely a theoretical construct, as all sorts of frictions introduces delays,
        which would break the assumptions.
        Nonetheless, the idea of a perfect CSA will be helpful to have as a reference,
        when introducing funding value adjustments.

        As a final remark on this topic,
        it is important to note that this thesis will have a rather idealized perception of collateralization.
        It is assumed that each collateral payment perfectly offsets the mark-to-market value,
        which is not necessarily the case in the real world,
        where there might be disputes over the value of different instruments.
        In addition, it is assumed that calls for collateral will instantly be answered
        by either the required collateral or the counterparty's default. 
        In reality, collateral will not arrive instantly and there will most likely be a period of time between
        the point where the counterparty does not post collateral to the point where it actually defaults.
        All of these frictions of the real world increase the credit risk to the counterparty
        and must be considered in practical applications. 
        In the context of funding value adjustments, 
        they would simply be part of a large pool of contributors to funding frictions,
        and there is no value lost in assuming them non-existing;
        however, a great amount of simplicity is gained.

    \subsection{Clearing through central counterparty}
        A Central Counterparty, CCP, is an entity 
        providing clearing for standardized OTC derivative contracts
        and therefore works as the intermediary in a transaction.
        A CCP is a counterparty to both the buyer and the seller in a transaction 
        and guarantees the terms of the trade by providing compensation to one party when the other party defaults.
        It does so by collecting collateral from buyer and seller to cover potential losses.
        From a collateralization perspective, central clearing is very similar to trading through CSA agreements,
        since collateral must also be posted to the CCP in order to support movements in the mark-to-market value.
        In the context of central clearing, this collateral is known as variation margin.
        Central clearing can also support the posting of an independent amount of collateral 
        known as an initial margin.
        Unlike the independent amount of a CSA, initial margin is possibly dynamic 
        as it can be dependent on the riskiness of the parties, 
        such that increases in risk measures might trigger calls for additional initial margin.

        When using a CCP, rehypothecation of collateral is not allowed,
        since the assets are in the possession of the clearing house.
        As will be evident later, the possibility to rehypothecate collateral
        has significant impact on funding value adjustments.
        When clearing through a CCP the variation margin is very similar 
        to a strong CSA agreement with daily call frequency, 
        and initial margin is very similar to an independent amount in a CSA.
        While there, for practical applications, 
        are important differences between central clearing and CSA agreements,
        the results of this paper are mostly dependent on the way collateral is posted.
        Since both can be modified in mostly the same ways,
        there will not be much difference in using one or the other;
        therefore, throughout the thesis, CSA agreements will be mentioned the most, 
        but similar results could be obtained for transactions made through CCPs.

    Many other forms of credit risk mitigation than collateralization and central clearing exists. 
    Some derivatives contain break clauses that allows a party to prematurely terminate the 
    contract with a cash settlement corresponding to the derivative's value at that time.
    It is also possible for firms to buy insurance against their counterparty's default,
    such that a notional amount is paid to the firm if the counterparty defaults.
    These types of mitigations will not be explored further or otherwise used in this paper.
\end{document}
        
% !TEX root = ./sub-main.tex
\documentclass[main.tex]{subfiles}

\begin{document}
    \subsection{Overnight Index Swap Rates}
    
    \subsubsection{The LIBOR-OIS spread}
    An index swap is a contract that exchanges a fixed cash flow, determined at inception,
    for a floating cash flow tied to some price index.
    As implied by the name, an Overnight Index Swap, OIS, is a special case of an index swap
    that uses an overnight rate index to calculate the floating leg.
    The floating payments of an OIS is the daily compounded overnight rate 
    over the floating coupon period,
    while the fixed rate is a rate referred to as the OIS rate.
    The price index that specifies the reference rate is typically the overnight 
    lending rate between banks published by the central bank, e.g. the Federal Funds Rate.
    At maturity of the OIS, the parties exchange the difference between the interest accrued 
    at the fixed rate and the interest accrued at the compounded floating index rate.
    There is no exchange of principal why an OIS generally carries very little credit risk. 
    Conceptually, the OIS rate can be thought of as representing a given country's
    central bank rate throughout a period set by the OIS' term. 

    The London Interbank Offered Rate, LIBOR, represents the average rate 
    at which major London based banks charge each other for short-term unsecured borrowing.
    Multiple other reference rates for the interbank market exists, 
    such as Euribor for Eurozone banks and TIBOR for Tokyo based banks.
    Some results in the following are not particularly dependent on the specific rate.
    When that is the case, the generic term xIBOR will be used 
    to refer to these interest rate benchmarks.

    A common measure of banking system's health is the difference between the OIS rate
    and the LIBOR, known as the LIBOR-OIS spread.
    This spread can be used as an indication of how bank's perceive the creditworthiness
    of other financial institutions. 
    The LIBOR-OIS spread should be a better indicator of credit risk in the interbank lending market
    than the LIBOR itself. 
    The LIBOR is influenced both by the rates set by central banks and 
    the general credit risk in the interbank lending market,
    while the OIS rate is only based on the rates set by central banks.
    Therefore, subtracting the OIS rate from LIBOR should isolate the credit premium.
    A higher spread could be interpreted as a low willingness to lend by major banks
    and therefore low liquidity in the money market.

    \subsubsection{OIS as the risk-free rate}
    
    Before the financial crisis in 2007, banks have typically been discounting
    derivatives using interest rate curves based on various interbank borrowing rates.
    \textcite[Section 8.6]{Green2015XVA} 
    mentions two possible reasons for banks using these indices.
    First, the xIBOR discount curves was considered and used as proxies for the risk-free rate;
    Pre-crisis the underlying banks in xIBOR generally had high credit ratings
    and were regarded as very safe counterparties.
    xIBOR rates were in fact very close to the yields of bonds issued by highly rated governments,
    and the xIBOR rates could be considered as being very close to risk-free 
    while being associated with highly liquid market.
    Second, the xIBOR curves represented the banks' own funding rate for derivatives, since
    xIBOR referenced unsecured borrowing rates.

    When the LIBOR-OIS spread, and other similar measures based on e.g. EURIBOR, widened, 
    it had significant influence on the banks' pricing and valuation of derivatives.
    The arguments for xIBOR discounting just mentioned
    were disproven as the LIBOR-OIS spreads reached very high levels.

    There are two sources of credit risk in an OIS.
    The first is the credit risk in the overnight borrowing of federal funds, which is very small. 
    The second is the risk of one of the swap counterparties defaulting.
    \textcite{HullWhiteOISvsLIBOR}
    argue that this second risk is negligible in collateralized transactions,
    which standard swaps like OIS are due to legislation.
    Since credit risk is of inconsiderable size it can be concluded 
    that the OIS rate is a good proxy for the risk-free rate.

\end{document}

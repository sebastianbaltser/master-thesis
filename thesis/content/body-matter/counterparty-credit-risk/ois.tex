% !TEX root = ./sub-main.tex
\documentclass[main.tex]{subfiles}

\begin{document}
    \subsection{Overnight Index Swap Rates}
    
    \subsubsection{The LIBOR-\OIS/ spread}
    An index swap is a contract that exchanges a fixed cash flow, determined at inception,
    for a floating cash flow tied to some price index.
    As implied by the name, an Overnight Index Swap, \OIS/, is a special case of an index swap
    that uses an overnight rate index to calculate the floating leg.
    The floating payments of an \OIS/ is the daily compounded overnight rate 
    over the floating coupon period,
    while the fixed rate is a rate referred to as the \OIS/ rate.
    The price index that specifies the reference rate is typically the overnight 
    lending rate between banks published by the central bank, e.g. the Federal Funds Rate.
    At maturity of the \OIS/, the parties exchange the difference between the interest accrued 
    at the fixed rate and the interest accrued at the compounded floating index rate.
    There is no exchange of principal; 
    therefore, an \OIS/ generally carries very little credit risk. 
    Conceptually, the \OIS/ rate can be thought of as representing a given country's
    central bank rate throughout a period set by the \OIS/' term. 

    The London Interbank Offered Rate, LIBOR, represents the average rate 
    at which major London based banks charge each other for short-term unsecured borrowing.
    Multiple other reference rates for the interbank market exists, 
    such as Euribor for Eurozone banks and TIBOR for Tokyo based banks.
    Some results in the following are not particularly dependent on the specific rate.
    When that is the case, the generic term xIBOR will be used 
    to refer to these interest rate benchmarks.

    A common measure of the banking system's health is the difference between the \OIS/ rate
    and the LIBOR, known as the LIBOR-\OIS/ spread.
    This spread can be used as an indication of how bank's perceive the creditworthiness
    of other financial institutions. 
    The LIBOR-\OIS/ spread should be a better indicator 
    of credit risk in the interbank lending market than the LIBOR itself,
    since the LIBOR is influenced both by the rates set by central banks and 
    the general credit risk in the interbank lending market,
    while the \OIS/ rate is only based on the rates set by central banks.
    Therefore, subtracting the \OIS/ rate from LIBOR should isolate the credit premium.
    A higher spread could be interpreted as a low willingness to lend by major banks
    and therefore low liquidity in the money market.

    \subsubsection{\OIS/ as the risk-free rate}
    
    Before the financial crisis in 2007, banks have typically been discounting
    derivatives using interest rate curves based on the various interbank borrowing rates.
    Typically, the interest rate curves were based on 3-Month xIBOR.
    \textcite[Section 8.6]{Green2015XVA} 
    mentions two possible reasons for banks using these indices.
    First, the xIBOR discount curves was considered and used as proxies for the risk-free rate;
    Pre-crisis the underlying banks in xIBOR generally had high credit ratings
    and were regarded as very safe counterparties.
    xIBOR rates were in fact very close to the yields of bonds issued by highly rated governments,
    and the xIBOR rates could be considered as being very close to risk-free 
    while being associated with highly liquid market.
    Second, the xIBOR curves represented the banks' own funding rate for derivatives, since
    xIBOR referenced unsecured borrowing rates.

    \begin{figure}
        \centering
        \resizebox{\textwidth}{!}{%
            %% Creator: Matplotlib, PGF backend
%%
%% To include the figure in your LaTeX document, write
%%   \input{<filename>.pgf}
%%
%% Make sure the required packages are loaded in your preamble
%%   \usepackage{pgf}
%%
%% Also ensure that all the required font packages are loaded; for instance,
%% the lmodern package is sometimes necessary when using math font.
%%   \usepackage{lmodern}
%%
%% Figures using additional raster images can only be included by \input if
%% they are in the same directory as the main LaTeX file. For loading figures
%% from other directories you can use the `import` package
%%   \usepackage{import}
%%
%% and then include the figures with
%%   \import{<path to file>}{<filename>.pgf}
%%
%% Matplotlib used the following preamble
%%
\begingroup%
\makeatletter%
\begin{pgfpicture}%
\pgfpathrectangle{\pgfpointorigin}{\pgfqpoint{7.000000in}{4.000000in}}%
\pgfusepath{use as bounding box, clip}%
\begin{pgfscope}%
\pgfsetbuttcap%
\pgfsetmiterjoin%
\definecolor{currentfill}{rgb}{1.000000,1.000000,1.000000}%
\pgfsetfillcolor{currentfill}%
\pgfsetlinewidth{0.000000pt}%
\definecolor{currentstroke}{rgb}{0.500000,0.500000,0.500000}%
\pgfsetstrokecolor{currentstroke}%
\pgfsetdash{}{0pt}%
\pgfpathmoveto{\pgfqpoint{0.000000in}{0.000000in}}%
\pgfpathlineto{\pgfqpoint{7.000000in}{0.000000in}}%
\pgfpathlineto{\pgfqpoint{7.000000in}{4.000000in}}%
\pgfpathlineto{\pgfqpoint{0.000000in}{4.000000in}}%
\pgfpathlineto{\pgfqpoint{0.000000in}{0.000000in}}%
\pgfpathclose%
\pgfusepath{fill}%
\end{pgfscope}%
\begin{pgfscope}%
\pgfsetbuttcap%
\pgfsetmiterjoin%
\definecolor{currentfill}{rgb}{0.898039,0.898039,0.898039}%
\pgfsetfillcolor{currentfill}%
\pgfsetlinewidth{0.000000pt}%
\definecolor{currentstroke}{rgb}{0.000000,0.000000,0.000000}%
\pgfsetstrokecolor{currentstroke}%
\pgfsetstrokeopacity{0.000000}%
\pgfsetdash{}{0pt}%
\pgfpathmoveto{\pgfqpoint{0.875000in}{0.440000in}}%
\pgfpathlineto{\pgfqpoint{6.300000in}{0.440000in}}%
\pgfpathlineto{\pgfqpoint{6.300000in}{3.520000in}}%
\pgfpathlineto{\pgfqpoint{0.875000in}{3.520000in}}%
\pgfpathlineto{\pgfqpoint{0.875000in}{0.440000in}}%
\pgfpathclose%
\pgfusepath{fill}%
\end{pgfscope}%
\begin{pgfscope}%
\pgfpathrectangle{\pgfqpoint{0.875000in}{0.440000in}}{\pgfqpoint{5.425000in}{3.080000in}}%
\pgfusepath{clip}%
\pgfsetrectcap%
\pgfsetroundjoin%
\pgfsetlinewidth{0.803000pt}%
\definecolor{currentstroke}{rgb}{1.000000,1.000000,1.000000}%
\pgfsetstrokecolor{currentstroke}%
\pgfsetdash{}{0pt}%
\pgfpathmoveto{\pgfqpoint{0.875000in}{0.440000in}}%
\pgfpathlineto{\pgfqpoint{0.875000in}{3.520000in}}%
\pgfusepath{stroke}%
\end{pgfscope}%
\begin{pgfscope}%
\pgfsetbuttcap%
\pgfsetroundjoin%
\definecolor{currentfill}{rgb}{0.333333,0.333333,0.333333}%
\pgfsetfillcolor{currentfill}%
\pgfsetlinewidth{0.803000pt}%
\definecolor{currentstroke}{rgb}{0.333333,0.333333,0.333333}%
\pgfsetstrokecolor{currentstroke}%
\pgfsetdash{}{0pt}%
\pgfsys@defobject{currentmarker}{\pgfqpoint{0.000000in}{-0.048611in}}{\pgfqpoint{0.000000in}{0.000000in}}{%
\pgfpathmoveto{\pgfqpoint{0.000000in}{0.000000in}}%
\pgfpathlineto{\pgfqpoint{0.000000in}{-0.048611in}}%
\pgfusepath{stroke,fill}%
}%
\begin{pgfscope}%
\pgfsys@transformshift{0.875000in}{0.440000in}%
\pgfsys@useobject{currentmarker}{}%
\end{pgfscope}%
\end{pgfscope}%
\begin{pgfscope}%
\definecolor{textcolor}{rgb}{0.333333,0.333333,0.333333}%
\pgfsetstrokecolor{textcolor}%
\pgfsetfillcolor{textcolor}%
\pgftext[x=0.875000in,y=0.342778in,,top]{\color{textcolor}\rmfamily\fontsize{10.000000}{12.000000}\selectfont \(\displaystyle {2004}\)}%
\end{pgfscope}%
\begin{pgfscope}%
\pgfpathrectangle{\pgfqpoint{0.875000in}{0.440000in}}{\pgfqpoint{5.425000in}{3.080000in}}%
\pgfusepath{clip}%
\pgfsetrectcap%
\pgfsetroundjoin%
\pgfsetlinewidth{0.803000pt}%
\definecolor{currentstroke}{rgb}{1.000000,1.000000,1.000000}%
\pgfsetstrokecolor{currentstroke}%
\pgfsetdash{}{0pt}%
\pgfpathmoveto{\pgfqpoint{1.476315in}{0.440000in}}%
\pgfpathlineto{\pgfqpoint{1.476315in}{3.520000in}}%
\pgfusepath{stroke}%
\end{pgfscope}%
\begin{pgfscope}%
\pgfsetbuttcap%
\pgfsetroundjoin%
\definecolor{currentfill}{rgb}{0.333333,0.333333,0.333333}%
\pgfsetfillcolor{currentfill}%
\pgfsetlinewidth{0.803000pt}%
\definecolor{currentstroke}{rgb}{0.333333,0.333333,0.333333}%
\pgfsetstrokecolor{currentstroke}%
\pgfsetdash{}{0pt}%
\pgfsys@defobject{currentmarker}{\pgfqpoint{0.000000in}{-0.048611in}}{\pgfqpoint{0.000000in}{0.000000in}}{%
\pgfpathmoveto{\pgfqpoint{0.000000in}{0.000000in}}%
\pgfpathlineto{\pgfqpoint{0.000000in}{-0.048611in}}%
\pgfusepath{stroke,fill}%
}%
\begin{pgfscope}%
\pgfsys@transformshift{1.476315in}{0.440000in}%
\pgfsys@useobject{currentmarker}{}%
\end{pgfscope}%
\end{pgfscope}%
\begin{pgfscope}%
\definecolor{textcolor}{rgb}{0.333333,0.333333,0.333333}%
\pgfsetstrokecolor{textcolor}%
\pgfsetfillcolor{textcolor}%
\pgftext[x=1.476315in,y=0.342778in,,top]{\color{textcolor}\rmfamily\fontsize{10.000000}{12.000000}\selectfont \(\displaystyle {2006}\)}%
\end{pgfscope}%
\begin{pgfscope}%
\pgfpathrectangle{\pgfqpoint{0.875000in}{0.440000in}}{\pgfqpoint{5.425000in}{3.080000in}}%
\pgfusepath{clip}%
\pgfsetrectcap%
\pgfsetroundjoin%
\pgfsetlinewidth{0.803000pt}%
\definecolor{currentstroke}{rgb}{1.000000,1.000000,1.000000}%
\pgfsetstrokecolor{currentstroke}%
\pgfsetdash{}{0pt}%
\pgfpathmoveto{\pgfqpoint{2.076808in}{0.440000in}}%
\pgfpathlineto{\pgfqpoint{2.076808in}{3.520000in}}%
\pgfusepath{stroke}%
\end{pgfscope}%
\begin{pgfscope}%
\pgfsetbuttcap%
\pgfsetroundjoin%
\definecolor{currentfill}{rgb}{0.333333,0.333333,0.333333}%
\pgfsetfillcolor{currentfill}%
\pgfsetlinewidth{0.803000pt}%
\definecolor{currentstroke}{rgb}{0.333333,0.333333,0.333333}%
\pgfsetstrokecolor{currentstroke}%
\pgfsetdash{}{0pt}%
\pgfsys@defobject{currentmarker}{\pgfqpoint{0.000000in}{-0.048611in}}{\pgfqpoint{0.000000in}{0.000000in}}{%
\pgfpathmoveto{\pgfqpoint{0.000000in}{0.000000in}}%
\pgfpathlineto{\pgfqpoint{0.000000in}{-0.048611in}}%
\pgfusepath{stroke,fill}%
}%
\begin{pgfscope}%
\pgfsys@transformshift{2.076808in}{0.440000in}%
\pgfsys@useobject{currentmarker}{}%
\end{pgfscope}%
\end{pgfscope}%
\begin{pgfscope}%
\definecolor{textcolor}{rgb}{0.333333,0.333333,0.333333}%
\pgfsetstrokecolor{textcolor}%
\pgfsetfillcolor{textcolor}%
\pgftext[x=2.076808in,y=0.342778in,,top]{\color{textcolor}\rmfamily\fontsize{10.000000}{12.000000}\selectfont \(\displaystyle {2008}\)}%
\end{pgfscope}%
\begin{pgfscope}%
\pgfpathrectangle{\pgfqpoint{0.875000in}{0.440000in}}{\pgfqpoint{5.425000in}{3.080000in}}%
\pgfusepath{clip}%
\pgfsetrectcap%
\pgfsetroundjoin%
\pgfsetlinewidth{0.803000pt}%
\definecolor{currentstroke}{rgb}{1.000000,1.000000,1.000000}%
\pgfsetstrokecolor{currentstroke}%
\pgfsetdash{}{0pt}%
\pgfpathmoveto{\pgfqpoint{2.678124in}{0.440000in}}%
\pgfpathlineto{\pgfqpoint{2.678124in}{3.520000in}}%
\pgfusepath{stroke}%
\end{pgfscope}%
\begin{pgfscope}%
\pgfsetbuttcap%
\pgfsetroundjoin%
\definecolor{currentfill}{rgb}{0.333333,0.333333,0.333333}%
\pgfsetfillcolor{currentfill}%
\pgfsetlinewidth{0.803000pt}%
\definecolor{currentstroke}{rgb}{0.333333,0.333333,0.333333}%
\pgfsetstrokecolor{currentstroke}%
\pgfsetdash{}{0pt}%
\pgfsys@defobject{currentmarker}{\pgfqpoint{0.000000in}{-0.048611in}}{\pgfqpoint{0.000000in}{0.000000in}}{%
\pgfpathmoveto{\pgfqpoint{0.000000in}{0.000000in}}%
\pgfpathlineto{\pgfqpoint{0.000000in}{-0.048611in}}%
\pgfusepath{stroke,fill}%
}%
\begin{pgfscope}%
\pgfsys@transformshift{2.678124in}{0.440000in}%
\pgfsys@useobject{currentmarker}{}%
\end{pgfscope}%
\end{pgfscope}%
\begin{pgfscope}%
\definecolor{textcolor}{rgb}{0.333333,0.333333,0.333333}%
\pgfsetstrokecolor{textcolor}%
\pgfsetfillcolor{textcolor}%
\pgftext[x=2.678124in,y=0.342778in,,top]{\color{textcolor}\rmfamily\fontsize{10.000000}{12.000000}\selectfont \(\displaystyle {2010}\)}%
\end{pgfscope}%
\begin{pgfscope}%
\pgfpathrectangle{\pgfqpoint{0.875000in}{0.440000in}}{\pgfqpoint{5.425000in}{3.080000in}}%
\pgfusepath{clip}%
\pgfsetrectcap%
\pgfsetroundjoin%
\pgfsetlinewidth{0.803000pt}%
\definecolor{currentstroke}{rgb}{1.000000,1.000000,1.000000}%
\pgfsetstrokecolor{currentstroke}%
\pgfsetdash{}{0pt}%
\pgfpathmoveto{\pgfqpoint{3.278616in}{0.440000in}}%
\pgfpathlineto{\pgfqpoint{3.278616in}{3.520000in}}%
\pgfusepath{stroke}%
\end{pgfscope}%
\begin{pgfscope}%
\pgfsetbuttcap%
\pgfsetroundjoin%
\definecolor{currentfill}{rgb}{0.333333,0.333333,0.333333}%
\pgfsetfillcolor{currentfill}%
\pgfsetlinewidth{0.803000pt}%
\definecolor{currentstroke}{rgb}{0.333333,0.333333,0.333333}%
\pgfsetstrokecolor{currentstroke}%
\pgfsetdash{}{0pt}%
\pgfsys@defobject{currentmarker}{\pgfqpoint{0.000000in}{-0.048611in}}{\pgfqpoint{0.000000in}{0.000000in}}{%
\pgfpathmoveto{\pgfqpoint{0.000000in}{0.000000in}}%
\pgfpathlineto{\pgfqpoint{0.000000in}{-0.048611in}}%
\pgfusepath{stroke,fill}%
}%
\begin{pgfscope}%
\pgfsys@transformshift{3.278616in}{0.440000in}%
\pgfsys@useobject{currentmarker}{}%
\end{pgfscope}%
\end{pgfscope}%
\begin{pgfscope}%
\definecolor{textcolor}{rgb}{0.333333,0.333333,0.333333}%
\pgfsetstrokecolor{textcolor}%
\pgfsetfillcolor{textcolor}%
\pgftext[x=3.278616in,y=0.342778in,,top]{\color{textcolor}\rmfamily\fontsize{10.000000}{12.000000}\selectfont \(\displaystyle {2012}\)}%
\end{pgfscope}%
\begin{pgfscope}%
\pgfpathrectangle{\pgfqpoint{0.875000in}{0.440000in}}{\pgfqpoint{5.425000in}{3.080000in}}%
\pgfusepath{clip}%
\pgfsetrectcap%
\pgfsetroundjoin%
\pgfsetlinewidth{0.803000pt}%
\definecolor{currentstroke}{rgb}{1.000000,1.000000,1.000000}%
\pgfsetstrokecolor{currentstroke}%
\pgfsetdash{}{0pt}%
\pgfpathmoveto{\pgfqpoint{3.879932in}{0.440000in}}%
\pgfpathlineto{\pgfqpoint{3.879932in}{3.520000in}}%
\pgfusepath{stroke}%
\end{pgfscope}%
\begin{pgfscope}%
\pgfsetbuttcap%
\pgfsetroundjoin%
\definecolor{currentfill}{rgb}{0.333333,0.333333,0.333333}%
\pgfsetfillcolor{currentfill}%
\pgfsetlinewidth{0.803000pt}%
\definecolor{currentstroke}{rgb}{0.333333,0.333333,0.333333}%
\pgfsetstrokecolor{currentstroke}%
\pgfsetdash{}{0pt}%
\pgfsys@defobject{currentmarker}{\pgfqpoint{0.000000in}{-0.048611in}}{\pgfqpoint{0.000000in}{0.000000in}}{%
\pgfpathmoveto{\pgfqpoint{0.000000in}{0.000000in}}%
\pgfpathlineto{\pgfqpoint{0.000000in}{-0.048611in}}%
\pgfusepath{stroke,fill}%
}%
\begin{pgfscope}%
\pgfsys@transformshift{3.879932in}{0.440000in}%
\pgfsys@useobject{currentmarker}{}%
\end{pgfscope}%
\end{pgfscope}%
\begin{pgfscope}%
\definecolor{textcolor}{rgb}{0.333333,0.333333,0.333333}%
\pgfsetstrokecolor{textcolor}%
\pgfsetfillcolor{textcolor}%
\pgftext[x=3.879932in,y=0.342778in,,top]{\color{textcolor}\rmfamily\fontsize{10.000000}{12.000000}\selectfont \(\displaystyle {2014}\)}%
\end{pgfscope}%
\begin{pgfscope}%
\pgfpathrectangle{\pgfqpoint{0.875000in}{0.440000in}}{\pgfqpoint{5.425000in}{3.080000in}}%
\pgfusepath{clip}%
\pgfsetrectcap%
\pgfsetroundjoin%
\pgfsetlinewidth{0.803000pt}%
\definecolor{currentstroke}{rgb}{1.000000,1.000000,1.000000}%
\pgfsetstrokecolor{currentstroke}%
\pgfsetdash{}{0pt}%
\pgfpathmoveto{\pgfqpoint{4.480425in}{0.440000in}}%
\pgfpathlineto{\pgfqpoint{4.480425in}{3.520000in}}%
\pgfusepath{stroke}%
\end{pgfscope}%
\begin{pgfscope}%
\pgfsetbuttcap%
\pgfsetroundjoin%
\definecolor{currentfill}{rgb}{0.333333,0.333333,0.333333}%
\pgfsetfillcolor{currentfill}%
\pgfsetlinewidth{0.803000pt}%
\definecolor{currentstroke}{rgb}{0.333333,0.333333,0.333333}%
\pgfsetstrokecolor{currentstroke}%
\pgfsetdash{}{0pt}%
\pgfsys@defobject{currentmarker}{\pgfqpoint{0.000000in}{-0.048611in}}{\pgfqpoint{0.000000in}{0.000000in}}{%
\pgfpathmoveto{\pgfqpoint{0.000000in}{0.000000in}}%
\pgfpathlineto{\pgfqpoint{0.000000in}{-0.048611in}}%
\pgfusepath{stroke,fill}%
}%
\begin{pgfscope}%
\pgfsys@transformshift{4.480425in}{0.440000in}%
\pgfsys@useobject{currentmarker}{}%
\end{pgfscope}%
\end{pgfscope}%
\begin{pgfscope}%
\definecolor{textcolor}{rgb}{0.333333,0.333333,0.333333}%
\pgfsetstrokecolor{textcolor}%
\pgfsetfillcolor{textcolor}%
\pgftext[x=4.480425in,y=0.342778in,,top]{\color{textcolor}\rmfamily\fontsize{10.000000}{12.000000}\selectfont \(\displaystyle {2016}\)}%
\end{pgfscope}%
\begin{pgfscope}%
\pgfpathrectangle{\pgfqpoint{0.875000in}{0.440000in}}{\pgfqpoint{5.425000in}{3.080000in}}%
\pgfusepath{clip}%
\pgfsetrectcap%
\pgfsetroundjoin%
\pgfsetlinewidth{0.803000pt}%
\definecolor{currentstroke}{rgb}{1.000000,1.000000,1.000000}%
\pgfsetstrokecolor{currentstroke}%
\pgfsetdash{}{0pt}%
\pgfpathmoveto{\pgfqpoint{5.081740in}{0.440000in}}%
\pgfpathlineto{\pgfqpoint{5.081740in}{3.520000in}}%
\pgfusepath{stroke}%
\end{pgfscope}%
\begin{pgfscope}%
\pgfsetbuttcap%
\pgfsetroundjoin%
\definecolor{currentfill}{rgb}{0.333333,0.333333,0.333333}%
\pgfsetfillcolor{currentfill}%
\pgfsetlinewidth{0.803000pt}%
\definecolor{currentstroke}{rgb}{0.333333,0.333333,0.333333}%
\pgfsetstrokecolor{currentstroke}%
\pgfsetdash{}{0pt}%
\pgfsys@defobject{currentmarker}{\pgfqpoint{0.000000in}{-0.048611in}}{\pgfqpoint{0.000000in}{0.000000in}}{%
\pgfpathmoveto{\pgfqpoint{0.000000in}{0.000000in}}%
\pgfpathlineto{\pgfqpoint{0.000000in}{-0.048611in}}%
\pgfusepath{stroke,fill}%
}%
\begin{pgfscope}%
\pgfsys@transformshift{5.081740in}{0.440000in}%
\pgfsys@useobject{currentmarker}{}%
\end{pgfscope}%
\end{pgfscope}%
\begin{pgfscope}%
\definecolor{textcolor}{rgb}{0.333333,0.333333,0.333333}%
\pgfsetstrokecolor{textcolor}%
\pgfsetfillcolor{textcolor}%
\pgftext[x=5.081740in,y=0.342778in,,top]{\color{textcolor}\rmfamily\fontsize{10.000000}{12.000000}\selectfont \(\displaystyle {2018}\)}%
\end{pgfscope}%
\begin{pgfscope}%
\pgfpathrectangle{\pgfqpoint{0.875000in}{0.440000in}}{\pgfqpoint{5.425000in}{3.080000in}}%
\pgfusepath{clip}%
\pgfsetrectcap%
\pgfsetroundjoin%
\pgfsetlinewidth{0.803000pt}%
\definecolor{currentstroke}{rgb}{1.000000,1.000000,1.000000}%
\pgfsetstrokecolor{currentstroke}%
\pgfsetdash{}{0pt}%
\pgfpathmoveto{\pgfqpoint{5.682233in}{0.440000in}}%
\pgfpathlineto{\pgfqpoint{5.682233in}{3.520000in}}%
\pgfusepath{stroke}%
\end{pgfscope}%
\begin{pgfscope}%
\pgfsetbuttcap%
\pgfsetroundjoin%
\definecolor{currentfill}{rgb}{0.333333,0.333333,0.333333}%
\pgfsetfillcolor{currentfill}%
\pgfsetlinewidth{0.803000pt}%
\definecolor{currentstroke}{rgb}{0.333333,0.333333,0.333333}%
\pgfsetstrokecolor{currentstroke}%
\pgfsetdash{}{0pt}%
\pgfsys@defobject{currentmarker}{\pgfqpoint{0.000000in}{-0.048611in}}{\pgfqpoint{0.000000in}{0.000000in}}{%
\pgfpathmoveto{\pgfqpoint{0.000000in}{0.000000in}}%
\pgfpathlineto{\pgfqpoint{0.000000in}{-0.048611in}}%
\pgfusepath{stroke,fill}%
}%
\begin{pgfscope}%
\pgfsys@transformshift{5.682233in}{0.440000in}%
\pgfsys@useobject{currentmarker}{}%
\end{pgfscope}%
\end{pgfscope}%
\begin{pgfscope}%
\definecolor{textcolor}{rgb}{0.333333,0.333333,0.333333}%
\pgfsetstrokecolor{textcolor}%
\pgfsetfillcolor{textcolor}%
\pgftext[x=5.682233in,y=0.342778in,,top]{\color{textcolor}\rmfamily\fontsize{10.000000}{12.000000}\selectfont \(\displaystyle {2020}\)}%
\end{pgfscope}%
\begin{pgfscope}%
\pgfpathrectangle{\pgfqpoint{0.875000in}{0.440000in}}{\pgfqpoint{5.425000in}{3.080000in}}%
\pgfusepath{clip}%
\pgfsetrectcap%
\pgfsetroundjoin%
\pgfsetlinewidth{0.803000pt}%
\definecolor{currentstroke}{rgb}{1.000000,1.000000,1.000000}%
\pgfsetstrokecolor{currentstroke}%
\pgfsetdash{}{0pt}%
\pgfpathmoveto{\pgfqpoint{6.283548in}{0.440000in}}%
\pgfpathlineto{\pgfqpoint{6.283548in}{3.520000in}}%
\pgfusepath{stroke}%
\end{pgfscope}%
\begin{pgfscope}%
\pgfsetbuttcap%
\pgfsetroundjoin%
\definecolor{currentfill}{rgb}{0.333333,0.333333,0.333333}%
\pgfsetfillcolor{currentfill}%
\pgfsetlinewidth{0.803000pt}%
\definecolor{currentstroke}{rgb}{0.333333,0.333333,0.333333}%
\pgfsetstrokecolor{currentstroke}%
\pgfsetdash{}{0pt}%
\pgfsys@defobject{currentmarker}{\pgfqpoint{0.000000in}{-0.048611in}}{\pgfqpoint{0.000000in}{0.000000in}}{%
\pgfpathmoveto{\pgfqpoint{0.000000in}{0.000000in}}%
\pgfpathlineto{\pgfqpoint{0.000000in}{-0.048611in}}%
\pgfusepath{stroke,fill}%
}%
\begin{pgfscope}%
\pgfsys@transformshift{6.283548in}{0.440000in}%
\pgfsys@useobject{currentmarker}{}%
\end{pgfscope}%
\end{pgfscope}%
\begin{pgfscope}%
\definecolor{textcolor}{rgb}{0.333333,0.333333,0.333333}%
\pgfsetstrokecolor{textcolor}%
\pgfsetfillcolor{textcolor}%
\pgftext[x=6.283548in,y=0.342778in,,top]{\color{textcolor}\rmfamily\fontsize{10.000000}{12.000000}\selectfont \(\displaystyle {2022}\)}%
\end{pgfscope}%
\begin{pgfscope}%
\definecolor{textcolor}{rgb}{0.333333,0.333333,0.333333}%
\pgfsetstrokecolor{textcolor}%
\pgfsetfillcolor{textcolor}%
\pgftext[x=3.587500in,y=0.163766in,,top]{\color{textcolor}\rmfamily\fontsize{12.000000}{14.400000}\selectfont Date}%
\end{pgfscope}%
\begin{pgfscope}%
\pgfpathrectangle{\pgfqpoint{0.875000in}{0.440000in}}{\pgfqpoint{5.425000in}{3.080000in}}%
\pgfusepath{clip}%
\pgfsetrectcap%
\pgfsetroundjoin%
\pgfsetlinewidth{0.803000pt}%
\definecolor{currentstroke}{rgb}{1.000000,1.000000,1.000000}%
\pgfsetstrokecolor{currentstroke}%
\pgfsetdash{}{0pt}%
\pgfpathmoveto{\pgfqpoint{0.875000in}{0.440000in}}%
\pgfpathlineto{\pgfqpoint{6.300000in}{0.440000in}}%
\pgfusepath{stroke}%
\end{pgfscope}%
\begin{pgfscope}%
\pgfsetbuttcap%
\pgfsetroundjoin%
\definecolor{currentfill}{rgb}{0.333333,0.333333,0.333333}%
\pgfsetfillcolor{currentfill}%
\pgfsetlinewidth{0.803000pt}%
\definecolor{currentstroke}{rgb}{0.333333,0.333333,0.333333}%
\pgfsetstrokecolor{currentstroke}%
\pgfsetdash{}{0pt}%
\pgfsys@defobject{currentmarker}{\pgfqpoint{-0.048611in}{0.000000in}}{\pgfqpoint{-0.000000in}{0.000000in}}{%
\pgfpathmoveto{\pgfqpoint{-0.000000in}{0.000000in}}%
\pgfpathlineto{\pgfqpoint{-0.048611in}{0.000000in}}%
\pgfusepath{stroke,fill}%
}%
\begin{pgfscope}%
\pgfsys@transformshift{0.875000in}{0.440000in}%
\pgfsys@useobject{currentmarker}{}%
\end{pgfscope}%
\end{pgfscope}%
\begin{pgfscope}%
\definecolor{textcolor}{rgb}{0.333333,0.333333,0.333333}%
\pgfsetstrokecolor{textcolor}%
\pgfsetfillcolor{textcolor}%
\pgftext[x=0.484567in, y=0.391775in, left, base]{\color{textcolor}\rmfamily\fontsize{10.000000}{12.000000}\selectfont 0.0\%}%
\end{pgfscope}%
\begin{pgfscope}%
\pgfpathrectangle{\pgfqpoint{0.875000in}{0.440000in}}{\pgfqpoint{5.425000in}{3.080000in}}%
\pgfusepath{clip}%
\pgfsetrectcap%
\pgfsetroundjoin%
\pgfsetlinewidth{0.803000pt}%
\definecolor{currentstroke}{rgb}{1.000000,1.000000,1.000000}%
\pgfsetstrokecolor{currentstroke}%
\pgfsetdash{}{0pt}%
\pgfpathmoveto{\pgfqpoint{0.875000in}{1.056000in}}%
\pgfpathlineto{\pgfqpoint{6.300000in}{1.056000in}}%
\pgfusepath{stroke}%
\end{pgfscope}%
\begin{pgfscope}%
\pgfsetbuttcap%
\pgfsetroundjoin%
\definecolor{currentfill}{rgb}{0.333333,0.333333,0.333333}%
\pgfsetfillcolor{currentfill}%
\pgfsetlinewidth{0.803000pt}%
\definecolor{currentstroke}{rgb}{0.333333,0.333333,0.333333}%
\pgfsetstrokecolor{currentstroke}%
\pgfsetdash{}{0pt}%
\pgfsys@defobject{currentmarker}{\pgfqpoint{-0.048611in}{0.000000in}}{\pgfqpoint{-0.000000in}{0.000000in}}{%
\pgfpathmoveto{\pgfqpoint{-0.000000in}{0.000000in}}%
\pgfpathlineto{\pgfqpoint{-0.048611in}{0.000000in}}%
\pgfusepath{stroke,fill}%
}%
\begin{pgfscope}%
\pgfsys@transformshift{0.875000in}{1.056000in}%
\pgfsys@useobject{currentmarker}{}%
\end{pgfscope}%
\end{pgfscope}%
\begin{pgfscope}%
\definecolor{textcolor}{rgb}{0.333333,0.333333,0.333333}%
\pgfsetstrokecolor{textcolor}%
\pgfsetfillcolor{textcolor}%
\pgftext[x=0.484567in, y=1.007775in, left, base]{\color{textcolor}\rmfamily\fontsize{10.000000}{12.000000}\selectfont 1.0\%}%
\end{pgfscope}%
\begin{pgfscope}%
\pgfpathrectangle{\pgfqpoint{0.875000in}{0.440000in}}{\pgfqpoint{5.425000in}{3.080000in}}%
\pgfusepath{clip}%
\pgfsetrectcap%
\pgfsetroundjoin%
\pgfsetlinewidth{0.803000pt}%
\definecolor{currentstroke}{rgb}{1.000000,1.000000,1.000000}%
\pgfsetstrokecolor{currentstroke}%
\pgfsetdash{}{0pt}%
\pgfpathmoveto{\pgfqpoint{0.875000in}{1.672000in}}%
\pgfpathlineto{\pgfqpoint{6.300000in}{1.672000in}}%
\pgfusepath{stroke}%
\end{pgfscope}%
\begin{pgfscope}%
\pgfsetbuttcap%
\pgfsetroundjoin%
\definecolor{currentfill}{rgb}{0.333333,0.333333,0.333333}%
\pgfsetfillcolor{currentfill}%
\pgfsetlinewidth{0.803000pt}%
\definecolor{currentstroke}{rgb}{0.333333,0.333333,0.333333}%
\pgfsetstrokecolor{currentstroke}%
\pgfsetdash{}{0pt}%
\pgfsys@defobject{currentmarker}{\pgfqpoint{-0.048611in}{0.000000in}}{\pgfqpoint{-0.000000in}{0.000000in}}{%
\pgfpathmoveto{\pgfqpoint{-0.000000in}{0.000000in}}%
\pgfpathlineto{\pgfqpoint{-0.048611in}{0.000000in}}%
\pgfusepath{stroke,fill}%
}%
\begin{pgfscope}%
\pgfsys@transformshift{0.875000in}{1.672000in}%
\pgfsys@useobject{currentmarker}{}%
\end{pgfscope}%
\end{pgfscope}%
\begin{pgfscope}%
\definecolor{textcolor}{rgb}{0.333333,0.333333,0.333333}%
\pgfsetstrokecolor{textcolor}%
\pgfsetfillcolor{textcolor}%
\pgftext[x=0.484567in, y=1.623775in, left, base]{\color{textcolor}\rmfamily\fontsize{10.000000}{12.000000}\selectfont 2.0\%}%
\end{pgfscope}%
\begin{pgfscope}%
\pgfpathrectangle{\pgfqpoint{0.875000in}{0.440000in}}{\pgfqpoint{5.425000in}{3.080000in}}%
\pgfusepath{clip}%
\pgfsetrectcap%
\pgfsetroundjoin%
\pgfsetlinewidth{0.803000pt}%
\definecolor{currentstroke}{rgb}{1.000000,1.000000,1.000000}%
\pgfsetstrokecolor{currentstroke}%
\pgfsetdash{}{0pt}%
\pgfpathmoveto{\pgfqpoint{0.875000in}{2.288000in}}%
\pgfpathlineto{\pgfqpoint{6.300000in}{2.288000in}}%
\pgfusepath{stroke}%
\end{pgfscope}%
\begin{pgfscope}%
\pgfsetbuttcap%
\pgfsetroundjoin%
\definecolor{currentfill}{rgb}{0.333333,0.333333,0.333333}%
\pgfsetfillcolor{currentfill}%
\pgfsetlinewidth{0.803000pt}%
\definecolor{currentstroke}{rgb}{0.333333,0.333333,0.333333}%
\pgfsetstrokecolor{currentstroke}%
\pgfsetdash{}{0pt}%
\pgfsys@defobject{currentmarker}{\pgfqpoint{-0.048611in}{0.000000in}}{\pgfqpoint{-0.000000in}{0.000000in}}{%
\pgfpathmoveto{\pgfqpoint{-0.000000in}{0.000000in}}%
\pgfpathlineto{\pgfqpoint{-0.048611in}{0.000000in}}%
\pgfusepath{stroke,fill}%
}%
\begin{pgfscope}%
\pgfsys@transformshift{0.875000in}{2.288000in}%
\pgfsys@useobject{currentmarker}{}%
\end{pgfscope}%
\end{pgfscope}%
\begin{pgfscope}%
\definecolor{textcolor}{rgb}{0.333333,0.333333,0.333333}%
\pgfsetstrokecolor{textcolor}%
\pgfsetfillcolor{textcolor}%
\pgftext[x=0.484567in, y=2.239775in, left, base]{\color{textcolor}\rmfamily\fontsize{10.000000}{12.000000}\selectfont 3.0\%}%
\end{pgfscope}%
\begin{pgfscope}%
\pgfpathrectangle{\pgfqpoint{0.875000in}{0.440000in}}{\pgfqpoint{5.425000in}{3.080000in}}%
\pgfusepath{clip}%
\pgfsetrectcap%
\pgfsetroundjoin%
\pgfsetlinewidth{0.803000pt}%
\definecolor{currentstroke}{rgb}{1.000000,1.000000,1.000000}%
\pgfsetstrokecolor{currentstroke}%
\pgfsetdash{}{0pt}%
\pgfpathmoveto{\pgfqpoint{0.875000in}{2.904000in}}%
\pgfpathlineto{\pgfqpoint{6.300000in}{2.904000in}}%
\pgfusepath{stroke}%
\end{pgfscope}%
\begin{pgfscope}%
\pgfsetbuttcap%
\pgfsetroundjoin%
\definecolor{currentfill}{rgb}{0.333333,0.333333,0.333333}%
\pgfsetfillcolor{currentfill}%
\pgfsetlinewidth{0.803000pt}%
\definecolor{currentstroke}{rgb}{0.333333,0.333333,0.333333}%
\pgfsetstrokecolor{currentstroke}%
\pgfsetdash{}{0pt}%
\pgfsys@defobject{currentmarker}{\pgfqpoint{-0.048611in}{0.000000in}}{\pgfqpoint{-0.000000in}{0.000000in}}{%
\pgfpathmoveto{\pgfqpoint{-0.000000in}{0.000000in}}%
\pgfpathlineto{\pgfqpoint{-0.048611in}{0.000000in}}%
\pgfusepath{stroke,fill}%
}%
\begin{pgfscope}%
\pgfsys@transformshift{0.875000in}{2.904000in}%
\pgfsys@useobject{currentmarker}{}%
\end{pgfscope}%
\end{pgfscope}%
\begin{pgfscope}%
\definecolor{textcolor}{rgb}{0.333333,0.333333,0.333333}%
\pgfsetstrokecolor{textcolor}%
\pgfsetfillcolor{textcolor}%
\pgftext[x=0.484567in, y=2.855775in, left, base]{\color{textcolor}\rmfamily\fontsize{10.000000}{12.000000}\selectfont 4.0\%}%
\end{pgfscope}%
\begin{pgfscope}%
\pgfpathrectangle{\pgfqpoint{0.875000in}{0.440000in}}{\pgfqpoint{5.425000in}{3.080000in}}%
\pgfusepath{clip}%
\pgfsetrectcap%
\pgfsetroundjoin%
\pgfsetlinewidth{0.803000pt}%
\definecolor{currentstroke}{rgb}{1.000000,1.000000,1.000000}%
\pgfsetstrokecolor{currentstroke}%
\pgfsetdash{}{0pt}%
\pgfpathmoveto{\pgfqpoint{0.875000in}{3.520000in}}%
\pgfpathlineto{\pgfqpoint{6.300000in}{3.520000in}}%
\pgfusepath{stroke}%
\end{pgfscope}%
\begin{pgfscope}%
\pgfsetbuttcap%
\pgfsetroundjoin%
\definecolor{currentfill}{rgb}{0.333333,0.333333,0.333333}%
\pgfsetfillcolor{currentfill}%
\pgfsetlinewidth{0.803000pt}%
\definecolor{currentstroke}{rgb}{0.333333,0.333333,0.333333}%
\pgfsetstrokecolor{currentstroke}%
\pgfsetdash{}{0pt}%
\pgfsys@defobject{currentmarker}{\pgfqpoint{-0.048611in}{0.000000in}}{\pgfqpoint{-0.000000in}{0.000000in}}{%
\pgfpathmoveto{\pgfqpoint{-0.000000in}{0.000000in}}%
\pgfpathlineto{\pgfqpoint{-0.048611in}{0.000000in}}%
\pgfusepath{stroke,fill}%
}%
\begin{pgfscope}%
\pgfsys@transformshift{0.875000in}{3.520000in}%
\pgfsys@useobject{currentmarker}{}%
\end{pgfscope}%
\end{pgfscope}%
\begin{pgfscope}%
\definecolor{textcolor}{rgb}{0.333333,0.333333,0.333333}%
\pgfsetstrokecolor{textcolor}%
\pgfsetfillcolor{textcolor}%
\pgftext[x=0.484567in, y=3.471775in, left, base]{\color{textcolor}\rmfamily\fontsize{10.000000}{12.000000}\selectfont 5.0\%}%
\end{pgfscope}%
\begin{pgfscope}%
\definecolor{textcolor}{rgb}{0.333333,0.333333,0.333333}%
\pgfsetstrokecolor{textcolor}%
\pgfsetfillcolor{textcolor}%
\pgftext[x=0.429011in,y=1.980000in,,bottom,rotate=90.000000]{\color{textcolor}\rmfamily\fontsize{12.000000}{14.400000}\selectfont Spread}%
\end{pgfscope}%
\begin{pgfscope}%
\pgfpathrectangle{\pgfqpoint{0.875000in}{0.440000in}}{\pgfqpoint{5.425000in}{3.080000in}}%
\pgfusepath{clip}%
\pgfsetrectcap%
\pgfsetroundjoin%
\pgfsetlinewidth{1.505625pt}%
\definecolor{currentstroke}{rgb}{1.000000,0.266667,0.239216}%
\pgfsetstrokecolor{currentstroke}%
\pgfsetdash{}{0pt}%
\pgfpathmoveto{\pgfqpoint{0.874177in}{0.575520in}}%
\pgfpathlineto{\pgfqpoint{0.875823in}{0.587840in}}%
\pgfpathlineto{\pgfqpoint{0.879113in}{0.594000in}}%
\pgfpathlineto{\pgfqpoint{0.879936in}{0.594000in}}%
\pgfpathlineto{\pgfqpoint{0.881581in}{0.618640in}}%
\pgfpathlineto{\pgfqpoint{0.884049in}{0.587840in}}%
\pgfpathlineto{\pgfqpoint{0.886516in}{0.600160in}}%
\pgfpathlineto{\pgfqpoint{0.887339in}{0.594000in}}%
\pgfpathlineto{\pgfqpoint{0.890629in}{0.587840in}}%
\pgfpathlineto{\pgfqpoint{0.892274in}{0.600160in}}%
\pgfpathlineto{\pgfqpoint{0.893097in}{0.587840in}}%
\pgfpathlineto{\pgfqpoint{0.896387in}{0.581680in}}%
\pgfpathlineto{\pgfqpoint{0.897210in}{0.563200in}}%
\pgfpathlineto{\pgfqpoint{0.898033in}{0.569360in}}%
\pgfpathlineto{\pgfqpoint{0.898855in}{0.581680in}}%
\pgfpathlineto{\pgfqpoint{0.901323in}{0.563200in}}%
\pgfpathlineto{\pgfqpoint{0.902968in}{0.569360in}}%
\pgfpathlineto{\pgfqpoint{0.903791in}{0.569360in}}%
\pgfpathlineto{\pgfqpoint{0.904613in}{0.575520in}}%
\pgfpathlineto{\pgfqpoint{0.907904in}{0.563200in}}%
\pgfpathlineto{\pgfqpoint{0.908726in}{0.575520in}}%
\pgfpathlineto{\pgfqpoint{0.909549in}{0.569360in}}%
\pgfpathlineto{\pgfqpoint{0.910371in}{0.575520in}}%
\pgfpathlineto{\pgfqpoint{0.913662in}{0.557040in}}%
\pgfpathlineto{\pgfqpoint{0.915307in}{0.563200in}}%
\pgfpathlineto{\pgfqpoint{0.916130in}{0.563200in}}%
\pgfpathlineto{\pgfqpoint{0.919420in}{0.544720in}}%
\pgfpathlineto{\pgfqpoint{0.921065in}{0.550880in}}%
\pgfpathlineto{\pgfqpoint{0.921888in}{0.550880in}}%
\pgfpathlineto{\pgfqpoint{0.925178in}{0.544720in}}%
\pgfpathlineto{\pgfqpoint{0.926001in}{0.544720in}}%
\pgfpathlineto{\pgfqpoint{0.927646in}{0.557040in}}%
\pgfpathlineto{\pgfqpoint{0.930114in}{0.538560in}}%
\pgfpathlineto{\pgfqpoint{0.931759in}{0.544720in}}%
\pgfpathlineto{\pgfqpoint{0.932582in}{0.538560in}}%
\pgfpathlineto{\pgfqpoint{0.933404in}{0.544720in}}%
\pgfpathlineto{\pgfqpoint{0.935872in}{0.538560in}}%
\pgfpathlineto{\pgfqpoint{0.939162in}{0.557040in}}%
\pgfpathlineto{\pgfqpoint{0.941630in}{0.544720in}}%
\pgfpathlineto{\pgfqpoint{0.944098in}{0.557040in}}%
\pgfpathlineto{\pgfqpoint{0.947388in}{0.538560in}}%
\pgfpathlineto{\pgfqpoint{0.949856in}{0.557040in}}%
\pgfpathlineto{\pgfqpoint{0.950679in}{0.550880in}}%
\pgfpathlineto{\pgfqpoint{0.953969in}{0.575520in}}%
\pgfpathlineto{\pgfqpoint{0.954792in}{0.569360in}}%
\pgfpathlineto{\pgfqpoint{0.955614in}{0.575520in}}%
\pgfpathlineto{\pgfqpoint{0.959727in}{0.569360in}}%
\pgfpathlineto{\pgfqpoint{0.960550in}{0.563200in}}%
\pgfpathlineto{\pgfqpoint{0.962195in}{0.581680in}}%
\pgfpathlineto{\pgfqpoint{0.966308in}{0.557040in}}%
\pgfpathlineto{\pgfqpoint{0.967130in}{0.575520in}}%
\pgfpathlineto{\pgfqpoint{0.967953in}{0.569360in}}%
\pgfpathlineto{\pgfqpoint{0.970421in}{0.557040in}}%
\pgfpathlineto{\pgfqpoint{0.972889in}{0.581680in}}%
\pgfpathlineto{\pgfqpoint{0.973711in}{0.575520in}}%
\pgfpathlineto{\pgfqpoint{0.977002in}{0.557040in}}%
\pgfpathlineto{\pgfqpoint{0.977824in}{0.569360in}}%
\pgfpathlineto{\pgfqpoint{0.978647in}{0.563200in}}%
\pgfpathlineto{\pgfqpoint{0.979469in}{0.526240in}}%
\pgfpathlineto{\pgfqpoint{0.981937in}{0.544720in}}%
\pgfpathlineto{\pgfqpoint{0.985227in}{0.618640in}}%
\pgfpathlineto{\pgfqpoint{0.987695in}{0.569360in}}%
\pgfpathlineto{\pgfqpoint{0.988518in}{0.575520in}}%
\pgfpathlineto{\pgfqpoint{0.990986in}{0.606320in}}%
\pgfpathlineto{\pgfqpoint{0.994276in}{0.581680in}}%
\pgfpathlineto{\pgfqpoint{0.995099in}{0.587840in}}%
\pgfpathlineto{\pgfqpoint{0.995921in}{0.606320in}}%
\pgfpathlineto{\pgfqpoint{1.000034in}{0.550880in}}%
\pgfpathlineto{\pgfqpoint{1.001679in}{0.569360in}}%
\pgfpathlineto{\pgfqpoint{1.002502in}{0.557040in}}%
\pgfpathlineto{\pgfqpoint{1.005792in}{0.538560in}}%
\pgfpathlineto{\pgfqpoint{1.007437in}{0.563200in}}%
\pgfpathlineto{\pgfqpoint{1.010728in}{0.526240in}}%
\pgfpathlineto{\pgfqpoint{1.012373in}{0.594000in}}%
\pgfpathlineto{\pgfqpoint{1.013196in}{0.624800in}}%
\pgfpathlineto{\pgfqpoint{1.014018in}{0.618640in}}%
\pgfpathlineto{\pgfqpoint{1.016486in}{0.587840in}}%
\pgfpathlineto{\pgfqpoint{1.018131in}{0.624800in}}%
\pgfpathlineto{\pgfqpoint{1.018954in}{0.643280in}}%
\pgfpathlineto{\pgfqpoint{1.022244in}{0.581680in}}%
\pgfpathlineto{\pgfqpoint{1.023067in}{0.587840in}}%
\pgfpathlineto{\pgfqpoint{1.024712in}{0.686400in}}%
\pgfpathlineto{\pgfqpoint{1.025534in}{0.643280in}}%
\pgfpathlineto{\pgfqpoint{1.028825in}{0.600160in}}%
\pgfpathlineto{\pgfqpoint{1.031293in}{0.649440in}}%
\pgfpathlineto{\pgfqpoint{1.034583in}{0.612480in}}%
\pgfpathlineto{\pgfqpoint{1.037051in}{0.630960in}}%
\pgfpathlineto{\pgfqpoint{1.039519in}{0.624800in}}%
\pgfpathlineto{\pgfqpoint{1.040341in}{0.618640in}}%
\pgfpathlineto{\pgfqpoint{1.041986in}{0.637120in}}%
\pgfpathlineto{\pgfqpoint{1.042809in}{0.630960in}}%
\pgfpathlineto{\pgfqpoint{1.046099in}{0.581680in}}%
\pgfpathlineto{\pgfqpoint{1.048567in}{0.612480in}}%
\pgfpathlineto{\pgfqpoint{1.051035in}{0.569360in}}%
\pgfpathlineto{\pgfqpoint{1.054325in}{0.624800in}}%
\pgfpathlineto{\pgfqpoint{1.056793in}{0.550880in}}%
\pgfpathlineto{\pgfqpoint{1.057616in}{0.563200in}}%
\pgfpathlineto{\pgfqpoint{1.059261in}{0.624800in}}%
\pgfpathlineto{\pgfqpoint{1.060083in}{0.630960in}}%
\pgfpathlineto{\pgfqpoint{1.062551in}{0.594000in}}%
\pgfpathlineto{\pgfqpoint{1.065019in}{0.618640in}}%
\pgfpathlineto{\pgfqpoint{1.065842in}{0.612480in}}%
\pgfpathlineto{\pgfqpoint{1.068309in}{0.581680in}}%
\pgfpathlineto{\pgfqpoint{1.071600in}{0.600160in}}%
\pgfpathlineto{\pgfqpoint{1.074890in}{0.581680in}}%
\pgfpathlineto{\pgfqpoint{1.075713in}{0.594000in}}%
\pgfpathlineto{\pgfqpoint{1.076535in}{0.587840in}}%
\pgfpathlineto{\pgfqpoint{1.077358in}{0.563200in}}%
\pgfpathlineto{\pgfqpoint{1.080648in}{0.575520in}}%
\pgfpathlineto{\pgfqpoint{1.082293in}{0.600160in}}%
\pgfpathlineto{\pgfqpoint{1.083116in}{0.587840in}}%
\pgfpathlineto{\pgfqpoint{1.085584in}{0.587840in}}%
\pgfpathlineto{\pgfqpoint{1.088052in}{0.606320in}}%
\pgfpathlineto{\pgfqpoint{1.088874in}{0.581680in}}%
\pgfpathlineto{\pgfqpoint{1.091342in}{0.575520in}}%
\pgfpathlineto{\pgfqpoint{1.093810in}{0.600160in}}%
\pgfpathlineto{\pgfqpoint{1.094632in}{0.600160in}}%
\pgfpathlineto{\pgfqpoint{1.097100in}{0.594000in}}%
\pgfpathlineto{\pgfqpoint{1.100390in}{0.661760in}}%
\pgfpathlineto{\pgfqpoint{1.102858in}{0.649440in}}%
\pgfpathlineto{\pgfqpoint{1.105326in}{0.680240in}}%
\pgfpathlineto{\pgfqpoint{1.106149in}{0.680240in}}%
\pgfpathlineto{\pgfqpoint{1.109439in}{0.667920in}}%
\pgfpathlineto{\pgfqpoint{1.110262in}{0.674080in}}%
\pgfpathlineto{\pgfqpoint{1.114375in}{0.624800in}}%
\pgfpathlineto{\pgfqpoint{1.115197in}{0.618640in}}%
\pgfpathlineto{\pgfqpoint{1.116020in}{0.624800in}}%
\pgfpathlineto{\pgfqpoint{1.116842in}{0.618640in}}%
\pgfpathlineto{\pgfqpoint{1.117665in}{0.624800in}}%
\pgfpathlineto{\pgfqpoint{1.120955in}{0.587840in}}%
\pgfpathlineto{\pgfqpoint{1.121778in}{0.594000in}}%
\pgfpathlineto{\pgfqpoint{1.123423in}{0.624800in}}%
\pgfpathlineto{\pgfqpoint{1.125891in}{0.575520in}}%
\pgfpathlineto{\pgfqpoint{1.127536in}{0.612480in}}%
\pgfpathlineto{\pgfqpoint{1.128359in}{0.606320in}}%
\pgfpathlineto{\pgfqpoint{1.129181in}{0.581680in}}%
\pgfpathlineto{\pgfqpoint{1.131649in}{0.575520in}}%
\pgfpathlineto{\pgfqpoint{1.134939in}{0.594000in}}%
\pgfpathlineto{\pgfqpoint{1.138230in}{0.569360in}}%
\pgfpathlineto{\pgfqpoint{1.139875in}{0.594000in}}%
\pgfpathlineto{\pgfqpoint{1.140697in}{0.594000in}}%
\pgfpathlineto{\pgfqpoint{1.143165in}{0.563200in}}%
\pgfpathlineto{\pgfqpoint{1.144810in}{0.587840in}}%
\pgfpathlineto{\pgfqpoint{1.146456in}{0.587840in}}%
\pgfpathlineto{\pgfqpoint{1.148923in}{0.563200in}}%
\pgfpathlineto{\pgfqpoint{1.152214in}{0.606320in}}%
\pgfpathlineto{\pgfqpoint{1.154682in}{0.575520in}}%
\pgfpathlineto{\pgfqpoint{1.157149in}{0.606320in}}%
\pgfpathlineto{\pgfqpoint{1.157972in}{0.606320in}}%
\pgfpathlineto{\pgfqpoint{1.160440in}{0.612480in}}%
\pgfpathlineto{\pgfqpoint{1.162908in}{0.655600in}}%
\pgfpathlineto{\pgfqpoint{1.163730in}{0.661760in}}%
\pgfpathlineto{\pgfqpoint{1.166198in}{0.643280in}}%
\pgfpathlineto{\pgfqpoint{1.168666in}{0.686400in}}%
\pgfpathlineto{\pgfqpoint{1.173601in}{0.667920in}}%
\pgfpathlineto{\pgfqpoint{1.175246in}{0.674080in}}%
\pgfpathlineto{\pgfqpoint{1.178537in}{0.606320in}}%
\pgfpathlineto{\pgfqpoint{1.180182in}{0.649440in}}%
\pgfpathlineto{\pgfqpoint{1.181005in}{0.643280in}}%
\pgfpathlineto{\pgfqpoint{1.183472in}{0.618640in}}%
\pgfpathlineto{\pgfqpoint{1.185940in}{0.655600in}}%
\pgfpathlineto{\pgfqpoint{1.190053in}{0.630960in}}%
\pgfpathlineto{\pgfqpoint{1.192521in}{0.680240in}}%
\pgfpathlineto{\pgfqpoint{1.198279in}{0.643280in}}%
\pgfpathlineto{\pgfqpoint{1.200747in}{0.606320in}}%
\pgfpathlineto{\pgfqpoint{1.202392in}{0.612480in}}%
\pgfpathlineto{\pgfqpoint{1.204037in}{0.649440in}}%
\pgfpathlineto{\pgfqpoint{1.207328in}{0.618640in}}%
\pgfpathlineto{\pgfqpoint{1.208973in}{0.637120in}}%
\pgfpathlineto{\pgfqpoint{1.209795in}{0.618640in}}%
\pgfpathlineto{\pgfqpoint{1.212263in}{0.594000in}}%
\pgfpathlineto{\pgfqpoint{1.213086in}{0.600160in}}%
\pgfpathlineto{\pgfqpoint{1.214731in}{0.637120in}}%
\pgfpathlineto{\pgfqpoint{1.215553in}{0.618640in}}%
\pgfpathlineto{\pgfqpoint{1.219666in}{0.587840in}}%
\pgfpathlineto{\pgfqpoint{1.220489in}{0.594000in}}%
\pgfpathlineto{\pgfqpoint{1.221312in}{0.575520in}}%
\pgfpathlineto{\pgfqpoint{1.223779in}{0.563200in}}%
\pgfpathlineto{\pgfqpoint{1.225425in}{0.600160in}}%
\pgfpathlineto{\pgfqpoint{1.226247in}{0.594000in}}%
\pgfpathlineto{\pgfqpoint{1.227070in}{0.600160in}}%
\pgfpathlineto{\pgfqpoint{1.229538in}{0.594000in}}%
\pgfpathlineto{\pgfqpoint{1.232828in}{0.630960in}}%
\pgfpathlineto{\pgfqpoint{1.235296in}{0.606320in}}%
\pgfpathlineto{\pgfqpoint{1.236941in}{0.624800in}}%
\pgfpathlineto{\pgfqpoint{1.237763in}{0.637120in}}%
\pgfpathlineto{\pgfqpoint{1.241876in}{0.581680in}}%
\pgfpathlineto{\pgfqpoint{1.242699in}{0.637120in}}%
\pgfpathlineto{\pgfqpoint{1.243522in}{0.630960in}}%
\pgfpathlineto{\pgfqpoint{1.247635in}{0.630960in}}%
\pgfpathlineto{\pgfqpoint{1.248457in}{0.643280in}}%
\pgfpathlineto{\pgfqpoint{1.249280in}{0.680240in}}%
\pgfpathlineto{\pgfqpoint{1.250102in}{0.674080in}}%
\pgfpathlineto{\pgfqpoint{1.252570in}{0.667920in}}%
\pgfpathlineto{\pgfqpoint{1.253393in}{0.674080in}}%
\pgfpathlineto{\pgfqpoint{1.255038in}{0.698720in}}%
\pgfpathlineto{\pgfqpoint{1.255861in}{0.686400in}}%
\pgfpathlineto{\pgfqpoint{1.259151in}{0.704880in}}%
\pgfpathlineto{\pgfqpoint{1.259973in}{0.698720in}}%
\pgfpathlineto{\pgfqpoint{1.260796in}{0.704880in}}%
\pgfpathlineto{\pgfqpoint{1.261619in}{0.698720in}}%
\pgfpathlineto{\pgfqpoint{1.264086in}{0.618640in}}%
\pgfpathlineto{\pgfqpoint{1.264909in}{0.624800in}}%
\pgfpathlineto{\pgfqpoint{1.266554in}{0.649440in}}%
\pgfpathlineto{\pgfqpoint{1.267377in}{0.624800in}}%
\pgfpathlineto{\pgfqpoint{1.269845in}{0.624800in}}%
\pgfpathlineto{\pgfqpoint{1.272312in}{0.680240in}}%
\pgfpathlineto{\pgfqpoint{1.273135in}{0.667920in}}%
\pgfpathlineto{\pgfqpoint{1.276425in}{0.667920in}}%
\pgfpathlineto{\pgfqpoint{1.278071in}{0.735680in}}%
\pgfpathlineto{\pgfqpoint{1.278893in}{0.704880in}}%
\pgfpathlineto{\pgfqpoint{1.281361in}{0.680240in}}%
\pgfpathlineto{\pgfqpoint{1.283829in}{0.717200in}}%
\pgfpathlineto{\pgfqpoint{1.284651in}{0.754160in}}%
\pgfpathlineto{\pgfqpoint{1.287119in}{0.704880in}}%
\pgfpathlineto{\pgfqpoint{1.288764in}{0.735680in}}%
\pgfpathlineto{\pgfqpoint{1.289587in}{0.729520in}}%
\pgfpathlineto{\pgfqpoint{1.290409in}{0.723360in}}%
\pgfpathlineto{\pgfqpoint{1.292877in}{0.674080in}}%
\pgfpathlineto{\pgfqpoint{1.294522in}{0.698720in}}%
\pgfpathlineto{\pgfqpoint{1.295345in}{0.711040in}}%
\pgfpathlineto{\pgfqpoint{1.296168in}{0.704880in}}%
\pgfpathlineto{\pgfqpoint{1.299458in}{0.692560in}}%
\pgfpathlineto{\pgfqpoint{1.301103in}{0.711040in}}%
\pgfpathlineto{\pgfqpoint{1.301926in}{0.698720in}}%
\pgfpathlineto{\pgfqpoint{1.305216in}{0.692560in}}%
\pgfpathlineto{\pgfqpoint{1.306039in}{0.698720in}}%
\pgfpathlineto{\pgfqpoint{1.307684in}{0.723360in}}%
\pgfpathlineto{\pgfqpoint{1.310152in}{0.711040in}}%
\pgfpathlineto{\pgfqpoint{1.313442in}{0.760320in}}%
\pgfpathlineto{\pgfqpoint{1.315910in}{0.735680in}}%
\pgfpathlineto{\pgfqpoint{1.317555in}{0.741840in}}%
\pgfpathlineto{\pgfqpoint{1.321668in}{0.680240in}}%
\pgfpathlineto{\pgfqpoint{1.324136in}{0.723360in}}%
\pgfpathlineto{\pgfqpoint{1.324958in}{0.704880in}}%
\pgfpathlineto{\pgfqpoint{1.328249in}{0.686400in}}%
\pgfpathlineto{\pgfqpoint{1.329894in}{0.729520in}}%
\pgfpathlineto{\pgfqpoint{1.333184in}{0.711040in}}%
\pgfpathlineto{\pgfqpoint{1.334007in}{0.711040in}}%
\pgfpathlineto{\pgfqpoint{1.335652in}{0.717200in}}%
\pgfpathlineto{\pgfqpoint{1.338942in}{0.680240in}}%
\pgfpathlineto{\pgfqpoint{1.339765in}{0.680240in}}%
\pgfpathlineto{\pgfqpoint{1.340588in}{0.692560in}}%
\pgfpathlineto{\pgfqpoint{1.344701in}{0.618640in}}%
\pgfpathlineto{\pgfqpoint{1.347991in}{0.661760in}}%
\pgfpathlineto{\pgfqpoint{1.350459in}{0.624800in}}%
\pgfpathlineto{\pgfqpoint{1.351281in}{0.630960in}}%
\pgfpathlineto{\pgfqpoint{1.352926in}{0.661760in}}%
\pgfpathlineto{\pgfqpoint{1.353749in}{0.630960in}}%
\pgfpathlineto{\pgfqpoint{1.356217in}{0.618640in}}%
\pgfpathlineto{\pgfqpoint{1.357862in}{0.655600in}}%
\pgfpathlineto{\pgfqpoint{1.358685in}{0.661760in}}%
\pgfpathlineto{\pgfqpoint{1.361975in}{0.643280in}}%
\pgfpathlineto{\pgfqpoint{1.362798in}{0.649440in}}%
\pgfpathlineto{\pgfqpoint{1.364443in}{0.686400in}}%
\pgfpathlineto{\pgfqpoint{1.365265in}{0.674080in}}%
\pgfpathlineto{\pgfqpoint{1.367733in}{0.667920in}}%
\pgfpathlineto{\pgfqpoint{1.370201in}{0.686400in}}%
\pgfpathlineto{\pgfqpoint{1.371024in}{0.680240in}}%
\pgfpathlineto{\pgfqpoint{1.374314in}{0.692560in}}%
\pgfpathlineto{\pgfqpoint{1.375959in}{0.723360in}}%
\pgfpathlineto{\pgfqpoint{1.376782in}{0.674080in}}%
\pgfpathlineto{\pgfqpoint{1.380072in}{0.649440in}}%
\pgfpathlineto{\pgfqpoint{1.382540in}{0.711040in}}%
\pgfpathlineto{\pgfqpoint{1.385008in}{0.698720in}}%
\pgfpathlineto{\pgfqpoint{1.387475in}{0.760320in}}%
\pgfpathlineto{\pgfqpoint{1.388298in}{0.729520in}}%
\pgfpathlineto{\pgfqpoint{1.391588in}{0.674080in}}%
\pgfpathlineto{\pgfqpoint{1.392411in}{0.821920in}}%
\pgfpathlineto{\pgfqpoint{1.393234in}{0.791120in}}%
\pgfpathlineto{\pgfqpoint{1.394056in}{0.791120in}}%
\pgfpathlineto{\pgfqpoint{1.396524in}{0.772640in}}%
\pgfpathlineto{\pgfqpoint{1.398169in}{0.828080in}}%
\pgfpathlineto{\pgfqpoint{1.398992in}{0.803440in}}%
\pgfpathlineto{\pgfqpoint{1.399814in}{0.809600in}}%
\pgfpathlineto{\pgfqpoint{1.403105in}{0.772640in}}%
\pgfpathlineto{\pgfqpoint{1.403927in}{0.809600in}}%
\pgfpathlineto{\pgfqpoint{1.404750in}{0.803440in}}%
\pgfpathlineto{\pgfqpoint{1.405572in}{0.797280in}}%
\pgfpathlineto{\pgfqpoint{1.408863in}{0.735680in}}%
\pgfpathlineto{\pgfqpoint{1.409685in}{0.760320in}}%
\pgfpathlineto{\pgfqpoint{1.413798in}{0.680240in}}%
\pgfpathlineto{\pgfqpoint{1.417089in}{0.704880in}}%
\pgfpathlineto{\pgfqpoint{1.419556in}{0.667920in}}%
\pgfpathlineto{\pgfqpoint{1.420379in}{0.674080in}}%
\pgfpathlineto{\pgfqpoint{1.422024in}{0.717200in}}%
\pgfpathlineto{\pgfqpoint{1.425315in}{0.667920in}}%
\pgfpathlineto{\pgfqpoint{1.427782in}{0.711040in}}%
\pgfpathlineto{\pgfqpoint{1.428605in}{0.698720in}}%
\pgfpathlineto{\pgfqpoint{1.431073in}{0.704880in}}%
\pgfpathlineto{\pgfqpoint{1.431895in}{0.711040in}}%
\pgfpathlineto{\pgfqpoint{1.432718in}{0.704880in}}%
\pgfpathlineto{\pgfqpoint{1.433541in}{0.717200in}}%
\pgfpathlineto{\pgfqpoint{1.436831in}{0.692560in}}%
\pgfpathlineto{\pgfqpoint{1.438476in}{0.723360in}}%
\pgfpathlineto{\pgfqpoint{1.439299in}{0.717200in}}%
\pgfpathlineto{\pgfqpoint{1.440121in}{0.723360in}}%
\pgfpathlineto{\pgfqpoint{1.442589in}{0.729520in}}%
\pgfpathlineto{\pgfqpoint{1.443412in}{0.778800in}}%
\pgfpathlineto{\pgfqpoint{1.444234in}{0.766480in}}%
\pgfpathlineto{\pgfqpoint{1.445879in}{0.772640in}}%
\pgfpathlineto{\pgfqpoint{1.449170in}{0.760320in}}%
\pgfpathlineto{\pgfqpoint{1.450815in}{0.784960in}}%
\pgfpathlineto{\pgfqpoint{1.451638in}{0.778800in}}%
\pgfpathlineto{\pgfqpoint{1.454105in}{0.748000in}}%
\pgfpathlineto{\pgfqpoint{1.455751in}{0.760320in}}%
\pgfpathlineto{\pgfqpoint{1.457396in}{0.828080in}}%
\pgfpathlineto{\pgfqpoint{1.461509in}{0.858880in}}%
\pgfpathlineto{\pgfqpoint{1.463154in}{0.846560in}}%
\pgfpathlineto{\pgfqpoint{1.465622in}{0.815760in}}%
\pgfpathlineto{\pgfqpoint{1.466444in}{0.821920in}}%
\pgfpathlineto{\pgfqpoint{1.467267in}{0.809600in}}%
\pgfpathlineto{\pgfqpoint{1.468089in}{0.828080in}}%
\pgfpathlineto{\pgfqpoint{1.468912in}{0.821920in}}%
\pgfpathlineto{\pgfqpoint{1.473025in}{0.846560in}}%
\pgfpathlineto{\pgfqpoint{1.474670in}{0.778800in}}%
\pgfpathlineto{\pgfqpoint{1.480428in}{0.704880in}}%
\pgfpathlineto{\pgfqpoint{1.482896in}{0.698720in}}%
\pgfpathlineto{\pgfqpoint{1.484541in}{0.674080in}}%
\pgfpathlineto{\pgfqpoint{1.485364in}{0.674080in}}%
\pgfpathlineto{\pgfqpoint{1.489477in}{0.637120in}}%
\pgfpathlineto{\pgfqpoint{1.491945in}{0.667920in}}%
\pgfpathlineto{\pgfqpoint{1.496880in}{0.637120in}}%
\pgfpathlineto{\pgfqpoint{1.497703in}{0.643280in}}%
\pgfpathlineto{\pgfqpoint{1.500171in}{0.624800in}}%
\pgfpathlineto{\pgfqpoint{1.503461in}{0.649440in}}%
\pgfpathlineto{\pgfqpoint{1.505929in}{0.649440in}}%
\pgfpathlineto{\pgfqpoint{1.507574in}{0.637120in}}%
\pgfpathlineto{\pgfqpoint{1.508397in}{0.643280in}}%
\pgfpathlineto{\pgfqpoint{1.511687in}{0.624800in}}%
\pgfpathlineto{\pgfqpoint{1.512509in}{0.624800in}}%
\pgfpathlineto{\pgfqpoint{1.514977in}{0.649440in}}%
\pgfpathlineto{\pgfqpoint{1.518268in}{0.630960in}}%
\pgfpathlineto{\pgfqpoint{1.519913in}{0.637120in}}%
\pgfpathlineto{\pgfqpoint{1.520735in}{0.637120in}}%
\pgfpathlineto{\pgfqpoint{1.524026in}{0.630960in}}%
\pgfpathlineto{\pgfqpoint{1.525671in}{0.649440in}}%
\pgfpathlineto{\pgfqpoint{1.531429in}{0.692560in}}%
\pgfpathlineto{\pgfqpoint{1.532252in}{0.686400in}}%
\pgfpathlineto{\pgfqpoint{1.534719in}{0.692560in}}%
\pgfpathlineto{\pgfqpoint{1.535542in}{0.711040in}}%
\pgfpathlineto{\pgfqpoint{1.536365in}{0.692560in}}%
\pgfpathlineto{\pgfqpoint{1.537187in}{0.704880in}}%
\pgfpathlineto{\pgfqpoint{1.538010in}{0.692560in}}%
\pgfpathlineto{\pgfqpoint{1.540478in}{0.680240in}}%
\pgfpathlineto{\pgfqpoint{1.541300in}{0.667920in}}%
\pgfpathlineto{\pgfqpoint{1.542123in}{0.674080in}}%
\pgfpathlineto{\pgfqpoint{1.543768in}{0.698720in}}%
\pgfpathlineto{\pgfqpoint{1.546236in}{0.717200in}}%
\pgfpathlineto{\pgfqpoint{1.547058in}{0.704880in}}%
\pgfpathlineto{\pgfqpoint{1.548704in}{0.741840in}}%
\pgfpathlineto{\pgfqpoint{1.552817in}{0.717200in}}%
\pgfpathlineto{\pgfqpoint{1.554462in}{0.723360in}}%
\pgfpathlineto{\pgfqpoint{1.557752in}{0.723360in}}%
\pgfpathlineto{\pgfqpoint{1.560220in}{0.748000in}}%
\pgfpathlineto{\pgfqpoint{1.565155in}{0.723360in}}%
\pgfpathlineto{\pgfqpoint{1.565978in}{0.735680in}}%
\pgfpathlineto{\pgfqpoint{1.566801in}{0.729520in}}%
\pgfpathlineto{\pgfqpoint{1.569268in}{0.729520in}}%
\pgfpathlineto{\pgfqpoint{1.570091in}{0.711040in}}%
\pgfpathlineto{\pgfqpoint{1.571736in}{0.741840in}}%
\pgfpathlineto{\pgfqpoint{1.572559in}{0.735680in}}%
\pgfpathlineto{\pgfqpoint{1.576672in}{0.717200in}}%
\pgfpathlineto{\pgfqpoint{1.577494in}{0.735680in}}%
\pgfpathlineto{\pgfqpoint{1.578317in}{0.729520in}}%
\pgfpathlineto{\pgfqpoint{1.581607in}{0.686400in}}%
\pgfpathlineto{\pgfqpoint{1.582430in}{0.686400in}}%
\pgfpathlineto{\pgfqpoint{1.583252in}{0.729520in}}%
\pgfpathlineto{\pgfqpoint{1.584075in}{0.717200in}}%
\pgfpathlineto{\pgfqpoint{1.586543in}{0.704880in}}%
\pgfpathlineto{\pgfqpoint{1.589011in}{0.735680in}}%
\pgfpathlineto{\pgfqpoint{1.589833in}{0.741840in}}%
\pgfpathlineto{\pgfqpoint{1.592301in}{0.748000in}}%
\pgfpathlineto{\pgfqpoint{1.593124in}{0.741840in}}%
\pgfpathlineto{\pgfqpoint{1.594769in}{0.760320in}}%
\pgfpathlineto{\pgfqpoint{1.595591in}{0.754160in}}%
\pgfpathlineto{\pgfqpoint{1.598882in}{0.754160in}}%
\pgfpathlineto{\pgfqpoint{1.599704in}{0.748000in}}%
\pgfpathlineto{\pgfqpoint{1.601350in}{0.797280in}}%
\pgfpathlineto{\pgfqpoint{1.603817in}{0.754160in}}%
\pgfpathlineto{\pgfqpoint{1.606285in}{0.784960in}}%
\pgfpathlineto{\pgfqpoint{1.607108in}{0.784960in}}%
\pgfpathlineto{\pgfqpoint{1.609575in}{0.754160in}}%
\pgfpathlineto{\pgfqpoint{1.612043in}{0.858880in}}%
\pgfpathlineto{\pgfqpoint{1.612866in}{0.846560in}}%
\pgfpathlineto{\pgfqpoint{1.615334in}{0.815760in}}%
\pgfpathlineto{\pgfqpoint{1.617801in}{0.852720in}}%
\pgfpathlineto{\pgfqpoint{1.621092in}{0.797280in}}%
\pgfpathlineto{\pgfqpoint{1.621914in}{0.797280in}}%
\pgfpathlineto{\pgfqpoint{1.623560in}{0.828080in}}%
\pgfpathlineto{\pgfqpoint{1.624382in}{0.815760in}}%
\pgfpathlineto{\pgfqpoint{1.626850in}{0.760320in}}%
\pgfpathlineto{\pgfqpoint{1.629318in}{0.834240in}}%
\pgfpathlineto{\pgfqpoint{1.632608in}{0.791120in}}%
\pgfpathlineto{\pgfqpoint{1.633431in}{0.797280in}}%
\pgfpathlineto{\pgfqpoint{1.634253in}{0.791120in}}%
\pgfpathlineto{\pgfqpoint{1.635076in}{0.803440in}}%
\pgfpathlineto{\pgfqpoint{1.639189in}{0.748000in}}%
\pgfpathlineto{\pgfqpoint{1.640834in}{0.784960in}}%
\pgfpathlineto{\pgfqpoint{1.641657in}{0.766480in}}%
\pgfpathlineto{\pgfqpoint{1.644124in}{0.760320in}}%
\pgfpathlineto{\pgfqpoint{1.644947in}{0.741840in}}%
\pgfpathlineto{\pgfqpoint{1.647415in}{0.778800in}}%
\pgfpathlineto{\pgfqpoint{1.650705in}{0.735680in}}%
\pgfpathlineto{\pgfqpoint{1.653173in}{0.772640in}}%
\pgfpathlineto{\pgfqpoint{1.655641in}{0.717200in}}%
\pgfpathlineto{\pgfqpoint{1.656463in}{0.735680in}}%
\pgfpathlineto{\pgfqpoint{1.657286in}{0.723360in}}%
\pgfpathlineto{\pgfqpoint{1.658108in}{0.735680in}}%
\pgfpathlineto{\pgfqpoint{1.658931in}{0.723360in}}%
\pgfpathlineto{\pgfqpoint{1.661399in}{0.704880in}}%
\pgfpathlineto{\pgfqpoint{1.663044in}{0.729520in}}%
\pgfpathlineto{\pgfqpoint{1.664689in}{0.704880in}}%
\pgfpathlineto{\pgfqpoint{1.667157in}{0.698720in}}%
\pgfpathlineto{\pgfqpoint{1.669625in}{0.711040in}}%
\pgfpathlineto{\pgfqpoint{1.670447in}{0.698720in}}%
\pgfpathlineto{\pgfqpoint{1.681141in}{0.778800in}}%
\pgfpathlineto{\pgfqpoint{1.681964in}{0.803440in}}%
\pgfpathlineto{\pgfqpoint{1.684431in}{0.797280in}}%
\pgfpathlineto{\pgfqpoint{1.685254in}{0.815760in}}%
\pgfpathlineto{\pgfqpoint{1.686077in}{0.809600in}}%
\pgfpathlineto{\pgfqpoint{1.687722in}{0.784960in}}%
\pgfpathlineto{\pgfqpoint{1.691835in}{0.797280in}}%
\pgfpathlineto{\pgfqpoint{1.692657in}{0.809600in}}%
\pgfpathlineto{\pgfqpoint{1.693480in}{0.791120in}}%
\pgfpathlineto{\pgfqpoint{1.695948in}{0.815760in}}%
\pgfpathlineto{\pgfqpoint{1.696770in}{0.809600in}}%
\pgfpathlineto{\pgfqpoint{1.698415in}{0.821920in}}%
\pgfpathlineto{\pgfqpoint{1.699238in}{0.809600in}}%
\pgfpathlineto{\pgfqpoint{1.701706in}{0.815760in}}%
\pgfpathlineto{\pgfqpoint{1.704996in}{0.778800in}}%
\pgfpathlineto{\pgfqpoint{1.709109in}{0.735680in}}%
\pgfpathlineto{\pgfqpoint{1.709932in}{0.711040in}}%
\pgfpathlineto{\pgfqpoint{1.710754in}{0.717200in}}%
\pgfpathlineto{\pgfqpoint{1.714045in}{0.692560in}}%
\pgfpathlineto{\pgfqpoint{1.714867in}{0.692560in}}%
\pgfpathlineto{\pgfqpoint{1.715690in}{0.686400in}}%
\pgfpathlineto{\pgfqpoint{1.716513in}{0.698720in}}%
\pgfpathlineto{\pgfqpoint{1.719803in}{0.674080in}}%
\pgfpathlineto{\pgfqpoint{1.722271in}{0.686400in}}%
\pgfpathlineto{\pgfqpoint{1.724738in}{0.686400in}}%
\pgfpathlineto{\pgfqpoint{1.726384in}{0.704880in}}%
\pgfpathlineto{\pgfqpoint{1.727206in}{0.704880in}}%
\pgfpathlineto{\pgfqpoint{1.728029in}{0.692560in}}%
\pgfpathlineto{\pgfqpoint{1.730497in}{0.698720in}}%
\pgfpathlineto{\pgfqpoint{1.731319in}{0.704880in}}%
\pgfpathlineto{\pgfqpoint{1.732142in}{0.686400in}}%
\pgfpathlineto{\pgfqpoint{1.732964in}{0.698720in}}%
\pgfpathlineto{\pgfqpoint{1.733787in}{0.686400in}}%
\pgfpathlineto{\pgfqpoint{1.736255in}{0.692560in}}%
\pgfpathlineto{\pgfqpoint{1.737077in}{0.698720in}}%
\pgfpathlineto{\pgfqpoint{1.737900in}{0.692560in}}%
\pgfpathlineto{\pgfqpoint{1.739545in}{0.698720in}}%
\pgfpathlineto{\pgfqpoint{1.742013in}{0.698720in}}%
\pgfpathlineto{\pgfqpoint{1.745303in}{0.723360in}}%
\pgfpathlineto{\pgfqpoint{1.747771in}{0.717200in}}%
\pgfpathlineto{\pgfqpoint{1.750239in}{0.729520in}}%
\pgfpathlineto{\pgfqpoint{1.753529in}{0.729520in}}%
\pgfpathlineto{\pgfqpoint{1.755174in}{0.741840in}}%
\pgfpathlineto{\pgfqpoint{1.760110in}{0.778800in}}%
\pgfpathlineto{\pgfqpoint{1.761755in}{0.766480in}}%
\pgfpathlineto{\pgfqpoint{1.762578in}{0.797280in}}%
\pgfpathlineto{\pgfqpoint{1.765868in}{0.766480in}}%
\pgfpathlineto{\pgfqpoint{1.767513in}{0.766480in}}%
\pgfpathlineto{\pgfqpoint{1.768336in}{0.748000in}}%
\pgfpathlineto{\pgfqpoint{1.772449in}{0.754160in}}%
\pgfpathlineto{\pgfqpoint{1.774094in}{0.729520in}}%
\pgfpathlineto{\pgfqpoint{1.777384in}{0.698720in}}%
\pgfpathlineto{\pgfqpoint{1.779030in}{0.717200in}}%
\pgfpathlineto{\pgfqpoint{1.784788in}{0.674080in}}%
\pgfpathlineto{\pgfqpoint{1.785610in}{0.686400in}}%
\pgfpathlineto{\pgfqpoint{1.790546in}{0.667920in}}%
\pgfpathlineto{\pgfqpoint{1.791368in}{0.655600in}}%
\pgfpathlineto{\pgfqpoint{1.793836in}{0.661760in}}%
\pgfpathlineto{\pgfqpoint{1.794659in}{0.655600in}}%
\pgfpathlineto{\pgfqpoint{1.795481in}{0.661760in}}%
\pgfpathlineto{\pgfqpoint{1.796304in}{0.655600in}}%
\pgfpathlineto{\pgfqpoint{1.797127in}{0.661760in}}%
\pgfpathlineto{\pgfqpoint{1.799594in}{0.655600in}}%
\pgfpathlineto{\pgfqpoint{1.801240in}{0.667920in}}%
\pgfpathlineto{\pgfqpoint{1.802885in}{0.655600in}}%
\pgfpathlineto{\pgfqpoint{1.805353in}{0.649440in}}%
\pgfpathlineto{\pgfqpoint{1.806175in}{0.655600in}}%
\pgfpathlineto{\pgfqpoint{1.807820in}{0.643280in}}%
\pgfpathlineto{\pgfqpoint{1.808643in}{0.649440in}}%
\pgfpathlineto{\pgfqpoint{1.811111in}{0.637120in}}%
\pgfpathlineto{\pgfqpoint{1.812756in}{0.649440in}}%
\pgfpathlineto{\pgfqpoint{1.814401in}{0.637120in}}%
\pgfpathlineto{\pgfqpoint{1.817691in}{0.630960in}}%
\pgfpathlineto{\pgfqpoint{1.818514in}{0.637120in}}%
\pgfpathlineto{\pgfqpoint{1.820159in}{0.630960in}}%
\pgfpathlineto{\pgfqpoint{1.822627in}{0.630960in}}%
\pgfpathlineto{\pgfqpoint{1.823450in}{0.667920in}}%
\pgfpathlineto{\pgfqpoint{1.824272in}{0.649440in}}%
\pgfpathlineto{\pgfqpoint{1.825095in}{0.655600in}}%
\pgfpathlineto{\pgfqpoint{1.825917in}{0.667920in}}%
\pgfpathlineto{\pgfqpoint{1.828385in}{0.661760in}}%
\pgfpathlineto{\pgfqpoint{1.829208in}{0.655600in}}%
\pgfpathlineto{\pgfqpoint{1.830853in}{0.680240in}}%
\pgfpathlineto{\pgfqpoint{1.831676in}{0.674080in}}%
\pgfpathlineto{\pgfqpoint{1.834966in}{0.692560in}}%
\pgfpathlineto{\pgfqpoint{1.837434in}{0.717200in}}%
\pgfpathlineto{\pgfqpoint{1.839901in}{0.698720in}}%
\pgfpathlineto{\pgfqpoint{1.841547in}{0.711040in}}%
\pgfpathlineto{\pgfqpoint{1.843192in}{0.692560in}}%
\pgfpathlineto{\pgfqpoint{1.847305in}{0.704880in}}%
\pgfpathlineto{\pgfqpoint{1.848950in}{0.717200in}}%
\pgfpathlineto{\pgfqpoint{1.853063in}{0.704880in}}%
\pgfpathlineto{\pgfqpoint{1.853886in}{0.717200in}}%
\pgfpathlineto{\pgfqpoint{1.857998in}{0.729520in}}%
\pgfpathlineto{\pgfqpoint{1.858821in}{0.723360in}}%
\pgfpathlineto{\pgfqpoint{1.860466in}{0.735680in}}%
\pgfpathlineto{\pgfqpoint{1.862934in}{0.735680in}}%
\pgfpathlineto{\pgfqpoint{1.866224in}{0.754160in}}%
\pgfpathlineto{\pgfqpoint{1.868692in}{0.754160in}}%
\pgfpathlineto{\pgfqpoint{1.876918in}{0.809600in}}%
\pgfpathlineto{\pgfqpoint{1.881031in}{0.809600in}}%
\pgfpathlineto{\pgfqpoint{1.882676in}{0.828080in}}%
\pgfpathlineto{\pgfqpoint{1.885967in}{0.828080in}}%
\pgfpathlineto{\pgfqpoint{1.886789in}{0.846560in}}%
\pgfpathlineto{\pgfqpoint{1.887612in}{0.895840in}}%
\pgfpathlineto{\pgfqpoint{1.889257in}{0.846560in}}%
\pgfpathlineto{\pgfqpoint{1.892547in}{0.791120in}}%
\pgfpathlineto{\pgfqpoint{1.895015in}{0.815760in}}%
\pgfpathlineto{\pgfqpoint{1.898306in}{0.809600in}}%
\pgfpathlineto{\pgfqpoint{1.899951in}{0.908160in}}%
\pgfpathlineto{\pgfqpoint{1.900773in}{0.871200in}}%
\pgfpathlineto{\pgfqpoint{1.903241in}{0.858880in}}%
\pgfpathlineto{\pgfqpoint{1.904064in}{0.846560in}}%
\pgfpathlineto{\pgfqpoint{1.906531in}{0.883520in}}%
\pgfpathlineto{\pgfqpoint{1.909822in}{0.914320in}}%
\pgfpathlineto{\pgfqpoint{1.912290in}{1.012880in}}%
\pgfpathlineto{\pgfqpoint{1.915580in}{0.957440in}}%
\pgfpathlineto{\pgfqpoint{1.916403in}{0.902000in}}%
\pgfpathlineto{\pgfqpoint{1.917225in}{0.926640in}}%
\pgfpathlineto{\pgfqpoint{1.918048in}{0.914320in}}%
\pgfpathlineto{\pgfqpoint{1.921338in}{0.852720in}}%
\pgfpathlineto{\pgfqpoint{1.922983in}{0.883520in}}%
\pgfpathlineto{\pgfqpoint{1.927096in}{0.778800in}}%
\pgfpathlineto{\pgfqpoint{1.929564in}{0.778800in}}%
\pgfpathlineto{\pgfqpoint{1.932032in}{0.766480in}}%
\pgfpathlineto{\pgfqpoint{1.933677in}{0.778800in}}%
\pgfpathlineto{\pgfqpoint{1.935322in}{0.766480in}}%
\pgfpathlineto{\pgfqpoint{1.937790in}{0.760320in}}%
\pgfpathlineto{\pgfqpoint{1.939435in}{0.772640in}}%
\pgfpathlineto{\pgfqpoint{1.940258in}{0.760320in}}%
\pgfpathlineto{\pgfqpoint{1.941080in}{0.766480in}}%
\pgfpathlineto{\pgfqpoint{1.944371in}{0.735680in}}%
\pgfpathlineto{\pgfqpoint{1.946016in}{0.797280in}}%
\pgfpathlineto{\pgfqpoint{1.946839in}{0.834240in}}%
\pgfpathlineto{\pgfqpoint{1.950129in}{0.772640in}}%
\pgfpathlineto{\pgfqpoint{1.951774in}{0.815760in}}%
\pgfpathlineto{\pgfqpoint{1.952597in}{0.834240in}}%
\pgfpathlineto{\pgfqpoint{1.955064in}{0.821920in}}%
\pgfpathlineto{\pgfqpoint{1.955887in}{0.784960in}}%
\pgfpathlineto{\pgfqpoint{1.956710in}{0.791120in}}%
\pgfpathlineto{\pgfqpoint{1.958355in}{1.136080in}}%
\pgfpathlineto{\pgfqpoint{1.960823in}{1.025200in}}%
\pgfpathlineto{\pgfqpoint{1.961645in}{1.068320in}}%
\pgfpathlineto{\pgfqpoint{1.964113in}{1.579600in}}%
\pgfpathlineto{\pgfqpoint{1.966581in}{1.930720in}}%
\pgfpathlineto{\pgfqpoint{1.968226in}{1.622720in}}%
\pgfpathlineto{\pgfqpoint{1.969871in}{1.290080in}}%
\pgfpathlineto{\pgfqpoint{1.973161in}{1.197680in}}%
\pgfpathlineto{\pgfqpoint{1.974807in}{1.573440in}}%
\pgfpathlineto{\pgfqpoint{1.978920in}{1.271600in}}%
\pgfpathlineto{\pgfqpoint{1.981387in}{1.524160in}}%
\pgfpathlineto{\pgfqpoint{1.983855in}{1.573440in}}%
\pgfpathlineto{\pgfqpoint{1.984678in}{1.487200in}}%
\pgfpathlineto{\pgfqpoint{1.985500in}{1.530320in}}%
\pgfpathlineto{\pgfqpoint{1.986323in}{1.499520in}}%
\pgfpathlineto{\pgfqpoint{1.987146in}{1.518000in}}%
\pgfpathlineto{\pgfqpoint{1.989613in}{1.394800in}}%
\pgfpathlineto{\pgfqpoint{1.990436in}{1.474880in}}%
\pgfpathlineto{\pgfqpoint{1.991259in}{1.308560in}}%
\pgfpathlineto{\pgfqpoint{1.992904in}{1.400960in}}%
\pgfpathlineto{\pgfqpoint{1.995371in}{1.327040in}}%
\pgfpathlineto{\pgfqpoint{1.997839in}{1.431760in}}%
\pgfpathlineto{\pgfqpoint{1.998662in}{1.370160in}}%
\pgfpathlineto{\pgfqpoint{2.001952in}{1.290080in}}%
\pgfpathlineto{\pgfqpoint{2.002775in}{1.290080in}}%
\pgfpathlineto{\pgfqpoint{2.003597in}{1.296240in}}%
\pgfpathlineto{\pgfqpoint{2.004420in}{1.271600in}}%
\pgfpathlineto{\pgfqpoint{2.007710in}{1.234640in}}%
\pgfpathlineto{\pgfqpoint{2.008533in}{1.240800in}}%
\pgfpathlineto{\pgfqpoint{2.012646in}{1.068320in}}%
\pgfpathlineto{\pgfqpoint{2.013469in}{1.099120in}}%
\pgfpathlineto{\pgfqpoint{2.015114in}{1.376320in}}%
\pgfpathlineto{\pgfqpoint{2.018404in}{1.173040in}}%
\pgfpathlineto{\pgfqpoint{2.019227in}{1.179200in}}%
\pgfpathlineto{\pgfqpoint{2.020049in}{1.253120in}}%
\pgfpathlineto{\pgfqpoint{2.021694in}{1.129920in}}%
\pgfpathlineto{\pgfqpoint{2.024985in}{1.086800in}}%
\pgfpathlineto{\pgfqpoint{2.025807in}{1.086800in}}%
\pgfpathlineto{\pgfqpoint{2.027453in}{1.271600in}}%
\pgfpathlineto{\pgfqpoint{2.030743in}{1.203840in}}%
\pgfpathlineto{\pgfqpoint{2.033211in}{1.481040in}}%
\pgfpathlineto{\pgfqpoint{2.036501in}{1.339360in}}%
\pgfpathlineto{\pgfqpoint{2.038146in}{1.474880in}}%
\pgfpathlineto{\pgfqpoint{2.038969in}{1.444080in}}%
\pgfpathlineto{\pgfqpoint{2.041437in}{1.462560in}}%
\pgfpathlineto{\pgfqpoint{2.043082in}{1.665840in}}%
\pgfpathlineto{\pgfqpoint{2.044727in}{1.604240in}}%
\pgfpathlineto{\pgfqpoint{2.048840in}{1.733600in}}%
\pgfpathlineto{\pgfqpoint{2.049663in}{1.795200in}}%
\pgfpathlineto{\pgfqpoint{2.050485in}{1.702800in}}%
\pgfpathlineto{\pgfqpoint{2.053776in}{1.764400in}}%
\pgfpathlineto{\pgfqpoint{2.054598in}{1.764400in}}%
\pgfpathlineto{\pgfqpoint{2.056243in}{1.733600in}}%
\pgfpathlineto{\pgfqpoint{2.058711in}{1.764400in}}%
\pgfpathlineto{\pgfqpoint{2.060356in}{1.826000in}}%
\pgfpathlineto{\pgfqpoint{2.064469in}{1.628880in}}%
\pgfpathlineto{\pgfqpoint{2.065292in}{1.641200in}}%
\pgfpathlineto{\pgfqpoint{2.066114in}{1.708960in}}%
\pgfpathlineto{\pgfqpoint{2.066937in}{1.690480in}}%
\pgfpathlineto{\pgfqpoint{2.070227in}{1.413280in}}%
\pgfpathlineto{\pgfqpoint{2.072695in}{1.511840in}}%
\pgfpathlineto{\pgfqpoint{2.075986in}{1.308560in}}%
\pgfpathlineto{\pgfqpoint{2.077631in}{1.364000in}}%
\pgfpathlineto{\pgfqpoint{2.078453in}{1.357840in}}%
\pgfpathlineto{\pgfqpoint{2.079276in}{1.364000in}}%
\pgfpathlineto{\pgfqpoint{2.084212in}{1.191520in}}%
\pgfpathlineto{\pgfqpoint{2.085034in}{1.203840in}}%
\pgfpathlineto{\pgfqpoint{2.088324in}{1.000560in}}%
\pgfpathlineto{\pgfqpoint{2.089147in}{0.988240in}}%
\pgfpathlineto{\pgfqpoint{2.089970in}{1.012880in}}%
\pgfpathlineto{\pgfqpoint{2.090792in}{1.117600in}}%
\pgfpathlineto{\pgfqpoint{2.094083in}{1.314720in}}%
\pgfpathlineto{\pgfqpoint{2.095728in}{1.006720in}}%
\pgfpathlineto{\pgfqpoint{2.096550in}{1.092960in}}%
\pgfpathlineto{\pgfqpoint{2.099018in}{1.031360in}}%
\pgfpathlineto{\pgfqpoint{2.100663in}{1.105280in}}%
\pgfpathlineto{\pgfqpoint{2.101486in}{1.173040in}}%
\pgfpathlineto{\pgfqpoint{2.102309in}{1.080640in}}%
\pgfpathlineto{\pgfqpoint{2.104776in}{1.006720in}}%
\pgfpathlineto{\pgfqpoint{2.106422in}{1.105280in}}%
\pgfpathlineto{\pgfqpoint{2.108067in}{1.000560in}}%
\pgfpathlineto{\pgfqpoint{2.111357in}{0.945120in}}%
\pgfpathlineto{\pgfqpoint{2.112180in}{0.957440in}}%
\pgfpathlineto{\pgfqpoint{2.113002in}{0.938960in}}%
\pgfpathlineto{\pgfqpoint{2.113825in}{1.000560in}}%
\pgfpathlineto{\pgfqpoint{2.117115in}{0.957440in}}%
\pgfpathlineto{\pgfqpoint{2.118760in}{1.019040in}}%
\pgfpathlineto{\pgfqpoint{2.119583in}{1.012880in}}%
\pgfpathlineto{\pgfqpoint{2.122051in}{1.019040in}}%
\pgfpathlineto{\pgfqpoint{2.123696in}{1.142240in}}%
\pgfpathlineto{\pgfqpoint{2.125341in}{1.210000in}}%
\pgfpathlineto{\pgfqpoint{2.128632in}{1.290080in}}%
\pgfpathlineto{\pgfqpoint{2.130277in}{1.437920in}}%
\pgfpathlineto{\pgfqpoint{2.131099in}{1.376320in}}%
\pgfpathlineto{\pgfqpoint{2.133567in}{1.407120in}}%
\pgfpathlineto{\pgfqpoint{2.135212in}{1.302400in}}%
\pgfpathlineto{\pgfqpoint{2.136857in}{1.425600in}}%
\pgfpathlineto{\pgfqpoint{2.139325in}{1.357840in}}%
\pgfpathlineto{\pgfqpoint{2.140148in}{1.444080in}}%
\pgfpathlineto{\pgfqpoint{2.140970in}{1.672000in}}%
\pgfpathlineto{\pgfqpoint{2.141793in}{1.665840in}}%
\pgfpathlineto{\pgfqpoint{2.145906in}{1.290080in}}%
\pgfpathlineto{\pgfqpoint{2.146729in}{1.302400in}}%
\pgfpathlineto{\pgfqpoint{2.147551in}{1.314720in}}%
\pgfpathlineto{\pgfqpoint{2.151664in}{1.240800in}}%
\pgfpathlineto{\pgfqpoint{2.152487in}{1.246960in}}%
\pgfpathlineto{\pgfqpoint{2.154132in}{1.302400in}}%
\pgfpathlineto{\pgfqpoint{2.156600in}{1.228480in}}%
\pgfpathlineto{\pgfqpoint{2.157422in}{1.253120in}}%
\pgfpathlineto{\pgfqpoint{2.159890in}{1.388640in}}%
\pgfpathlineto{\pgfqpoint{2.162358in}{1.450240in}}%
\pgfpathlineto{\pgfqpoint{2.163180in}{1.444080in}}%
\pgfpathlineto{\pgfqpoint{2.165648in}{1.419440in}}%
\pgfpathlineto{\pgfqpoint{2.168116in}{1.419440in}}%
\pgfpathlineto{\pgfqpoint{2.169761in}{1.474880in}}%
\pgfpathlineto{\pgfqpoint{2.170584in}{1.468720in}}%
\pgfpathlineto{\pgfqpoint{2.173874in}{1.351680in}}%
\pgfpathlineto{\pgfqpoint{2.177165in}{1.240800in}}%
\pgfpathlineto{\pgfqpoint{2.186213in}{0.982080in}}%
\pgfpathlineto{\pgfqpoint{2.187036in}{1.012880in}}%
\pgfpathlineto{\pgfqpoint{2.187858in}{1.006720in}}%
\pgfpathlineto{\pgfqpoint{2.188681in}{0.982080in}}%
\pgfpathlineto{\pgfqpoint{2.191149in}{0.988240in}}%
\pgfpathlineto{\pgfqpoint{2.192794in}{0.938960in}}%
\pgfpathlineto{\pgfqpoint{2.193616in}{0.938960in}}%
\pgfpathlineto{\pgfqpoint{2.194439in}{0.951280in}}%
\pgfpathlineto{\pgfqpoint{2.197729in}{0.920480in}}%
\pgfpathlineto{\pgfqpoint{2.199375in}{0.938960in}}%
\pgfpathlineto{\pgfqpoint{2.200197in}{0.951280in}}%
\pgfpathlineto{\pgfqpoint{2.202665in}{0.969760in}}%
\pgfpathlineto{\pgfqpoint{2.203487in}{0.963600in}}%
\pgfpathlineto{\pgfqpoint{2.205955in}{0.988240in}}%
\pgfpathlineto{\pgfqpoint{2.209246in}{0.938960in}}%
\pgfpathlineto{\pgfqpoint{2.210068in}{0.975920in}}%
\pgfpathlineto{\pgfqpoint{2.210891in}{0.945120in}}%
\pgfpathlineto{\pgfqpoint{2.211713in}{0.969760in}}%
\pgfpathlineto{\pgfqpoint{2.214181in}{0.914320in}}%
\pgfpathlineto{\pgfqpoint{2.216649in}{1.012880in}}%
\pgfpathlineto{\pgfqpoint{2.217472in}{1.037520in}}%
\pgfpathlineto{\pgfqpoint{2.219939in}{1.025200in}}%
\pgfpathlineto{\pgfqpoint{2.223230in}{1.148400in}}%
\pgfpathlineto{\pgfqpoint{2.225697in}{1.000560in}}%
\pgfpathlineto{\pgfqpoint{2.227343in}{1.056000in}}%
\pgfpathlineto{\pgfqpoint{2.228165in}{1.043680in}}%
\pgfpathlineto{\pgfqpoint{2.231456in}{1.025200in}}%
\pgfpathlineto{\pgfqpoint{2.232278in}{1.031360in}}%
\pgfpathlineto{\pgfqpoint{2.233101in}{1.056000in}}%
\pgfpathlineto{\pgfqpoint{2.234746in}{1.185360in}}%
\pgfpathlineto{\pgfqpoint{2.237214in}{1.234640in}}%
\pgfpathlineto{\pgfqpoint{2.238859in}{1.327040in}}%
\pgfpathlineto{\pgfqpoint{2.240504in}{1.265440in}}%
\pgfpathlineto{\pgfqpoint{2.242972in}{1.265440in}}%
\pgfpathlineto{\pgfqpoint{2.246262in}{1.105280in}}%
\pgfpathlineto{\pgfqpoint{2.249553in}{1.123760in}}%
\pgfpathlineto{\pgfqpoint{2.252020in}{1.154560in}}%
\pgfpathlineto{\pgfqpoint{2.255311in}{1.105280in}}%
\pgfpathlineto{\pgfqpoint{2.256133in}{1.173040in}}%
\pgfpathlineto{\pgfqpoint{2.256956in}{1.160720in}}%
\pgfpathlineto{\pgfqpoint{2.260246in}{1.031360in}}%
\pgfpathlineto{\pgfqpoint{2.261892in}{1.043680in}}%
\pgfpathlineto{\pgfqpoint{2.262714in}{1.031360in}}%
\pgfpathlineto{\pgfqpoint{2.263537in}{1.056000in}}%
\pgfpathlineto{\pgfqpoint{2.266005in}{1.068320in}}%
\pgfpathlineto{\pgfqpoint{2.266827in}{1.080640in}}%
\pgfpathlineto{\pgfqpoint{2.267650in}{1.136080in}}%
\pgfpathlineto{\pgfqpoint{2.268472in}{1.129920in}}%
\pgfpathlineto{\pgfqpoint{2.269295in}{1.142240in}}%
\pgfpathlineto{\pgfqpoint{2.272585in}{1.136080in}}%
\pgfpathlineto{\pgfqpoint{2.273408in}{1.160720in}}%
\pgfpathlineto{\pgfqpoint{2.274230in}{1.117600in}}%
\pgfpathlineto{\pgfqpoint{2.275053in}{1.129920in}}%
\pgfpathlineto{\pgfqpoint{2.278343in}{1.129920in}}%
\pgfpathlineto{\pgfqpoint{2.280811in}{1.154560in}}%
\pgfpathlineto{\pgfqpoint{2.283279in}{1.136080in}}%
\pgfpathlineto{\pgfqpoint{2.286569in}{1.277760in}}%
\pgfpathlineto{\pgfqpoint{2.289860in}{1.696640in}}%
\pgfpathlineto{\pgfqpoint{2.290682in}{2.306480in}}%
\pgfpathlineto{\pgfqpoint{2.291505in}{2.269520in}}%
\pgfpathlineto{\pgfqpoint{2.292328in}{1.819840in}}%
\pgfpathlineto{\pgfqpoint{2.294795in}{1.616560in}}%
\pgfpathlineto{\pgfqpoint{2.297263in}{2.300320in}}%
\pgfpathlineto{\pgfqpoint{2.298086in}{2.226400in}}%
\pgfpathlineto{\pgfqpoint{2.300553in}{2.238720in}}%
\pgfpathlineto{\pgfqpoint{2.303844in}{2.799280in}}%
\pgfpathlineto{\pgfqpoint{2.306312in}{2.750000in}}%
\pgfpathlineto{\pgfqpoint{2.307134in}{2.602160in}}%
\pgfpathlineto{\pgfqpoint{2.309602in}{3.261280in}}%
\pgfpathlineto{\pgfqpoint{2.312892in}{3.088800in}}%
\pgfpathlineto{\pgfqpoint{2.313715in}{3.107280in}}%
\pgfpathlineto{\pgfqpoint{2.319473in}{1.986160in}}%
\pgfpathlineto{\pgfqpoint{2.321118in}{2.078560in}}%
\pgfpathlineto{\pgfqpoint{2.323586in}{2.084720in}}%
\pgfpathlineto{\pgfqpoint{2.324409in}{2.103200in}}%
\pgfpathlineto{\pgfqpoint{2.325231in}{2.170960in}}%
\pgfpathlineto{\pgfqpoint{2.326054in}{2.158640in}}%
\pgfpathlineto{\pgfqpoint{2.326876in}{2.035440in}}%
\pgfpathlineto{\pgfqpoint{2.330167in}{1.813680in}}%
\pgfpathlineto{\pgfqpoint{2.332635in}{1.665840in}}%
\pgfpathlineto{\pgfqpoint{2.335102in}{1.635040in}}%
\pgfpathlineto{\pgfqpoint{2.336748in}{1.641200in}}%
\pgfpathlineto{\pgfqpoint{2.337570in}{1.635040in}}%
\pgfpathlineto{\pgfqpoint{2.338393in}{1.733600in}}%
\pgfpathlineto{\pgfqpoint{2.340861in}{1.745920in}}%
\pgfpathlineto{\pgfqpoint{2.342506in}{1.733600in}}%
\pgfpathlineto{\pgfqpoint{2.344151in}{1.758240in}}%
\pgfpathlineto{\pgfqpoint{2.346619in}{1.696640in}}%
\pgfpathlineto{\pgfqpoint{2.349909in}{1.801360in}}%
\pgfpathlineto{\pgfqpoint{2.353199in}{1.764400in}}%
\pgfpathlineto{\pgfqpoint{2.354022in}{1.782880in}}%
\pgfpathlineto{\pgfqpoint{2.354845in}{1.776720in}}%
\pgfpathlineto{\pgfqpoint{2.355667in}{1.776720in}}%
\pgfpathlineto{\pgfqpoint{2.358135in}{1.770560in}}%
\pgfpathlineto{\pgfqpoint{2.359780in}{1.733600in}}%
\pgfpathlineto{\pgfqpoint{2.361425in}{1.610400in}}%
\pgfpathlineto{\pgfqpoint{2.364716in}{1.554960in}}%
\pgfpathlineto{\pgfqpoint{2.366361in}{1.376320in}}%
\pgfpathlineto{\pgfqpoint{2.367183in}{1.351680in}}%
\pgfpathlineto{\pgfqpoint{2.370474in}{1.333200in}}%
\pgfpathlineto{\pgfqpoint{2.371296in}{1.345520in}}%
\pgfpathlineto{\pgfqpoint{2.375409in}{1.302400in}}%
\pgfpathlineto{\pgfqpoint{2.377055in}{1.253120in}}%
\pgfpathlineto{\pgfqpoint{2.378700in}{1.259280in}}%
\pgfpathlineto{\pgfqpoint{2.381990in}{1.222320in}}%
\pgfpathlineto{\pgfqpoint{2.382813in}{1.234640in}}%
\pgfpathlineto{\pgfqpoint{2.384458in}{1.173040in}}%
\pgfpathlineto{\pgfqpoint{2.388571in}{1.031360in}}%
\pgfpathlineto{\pgfqpoint{2.390216in}{1.068320in}}%
\pgfpathlineto{\pgfqpoint{2.393506in}{1.049840in}}%
\pgfpathlineto{\pgfqpoint{2.394329in}{1.062160in}}%
\pgfpathlineto{\pgfqpoint{2.395974in}{1.092960in}}%
\pgfpathlineto{\pgfqpoint{2.398442in}{1.074480in}}%
\pgfpathlineto{\pgfqpoint{2.399265in}{1.086800in}}%
\pgfpathlineto{\pgfqpoint{2.400910in}{1.019040in}}%
\pgfpathlineto{\pgfqpoint{2.401732in}{1.019040in}}%
\pgfpathlineto{\pgfqpoint{2.404200in}{1.031360in}}%
\pgfpathlineto{\pgfqpoint{2.405023in}{1.000560in}}%
\pgfpathlineto{\pgfqpoint{2.407491in}{1.031360in}}%
\pgfpathlineto{\pgfqpoint{2.410781in}{1.000560in}}%
\pgfpathlineto{\pgfqpoint{2.413249in}{1.025200in}}%
\pgfpathlineto{\pgfqpoint{2.416539in}{1.012880in}}%
\pgfpathlineto{\pgfqpoint{2.418184in}{1.025200in}}%
\pgfpathlineto{\pgfqpoint{2.419007in}{1.043680in}}%
\pgfpathlineto{\pgfqpoint{2.422297in}{1.012880in}}%
\pgfpathlineto{\pgfqpoint{2.424765in}{1.056000in}}%
\pgfpathlineto{\pgfqpoint{2.427233in}{1.049840in}}%
\pgfpathlineto{\pgfqpoint{2.428878in}{1.068320in}}%
\pgfpathlineto{\pgfqpoint{2.430523in}{1.111440in}}%
\pgfpathlineto{\pgfqpoint{2.432991in}{1.105280in}}%
\pgfpathlineto{\pgfqpoint{2.436281in}{1.136080in}}%
\pgfpathlineto{\pgfqpoint{2.439572in}{1.092960in}}%
\pgfpathlineto{\pgfqpoint{2.440394in}{1.105280in}}%
\pgfpathlineto{\pgfqpoint{2.442039in}{1.062160in}}%
\pgfpathlineto{\pgfqpoint{2.444507in}{1.056000in}}%
\pgfpathlineto{\pgfqpoint{2.447798in}{1.111440in}}%
\pgfpathlineto{\pgfqpoint{2.450265in}{1.074480in}}%
\pgfpathlineto{\pgfqpoint{2.451911in}{1.031360in}}%
\pgfpathlineto{\pgfqpoint{2.453556in}{1.025200in}}%
\pgfpathlineto{\pgfqpoint{2.456024in}{1.031360in}}%
\pgfpathlineto{\pgfqpoint{2.456846in}{1.025200in}}%
\pgfpathlineto{\pgfqpoint{2.457669in}{1.031360in}}%
\pgfpathlineto{\pgfqpoint{2.458491in}{1.025200in}}%
\pgfpathlineto{\pgfqpoint{2.462604in}{1.025200in}}%
\pgfpathlineto{\pgfqpoint{2.464249in}{1.043680in}}%
\pgfpathlineto{\pgfqpoint{2.465072in}{1.031360in}}%
\pgfpathlineto{\pgfqpoint{2.467540in}{1.031360in}}%
\pgfpathlineto{\pgfqpoint{2.468362in}{1.025200in}}%
\pgfpathlineto{\pgfqpoint{2.469185in}{1.031360in}}%
\pgfpathlineto{\pgfqpoint{2.470008in}{1.049840in}}%
\pgfpathlineto{\pgfqpoint{2.474121in}{1.000560in}}%
\pgfpathlineto{\pgfqpoint{2.474943in}{1.006720in}}%
\pgfpathlineto{\pgfqpoint{2.476588in}{0.969760in}}%
\pgfpathlineto{\pgfqpoint{2.484814in}{0.889680in}}%
\pgfpathlineto{\pgfqpoint{2.487282in}{0.865040in}}%
\pgfpathlineto{\pgfqpoint{2.493040in}{0.729520in}}%
\pgfpathlineto{\pgfqpoint{2.493863in}{0.735680in}}%
\pgfpathlineto{\pgfqpoint{2.497153in}{0.735680in}}%
\pgfpathlineto{\pgfqpoint{2.499621in}{0.760320in}}%
\pgfpathlineto{\pgfqpoint{2.502089in}{0.760320in}}%
\pgfpathlineto{\pgfqpoint{2.504556in}{0.741840in}}%
\pgfpathlineto{\pgfqpoint{2.505379in}{0.711040in}}%
\pgfpathlineto{\pgfqpoint{2.508669in}{0.723360in}}%
\pgfpathlineto{\pgfqpoint{2.509492in}{0.723360in}}%
\pgfpathlineto{\pgfqpoint{2.511137in}{0.711040in}}%
\pgfpathlineto{\pgfqpoint{2.515250in}{0.711040in}}%
\pgfpathlineto{\pgfqpoint{2.516895in}{0.704880in}}%
\pgfpathlineto{\pgfqpoint{2.520186in}{0.692560in}}%
\pgfpathlineto{\pgfqpoint{2.521008in}{0.692560in}}%
\pgfpathlineto{\pgfqpoint{2.521831in}{0.698720in}}%
\pgfpathlineto{\pgfqpoint{2.522654in}{0.692560in}}%
\pgfpathlineto{\pgfqpoint{2.525944in}{0.686400in}}%
\pgfpathlineto{\pgfqpoint{2.526766in}{0.698720in}}%
\pgfpathlineto{\pgfqpoint{2.532525in}{0.649440in}}%
\pgfpathlineto{\pgfqpoint{2.534170in}{0.637120in}}%
\pgfpathlineto{\pgfqpoint{2.537460in}{0.643280in}}%
\pgfpathlineto{\pgfqpoint{2.538283in}{0.643280in}}%
\pgfpathlineto{\pgfqpoint{2.539105in}{0.637120in}}%
\pgfpathlineto{\pgfqpoint{2.539928in}{0.643280in}}%
\pgfpathlineto{\pgfqpoint{2.542396in}{0.637120in}}%
\pgfpathlineto{\pgfqpoint{2.544041in}{0.630960in}}%
\pgfpathlineto{\pgfqpoint{2.544864in}{0.630960in}}%
\pgfpathlineto{\pgfqpoint{2.545686in}{0.637120in}}%
\pgfpathlineto{\pgfqpoint{2.548154in}{0.630960in}}%
\pgfpathlineto{\pgfqpoint{2.549799in}{0.624800in}}%
\pgfpathlineto{\pgfqpoint{2.551444in}{0.624800in}}%
\pgfpathlineto{\pgfqpoint{2.553912in}{0.612480in}}%
\pgfpathlineto{\pgfqpoint{2.555557in}{0.618640in}}%
\pgfpathlineto{\pgfqpoint{2.556380in}{0.618640in}}%
\pgfpathlineto{\pgfqpoint{2.557202in}{0.612480in}}%
\pgfpathlineto{\pgfqpoint{2.560493in}{0.606320in}}%
\pgfpathlineto{\pgfqpoint{2.561315in}{0.612480in}}%
\pgfpathlineto{\pgfqpoint{2.562961in}{0.594000in}}%
\pgfpathlineto{\pgfqpoint{2.565428in}{0.594000in}}%
\pgfpathlineto{\pgfqpoint{2.566251in}{0.587840in}}%
\pgfpathlineto{\pgfqpoint{2.567074in}{0.594000in}}%
\pgfpathlineto{\pgfqpoint{2.568719in}{0.575520in}}%
\pgfpathlineto{\pgfqpoint{2.571187in}{0.575520in}}%
\pgfpathlineto{\pgfqpoint{2.572009in}{0.569360in}}%
\pgfpathlineto{\pgfqpoint{2.572832in}{0.575520in}}%
\pgfpathlineto{\pgfqpoint{2.574477in}{0.563200in}}%
\pgfpathlineto{\pgfqpoint{2.578590in}{0.557040in}}%
\pgfpathlineto{\pgfqpoint{2.580235in}{0.544720in}}%
\pgfpathlineto{\pgfqpoint{2.584348in}{0.538560in}}%
\pgfpathlineto{\pgfqpoint{2.585993in}{0.538560in}}%
\pgfpathlineto{\pgfqpoint{2.588461in}{0.532400in}}%
\pgfpathlineto{\pgfqpoint{2.589284in}{0.538560in}}%
\pgfpathlineto{\pgfqpoint{2.591751in}{0.569360in}}%
\pgfpathlineto{\pgfqpoint{2.595042in}{0.550880in}}%
\pgfpathlineto{\pgfqpoint{2.597509in}{0.550880in}}%
\pgfpathlineto{\pgfqpoint{2.599977in}{0.544720in}}%
\pgfpathlineto{\pgfqpoint{2.601622in}{0.532400in}}%
\pgfpathlineto{\pgfqpoint{2.603268in}{0.550880in}}%
\pgfpathlineto{\pgfqpoint{2.606558in}{0.563200in}}%
\pgfpathlineto{\pgfqpoint{2.607381in}{0.557040in}}%
\pgfpathlineto{\pgfqpoint{2.608203in}{0.575520in}}%
\pgfpathlineto{\pgfqpoint{2.609026in}{0.569360in}}%
\pgfpathlineto{\pgfqpoint{2.614784in}{0.569360in}}%
\pgfpathlineto{\pgfqpoint{2.618074in}{0.563200in}}%
\pgfpathlineto{\pgfqpoint{2.619719in}{0.575520in}}%
\pgfpathlineto{\pgfqpoint{2.620542in}{0.569360in}}%
\pgfpathlineto{\pgfqpoint{2.623832in}{0.563200in}}%
\pgfpathlineto{\pgfqpoint{2.626300in}{0.581680in}}%
\pgfpathlineto{\pgfqpoint{2.629591in}{0.575520in}}%
\pgfpathlineto{\pgfqpoint{2.631236in}{0.587840in}}%
\pgfpathlineto{\pgfqpoint{2.632058in}{0.569360in}}%
\pgfpathlineto{\pgfqpoint{2.635349in}{0.563200in}}%
\pgfpathlineto{\pgfqpoint{2.636994in}{0.563200in}}%
\pgfpathlineto{\pgfqpoint{2.637817in}{0.569360in}}%
\pgfpathlineto{\pgfqpoint{2.640284in}{0.563200in}}%
\pgfpathlineto{\pgfqpoint{2.641929in}{0.581680in}}%
\pgfpathlineto{\pgfqpoint{2.642752in}{0.594000in}}%
\pgfpathlineto{\pgfqpoint{2.646865in}{0.569360in}}%
\pgfpathlineto{\pgfqpoint{2.647688in}{0.569360in}}%
\pgfpathlineto{\pgfqpoint{2.649333in}{0.581680in}}%
\pgfpathlineto{\pgfqpoint{2.652623in}{0.563200in}}%
\pgfpathlineto{\pgfqpoint{2.655091in}{0.563200in}}%
\pgfpathlineto{\pgfqpoint{2.659204in}{0.581680in}}%
\pgfpathlineto{\pgfqpoint{2.660027in}{0.581680in}}%
\pgfpathlineto{\pgfqpoint{2.660849in}{0.575520in}}%
\pgfpathlineto{\pgfqpoint{2.663317in}{0.569360in}}%
\pgfpathlineto{\pgfqpoint{2.664139in}{0.563200in}}%
\pgfpathlineto{\pgfqpoint{2.665785in}{0.569360in}}%
\pgfpathlineto{\pgfqpoint{2.669898in}{0.544720in}}%
\pgfpathlineto{\pgfqpoint{2.671543in}{0.563200in}}%
\pgfpathlineto{\pgfqpoint{2.675656in}{0.532400in}}%
\pgfpathlineto{\pgfqpoint{2.676478in}{0.563200in}}%
\pgfpathlineto{\pgfqpoint{2.677301in}{0.557040in}}%
\pgfpathlineto{\pgfqpoint{2.680591in}{0.544720in}}%
\pgfpathlineto{\pgfqpoint{2.683882in}{0.563200in}}%
\pgfpathlineto{\pgfqpoint{2.686350in}{0.569360in}}%
\pgfpathlineto{\pgfqpoint{2.687995in}{0.557040in}}%
\pgfpathlineto{\pgfqpoint{2.688817in}{0.563200in}}%
\pgfpathlineto{\pgfqpoint{2.689640in}{0.557040in}}%
\pgfpathlineto{\pgfqpoint{2.692930in}{0.557040in}}%
\pgfpathlineto{\pgfqpoint{2.693753in}{0.563200in}}%
\pgfpathlineto{\pgfqpoint{2.695398in}{0.557040in}}%
\pgfpathlineto{\pgfqpoint{2.697866in}{0.557040in}}%
\pgfpathlineto{\pgfqpoint{2.700334in}{0.544720in}}%
\pgfpathlineto{\pgfqpoint{2.701156in}{0.544720in}}%
\pgfpathlineto{\pgfqpoint{2.704447in}{0.532400in}}%
\pgfpathlineto{\pgfqpoint{2.705269in}{0.532400in}}%
\pgfpathlineto{\pgfqpoint{2.706092in}{0.538560in}}%
\pgfpathlineto{\pgfqpoint{2.709382in}{0.520080in}}%
\pgfpathlineto{\pgfqpoint{2.710205in}{0.520080in}}%
\pgfpathlineto{\pgfqpoint{2.712672in}{0.532400in}}%
\pgfpathlineto{\pgfqpoint{2.717608in}{0.532400in}}%
\pgfpathlineto{\pgfqpoint{2.718431in}{0.526240in}}%
\pgfpathlineto{\pgfqpoint{2.720898in}{0.526240in}}%
\pgfpathlineto{\pgfqpoint{2.723366in}{0.513920in}}%
\pgfpathlineto{\pgfqpoint{2.726657in}{0.513920in}}%
\pgfpathlineto{\pgfqpoint{2.728302in}{0.507760in}}%
\pgfpathlineto{\pgfqpoint{2.729124in}{0.507760in}}%
\pgfpathlineto{\pgfqpoint{2.729947in}{0.501600in}}%
\pgfpathlineto{\pgfqpoint{2.732415in}{0.495440in}}%
\pgfpathlineto{\pgfqpoint{2.734060in}{0.507760in}}%
\pgfpathlineto{\pgfqpoint{2.734882in}{0.501600in}}%
\pgfpathlineto{\pgfqpoint{2.735705in}{0.507760in}}%
\pgfpathlineto{\pgfqpoint{2.738173in}{0.495440in}}%
\pgfpathlineto{\pgfqpoint{2.741463in}{0.513920in}}%
\pgfpathlineto{\pgfqpoint{2.743931in}{0.520080in}}%
\pgfpathlineto{\pgfqpoint{2.746399in}{0.532400in}}%
\pgfpathlineto{\pgfqpoint{2.747221in}{0.532400in}}%
\pgfpathlineto{\pgfqpoint{2.749689in}{0.526240in}}%
\pgfpathlineto{\pgfqpoint{2.751334in}{0.520080in}}%
\pgfpathlineto{\pgfqpoint{2.752157in}{0.520080in}}%
\pgfpathlineto{\pgfqpoint{2.756270in}{0.513920in}}%
\pgfpathlineto{\pgfqpoint{2.757092in}{0.520080in}}%
\pgfpathlineto{\pgfqpoint{2.757915in}{0.513920in}}%
\pgfpathlineto{\pgfqpoint{2.758738in}{0.526240in}}%
\pgfpathlineto{\pgfqpoint{2.763673in}{0.526240in}}%
\pgfpathlineto{\pgfqpoint{2.764496in}{0.532400in}}%
\pgfpathlineto{\pgfqpoint{2.767786in}{0.532400in}}%
\pgfpathlineto{\pgfqpoint{2.769431in}{0.538560in}}%
\pgfpathlineto{\pgfqpoint{2.772722in}{0.538560in}}%
\pgfpathlineto{\pgfqpoint{2.774367in}{0.550880in}}%
\pgfpathlineto{\pgfqpoint{2.775190in}{0.544720in}}%
\pgfpathlineto{\pgfqpoint{2.776012in}{0.557040in}}%
\pgfpathlineto{\pgfqpoint{2.779303in}{0.557040in}}%
\pgfpathlineto{\pgfqpoint{2.780125in}{0.569360in}}%
\pgfpathlineto{\pgfqpoint{2.781770in}{0.624800in}}%
\pgfpathlineto{\pgfqpoint{2.785061in}{0.600160in}}%
\pgfpathlineto{\pgfqpoint{2.787528in}{0.618640in}}%
\pgfpathlineto{\pgfqpoint{2.789996in}{0.624800in}}%
\pgfpathlineto{\pgfqpoint{2.790819in}{0.618640in}}%
\pgfpathlineto{\pgfqpoint{2.793287in}{0.643280in}}%
\pgfpathlineto{\pgfqpoint{2.795754in}{0.649440in}}%
\pgfpathlineto{\pgfqpoint{2.797400in}{0.667920in}}%
\pgfpathlineto{\pgfqpoint{2.798222in}{0.667920in}}%
\pgfpathlineto{\pgfqpoint{2.799045in}{0.674080in}}%
\pgfpathlineto{\pgfqpoint{2.803158in}{0.674080in}}%
\pgfpathlineto{\pgfqpoint{2.804803in}{0.686400in}}%
\pgfpathlineto{\pgfqpoint{2.809738in}{0.711040in}}%
\pgfpathlineto{\pgfqpoint{2.810561in}{0.723360in}}%
\pgfpathlineto{\pgfqpoint{2.813029in}{0.729520in}}%
\pgfpathlineto{\pgfqpoint{2.814674in}{0.711040in}}%
\pgfpathlineto{\pgfqpoint{2.815497in}{0.717200in}}%
\pgfpathlineto{\pgfqpoint{2.816319in}{0.704880in}}%
\pgfpathlineto{\pgfqpoint{2.818787in}{0.698720in}}%
\pgfpathlineto{\pgfqpoint{2.820432in}{0.692560in}}%
\pgfpathlineto{\pgfqpoint{2.821255in}{0.692560in}}%
\pgfpathlineto{\pgfqpoint{2.824545in}{0.661760in}}%
\pgfpathlineto{\pgfqpoint{2.825368in}{0.674080in}}%
\pgfpathlineto{\pgfqpoint{2.826190in}{0.655600in}}%
\pgfpathlineto{\pgfqpoint{2.827013in}{0.661760in}}%
\pgfpathlineto{\pgfqpoint{2.831126in}{0.661760in}}%
\pgfpathlineto{\pgfqpoint{2.832771in}{0.674080in}}%
\pgfpathlineto{\pgfqpoint{2.833594in}{0.667920in}}%
\pgfpathlineto{\pgfqpoint{2.836061in}{0.667920in}}%
\pgfpathlineto{\pgfqpoint{2.837707in}{0.674080in}}%
\pgfpathlineto{\pgfqpoint{2.839352in}{0.667920in}}%
\pgfpathlineto{\pgfqpoint{2.842642in}{0.655600in}}%
\pgfpathlineto{\pgfqpoint{2.843465in}{0.655600in}}%
\pgfpathlineto{\pgfqpoint{2.845110in}{0.643280in}}%
\pgfpathlineto{\pgfqpoint{2.848400in}{0.643280in}}%
\pgfpathlineto{\pgfqpoint{2.854981in}{0.600160in}}%
\pgfpathlineto{\pgfqpoint{2.855804in}{0.606320in}}%
\pgfpathlineto{\pgfqpoint{2.856626in}{0.600160in}}%
\pgfpathlineto{\pgfqpoint{2.859917in}{0.594000in}}%
\pgfpathlineto{\pgfqpoint{2.861562in}{0.575520in}}%
\pgfpathlineto{\pgfqpoint{2.862384in}{0.575520in}}%
\pgfpathlineto{\pgfqpoint{2.864852in}{0.563200in}}%
\pgfpathlineto{\pgfqpoint{2.865675in}{0.550880in}}%
\pgfpathlineto{\pgfqpoint{2.866497in}{0.557040in}}%
\pgfpathlineto{\pgfqpoint{2.868143in}{0.550880in}}%
\pgfpathlineto{\pgfqpoint{2.873078in}{0.526240in}}%
\pgfpathlineto{\pgfqpoint{2.873901in}{0.532400in}}%
\pgfpathlineto{\pgfqpoint{2.877191in}{0.538560in}}%
\pgfpathlineto{\pgfqpoint{2.878014in}{0.544720in}}%
\pgfpathlineto{\pgfqpoint{2.879659in}{0.532400in}}%
\pgfpathlineto{\pgfqpoint{2.885417in}{0.532400in}}%
\pgfpathlineto{\pgfqpoint{2.888707in}{0.526240in}}%
\pgfpathlineto{\pgfqpoint{2.889530in}{0.526240in}}%
\pgfpathlineto{\pgfqpoint{2.891175in}{0.520080in}}%
\pgfpathlineto{\pgfqpoint{2.894466in}{0.513920in}}%
\pgfpathlineto{\pgfqpoint{2.896933in}{0.526240in}}%
\pgfpathlineto{\pgfqpoint{2.900224in}{0.520080in}}%
\pgfpathlineto{\pgfqpoint{2.902691in}{0.520080in}}%
\pgfpathlineto{\pgfqpoint{2.905982in}{0.544720in}}%
\pgfpathlineto{\pgfqpoint{2.907627in}{0.538560in}}%
\pgfpathlineto{\pgfqpoint{2.908450in}{0.544720in}}%
\pgfpathlineto{\pgfqpoint{2.912563in}{0.538560in}}%
\pgfpathlineto{\pgfqpoint{2.914208in}{0.532400in}}%
\pgfpathlineto{\pgfqpoint{2.916676in}{0.526240in}}%
\pgfpathlineto{\pgfqpoint{2.919143in}{0.538560in}}%
\pgfpathlineto{\pgfqpoint{2.919966in}{0.538560in}}%
\pgfpathlineto{\pgfqpoint{2.923256in}{0.532400in}}%
\pgfpathlineto{\pgfqpoint{2.924079in}{0.532400in}}%
\pgfpathlineto{\pgfqpoint{2.925724in}{0.544720in}}%
\pgfpathlineto{\pgfqpoint{2.929014in}{0.538560in}}%
\pgfpathlineto{\pgfqpoint{2.935595in}{0.538560in}}%
\pgfpathlineto{\pgfqpoint{2.940531in}{0.520080in}}%
\pgfpathlineto{\pgfqpoint{2.941353in}{0.526240in}}%
\pgfpathlineto{\pgfqpoint{2.942176in}{0.520080in}}%
\pgfpathlineto{\pgfqpoint{2.942998in}{0.526240in}}%
\pgfpathlineto{\pgfqpoint{2.946289in}{0.520080in}}%
\pgfpathlineto{\pgfqpoint{2.948757in}{0.520080in}}%
\pgfpathlineto{\pgfqpoint{2.951224in}{0.513920in}}%
\pgfpathlineto{\pgfqpoint{2.954515in}{0.538560in}}%
\pgfpathlineto{\pgfqpoint{2.956983in}{0.532400in}}%
\pgfpathlineto{\pgfqpoint{2.957805in}{0.538560in}}%
\pgfpathlineto{\pgfqpoint{2.958628in}{0.532400in}}%
\pgfpathlineto{\pgfqpoint{2.960273in}{0.544720in}}%
\pgfpathlineto{\pgfqpoint{2.963563in}{0.532400in}}%
\pgfpathlineto{\pgfqpoint{2.966031in}{0.557040in}}%
\pgfpathlineto{\pgfqpoint{2.969321in}{0.538560in}}%
\pgfpathlineto{\pgfqpoint{2.970967in}{0.538560in}}%
\pgfpathlineto{\pgfqpoint{2.975902in}{0.544720in}}%
\pgfpathlineto{\pgfqpoint{2.977547in}{0.550880in}}%
\pgfpathlineto{\pgfqpoint{2.982483in}{0.532400in}}%
\pgfpathlineto{\pgfqpoint{2.983306in}{0.538560in}}%
\pgfpathlineto{\pgfqpoint{2.986596in}{0.532400in}}%
\pgfpathlineto{\pgfqpoint{2.989064in}{0.532400in}}%
\pgfpathlineto{\pgfqpoint{2.993177in}{0.526240in}}%
\pgfpathlineto{\pgfqpoint{2.998935in}{0.526240in}}%
\pgfpathlineto{\pgfqpoint{3.000580in}{0.532400in}}%
\pgfpathlineto{\pgfqpoint{3.003048in}{0.532400in}}%
\pgfpathlineto{\pgfqpoint{3.003870in}{0.538560in}}%
\pgfpathlineto{\pgfqpoint{3.004693in}{0.532400in}}%
\pgfpathlineto{\pgfqpoint{3.005516in}{0.544720in}}%
\pgfpathlineto{\pgfqpoint{3.006338in}{0.538560in}}%
\pgfpathlineto{\pgfqpoint{3.008806in}{0.532400in}}%
\pgfpathlineto{\pgfqpoint{3.012096in}{0.557040in}}%
\pgfpathlineto{\pgfqpoint{3.015387in}{0.550880in}}%
\pgfpathlineto{\pgfqpoint{3.016209in}{0.557040in}}%
\pgfpathlineto{\pgfqpoint{3.017032in}{0.575520in}}%
\pgfpathlineto{\pgfqpoint{3.017854in}{0.569360in}}%
\pgfpathlineto{\pgfqpoint{3.026080in}{0.538560in}}%
\pgfpathlineto{\pgfqpoint{3.029371in}{0.557040in}}%
\pgfpathlineto{\pgfqpoint{3.033484in}{0.569360in}}%
\pgfpathlineto{\pgfqpoint{3.035129in}{0.581680in}}%
\pgfpathlineto{\pgfqpoint{3.037597in}{0.575520in}}%
\pgfpathlineto{\pgfqpoint{3.039242in}{0.569360in}}%
\pgfpathlineto{\pgfqpoint{3.040887in}{0.587840in}}%
\pgfpathlineto{\pgfqpoint{3.044177in}{0.569360in}}%
\pgfpathlineto{\pgfqpoint{3.045823in}{0.575520in}}%
\pgfpathlineto{\pgfqpoint{3.046645in}{0.575520in}}%
\pgfpathlineto{\pgfqpoint{3.049113in}{0.563200in}}%
\pgfpathlineto{\pgfqpoint{3.049936in}{0.569360in}}%
\pgfpathlineto{\pgfqpoint{3.050758in}{0.563200in}}%
\pgfpathlineto{\pgfqpoint{3.054871in}{0.587840in}}%
\pgfpathlineto{\pgfqpoint{3.055694in}{0.575520in}}%
\pgfpathlineto{\pgfqpoint{3.058161in}{0.594000in}}%
\pgfpathlineto{\pgfqpoint{3.063920in}{0.563200in}}%
\pgfpathlineto{\pgfqpoint{3.066387in}{0.563200in}}%
\pgfpathlineto{\pgfqpoint{3.068033in}{0.569360in}}%
\pgfpathlineto{\pgfqpoint{3.068855in}{0.569360in}}%
\pgfpathlineto{\pgfqpoint{3.072968in}{0.563200in}}%
\pgfpathlineto{\pgfqpoint{3.074613in}{0.581680in}}%
\pgfpathlineto{\pgfqpoint{3.079549in}{0.587840in}}%
\pgfpathlineto{\pgfqpoint{3.081194in}{0.594000in}}%
\pgfpathlineto{\pgfqpoint{3.083662in}{0.587840in}}%
\pgfpathlineto{\pgfqpoint{3.085307in}{0.581680in}}%
\pgfpathlineto{\pgfqpoint{3.086130in}{0.587840in}}%
\pgfpathlineto{\pgfqpoint{3.086952in}{0.581680in}}%
\pgfpathlineto{\pgfqpoint{3.095178in}{0.563200in}}%
\pgfpathlineto{\pgfqpoint{3.096001in}{0.563200in}}%
\pgfpathlineto{\pgfqpoint{3.096823in}{0.557040in}}%
\pgfpathlineto{\pgfqpoint{3.098469in}{0.563200in}}%
\pgfpathlineto{\pgfqpoint{3.101759in}{0.557040in}}%
\pgfpathlineto{\pgfqpoint{3.104227in}{0.569360in}}%
\pgfpathlineto{\pgfqpoint{3.107517in}{0.563200in}}%
\pgfpathlineto{\pgfqpoint{3.108340in}{0.569360in}}%
\pgfpathlineto{\pgfqpoint{3.109985in}{0.563200in}}%
\pgfpathlineto{\pgfqpoint{3.114920in}{0.563200in}}%
\pgfpathlineto{\pgfqpoint{3.115743in}{0.569360in}}%
\pgfpathlineto{\pgfqpoint{3.119856in}{0.581680in}}%
\pgfpathlineto{\pgfqpoint{3.120679in}{0.587840in}}%
\pgfpathlineto{\pgfqpoint{3.121501in}{0.581680in}}%
\pgfpathlineto{\pgfqpoint{3.123969in}{0.581680in}}%
\pgfpathlineto{\pgfqpoint{3.124792in}{0.575520in}}%
\pgfpathlineto{\pgfqpoint{3.125614in}{0.581680in}}%
\pgfpathlineto{\pgfqpoint{3.126437in}{0.575520in}}%
\pgfpathlineto{\pgfqpoint{3.127259in}{0.581680in}}%
\pgfpathlineto{\pgfqpoint{3.130550in}{0.581680in}}%
\pgfpathlineto{\pgfqpoint{3.131372in}{0.587840in}}%
\pgfpathlineto{\pgfqpoint{3.133017in}{0.575520in}}%
\pgfpathlineto{\pgfqpoint{3.136308in}{0.575520in}}%
\pgfpathlineto{\pgfqpoint{3.137953in}{0.587840in}}%
\pgfpathlineto{\pgfqpoint{3.138776in}{0.581680in}}%
\pgfpathlineto{\pgfqpoint{3.141243in}{0.581680in}}%
\pgfpathlineto{\pgfqpoint{3.142066in}{0.575520in}}%
\pgfpathlineto{\pgfqpoint{3.142889in}{0.581680in}}%
\pgfpathlineto{\pgfqpoint{3.144534in}{0.563200in}}%
\pgfpathlineto{\pgfqpoint{3.147002in}{0.563200in}}%
\pgfpathlineto{\pgfqpoint{3.148647in}{0.544720in}}%
\pgfpathlineto{\pgfqpoint{3.149469in}{0.550880in}}%
\pgfpathlineto{\pgfqpoint{3.150292in}{0.538560in}}%
\pgfpathlineto{\pgfqpoint{3.152760in}{0.538560in}}%
\pgfpathlineto{\pgfqpoint{3.155227in}{0.594000in}}%
\pgfpathlineto{\pgfqpoint{3.156050in}{0.600160in}}%
\pgfpathlineto{\pgfqpoint{3.158518in}{0.575520in}}%
\pgfpathlineto{\pgfqpoint{3.160163in}{0.600160in}}%
\pgfpathlineto{\pgfqpoint{3.160986in}{0.600160in}}%
\pgfpathlineto{\pgfqpoint{3.161808in}{0.606320in}}%
\pgfpathlineto{\pgfqpoint{3.164276in}{0.606320in}}%
\pgfpathlineto{\pgfqpoint{3.165099in}{0.600160in}}%
\pgfpathlineto{\pgfqpoint{3.166744in}{0.618640in}}%
\pgfpathlineto{\pgfqpoint{3.167566in}{0.612480in}}%
\pgfpathlineto{\pgfqpoint{3.170857in}{0.624800in}}%
\pgfpathlineto{\pgfqpoint{3.171679in}{0.618640in}}%
\pgfpathlineto{\pgfqpoint{3.173324in}{0.630960in}}%
\pgfpathlineto{\pgfqpoint{3.176615in}{0.637120in}}%
\pgfpathlineto{\pgfqpoint{3.178260in}{0.630960in}}%
\pgfpathlineto{\pgfqpoint{3.179083in}{0.630960in}}%
\pgfpathlineto{\pgfqpoint{3.183196in}{0.637120in}}%
\pgfpathlineto{\pgfqpoint{3.184018in}{0.637120in}}%
\pgfpathlineto{\pgfqpoint{3.184841in}{0.643280in}}%
\pgfpathlineto{\pgfqpoint{3.187309in}{0.643280in}}%
\pgfpathlineto{\pgfqpoint{3.188954in}{0.649440in}}%
\pgfpathlineto{\pgfqpoint{3.193889in}{0.649440in}}%
\pgfpathlineto{\pgfqpoint{3.195534in}{0.661760in}}%
\pgfpathlineto{\pgfqpoint{3.196357in}{0.655600in}}%
\pgfpathlineto{\pgfqpoint{3.198825in}{0.649440in}}%
\pgfpathlineto{\pgfqpoint{3.199647in}{0.661760in}}%
\pgfpathlineto{\pgfqpoint{3.201293in}{0.655600in}}%
\pgfpathlineto{\pgfqpoint{3.202115in}{0.655600in}}%
\pgfpathlineto{\pgfqpoint{3.204583in}{0.661760in}}%
\pgfpathlineto{\pgfqpoint{3.207051in}{0.674080in}}%
\pgfpathlineto{\pgfqpoint{3.213632in}{0.674080in}}%
\pgfpathlineto{\pgfqpoint{3.216922in}{0.667920in}}%
\pgfpathlineto{\pgfqpoint{3.219390in}{0.686400in}}%
\pgfpathlineto{\pgfqpoint{3.221857in}{0.686400in}}%
\pgfpathlineto{\pgfqpoint{3.222680in}{0.692560in}}%
\pgfpathlineto{\pgfqpoint{3.223503in}{0.686400in}}%
\pgfpathlineto{\pgfqpoint{3.225148in}{0.698720in}}%
\pgfpathlineto{\pgfqpoint{3.230083in}{0.698720in}}%
\pgfpathlineto{\pgfqpoint{3.230906in}{0.704880in}}%
\pgfpathlineto{\pgfqpoint{3.234196in}{0.704880in}}%
\pgfpathlineto{\pgfqpoint{3.235842in}{0.711040in}}%
\pgfpathlineto{\pgfqpoint{3.239132in}{0.717200in}}%
\pgfpathlineto{\pgfqpoint{3.240777in}{0.723360in}}%
\pgfpathlineto{\pgfqpoint{3.241600in}{0.723360in}}%
\pgfpathlineto{\pgfqpoint{3.242422in}{0.735680in}}%
\pgfpathlineto{\pgfqpoint{3.245713in}{0.735680in}}%
\pgfpathlineto{\pgfqpoint{3.248180in}{0.748000in}}%
\pgfpathlineto{\pgfqpoint{3.250648in}{0.741840in}}%
\pgfpathlineto{\pgfqpoint{3.252293in}{0.760320in}}%
\pgfpathlineto{\pgfqpoint{3.253116in}{0.760320in}}%
\pgfpathlineto{\pgfqpoint{3.253939in}{0.754160in}}%
\pgfpathlineto{\pgfqpoint{3.256406in}{0.760320in}}%
\pgfpathlineto{\pgfqpoint{3.257229in}{0.766480in}}%
\pgfpathlineto{\pgfqpoint{3.258052in}{0.760320in}}%
\pgfpathlineto{\pgfqpoint{3.259697in}{0.766480in}}%
\pgfpathlineto{\pgfqpoint{3.262165in}{0.766480in}}%
\pgfpathlineto{\pgfqpoint{3.265455in}{0.784960in}}%
\pgfpathlineto{\pgfqpoint{3.270390in}{0.784960in}}%
\pgfpathlineto{\pgfqpoint{3.271213in}{0.791120in}}%
\pgfpathlineto{\pgfqpoint{3.275326in}{0.791120in}}%
\pgfpathlineto{\pgfqpoint{3.276971in}{0.784960in}}%
\pgfpathlineto{\pgfqpoint{3.282729in}{0.784960in}}%
\pgfpathlineto{\pgfqpoint{3.285197in}{0.791120in}}%
\pgfpathlineto{\pgfqpoint{3.288487in}{0.772640in}}%
\pgfpathlineto{\pgfqpoint{3.292600in}{0.766480in}}%
\pgfpathlineto{\pgfqpoint{3.294246in}{0.754160in}}%
\pgfpathlineto{\pgfqpoint{3.297536in}{0.760320in}}%
\pgfpathlineto{\pgfqpoint{3.298359in}{0.760320in}}%
\pgfpathlineto{\pgfqpoint{3.300004in}{0.741840in}}%
\pgfpathlineto{\pgfqpoint{3.302472in}{0.748000in}}%
\pgfpathlineto{\pgfqpoint{3.304939in}{0.717200in}}%
\pgfpathlineto{\pgfqpoint{3.305762in}{0.717200in}}%
\pgfpathlineto{\pgfqpoint{3.309052in}{0.711040in}}%
\pgfpathlineto{\pgfqpoint{3.310697in}{0.698720in}}%
\pgfpathlineto{\pgfqpoint{3.311520in}{0.698720in}}%
\pgfpathlineto{\pgfqpoint{3.314810in}{0.674080in}}%
\pgfpathlineto{\pgfqpoint{3.317278in}{0.686400in}}%
\pgfpathlineto{\pgfqpoint{3.320569in}{0.692560in}}%
\pgfpathlineto{\pgfqpoint{3.323036in}{0.680240in}}%
\pgfpathlineto{\pgfqpoint{3.325504in}{0.667920in}}%
\pgfpathlineto{\pgfqpoint{3.327149in}{0.686400in}}%
\pgfpathlineto{\pgfqpoint{3.327972in}{0.686400in}}%
\pgfpathlineto{\pgfqpoint{3.328795in}{0.692560in}}%
\pgfpathlineto{\pgfqpoint{3.332085in}{0.680240in}}%
\pgfpathlineto{\pgfqpoint{3.333730in}{0.680240in}}%
\pgfpathlineto{\pgfqpoint{3.334553in}{0.674080in}}%
\pgfpathlineto{\pgfqpoint{3.337020in}{0.674080in}}%
\pgfpathlineto{\pgfqpoint{3.337843in}{0.680240in}}%
\pgfpathlineto{\pgfqpoint{3.338666in}{0.674080in}}%
\pgfpathlineto{\pgfqpoint{3.339488in}{0.680240in}}%
\pgfpathlineto{\pgfqpoint{3.340311in}{0.674080in}}%
\pgfpathlineto{\pgfqpoint{3.343601in}{0.667920in}}%
\pgfpathlineto{\pgfqpoint{3.346069in}{0.680240in}}%
\pgfpathlineto{\pgfqpoint{3.349359in}{0.674080in}}%
\pgfpathlineto{\pgfqpoint{3.350182in}{0.674080in}}%
\pgfpathlineto{\pgfqpoint{3.351827in}{0.686400in}}%
\pgfpathlineto{\pgfqpoint{3.355118in}{0.680240in}}%
\pgfpathlineto{\pgfqpoint{3.356763in}{0.680240in}}%
\pgfpathlineto{\pgfqpoint{3.361698in}{0.674080in}}%
\pgfpathlineto{\pgfqpoint{3.363343in}{0.674080in}}%
\pgfpathlineto{\pgfqpoint{3.368279in}{0.686400in}}%
\pgfpathlineto{\pgfqpoint{3.369102in}{0.680240in}}%
\pgfpathlineto{\pgfqpoint{3.371569in}{0.680240in}}%
\pgfpathlineto{\pgfqpoint{3.373215in}{0.674080in}}%
\pgfpathlineto{\pgfqpoint{3.374860in}{0.674080in}}%
\pgfpathlineto{\pgfqpoint{3.377328in}{0.667920in}}%
\pgfpathlineto{\pgfqpoint{3.378973in}{0.680240in}}%
\pgfpathlineto{\pgfqpoint{3.379795in}{0.674080in}}%
\pgfpathlineto{\pgfqpoint{3.380618in}{0.686400in}}%
\pgfpathlineto{\pgfqpoint{3.386376in}{0.667920in}}%
\pgfpathlineto{\pgfqpoint{3.388844in}{0.667920in}}%
\pgfpathlineto{\pgfqpoint{3.389666in}{0.674080in}}%
\pgfpathlineto{\pgfqpoint{3.391312in}{0.667920in}}%
\pgfpathlineto{\pgfqpoint{3.392134in}{0.680240in}}%
\pgfpathlineto{\pgfqpoint{3.395425in}{0.674080in}}%
\pgfpathlineto{\pgfqpoint{3.396247in}{0.674080in}}%
\pgfpathlineto{\pgfqpoint{3.397070in}{0.667920in}}%
\pgfpathlineto{\pgfqpoint{3.397892in}{0.674080in}}%
\pgfpathlineto{\pgfqpoint{3.401183in}{0.674080in}}%
\pgfpathlineto{\pgfqpoint{3.402828in}{0.686400in}}%
\pgfpathlineto{\pgfqpoint{3.403650in}{0.686400in}}%
\pgfpathlineto{\pgfqpoint{3.408586in}{0.674080in}}%
\pgfpathlineto{\pgfqpoint{3.411876in}{0.674080in}}%
\pgfpathlineto{\pgfqpoint{3.414344in}{0.661760in}}%
\pgfpathlineto{\pgfqpoint{3.415167in}{0.674080in}}%
\pgfpathlineto{\pgfqpoint{3.417635in}{0.667920in}}%
\pgfpathlineto{\pgfqpoint{3.419280in}{0.674080in}}%
\pgfpathlineto{\pgfqpoint{3.420102in}{0.674080in}}%
\pgfpathlineto{\pgfqpoint{3.420925in}{0.667920in}}%
\pgfpathlineto{\pgfqpoint{3.424215in}{0.661760in}}%
\pgfpathlineto{\pgfqpoint{3.425861in}{0.667920in}}%
\pgfpathlineto{\pgfqpoint{3.426683in}{0.667920in}}%
\pgfpathlineto{\pgfqpoint{3.429151in}{0.661760in}}%
\pgfpathlineto{\pgfqpoint{3.431619in}{0.674080in}}%
\pgfpathlineto{\pgfqpoint{3.432441in}{0.674080in}}%
\pgfpathlineto{\pgfqpoint{3.438199in}{0.661760in}}%
\pgfpathlineto{\pgfqpoint{3.441490in}{0.661760in}}%
\pgfpathlineto{\pgfqpoint{3.442312in}{0.667920in}}%
\pgfpathlineto{\pgfqpoint{3.443958in}{0.661760in}}%
\pgfpathlineto{\pgfqpoint{3.447248in}{0.655600in}}%
\pgfpathlineto{\pgfqpoint{3.448071in}{0.655600in}}%
\pgfpathlineto{\pgfqpoint{3.449716in}{0.649440in}}%
\pgfpathlineto{\pgfqpoint{3.453006in}{0.643280in}}%
\pgfpathlineto{\pgfqpoint{3.455474in}{0.655600in}}%
\pgfpathlineto{\pgfqpoint{3.457942in}{0.649440in}}%
\pgfpathlineto{\pgfqpoint{3.459587in}{0.643280in}}%
\pgfpathlineto{\pgfqpoint{3.460409in}{0.643280in}}%
\pgfpathlineto{\pgfqpoint{3.461232in}{0.649440in}}%
\pgfpathlineto{\pgfqpoint{3.463700in}{0.637120in}}%
\pgfpathlineto{\pgfqpoint{3.466168in}{0.649440in}}%
\pgfpathlineto{\pgfqpoint{3.466990in}{0.649440in}}%
\pgfpathlineto{\pgfqpoint{3.470281in}{0.637120in}}%
\pgfpathlineto{\pgfqpoint{3.476039in}{0.637120in}}%
\pgfpathlineto{\pgfqpoint{3.476861in}{0.624800in}}%
\pgfpathlineto{\pgfqpoint{3.478506in}{0.643280in}}%
\pgfpathlineto{\pgfqpoint{3.484265in}{0.624800in}}%
\pgfpathlineto{\pgfqpoint{3.487555in}{0.624800in}}%
\pgfpathlineto{\pgfqpoint{3.490023in}{0.612480in}}%
\pgfpathlineto{\pgfqpoint{3.492491in}{0.606320in}}%
\pgfpathlineto{\pgfqpoint{3.493313in}{0.612480in}}%
\pgfpathlineto{\pgfqpoint{3.495781in}{0.600160in}}%
\pgfpathlineto{\pgfqpoint{3.498249in}{0.600160in}}%
\pgfpathlineto{\pgfqpoint{3.499894in}{0.594000in}}%
\pgfpathlineto{\pgfqpoint{3.500716in}{0.606320in}}%
\pgfpathlineto{\pgfqpoint{3.501539in}{0.600160in}}%
\pgfpathlineto{\pgfqpoint{3.504007in}{0.606320in}}%
\pgfpathlineto{\pgfqpoint{3.507297in}{0.587840in}}%
\pgfpathlineto{\pgfqpoint{3.510588in}{0.594000in}}%
\pgfpathlineto{\pgfqpoint{3.513055in}{0.575520in}}%
\pgfpathlineto{\pgfqpoint{3.515523in}{0.575520in}}%
\pgfpathlineto{\pgfqpoint{3.516346in}{0.581680in}}%
\pgfpathlineto{\pgfqpoint{3.517168in}{0.569360in}}%
\pgfpathlineto{\pgfqpoint{3.518813in}{0.575520in}}%
\pgfpathlineto{\pgfqpoint{3.521281in}{0.575520in}}%
\pgfpathlineto{\pgfqpoint{3.524572in}{0.557040in}}%
\pgfpathlineto{\pgfqpoint{3.527039in}{0.544720in}}%
\pgfpathlineto{\pgfqpoint{3.530330in}{0.575520in}}%
\pgfpathlineto{\pgfqpoint{3.532798in}{0.563200in}}%
\pgfpathlineto{\pgfqpoint{3.534443in}{0.569360in}}%
\pgfpathlineto{\pgfqpoint{3.535265in}{0.569360in}}%
\pgfpathlineto{\pgfqpoint{3.536088in}{0.575520in}}%
\pgfpathlineto{\pgfqpoint{3.539378in}{0.563200in}}%
\pgfpathlineto{\pgfqpoint{3.541024in}{0.581680in}}%
\pgfpathlineto{\pgfqpoint{3.541846in}{0.594000in}}%
\pgfpathlineto{\pgfqpoint{3.544314in}{0.575520in}}%
\pgfpathlineto{\pgfqpoint{3.545136in}{0.581680in}}%
\pgfpathlineto{\pgfqpoint{3.545959in}{0.569360in}}%
\pgfpathlineto{\pgfqpoint{3.551717in}{0.569360in}}%
\pgfpathlineto{\pgfqpoint{3.553362in}{0.581680in}}%
\pgfpathlineto{\pgfqpoint{3.556653in}{0.569360in}}%
\pgfpathlineto{\pgfqpoint{3.558298in}{0.569360in}}%
\pgfpathlineto{\pgfqpoint{3.559121in}{0.575520in}}%
\pgfpathlineto{\pgfqpoint{3.561588in}{0.575520in}}%
\pgfpathlineto{\pgfqpoint{3.564879in}{0.606320in}}%
\pgfpathlineto{\pgfqpoint{3.567346in}{0.600160in}}%
\pgfpathlineto{\pgfqpoint{3.568169in}{0.594000in}}%
\pgfpathlineto{\pgfqpoint{3.568992in}{0.600160in}}%
\pgfpathlineto{\pgfqpoint{3.570637in}{0.594000in}}%
\pgfpathlineto{\pgfqpoint{3.573105in}{0.594000in}}%
\pgfpathlineto{\pgfqpoint{3.575572in}{0.581680in}}%
\pgfpathlineto{\pgfqpoint{3.576395in}{0.624800in}}%
\pgfpathlineto{\pgfqpoint{3.581331in}{0.575520in}}%
\pgfpathlineto{\pgfqpoint{3.582153in}{0.587840in}}%
\pgfpathlineto{\pgfqpoint{3.587089in}{0.587840in}}%
\pgfpathlineto{\pgfqpoint{3.587911in}{0.581680in}}%
\pgfpathlineto{\pgfqpoint{3.590379in}{0.575520in}}%
\pgfpathlineto{\pgfqpoint{3.591202in}{0.569360in}}%
\pgfpathlineto{\pgfqpoint{3.592847in}{0.581680in}}%
\pgfpathlineto{\pgfqpoint{3.593669in}{0.575520in}}%
\pgfpathlineto{\pgfqpoint{3.599428in}{0.575520in}}%
\pgfpathlineto{\pgfqpoint{3.602718in}{0.581680in}}%
\pgfpathlineto{\pgfqpoint{3.604363in}{0.581680in}}%
\pgfpathlineto{\pgfqpoint{3.605186in}{0.587840in}}%
\pgfpathlineto{\pgfqpoint{3.610944in}{0.575520in}}%
\pgfpathlineto{\pgfqpoint{3.615879in}{0.550880in}}%
\pgfpathlineto{\pgfqpoint{3.616702in}{0.557040in}}%
\pgfpathlineto{\pgfqpoint{3.622460in}{0.538560in}}%
\pgfpathlineto{\pgfqpoint{3.624928in}{0.544720in}}%
\pgfpathlineto{\pgfqpoint{3.625751in}{0.532400in}}%
\pgfpathlineto{\pgfqpoint{3.627396in}{0.550880in}}%
\pgfpathlineto{\pgfqpoint{3.628218in}{0.544720in}}%
\pgfpathlineto{\pgfqpoint{3.630686in}{0.544720in}}%
\pgfpathlineto{\pgfqpoint{3.631509in}{0.563200in}}%
\pgfpathlineto{\pgfqpoint{3.633154in}{0.550880in}}%
\pgfpathlineto{\pgfqpoint{3.637267in}{0.550880in}}%
\pgfpathlineto{\pgfqpoint{3.638089in}{0.557040in}}%
\pgfpathlineto{\pgfqpoint{3.638912in}{0.550880in}}%
\pgfpathlineto{\pgfqpoint{3.642202in}{0.569360in}}%
\pgfpathlineto{\pgfqpoint{3.645493in}{0.569360in}}%
\pgfpathlineto{\pgfqpoint{3.647961in}{0.563200in}}%
\pgfpathlineto{\pgfqpoint{3.648783in}{0.569360in}}%
\pgfpathlineto{\pgfqpoint{3.649606in}{0.557040in}}%
\pgfpathlineto{\pgfqpoint{3.650428in}{0.569360in}}%
\pgfpathlineto{\pgfqpoint{3.654541in}{0.569360in}}%
\pgfpathlineto{\pgfqpoint{3.655364in}{0.575520in}}%
\pgfpathlineto{\pgfqpoint{3.657009in}{0.569360in}}%
\pgfpathlineto{\pgfqpoint{3.659477in}{0.569360in}}%
\pgfpathlineto{\pgfqpoint{3.660299in}{0.575520in}}%
\pgfpathlineto{\pgfqpoint{3.661945in}{0.569360in}}%
\pgfpathlineto{\pgfqpoint{3.662767in}{0.575520in}}%
\pgfpathlineto{\pgfqpoint{3.666880in}{0.575520in}}%
\pgfpathlineto{\pgfqpoint{3.668525in}{0.581680in}}%
\pgfpathlineto{\pgfqpoint{3.671816in}{0.581680in}}%
\pgfpathlineto{\pgfqpoint{3.672638in}{0.575520in}}%
\pgfpathlineto{\pgfqpoint{3.674284in}{0.581680in}}%
\pgfpathlineto{\pgfqpoint{3.678397in}{0.569360in}}%
\pgfpathlineto{\pgfqpoint{3.680042in}{0.581680in}}%
\pgfpathlineto{\pgfqpoint{3.684155in}{0.587840in}}%
\pgfpathlineto{\pgfqpoint{3.685800in}{0.587840in}}%
\pgfpathlineto{\pgfqpoint{3.688268in}{0.581680in}}%
\pgfpathlineto{\pgfqpoint{3.689090in}{0.575520in}}%
\pgfpathlineto{\pgfqpoint{3.690735in}{0.587840in}}%
\pgfpathlineto{\pgfqpoint{3.691558in}{0.581680in}}%
\pgfpathlineto{\pgfqpoint{3.694026in}{0.575520in}}%
\pgfpathlineto{\pgfqpoint{3.695671in}{0.581680in}}%
\pgfpathlineto{\pgfqpoint{3.696494in}{0.575520in}}%
\pgfpathlineto{\pgfqpoint{3.697316in}{0.581680in}}%
\pgfpathlineto{\pgfqpoint{3.700607in}{0.575520in}}%
\pgfpathlineto{\pgfqpoint{3.703074in}{0.587840in}}%
\pgfpathlineto{\pgfqpoint{3.705542in}{0.575520in}}%
\pgfpathlineto{\pgfqpoint{3.706365in}{0.581680in}}%
\pgfpathlineto{\pgfqpoint{3.708010in}{0.575520in}}%
\pgfpathlineto{\pgfqpoint{3.708832in}{0.587840in}}%
\pgfpathlineto{\pgfqpoint{3.712123in}{0.575520in}}%
\pgfpathlineto{\pgfqpoint{3.724462in}{0.575520in}}%
\pgfpathlineto{\pgfqpoint{3.725284in}{0.569360in}}%
\pgfpathlineto{\pgfqpoint{3.726107in}{0.581680in}}%
\pgfpathlineto{\pgfqpoint{3.728575in}{0.581680in}}%
\pgfpathlineto{\pgfqpoint{3.729397in}{0.587840in}}%
\pgfpathlineto{\pgfqpoint{3.730220in}{0.575520in}}%
\pgfpathlineto{\pgfqpoint{3.731865in}{0.581680in}}%
\pgfpathlineto{\pgfqpoint{3.734333in}{0.575520in}}%
\pgfpathlineto{\pgfqpoint{3.735978in}{0.581680in}}%
\pgfpathlineto{\pgfqpoint{3.740091in}{0.581680in}}%
\pgfpathlineto{\pgfqpoint{3.741736in}{0.587840in}}%
\pgfpathlineto{\pgfqpoint{3.742559in}{0.587840in}}%
\pgfpathlineto{\pgfqpoint{3.743381in}{0.581680in}}%
\pgfpathlineto{\pgfqpoint{3.745849in}{0.575520in}}%
\pgfpathlineto{\pgfqpoint{3.746672in}{0.594000in}}%
\pgfpathlineto{\pgfqpoint{3.747494in}{0.581680in}}%
\pgfpathlineto{\pgfqpoint{3.749139in}{0.587840in}}%
\pgfpathlineto{\pgfqpoint{3.751607in}{0.587840in}}%
\pgfpathlineto{\pgfqpoint{3.752430in}{0.575520in}}%
\pgfpathlineto{\pgfqpoint{3.754075in}{0.581680in}}%
\pgfpathlineto{\pgfqpoint{3.754898in}{0.581680in}}%
\pgfpathlineto{\pgfqpoint{3.757365in}{0.575520in}}%
\pgfpathlineto{\pgfqpoint{3.758188in}{0.581680in}}%
\pgfpathlineto{\pgfqpoint{3.760656in}{0.569360in}}%
\pgfpathlineto{\pgfqpoint{3.763946in}{0.563200in}}%
\pgfpathlineto{\pgfqpoint{3.765591in}{0.569360in}}%
\pgfpathlineto{\pgfqpoint{3.766414in}{0.569360in}}%
\pgfpathlineto{\pgfqpoint{3.768882in}{0.563200in}}%
\pgfpathlineto{\pgfqpoint{3.770527in}{0.575520in}}%
\pgfpathlineto{\pgfqpoint{3.772172in}{0.581680in}}%
\pgfpathlineto{\pgfqpoint{3.775462in}{0.569360in}}%
\pgfpathlineto{\pgfqpoint{3.777108in}{0.587840in}}%
\pgfpathlineto{\pgfqpoint{3.777930in}{0.581680in}}%
\pgfpathlineto{\pgfqpoint{3.782043in}{0.587840in}}%
\pgfpathlineto{\pgfqpoint{3.786979in}{0.587840in}}%
\pgfpathlineto{\pgfqpoint{3.787801in}{0.581680in}}%
\pgfpathlineto{\pgfqpoint{3.789447in}{0.587840in}}%
\pgfpathlineto{\pgfqpoint{3.791914in}{0.581680in}}%
\pgfpathlineto{\pgfqpoint{3.793560in}{0.587840in}}%
\pgfpathlineto{\pgfqpoint{3.795205in}{0.587840in}}%
\pgfpathlineto{\pgfqpoint{3.798495in}{0.581680in}}%
\pgfpathlineto{\pgfqpoint{3.799318in}{0.581680in}}%
\pgfpathlineto{\pgfqpoint{3.800140in}{0.594000in}}%
\pgfpathlineto{\pgfqpoint{3.800963in}{0.581680in}}%
\pgfpathlineto{\pgfqpoint{3.804253in}{0.581680in}}%
\pgfpathlineto{\pgfqpoint{3.806721in}{0.569360in}}%
\pgfpathlineto{\pgfqpoint{3.809189in}{0.569360in}}%
\pgfpathlineto{\pgfqpoint{3.810011in}{0.557040in}}%
\pgfpathlineto{\pgfqpoint{3.810834in}{0.563200in}}%
\pgfpathlineto{\pgfqpoint{3.811657in}{0.557040in}}%
\pgfpathlineto{\pgfqpoint{3.812479in}{0.538560in}}%
\pgfpathlineto{\pgfqpoint{3.815770in}{0.501600in}}%
\pgfpathlineto{\pgfqpoint{3.818237in}{0.563200in}}%
\pgfpathlineto{\pgfqpoint{3.822350in}{0.563200in}}%
\pgfpathlineto{\pgfqpoint{3.823173in}{0.569360in}}%
\pgfpathlineto{\pgfqpoint{3.823995in}{0.563200in}}%
\pgfpathlineto{\pgfqpoint{3.829754in}{0.563200in}}%
\pgfpathlineto{\pgfqpoint{3.833044in}{0.557040in}}%
\pgfpathlineto{\pgfqpoint{3.834689in}{0.557040in}}%
\pgfpathlineto{\pgfqpoint{3.835512in}{0.550880in}}%
\pgfpathlineto{\pgfqpoint{3.839625in}{0.538560in}}%
\pgfpathlineto{\pgfqpoint{3.841270in}{0.538560in}}%
\pgfpathlineto{\pgfqpoint{3.843738in}{0.532400in}}%
\pgfpathlineto{\pgfqpoint{3.846205in}{0.544720in}}%
\pgfpathlineto{\pgfqpoint{3.847028in}{0.544720in}}%
\pgfpathlineto{\pgfqpoint{3.849496in}{0.538560in}}%
\pgfpathlineto{\pgfqpoint{3.851141in}{0.544720in}}%
\pgfpathlineto{\pgfqpoint{3.855254in}{0.557040in}}%
\pgfpathlineto{\pgfqpoint{3.856899in}{0.550880in}}%
\pgfpathlineto{\pgfqpoint{3.858544in}{0.550880in}}%
\pgfpathlineto{\pgfqpoint{3.861835in}{0.544720in}}%
\pgfpathlineto{\pgfqpoint{3.867593in}{0.544720in}}%
\pgfpathlineto{\pgfqpoint{3.869238in}{0.557040in}}%
\pgfpathlineto{\pgfqpoint{3.870061in}{0.550880in}}%
\pgfpathlineto{\pgfqpoint{3.879109in}{0.550880in}}%
\pgfpathlineto{\pgfqpoint{3.881577in}{0.544720in}}%
\pgfpathlineto{\pgfqpoint{3.884867in}{0.563200in}}%
\pgfpathlineto{\pgfqpoint{3.885690in}{0.557040in}}%
\pgfpathlineto{\pgfqpoint{3.886513in}{0.563200in}}%
\pgfpathlineto{\pgfqpoint{3.887335in}{0.557040in}}%
\pgfpathlineto{\pgfqpoint{3.889803in}{0.569360in}}%
\pgfpathlineto{\pgfqpoint{3.891448in}{0.563200in}}%
\pgfpathlineto{\pgfqpoint{3.892271in}{0.563200in}}%
\pgfpathlineto{\pgfqpoint{3.893093in}{0.557040in}}%
\pgfpathlineto{\pgfqpoint{3.897206in}{0.563200in}}%
\pgfpathlineto{\pgfqpoint{3.898851in}{0.563200in}}%
\pgfpathlineto{\pgfqpoint{3.902142in}{0.557040in}}%
\pgfpathlineto{\pgfqpoint{3.904610in}{0.575520in}}%
\pgfpathlineto{\pgfqpoint{3.910368in}{0.532400in}}%
\pgfpathlineto{\pgfqpoint{3.912835in}{0.538560in}}%
\pgfpathlineto{\pgfqpoint{3.915303in}{0.569360in}}%
\pgfpathlineto{\pgfqpoint{3.916126in}{0.575520in}}%
\pgfpathlineto{\pgfqpoint{3.920239in}{0.544720in}}%
\pgfpathlineto{\pgfqpoint{3.921061in}{0.557040in}}%
\pgfpathlineto{\pgfqpoint{3.921884in}{0.550880in}}%
\pgfpathlineto{\pgfqpoint{3.925997in}{0.550880in}}%
\pgfpathlineto{\pgfqpoint{3.926820in}{0.563200in}}%
\pgfpathlineto{\pgfqpoint{3.927642in}{0.557040in}}%
\pgfpathlineto{\pgfqpoint{3.930933in}{0.557040in}}%
\pgfpathlineto{\pgfqpoint{3.931755in}{0.544720in}}%
\pgfpathlineto{\pgfqpoint{3.932578in}{0.557040in}}%
\pgfpathlineto{\pgfqpoint{3.933400in}{0.550880in}}%
\pgfpathlineto{\pgfqpoint{3.939158in}{0.550880in}}%
\pgfpathlineto{\pgfqpoint{3.941626in}{0.544720in}}%
\pgfpathlineto{\pgfqpoint{3.942449in}{0.550880in}}%
\pgfpathlineto{\pgfqpoint{3.944094in}{0.544720in}}%
\pgfpathlineto{\pgfqpoint{3.944917in}{0.544720in}}%
\pgfpathlineto{\pgfqpoint{3.948207in}{0.550880in}}%
\pgfpathlineto{\pgfqpoint{3.949030in}{0.550880in}}%
\pgfpathlineto{\pgfqpoint{3.950675in}{0.557040in}}%
\pgfpathlineto{\pgfqpoint{3.953143in}{0.550880in}}%
\pgfpathlineto{\pgfqpoint{3.955610in}{0.569360in}}%
\pgfpathlineto{\pgfqpoint{3.956433in}{0.563200in}}%
\pgfpathlineto{\pgfqpoint{3.959723in}{0.563200in}}%
\pgfpathlineto{\pgfqpoint{3.961368in}{0.557040in}}%
\pgfpathlineto{\pgfqpoint{3.966304in}{0.557040in}}%
\pgfpathlineto{\pgfqpoint{3.967127in}{0.563200in}}%
\pgfpathlineto{\pgfqpoint{3.971240in}{0.563200in}}%
\pgfpathlineto{\pgfqpoint{3.972885in}{0.575520in}}%
\pgfpathlineto{\pgfqpoint{3.973707in}{0.563200in}}%
\pgfpathlineto{\pgfqpoint{3.976175in}{0.557040in}}%
\pgfpathlineto{\pgfqpoint{3.976998in}{0.569360in}}%
\pgfpathlineto{\pgfqpoint{3.978643in}{0.557040in}}%
\pgfpathlineto{\pgfqpoint{3.979466in}{0.563200in}}%
\pgfpathlineto{\pgfqpoint{3.983578in}{0.557040in}}%
\pgfpathlineto{\pgfqpoint{3.985224in}{0.557040in}}%
\pgfpathlineto{\pgfqpoint{3.987691in}{0.563200in}}%
\pgfpathlineto{\pgfqpoint{3.988514in}{0.557040in}}%
\pgfpathlineto{\pgfqpoint{3.990159in}{0.563200in}}%
\pgfpathlineto{\pgfqpoint{3.994272in}{0.563200in}}%
\pgfpathlineto{\pgfqpoint{3.995095in}{0.557040in}}%
\pgfpathlineto{\pgfqpoint{3.995917in}{0.563200in}}%
\pgfpathlineto{\pgfqpoint{3.996740in}{0.557040in}}%
\pgfpathlineto{\pgfqpoint{4.017305in}{0.557040in}}%
\pgfpathlineto{\pgfqpoint{4.019773in}{0.569360in}}%
\pgfpathlineto{\pgfqpoint{4.023063in}{0.563200in}}%
\pgfpathlineto{\pgfqpoint{4.023886in}{0.563200in}}%
\pgfpathlineto{\pgfqpoint{4.024708in}{0.557040in}}%
\pgfpathlineto{\pgfqpoint{4.025531in}{0.563200in}}%
\pgfpathlineto{\pgfqpoint{4.027998in}{0.557040in}}%
\pgfpathlineto{\pgfqpoint{4.029644in}{0.569360in}}%
\pgfpathlineto{\pgfqpoint{4.030466in}{0.575520in}}%
\pgfpathlineto{\pgfqpoint{4.033757in}{0.557040in}}%
\pgfpathlineto{\pgfqpoint{4.036224in}{0.569360in}}%
\pgfpathlineto{\pgfqpoint{4.037047in}{0.569360in}}%
\pgfpathlineto{\pgfqpoint{4.039515in}{0.563200in}}%
\pgfpathlineto{\pgfqpoint{4.041160in}{0.569360in}}%
\pgfpathlineto{\pgfqpoint{4.042805in}{0.569360in}}%
\pgfpathlineto{\pgfqpoint{4.046096in}{0.563200in}}%
\pgfpathlineto{\pgfqpoint{4.046918in}{0.563200in}}%
\pgfpathlineto{\pgfqpoint{4.047741in}{0.569360in}}%
\pgfpathlineto{\pgfqpoint{4.048563in}{0.563200in}}%
\pgfpathlineto{\pgfqpoint{4.051031in}{0.563200in}}%
\pgfpathlineto{\pgfqpoint{4.051854in}{0.575520in}}%
\pgfpathlineto{\pgfqpoint{4.052676in}{0.563200in}}%
\pgfpathlineto{\pgfqpoint{4.054321in}{0.569360in}}%
\pgfpathlineto{\pgfqpoint{4.056789in}{0.563200in}}%
\pgfpathlineto{\pgfqpoint{4.057612in}{0.569360in}}%
\pgfpathlineto{\pgfqpoint{4.059257in}{0.563200in}}%
\pgfpathlineto{\pgfqpoint{4.060080in}{0.569360in}}%
\pgfpathlineto{\pgfqpoint{4.062547in}{0.557040in}}%
\pgfpathlineto{\pgfqpoint{4.063370in}{0.563200in}}%
\pgfpathlineto{\pgfqpoint{4.065015in}{0.557040in}}%
\pgfpathlineto{\pgfqpoint{4.065838in}{0.563200in}}%
\pgfpathlineto{\pgfqpoint{4.069128in}{0.563200in}}%
\pgfpathlineto{\pgfqpoint{4.069951in}{0.557040in}}%
\pgfpathlineto{\pgfqpoint{4.071596in}{0.569360in}}%
\pgfpathlineto{\pgfqpoint{4.074886in}{0.569360in}}%
\pgfpathlineto{\pgfqpoint{4.075709in}{0.557040in}}%
\pgfpathlineto{\pgfqpoint{4.077354in}{0.563200in}}%
\pgfpathlineto{\pgfqpoint{4.083112in}{0.563200in}}%
\pgfpathlineto{\pgfqpoint{4.086403in}{0.569360in}}%
\pgfpathlineto{\pgfqpoint{4.094629in}{0.569360in}}%
\pgfpathlineto{\pgfqpoint{4.097096in}{0.581680in}}%
\pgfpathlineto{\pgfqpoint{4.098741in}{0.575520in}}%
\pgfpathlineto{\pgfqpoint{4.103677in}{0.575520in}}%
\pgfpathlineto{\pgfqpoint{4.104500in}{0.569360in}}%
\pgfpathlineto{\pgfqpoint{4.106145in}{0.575520in}}%
\pgfpathlineto{\pgfqpoint{4.109435in}{0.569360in}}%
\pgfpathlineto{\pgfqpoint{4.111080in}{0.575520in}}%
\pgfpathlineto{\pgfqpoint{4.111903in}{0.575520in}}%
\pgfpathlineto{\pgfqpoint{4.116016in}{0.569360in}}%
\pgfpathlineto{\pgfqpoint{4.116839in}{0.563200in}}%
\pgfpathlineto{\pgfqpoint{4.117661in}{0.569360in}}%
\pgfpathlineto{\pgfqpoint{4.121774in}{0.569360in}}%
\pgfpathlineto{\pgfqpoint{4.123419in}{0.575520in}}%
\pgfpathlineto{\pgfqpoint{4.127532in}{0.563200in}}%
\pgfpathlineto{\pgfqpoint{4.129177in}{0.575520in}}%
\pgfpathlineto{\pgfqpoint{4.131645in}{0.569360in}}%
\pgfpathlineto{\pgfqpoint{4.133290in}{0.563200in}}%
\pgfpathlineto{\pgfqpoint{4.134936in}{0.563200in}}%
\pgfpathlineto{\pgfqpoint{4.137403in}{0.569360in}}%
\pgfpathlineto{\pgfqpoint{4.140694in}{0.569360in}}%
\pgfpathlineto{\pgfqpoint{4.143161in}{0.563200in}}%
\pgfpathlineto{\pgfqpoint{4.144807in}{0.575520in}}%
\pgfpathlineto{\pgfqpoint{4.145629in}{0.569360in}}%
\pgfpathlineto{\pgfqpoint{4.146452in}{0.575520in}}%
\pgfpathlineto{\pgfqpoint{4.148920in}{0.569360in}}%
\pgfpathlineto{\pgfqpoint{4.150565in}{0.575520in}}%
\pgfpathlineto{\pgfqpoint{4.154678in}{0.563200in}}%
\pgfpathlineto{\pgfqpoint{4.155500in}{0.563200in}}%
\pgfpathlineto{\pgfqpoint{4.157146in}{0.575520in}}%
\pgfpathlineto{\pgfqpoint{4.157968in}{0.575520in}}%
\pgfpathlineto{\pgfqpoint{4.160436in}{0.569360in}}%
\pgfpathlineto{\pgfqpoint{4.161259in}{0.563200in}}%
\pgfpathlineto{\pgfqpoint{4.163726in}{0.575520in}}%
\pgfpathlineto{\pgfqpoint{4.166194in}{0.563200in}}%
\pgfpathlineto{\pgfqpoint{4.167839in}{0.575520in}}%
\pgfpathlineto{\pgfqpoint{4.169484in}{0.569360in}}%
\pgfpathlineto{\pgfqpoint{4.171952in}{0.569360in}}%
\pgfpathlineto{\pgfqpoint{4.172775in}{0.575520in}}%
\pgfpathlineto{\pgfqpoint{4.173597in}{0.594000in}}%
\pgfpathlineto{\pgfqpoint{4.179356in}{0.575520in}}%
\pgfpathlineto{\pgfqpoint{4.181001in}{0.587840in}}%
\pgfpathlineto{\pgfqpoint{4.184291in}{0.575520in}}%
\pgfpathlineto{\pgfqpoint{4.185936in}{0.575520in}}%
\pgfpathlineto{\pgfqpoint{4.186759in}{0.581680in}}%
\pgfpathlineto{\pgfqpoint{4.190872in}{0.569360in}}%
\pgfpathlineto{\pgfqpoint{4.192517in}{0.581680in}}%
\pgfpathlineto{\pgfqpoint{4.197453in}{0.581680in}}%
\pgfpathlineto{\pgfqpoint{4.198275in}{0.587840in}}%
\pgfpathlineto{\pgfqpoint{4.201566in}{0.581680in}}%
\pgfpathlineto{\pgfqpoint{4.202388in}{0.581680in}}%
\pgfpathlineto{\pgfqpoint{4.203211in}{0.575520in}}%
\pgfpathlineto{\pgfqpoint{4.204033in}{0.581680in}}%
\pgfpathlineto{\pgfqpoint{4.206501in}{0.581680in}}%
\pgfpathlineto{\pgfqpoint{4.208146in}{0.594000in}}%
\pgfpathlineto{\pgfqpoint{4.209792in}{0.587840in}}%
\pgfpathlineto{\pgfqpoint{4.213082in}{0.594000in}}%
\pgfpathlineto{\pgfqpoint{4.213904in}{0.594000in}}%
\pgfpathlineto{\pgfqpoint{4.214727in}{0.587840in}}%
\pgfpathlineto{\pgfqpoint{4.215550in}{0.594000in}}%
\pgfpathlineto{\pgfqpoint{4.219663in}{0.587840in}}%
\pgfpathlineto{\pgfqpoint{4.225421in}{0.587840in}}%
\pgfpathlineto{\pgfqpoint{4.226243in}{0.581680in}}%
\pgfpathlineto{\pgfqpoint{4.227066in}{0.587840in}}%
\pgfpathlineto{\pgfqpoint{4.229534in}{0.587840in}}%
\pgfpathlineto{\pgfqpoint{4.231179in}{0.594000in}}%
\pgfpathlineto{\pgfqpoint{4.232002in}{0.587840in}}%
\pgfpathlineto{\pgfqpoint{4.232824in}{0.594000in}}%
\pgfpathlineto{\pgfqpoint{4.236114in}{0.594000in}}%
\pgfpathlineto{\pgfqpoint{4.237760in}{0.587840in}}%
\pgfpathlineto{\pgfqpoint{4.238582in}{0.587840in}}%
\pgfpathlineto{\pgfqpoint{4.241873in}{0.575520in}}%
\pgfpathlineto{\pgfqpoint{4.242695in}{0.587840in}}%
\pgfpathlineto{\pgfqpoint{4.243518in}{0.581680in}}%
\pgfpathlineto{\pgfqpoint{4.244340in}{0.600160in}}%
\pgfpathlineto{\pgfqpoint{4.246808in}{0.587840in}}%
\pgfpathlineto{\pgfqpoint{4.247631in}{0.594000in}}%
\pgfpathlineto{\pgfqpoint{4.248453in}{0.581680in}}%
\pgfpathlineto{\pgfqpoint{4.250099in}{0.587840in}}%
\pgfpathlineto{\pgfqpoint{4.252566in}{0.581680in}}%
\pgfpathlineto{\pgfqpoint{4.255034in}{0.594000in}}%
\pgfpathlineto{\pgfqpoint{4.259147in}{0.594000in}}%
\pgfpathlineto{\pgfqpoint{4.259970in}{0.587840in}}%
\pgfpathlineto{\pgfqpoint{4.261615in}{0.600160in}}%
\pgfpathlineto{\pgfqpoint{4.264083in}{0.594000in}}%
\pgfpathlineto{\pgfqpoint{4.264905in}{0.600160in}}%
\pgfpathlineto{\pgfqpoint{4.265728in}{0.594000in}}%
\pgfpathlineto{\pgfqpoint{4.267373in}{0.606320in}}%
\pgfpathlineto{\pgfqpoint{4.270663in}{0.594000in}}%
\pgfpathlineto{\pgfqpoint{4.273131in}{0.594000in}}%
\pgfpathlineto{\pgfqpoint{4.278889in}{0.606320in}}%
\pgfpathlineto{\pgfqpoint{4.282180in}{0.606320in}}%
\pgfpathlineto{\pgfqpoint{4.283002in}{0.600160in}}%
\pgfpathlineto{\pgfqpoint{4.284647in}{0.606320in}}%
\pgfpathlineto{\pgfqpoint{4.287115in}{0.600160in}}%
\pgfpathlineto{\pgfqpoint{4.288760in}{0.594000in}}%
\pgfpathlineto{\pgfqpoint{4.289583in}{0.606320in}}%
\pgfpathlineto{\pgfqpoint{4.290406in}{0.600160in}}%
\pgfpathlineto{\pgfqpoint{4.296164in}{0.600160in}}%
\pgfpathlineto{\pgfqpoint{4.300277in}{0.606320in}}%
\pgfpathlineto{\pgfqpoint{4.301922in}{0.606320in}}%
\pgfpathlineto{\pgfqpoint{4.304390in}{0.600160in}}%
\pgfpathlineto{\pgfqpoint{4.305212in}{0.606320in}}%
\pgfpathlineto{\pgfqpoint{4.307680in}{0.594000in}}%
\pgfpathlineto{\pgfqpoint{4.310148in}{0.600160in}}%
\pgfpathlineto{\pgfqpoint{4.312616in}{0.612480in}}%
\pgfpathlineto{\pgfqpoint{4.313438in}{0.606320in}}%
\pgfpathlineto{\pgfqpoint{4.315906in}{0.600160in}}%
\pgfpathlineto{\pgfqpoint{4.317551in}{0.612480in}}%
\pgfpathlineto{\pgfqpoint{4.319196in}{0.606320in}}%
\pgfpathlineto{\pgfqpoint{4.324955in}{0.606320in}}%
\pgfpathlineto{\pgfqpoint{4.327422in}{0.600160in}}%
\pgfpathlineto{\pgfqpoint{4.329067in}{0.606320in}}%
\pgfpathlineto{\pgfqpoint{4.329890in}{0.606320in}}%
\pgfpathlineto{\pgfqpoint{4.334003in}{0.600160in}}%
\pgfpathlineto{\pgfqpoint{4.335648in}{0.600160in}}%
\pgfpathlineto{\pgfqpoint{4.336471in}{0.612480in}}%
\pgfpathlineto{\pgfqpoint{4.338939in}{0.606320in}}%
\pgfpathlineto{\pgfqpoint{4.339761in}{0.612480in}}%
\pgfpathlineto{\pgfqpoint{4.342229in}{0.600160in}}%
\pgfpathlineto{\pgfqpoint{4.345519in}{0.600160in}}%
\pgfpathlineto{\pgfqpoint{4.346342in}{0.594000in}}%
\pgfpathlineto{\pgfqpoint{4.347165in}{0.606320in}}%
\pgfpathlineto{\pgfqpoint{4.347987in}{0.594000in}}%
\pgfpathlineto{\pgfqpoint{4.350455in}{0.587840in}}%
\pgfpathlineto{\pgfqpoint{4.351277in}{0.594000in}}%
\pgfpathlineto{\pgfqpoint{4.353745in}{0.581680in}}%
\pgfpathlineto{\pgfqpoint{4.357036in}{0.575520in}}%
\pgfpathlineto{\pgfqpoint{4.357858in}{0.581680in}}%
\pgfpathlineto{\pgfqpoint{4.358681in}{0.606320in}}%
\pgfpathlineto{\pgfqpoint{4.361971in}{0.557040in}}%
\pgfpathlineto{\pgfqpoint{4.363616in}{0.569360in}}%
\pgfpathlineto{\pgfqpoint{4.365262in}{0.581680in}}%
\pgfpathlineto{\pgfqpoint{4.367729in}{0.581680in}}%
\pgfpathlineto{\pgfqpoint{4.370197in}{0.630960in}}%
\pgfpathlineto{\pgfqpoint{4.374310in}{0.600160in}}%
\pgfpathlineto{\pgfqpoint{4.375133in}{0.606320in}}%
\pgfpathlineto{\pgfqpoint{4.375955in}{0.600160in}}%
\pgfpathlineto{\pgfqpoint{4.380068in}{0.624800in}}%
\pgfpathlineto{\pgfqpoint{4.380891in}{0.624800in}}%
\pgfpathlineto{\pgfqpoint{4.382536in}{0.630960in}}%
\pgfpathlineto{\pgfqpoint{4.385826in}{0.606320in}}%
\pgfpathlineto{\pgfqpoint{4.387472in}{0.637120in}}%
\pgfpathlineto{\pgfqpoint{4.388294in}{0.624800in}}%
\pgfpathlineto{\pgfqpoint{4.391585in}{0.600160in}}%
\pgfpathlineto{\pgfqpoint{4.392407in}{0.612480in}}%
\pgfpathlineto{\pgfqpoint{4.393230in}{0.649440in}}%
\pgfpathlineto{\pgfqpoint{4.394052in}{0.643280in}}%
\pgfpathlineto{\pgfqpoint{4.396520in}{0.637120in}}%
\pgfpathlineto{\pgfqpoint{4.397343in}{0.649440in}}%
\pgfpathlineto{\pgfqpoint{4.398988in}{0.637120in}}%
\pgfpathlineto{\pgfqpoint{4.399810in}{0.649440in}}%
\pgfpathlineto{\pgfqpoint{4.403101in}{0.637120in}}%
\pgfpathlineto{\pgfqpoint{4.404746in}{0.649440in}}%
\pgfpathlineto{\pgfqpoint{4.408036in}{0.630960in}}%
\pgfpathlineto{\pgfqpoint{4.410504in}{0.643280in}}%
\pgfpathlineto{\pgfqpoint{4.411327in}{0.630960in}}%
\pgfpathlineto{\pgfqpoint{4.414617in}{0.630960in}}%
\pgfpathlineto{\pgfqpoint{4.415440in}{0.637120in}}%
\pgfpathlineto{\pgfqpoint{4.417085in}{0.630960in}}%
\pgfpathlineto{\pgfqpoint{4.420375in}{0.624800in}}%
\pgfpathlineto{\pgfqpoint{4.422020in}{0.637120in}}%
\pgfpathlineto{\pgfqpoint{4.422843in}{0.630960in}}%
\pgfpathlineto{\pgfqpoint{4.425311in}{0.624800in}}%
\pgfpathlineto{\pgfqpoint{4.428601in}{0.594000in}}%
\pgfpathlineto{\pgfqpoint{4.431069in}{0.594000in}}%
\pgfpathlineto{\pgfqpoint{4.433537in}{0.618640in}}%
\pgfpathlineto{\pgfqpoint{4.434359in}{0.600160in}}%
\pgfpathlineto{\pgfqpoint{4.436827in}{0.575520in}}%
\pgfpathlineto{\pgfqpoint{4.437650in}{0.581680in}}%
\pgfpathlineto{\pgfqpoint{4.439295in}{0.575520in}}%
\pgfpathlineto{\pgfqpoint{4.440118in}{0.581680in}}%
\pgfpathlineto{\pgfqpoint{4.442585in}{0.569360in}}%
\pgfpathlineto{\pgfqpoint{4.445876in}{0.606320in}}%
\pgfpathlineto{\pgfqpoint{4.449166in}{0.587840in}}%
\pgfpathlineto{\pgfqpoint{4.449989in}{0.575520in}}%
\pgfpathlineto{\pgfqpoint{4.451634in}{0.587840in}}%
\pgfpathlineto{\pgfqpoint{4.454102in}{0.563200in}}%
\pgfpathlineto{\pgfqpoint{4.457392in}{0.587840in}}%
\pgfpathlineto{\pgfqpoint{4.459860in}{0.557040in}}%
\pgfpathlineto{\pgfqpoint{4.463150in}{0.618640in}}%
\pgfpathlineto{\pgfqpoint{4.465618in}{0.600160in}}%
\pgfpathlineto{\pgfqpoint{4.466440in}{0.612480in}}%
\pgfpathlineto{\pgfqpoint{4.467263in}{0.600160in}}%
\pgfpathlineto{\pgfqpoint{4.468908in}{0.686400in}}%
\pgfpathlineto{\pgfqpoint{4.471376in}{0.655600in}}%
\pgfpathlineto{\pgfqpoint{4.473844in}{0.686400in}}%
\pgfpathlineto{\pgfqpoint{4.477957in}{0.674080in}}%
\pgfpathlineto{\pgfqpoint{4.478779in}{0.686400in}}%
\pgfpathlineto{\pgfqpoint{4.479602in}{0.717200in}}%
\pgfpathlineto{\pgfqpoint{4.482892in}{0.680240in}}%
\pgfpathlineto{\pgfqpoint{4.483715in}{0.698720in}}%
\pgfpathlineto{\pgfqpoint{4.484538in}{0.692560in}}%
\pgfpathlineto{\pgfqpoint{4.486183in}{0.698720in}}%
\pgfpathlineto{\pgfqpoint{4.490296in}{0.686400in}}%
\pgfpathlineto{\pgfqpoint{4.491118in}{0.667920in}}%
\pgfpathlineto{\pgfqpoint{4.491941in}{0.680240in}}%
\pgfpathlineto{\pgfqpoint{4.496876in}{0.649440in}}%
\pgfpathlineto{\pgfqpoint{4.497699in}{0.637120in}}%
\pgfpathlineto{\pgfqpoint{4.501812in}{0.624800in}}%
\pgfpathlineto{\pgfqpoint{4.502635in}{0.612480in}}%
\pgfpathlineto{\pgfqpoint{4.503457in}{0.618640in}}%
\pgfpathlineto{\pgfqpoint{4.505925in}{0.606320in}}%
\pgfpathlineto{\pgfqpoint{4.507570in}{0.618640in}}%
\pgfpathlineto{\pgfqpoint{4.509215in}{0.643280in}}%
\pgfpathlineto{\pgfqpoint{4.511683in}{0.624800in}}%
\pgfpathlineto{\pgfqpoint{4.512506in}{0.637120in}}%
\pgfpathlineto{\pgfqpoint{4.513328in}{0.630960in}}%
\pgfpathlineto{\pgfqpoint{4.514151in}{0.649440in}}%
\pgfpathlineto{\pgfqpoint{4.514973in}{0.643280in}}%
\pgfpathlineto{\pgfqpoint{4.519086in}{0.637120in}}%
\pgfpathlineto{\pgfqpoint{4.520732in}{0.637120in}}%
\pgfpathlineto{\pgfqpoint{4.523199in}{0.618640in}}%
\pgfpathlineto{\pgfqpoint{4.524022in}{0.630960in}}%
\pgfpathlineto{\pgfqpoint{4.524845in}{0.624800in}}%
\pgfpathlineto{\pgfqpoint{4.526490in}{0.637120in}}%
\pgfpathlineto{\pgfqpoint{4.530603in}{0.612480in}}%
\pgfpathlineto{\pgfqpoint{4.531425in}{0.667920in}}%
\pgfpathlineto{\pgfqpoint{4.532248in}{0.655600in}}%
\pgfpathlineto{\pgfqpoint{4.534716in}{0.637120in}}%
\pgfpathlineto{\pgfqpoint{4.535538in}{0.655600in}}%
\pgfpathlineto{\pgfqpoint{4.538006in}{0.630960in}}%
\pgfpathlineto{\pgfqpoint{4.541296in}{0.624800in}}%
\pgfpathlineto{\pgfqpoint{4.542942in}{0.643280in}}%
\pgfpathlineto{\pgfqpoint{4.543764in}{0.643280in}}%
\pgfpathlineto{\pgfqpoint{4.546232in}{0.630960in}}%
\pgfpathlineto{\pgfqpoint{4.547877in}{0.643280in}}%
\pgfpathlineto{\pgfqpoint{4.552813in}{0.686400in}}%
\pgfpathlineto{\pgfqpoint{4.553635in}{0.704880in}}%
\pgfpathlineto{\pgfqpoint{4.554458in}{0.698720in}}%
\pgfpathlineto{\pgfqpoint{4.555281in}{0.686400in}}%
\pgfpathlineto{\pgfqpoint{4.563506in}{0.686400in}}%
\pgfpathlineto{\pgfqpoint{4.564329in}{0.692560in}}%
\pgfpathlineto{\pgfqpoint{4.565152in}{0.686400in}}%
\pgfpathlineto{\pgfqpoint{4.566797in}{0.692560in}}%
\pgfpathlineto{\pgfqpoint{4.569265in}{0.692560in}}%
\pgfpathlineto{\pgfqpoint{4.570087in}{0.698720in}}%
\pgfpathlineto{\pgfqpoint{4.571732in}{0.692560in}}%
\pgfpathlineto{\pgfqpoint{4.572555in}{0.692560in}}%
\pgfpathlineto{\pgfqpoint{4.575023in}{0.674080in}}%
\pgfpathlineto{\pgfqpoint{4.578313in}{0.698720in}}%
\pgfpathlineto{\pgfqpoint{4.581603in}{0.698720in}}%
\pgfpathlineto{\pgfqpoint{4.582426in}{0.711040in}}%
\pgfpathlineto{\pgfqpoint{4.583249in}{0.704880in}}%
\pgfpathlineto{\pgfqpoint{4.584071in}{0.711040in}}%
\pgfpathlineto{\pgfqpoint{4.589007in}{0.661760in}}%
\pgfpathlineto{\pgfqpoint{4.589829in}{0.655600in}}%
\pgfpathlineto{\pgfqpoint{4.593120in}{0.655600in}}%
\pgfpathlineto{\pgfqpoint{4.593942in}{0.649440in}}%
\pgfpathlineto{\pgfqpoint{4.594765in}{0.655600in}}%
\pgfpathlineto{\pgfqpoint{4.598055in}{0.630960in}}%
\pgfpathlineto{\pgfqpoint{4.599701in}{0.649440in}}%
\pgfpathlineto{\pgfqpoint{4.600523in}{0.667920in}}%
\pgfpathlineto{\pgfqpoint{4.601346in}{0.661760in}}%
\pgfpathlineto{\pgfqpoint{4.604636in}{0.655600in}}%
\pgfpathlineto{\pgfqpoint{4.606281in}{0.686400in}}%
\pgfpathlineto{\pgfqpoint{4.607104in}{0.680240in}}%
\pgfpathlineto{\pgfqpoint{4.610394in}{0.674080in}}%
\pgfpathlineto{\pgfqpoint{4.611217in}{0.698720in}}%
\pgfpathlineto{\pgfqpoint{4.612039in}{0.692560in}}%
\pgfpathlineto{\pgfqpoint{4.612862in}{0.692560in}}%
\pgfpathlineto{\pgfqpoint{4.616152in}{0.674080in}}%
\pgfpathlineto{\pgfqpoint{4.616975in}{0.686400in}}%
\pgfpathlineto{\pgfqpoint{4.618620in}{0.674080in}}%
\pgfpathlineto{\pgfqpoint{4.621911in}{0.667920in}}%
\pgfpathlineto{\pgfqpoint{4.622733in}{0.667920in}}%
\pgfpathlineto{\pgfqpoint{4.623556in}{0.649440in}}%
\pgfpathlineto{\pgfqpoint{4.624378in}{0.661760in}}%
\pgfpathlineto{\pgfqpoint{4.626846in}{0.661760in}}%
\pgfpathlineto{\pgfqpoint{4.629314in}{0.680240in}}%
\pgfpathlineto{\pgfqpoint{4.630136in}{0.674080in}}%
\pgfpathlineto{\pgfqpoint{4.633427in}{0.674080in}}%
\pgfpathlineto{\pgfqpoint{4.634249in}{0.680240in}}%
\pgfpathlineto{\pgfqpoint{4.635072in}{0.667920in}}%
\pgfpathlineto{\pgfqpoint{4.635895in}{0.686400in}}%
\pgfpathlineto{\pgfqpoint{4.638362in}{0.661760in}}%
\pgfpathlineto{\pgfqpoint{4.639185in}{0.674080in}}%
\pgfpathlineto{\pgfqpoint{4.640830in}{0.667920in}}%
\pgfpathlineto{\pgfqpoint{4.641653in}{0.674080in}}%
\pgfpathlineto{\pgfqpoint{4.644121in}{0.674080in}}%
\pgfpathlineto{\pgfqpoint{4.644943in}{0.680240in}}%
\pgfpathlineto{\pgfqpoint{4.645766in}{0.674080in}}%
\pgfpathlineto{\pgfqpoint{4.647411in}{0.686400in}}%
\pgfpathlineto{\pgfqpoint{4.649879in}{0.692560in}}%
\pgfpathlineto{\pgfqpoint{4.651524in}{0.711040in}}%
\pgfpathlineto{\pgfqpoint{4.652346in}{0.754160in}}%
\pgfpathlineto{\pgfqpoint{4.653169in}{0.741840in}}%
\pgfpathlineto{\pgfqpoint{4.655637in}{0.729520in}}%
\pgfpathlineto{\pgfqpoint{4.657282in}{0.748000in}}%
\pgfpathlineto{\pgfqpoint{4.658105in}{0.766480in}}%
\pgfpathlineto{\pgfqpoint{4.658927in}{0.760320in}}%
\pgfpathlineto{\pgfqpoint{4.661395in}{0.748000in}}%
\pgfpathlineto{\pgfqpoint{4.663040in}{0.772640in}}%
\pgfpathlineto{\pgfqpoint{4.664685in}{0.772640in}}%
\pgfpathlineto{\pgfqpoint{4.667153in}{0.741840in}}%
\pgfpathlineto{\pgfqpoint{4.667976in}{0.766480in}}%
\pgfpathlineto{\pgfqpoint{4.668798in}{0.754160in}}%
\pgfpathlineto{\pgfqpoint{4.670444in}{0.766480in}}%
\pgfpathlineto{\pgfqpoint{4.672911in}{0.772640in}}%
\pgfpathlineto{\pgfqpoint{4.676202in}{0.748000in}}%
\pgfpathlineto{\pgfqpoint{4.680315in}{0.754160in}}%
\pgfpathlineto{\pgfqpoint{4.681137in}{0.754160in}}%
\pgfpathlineto{\pgfqpoint{4.681960in}{0.760320in}}%
\pgfpathlineto{\pgfqpoint{4.685250in}{0.760320in}}%
\pgfpathlineto{\pgfqpoint{4.686073in}{0.741840in}}%
\pgfpathlineto{\pgfqpoint{4.687718in}{0.754160in}}%
\pgfpathlineto{\pgfqpoint{4.691008in}{0.741840in}}%
\pgfpathlineto{\pgfqpoint{4.693476in}{0.791120in}}%
\pgfpathlineto{\pgfqpoint{4.695944in}{0.784960in}}%
\pgfpathlineto{\pgfqpoint{4.696766in}{0.791120in}}%
\pgfpathlineto{\pgfqpoint{4.698412in}{0.858880in}}%
\pgfpathlineto{\pgfqpoint{4.699234in}{0.852720in}}%
\pgfpathlineto{\pgfqpoint{4.703347in}{0.791120in}}%
\pgfpathlineto{\pgfqpoint{4.704170in}{0.803440in}}%
\pgfpathlineto{\pgfqpoint{4.704992in}{0.791120in}}%
\pgfpathlineto{\pgfqpoint{4.708283in}{0.766480in}}%
\pgfpathlineto{\pgfqpoint{4.709928in}{0.778800in}}%
\pgfpathlineto{\pgfqpoint{4.710751in}{0.784960in}}%
\pgfpathlineto{\pgfqpoint{4.714864in}{0.760320in}}%
\pgfpathlineto{\pgfqpoint{4.715686in}{0.803440in}}%
\pgfpathlineto{\pgfqpoint{4.716509in}{0.791120in}}%
\pgfpathlineto{\pgfqpoint{4.719799in}{0.772640in}}%
\pgfpathlineto{\pgfqpoint{4.721444in}{0.772640in}}%
\pgfpathlineto{\pgfqpoint{4.722267in}{0.778800in}}%
\pgfpathlineto{\pgfqpoint{4.725557in}{0.778800in}}%
\pgfpathlineto{\pgfqpoint{4.726380in}{0.784960in}}%
\pgfpathlineto{\pgfqpoint{4.728025in}{0.809600in}}%
\pgfpathlineto{\pgfqpoint{4.732138in}{0.760320in}}%
\pgfpathlineto{\pgfqpoint{4.738719in}{0.704880in}}%
\pgfpathlineto{\pgfqpoint{4.742009in}{0.667920in}}%
\pgfpathlineto{\pgfqpoint{4.745299in}{0.741840in}}%
\pgfpathlineto{\pgfqpoint{4.749412in}{0.704880in}}%
\pgfpathlineto{\pgfqpoint{4.751058in}{0.723360in}}%
\pgfpathlineto{\pgfqpoint{4.753525in}{0.723360in}}%
\pgfpathlineto{\pgfqpoint{4.755171in}{0.717200in}}%
\pgfpathlineto{\pgfqpoint{4.756816in}{0.729520in}}%
\pgfpathlineto{\pgfqpoint{4.760106in}{0.723360in}}%
\pgfpathlineto{\pgfqpoint{4.760929in}{0.711040in}}%
\pgfpathlineto{\pgfqpoint{4.761751in}{0.717200in}}%
\pgfpathlineto{\pgfqpoint{4.762574in}{0.704880in}}%
\pgfpathlineto{\pgfqpoint{4.765042in}{0.717200in}}%
\pgfpathlineto{\pgfqpoint{4.765864in}{0.698720in}}%
\pgfpathlineto{\pgfqpoint{4.766687in}{0.704880in}}%
\pgfpathlineto{\pgfqpoint{4.768332in}{0.748000in}}%
\pgfpathlineto{\pgfqpoint{4.770800in}{0.729520in}}%
\pgfpathlineto{\pgfqpoint{4.773268in}{0.748000in}}%
\pgfpathlineto{\pgfqpoint{4.774090in}{0.741840in}}%
\pgfpathlineto{\pgfqpoint{4.778203in}{0.735680in}}%
\pgfpathlineto{\pgfqpoint{4.779026in}{0.772640in}}%
\pgfpathlineto{\pgfqpoint{4.779848in}{0.748000in}}%
\pgfpathlineto{\pgfqpoint{4.783139in}{0.729520in}}%
\pgfpathlineto{\pgfqpoint{4.784784in}{0.748000in}}%
\pgfpathlineto{\pgfqpoint{4.785607in}{0.741840in}}%
\pgfpathlineto{\pgfqpoint{4.788897in}{0.754160in}}%
\pgfpathlineto{\pgfqpoint{4.790542in}{0.754160in}}%
\pgfpathlineto{\pgfqpoint{4.791365in}{0.748000in}}%
\pgfpathlineto{\pgfqpoint{4.794655in}{0.735680in}}%
\pgfpathlineto{\pgfqpoint{4.797123in}{0.778800in}}%
\pgfpathlineto{\pgfqpoint{4.800413in}{0.766480in}}%
\pgfpathlineto{\pgfqpoint{4.801236in}{0.778800in}}%
\pgfpathlineto{\pgfqpoint{4.802881in}{0.766480in}}%
\pgfpathlineto{\pgfqpoint{4.805349in}{0.760320in}}%
\pgfpathlineto{\pgfqpoint{4.806171in}{0.754160in}}%
\pgfpathlineto{\pgfqpoint{4.806994in}{0.766480in}}%
\pgfpathlineto{\pgfqpoint{4.807817in}{0.760320in}}%
\pgfpathlineto{\pgfqpoint{4.808639in}{0.766480in}}%
\pgfpathlineto{\pgfqpoint{4.812752in}{0.748000in}}%
\pgfpathlineto{\pgfqpoint{4.814397in}{0.748000in}}%
\pgfpathlineto{\pgfqpoint{4.816865in}{0.760320in}}%
\pgfpathlineto{\pgfqpoint{4.817688in}{0.748000in}}%
\pgfpathlineto{\pgfqpoint{4.818510in}{0.754160in}}%
\pgfpathlineto{\pgfqpoint{4.819333in}{0.772640in}}%
\pgfpathlineto{\pgfqpoint{4.820155in}{0.766480in}}%
\pgfpathlineto{\pgfqpoint{4.823446in}{0.760320in}}%
\pgfpathlineto{\pgfqpoint{4.825914in}{0.772640in}}%
\pgfpathlineto{\pgfqpoint{4.828381in}{0.778800in}}%
\pgfpathlineto{\pgfqpoint{4.829204in}{0.766480in}}%
\pgfpathlineto{\pgfqpoint{4.831672in}{0.686400in}}%
\pgfpathlineto{\pgfqpoint{4.834139in}{0.674080in}}%
\pgfpathlineto{\pgfqpoint{4.834962in}{0.661760in}}%
\pgfpathlineto{\pgfqpoint{4.836607in}{0.686400in}}%
\pgfpathlineto{\pgfqpoint{4.839898in}{0.655600in}}%
\pgfpathlineto{\pgfqpoint{4.840720in}{0.667920in}}%
\pgfpathlineto{\pgfqpoint{4.842365in}{0.704880in}}%
\pgfpathlineto{\pgfqpoint{4.843188in}{0.704880in}}%
\pgfpathlineto{\pgfqpoint{4.847301in}{0.686400in}}%
\pgfpathlineto{\pgfqpoint{4.848124in}{0.686400in}}%
\pgfpathlineto{\pgfqpoint{4.848946in}{0.674080in}}%
\pgfpathlineto{\pgfqpoint{4.853882in}{0.674080in}}%
\pgfpathlineto{\pgfqpoint{4.854704in}{0.686400in}}%
\pgfpathlineto{\pgfqpoint{4.858817in}{0.661760in}}%
\pgfpathlineto{\pgfqpoint{4.859640in}{0.674080in}}%
\pgfpathlineto{\pgfqpoint{4.860462in}{0.655600in}}%
\pgfpathlineto{\pgfqpoint{4.863753in}{0.655600in}}%
\pgfpathlineto{\pgfqpoint{4.865398in}{0.661760in}}%
\pgfpathlineto{\pgfqpoint{4.869511in}{0.655600in}}%
\pgfpathlineto{\pgfqpoint{4.871979in}{0.674080in}}%
\pgfpathlineto{\pgfqpoint{4.874447in}{0.667920in}}%
\pgfpathlineto{\pgfqpoint{4.876092in}{0.655600in}}%
\pgfpathlineto{\pgfqpoint{4.877737in}{0.674080in}}%
\pgfpathlineto{\pgfqpoint{4.881027in}{0.661760in}}%
\pgfpathlineto{\pgfqpoint{4.883495in}{0.624800in}}%
\pgfpathlineto{\pgfqpoint{4.886785in}{0.612480in}}%
\pgfpathlineto{\pgfqpoint{4.889253in}{0.637120in}}%
\pgfpathlineto{\pgfqpoint{4.892544in}{0.618640in}}%
\pgfpathlineto{\pgfqpoint{4.893366in}{0.618640in}}%
\pgfpathlineto{\pgfqpoint{4.894189in}{0.600160in}}%
\pgfpathlineto{\pgfqpoint{4.895011in}{0.618640in}}%
\pgfpathlineto{\pgfqpoint{4.897479in}{0.606320in}}%
\pgfpathlineto{\pgfqpoint{4.899124in}{0.612480in}}%
\pgfpathlineto{\pgfqpoint{4.904060in}{0.612480in}}%
\pgfpathlineto{\pgfqpoint{4.904882in}{0.594000in}}%
\pgfpathlineto{\pgfqpoint{4.905705in}{0.600160in}}%
\pgfpathlineto{\pgfqpoint{4.906528in}{0.600160in}}%
\pgfpathlineto{\pgfqpoint{4.908995in}{0.606320in}}%
\pgfpathlineto{\pgfqpoint{4.911463in}{0.587840in}}%
\pgfpathlineto{\pgfqpoint{4.914754in}{0.606320in}}%
\pgfpathlineto{\pgfqpoint{4.916399in}{0.600160in}}%
\pgfpathlineto{\pgfqpoint{4.917221in}{0.606320in}}%
\pgfpathlineto{\pgfqpoint{4.918044in}{0.600160in}}%
\pgfpathlineto{\pgfqpoint{4.920512in}{0.606320in}}%
\pgfpathlineto{\pgfqpoint{4.922157in}{0.637120in}}%
\pgfpathlineto{\pgfqpoint{4.922980in}{0.661760in}}%
\pgfpathlineto{\pgfqpoint{4.923802in}{0.649440in}}%
\pgfpathlineto{\pgfqpoint{4.928738in}{0.612480in}}%
\pgfpathlineto{\pgfqpoint{4.929560in}{0.618640in}}%
\pgfpathlineto{\pgfqpoint{4.932028in}{0.600160in}}%
\pgfpathlineto{\pgfqpoint{4.935318in}{0.612480in}}%
\pgfpathlineto{\pgfqpoint{4.938609in}{0.606320in}}%
\pgfpathlineto{\pgfqpoint{4.940254in}{0.606320in}}%
\pgfpathlineto{\pgfqpoint{4.941077in}{0.612480in}}%
\pgfpathlineto{\pgfqpoint{4.944367in}{0.600160in}}%
\pgfpathlineto{\pgfqpoint{4.946835in}{0.544720in}}%
\pgfpathlineto{\pgfqpoint{4.950125in}{0.538560in}}%
\pgfpathlineto{\pgfqpoint{4.952593in}{0.594000in}}%
\pgfpathlineto{\pgfqpoint{4.955061in}{0.600160in}}%
\pgfpathlineto{\pgfqpoint{4.956706in}{0.594000in}}%
\pgfpathlineto{\pgfqpoint{4.958351in}{0.594000in}}%
\pgfpathlineto{\pgfqpoint{4.960819in}{0.624800in}}%
\pgfpathlineto{\pgfqpoint{4.962464in}{0.606320in}}%
\pgfpathlineto{\pgfqpoint{4.964109in}{0.624800in}}%
\pgfpathlineto{\pgfqpoint{4.966577in}{0.624800in}}%
\pgfpathlineto{\pgfqpoint{4.967400in}{0.618640in}}%
\pgfpathlineto{\pgfqpoint{4.969045in}{0.649440in}}%
\pgfpathlineto{\pgfqpoint{4.969867in}{0.630960in}}%
\pgfpathlineto{\pgfqpoint{4.972335in}{0.637120in}}%
\pgfpathlineto{\pgfqpoint{4.973980in}{0.649440in}}%
\pgfpathlineto{\pgfqpoint{4.975625in}{0.630960in}}%
\pgfpathlineto{\pgfqpoint{4.979738in}{0.630960in}}%
\pgfpathlineto{\pgfqpoint{4.980561in}{0.643280in}}%
\pgfpathlineto{\pgfqpoint{4.981384in}{0.637120in}}%
\pgfpathlineto{\pgfqpoint{4.984674in}{0.630960in}}%
\pgfpathlineto{\pgfqpoint{4.985497in}{0.606320in}}%
\pgfpathlineto{\pgfqpoint{4.987142in}{0.618640in}}%
\pgfpathlineto{\pgfqpoint{4.989610in}{0.612480in}}%
\pgfpathlineto{\pgfqpoint{4.991255in}{0.624800in}}%
\pgfpathlineto{\pgfqpoint{4.992900in}{0.618640in}}%
\pgfpathlineto{\pgfqpoint{4.995368in}{0.612480in}}%
\pgfpathlineto{\pgfqpoint{4.997013in}{0.624800in}}%
\pgfpathlineto{\pgfqpoint{4.998658in}{0.637120in}}%
\pgfpathlineto{\pgfqpoint{5.002771in}{0.612480in}}%
\pgfpathlineto{\pgfqpoint{5.006884in}{0.649440in}}%
\pgfpathlineto{\pgfqpoint{5.008529in}{0.618640in}}%
\pgfpathlineto{\pgfqpoint{5.010174in}{0.624800in}}%
\pgfpathlineto{\pgfqpoint{5.013465in}{0.624800in}}%
\pgfpathlineto{\pgfqpoint{5.014287in}{0.612480in}}%
\pgfpathlineto{\pgfqpoint{5.015110in}{0.618640in}}%
\pgfpathlineto{\pgfqpoint{5.018400in}{0.600160in}}%
\pgfpathlineto{\pgfqpoint{5.020045in}{0.618640in}}%
\pgfpathlineto{\pgfqpoint{5.021691in}{0.606320in}}%
\pgfpathlineto{\pgfqpoint{5.024158in}{0.618640in}}%
\pgfpathlineto{\pgfqpoint{5.025804in}{0.606320in}}%
\pgfpathlineto{\pgfqpoint{5.027449in}{0.624800in}}%
\pgfpathlineto{\pgfqpoint{5.029917in}{0.612480in}}%
\pgfpathlineto{\pgfqpoint{5.031562in}{0.575520in}}%
\pgfpathlineto{\pgfqpoint{5.032384in}{0.587840in}}%
\pgfpathlineto{\pgfqpoint{5.033207in}{0.581680in}}%
\pgfpathlineto{\pgfqpoint{5.035675in}{0.581680in}}%
\pgfpathlineto{\pgfqpoint{5.037320in}{0.563200in}}%
\pgfpathlineto{\pgfqpoint{5.038143in}{0.557040in}}%
\pgfpathlineto{\pgfqpoint{5.038965in}{0.563200in}}%
\pgfpathlineto{\pgfqpoint{5.041433in}{0.563200in}}%
\pgfpathlineto{\pgfqpoint{5.042255in}{0.550880in}}%
\pgfpathlineto{\pgfqpoint{5.043901in}{0.557040in}}%
\pgfpathlineto{\pgfqpoint{5.044723in}{0.550880in}}%
\pgfpathlineto{\pgfqpoint{5.048014in}{0.544720in}}%
\pgfpathlineto{\pgfqpoint{5.050481in}{0.569360in}}%
\pgfpathlineto{\pgfqpoint{5.052949in}{0.581680in}}%
\pgfpathlineto{\pgfqpoint{5.053772in}{0.563200in}}%
\pgfpathlineto{\pgfqpoint{5.054594in}{0.569360in}}%
\pgfpathlineto{\pgfqpoint{5.056240in}{0.587840in}}%
\pgfpathlineto{\pgfqpoint{5.058707in}{0.581680in}}%
\pgfpathlineto{\pgfqpoint{5.061175in}{0.606320in}}%
\pgfpathlineto{\pgfqpoint{5.061998in}{0.618640in}}%
\pgfpathlineto{\pgfqpoint{5.065288in}{0.594000in}}%
\pgfpathlineto{\pgfqpoint{5.066111in}{0.630960in}}%
\pgfpathlineto{\pgfqpoint{5.066933in}{0.624800in}}%
\pgfpathlineto{\pgfqpoint{5.067756in}{0.637120in}}%
\pgfpathlineto{\pgfqpoint{5.070224in}{0.606320in}}%
\pgfpathlineto{\pgfqpoint{5.072691in}{0.649440in}}%
\pgfpathlineto{\pgfqpoint{5.073514in}{0.674080in}}%
\pgfpathlineto{\pgfqpoint{5.077627in}{0.606320in}}%
\pgfpathlineto{\pgfqpoint{5.079272in}{0.637120in}}%
\pgfpathlineto{\pgfqpoint{5.082563in}{0.612480in}}%
\pgfpathlineto{\pgfqpoint{5.085030in}{0.643280in}}%
\pgfpathlineto{\pgfqpoint{5.088321in}{0.612480in}}%
\pgfpathlineto{\pgfqpoint{5.089966in}{0.630960in}}%
\pgfpathlineto{\pgfqpoint{5.090788in}{0.630960in}}%
\pgfpathlineto{\pgfqpoint{5.094079in}{0.624800in}}%
\pgfpathlineto{\pgfqpoint{5.094901in}{0.637120in}}%
\pgfpathlineto{\pgfqpoint{5.095724in}{0.630960in}}%
\pgfpathlineto{\pgfqpoint{5.096547in}{0.637120in}}%
\pgfpathlineto{\pgfqpoint{5.099014in}{0.630960in}}%
\pgfpathlineto{\pgfqpoint{5.102305in}{0.674080in}}%
\pgfpathlineto{\pgfqpoint{5.104773in}{0.649440in}}%
\pgfpathlineto{\pgfqpoint{5.105595in}{0.655600in}}%
\pgfpathlineto{\pgfqpoint{5.112176in}{0.606320in}}%
\pgfpathlineto{\pgfqpoint{5.113821in}{0.618640in}}%
\pgfpathlineto{\pgfqpoint{5.116289in}{0.575520in}}%
\pgfpathlineto{\pgfqpoint{5.118757in}{0.630960in}}%
\pgfpathlineto{\pgfqpoint{5.119579in}{0.618640in}}%
\pgfpathlineto{\pgfqpoint{5.122870in}{0.600160in}}%
\pgfpathlineto{\pgfqpoint{5.125337in}{0.655600in}}%
\pgfpathlineto{\pgfqpoint{5.127805in}{0.643280in}}%
\pgfpathlineto{\pgfqpoint{5.130273in}{0.698720in}}%
\pgfpathlineto{\pgfqpoint{5.131096in}{0.692560in}}%
\pgfpathlineto{\pgfqpoint{5.133563in}{0.661760in}}%
\pgfpathlineto{\pgfqpoint{5.136031in}{0.704880in}}%
\pgfpathlineto{\pgfqpoint{5.136854in}{0.717200in}}%
\pgfpathlineto{\pgfqpoint{5.140967in}{0.698720in}}%
\pgfpathlineto{\pgfqpoint{5.142612in}{0.717200in}}%
\pgfpathlineto{\pgfqpoint{5.145080in}{0.717200in}}%
\pgfpathlineto{\pgfqpoint{5.145902in}{0.729520in}}%
\pgfpathlineto{\pgfqpoint{5.147547in}{0.809600in}}%
\pgfpathlineto{\pgfqpoint{5.150838in}{0.766480in}}%
\pgfpathlineto{\pgfqpoint{5.153306in}{0.815760in}}%
\pgfpathlineto{\pgfqpoint{5.157418in}{0.809600in}}%
\pgfpathlineto{\pgfqpoint{5.159064in}{0.834240in}}%
\pgfpathlineto{\pgfqpoint{5.159886in}{0.834240in}}%
\pgfpathlineto{\pgfqpoint{5.162354in}{0.815760in}}%
\pgfpathlineto{\pgfqpoint{5.163999in}{0.834240in}}%
\pgfpathlineto{\pgfqpoint{5.170580in}{0.791120in}}%
\pgfpathlineto{\pgfqpoint{5.171403in}{0.797280in}}%
\pgfpathlineto{\pgfqpoint{5.174693in}{0.760320in}}%
\pgfpathlineto{\pgfqpoint{5.177161in}{0.797280in}}%
\pgfpathlineto{\pgfqpoint{5.179629in}{0.760320in}}%
\pgfpathlineto{\pgfqpoint{5.182919in}{0.791120in}}%
\pgfpathlineto{\pgfqpoint{5.191145in}{0.704880in}}%
\pgfpathlineto{\pgfqpoint{5.191967in}{0.704880in}}%
\pgfpathlineto{\pgfqpoint{5.193613in}{0.717200in}}%
\pgfpathlineto{\pgfqpoint{5.194435in}{0.723360in}}%
\pgfpathlineto{\pgfqpoint{5.197726in}{0.704880in}}%
\pgfpathlineto{\pgfqpoint{5.199371in}{0.717200in}}%
\pgfpathlineto{\pgfqpoint{5.200193in}{0.723360in}}%
\pgfpathlineto{\pgfqpoint{5.204306in}{0.686400in}}%
\pgfpathlineto{\pgfqpoint{5.205951in}{0.711040in}}%
\pgfpathlineto{\pgfqpoint{5.208419in}{0.686400in}}%
\pgfpathlineto{\pgfqpoint{5.211710in}{0.711040in}}%
\pgfpathlineto{\pgfqpoint{5.214177in}{0.698720in}}%
\pgfpathlineto{\pgfqpoint{5.215000in}{0.717200in}}%
\pgfpathlineto{\pgfqpoint{5.215823in}{0.711040in}}%
\pgfpathlineto{\pgfqpoint{5.217468in}{0.704880in}}%
\pgfpathlineto{\pgfqpoint{5.219936in}{0.692560in}}%
\pgfpathlineto{\pgfqpoint{5.223226in}{0.717200in}}%
\pgfpathlineto{\pgfqpoint{5.226516in}{0.711040in}}%
\pgfpathlineto{\pgfqpoint{5.227339in}{0.711040in}}%
\pgfpathlineto{\pgfqpoint{5.228984in}{0.717200in}}%
\pgfpathlineto{\pgfqpoint{5.231452in}{0.680240in}}%
\pgfpathlineto{\pgfqpoint{5.233920in}{0.698720in}}%
\pgfpathlineto{\pgfqpoint{5.234742in}{0.686400in}}%
\pgfpathlineto{\pgfqpoint{5.237210in}{0.674080in}}%
\pgfpathlineto{\pgfqpoint{5.238855in}{0.692560in}}%
\pgfpathlineto{\pgfqpoint{5.239678in}{0.686400in}}%
\pgfpathlineto{\pgfqpoint{5.240500in}{0.686400in}}%
\pgfpathlineto{\pgfqpoint{5.242968in}{0.655600in}}%
\pgfpathlineto{\pgfqpoint{5.243791in}{0.661760in}}%
\pgfpathlineto{\pgfqpoint{5.245436in}{0.680240in}}%
\pgfpathlineto{\pgfqpoint{5.246259in}{0.680240in}}%
\pgfpathlineto{\pgfqpoint{5.248726in}{0.674080in}}%
\pgfpathlineto{\pgfqpoint{5.249549in}{0.655600in}}%
\pgfpathlineto{\pgfqpoint{5.251194in}{0.680240in}}%
\pgfpathlineto{\pgfqpoint{5.252017in}{0.674080in}}%
\pgfpathlineto{\pgfqpoint{5.254484in}{0.643280in}}%
\pgfpathlineto{\pgfqpoint{5.256130in}{0.661760in}}%
\pgfpathlineto{\pgfqpoint{5.256952in}{0.661760in}}%
\pgfpathlineto{\pgfqpoint{5.257775in}{0.667920in}}%
\pgfpathlineto{\pgfqpoint{5.261065in}{0.637120in}}%
\pgfpathlineto{\pgfqpoint{5.262710in}{0.637120in}}%
\pgfpathlineto{\pgfqpoint{5.266823in}{0.612480in}}%
\pgfpathlineto{\pgfqpoint{5.267646in}{0.612480in}}%
\pgfpathlineto{\pgfqpoint{5.269291in}{0.624800in}}%
\pgfpathlineto{\pgfqpoint{5.273404in}{0.600160in}}%
\pgfpathlineto{\pgfqpoint{5.275049in}{0.606320in}}%
\pgfpathlineto{\pgfqpoint{5.279162in}{0.575520in}}%
\pgfpathlineto{\pgfqpoint{5.280807in}{0.594000in}}%
\pgfpathlineto{\pgfqpoint{5.284920in}{0.575520in}}%
\pgfpathlineto{\pgfqpoint{5.285743in}{0.587840in}}%
\pgfpathlineto{\pgfqpoint{5.286566in}{0.581680in}}%
\pgfpathlineto{\pgfqpoint{5.290679in}{0.569360in}}%
\pgfpathlineto{\pgfqpoint{5.292324in}{0.575520in}}%
\pgfpathlineto{\pgfqpoint{5.295614in}{0.569360in}}%
\pgfpathlineto{\pgfqpoint{5.297259in}{0.587840in}}%
\pgfpathlineto{\pgfqpoint{5.300550in}{0.557040in}}%
\pgfpathlineto{\pgfqpoint{5.303017in}{0.600160in}}%
\pgfpathlineto{\pgfqpoint{5.303840in}{0.594000in}}%
\pgfpathlineto{\pgfqpoint{5.306308in}{0.569360in}}%
\pgfpathlineto{\pgfqpoint{5.308776in}{0.581680in}}%
\pgfpathlineto{\pgfqpoint{5.309598in}{0.581680in}}%
\pgfpathlineto{\pgfqpoint{5.313711in}{0.563200in}}%
\pgfpathlineto{\pgfqpoint{5.314534in}{0.575520in}}%
\pgfpathlineto{\pgfqpoint{5.318647in}{0.550880in}}%
\pgfpathlineto{\pgfqpoint{5.319469in}{0.550880in}}%
\pgfpathlineto{\pgfqpoint{5.321114in}{0.575520in}}%
\pgfpathlineto{\pgfqpoint{5.323582in}{0.557040in}}%
\pgfpathlineto{\pgfqpoint{5.326873in}{0.587840in}}%
\pgfpathlineto{\pgfqpoint{5.329340in}{0.581680in}}%
\pgfpathlineto{\pgfqpoint{5.332631in}{0.630960in}}%
\pgfpathlineto{\pgfqpoint{5.335099in}{0.606320in}}%
\pgfpathlineto{\pgfqpoint{5.338389in}{0.630960in}}%
\pgfpathlineto{\pgfqpoint{5.341679in}{0.612480in}}%
\pgfpathlineto{\pgfqpoint{5.344147in}{0.643280in}}%
\pgfpathlineto{\pgfqpoint{5.347437in}{0.630960in}}%
\pgfpathlineto{\pgfqpoint{5.349905in}{0.643280in}}%
\pgfpathlineto{\pgfqpoint{5.352373in}{0.649440in}}%
\pgfpathlineto{\pgfqpoint{5.354018in}{0.661760in}}%
\pgfpathlineto{\pgfqpoint{5.355663in}{0.698720in}}%
\pgfpathlineto{\pgfqpoint{5.358131in}{0.692560in}}%
\pgfpathlineto{\pgfqpoint{5.358954in}{0.661760in}}%
\pgfpathlineto{\pgfqpoint{5.361422in}{0.698720in}}%
\pgfpathlineto{\pgfqpoint{5.364712in}{0.692560in}}%
\pgfpathlineto{\pgfqpoint{5.365534in}{0.680240in}}%
\pgfpathlineto{\pgfqpoint{5.367180in}{0.704880in}}%
\pgfpathlineto{\pgfqpoint{5.369647in}{0.711040in}}%
\pgfpathlineto{\pgfqpoint{5.370470in}{0.717200in}}%
\pgfpathlineto{\pgfqpoint{5.371293in}{0.711040in}}%
\pgfpathlineto{\pgfqpoint{5.372938in}{0.735680in}}%
\pgfpathlineto{\pgfqpoint{5.375406in}{0.686400in}}%
\pgfpathlineto{\pgfqpoint{5.378696in}{0.717200in}}%
\pgfpathlineto{\pgfqpoint{5.381164in}{0.692560in}}%
\pgfpathlineto{\pgfqpoint{5.382809in}{0.698720in}}%
\pgfpathlineto{\pgfqpoint{5.383632in}{0.711040in}}%
\pgfpathlineto{\pgfqpoint{5.386922in}{0.680240in}}%
\pgfpathlineto{\pgfqpoint{5.387745in}{0.667920in}}%
\pgfpathlineto{\pgfqpoint{5.389390in}{0.698720in}}%
\pgfpathlineto{\pgfqpoint{5.393503in}{0.667920in}}%
\pgfpathlineto{\pgfqpoint{5.395148in}{0.692560in}}%
\pgfpathlineto{\pgfqpoint{5.395970in}{0.686400in}}%
\pgfpathlineto{\pgfqpoint{5.399261in}{0.686400in}}%
\pgfpathlineto{\pgfqpoint{5.400083in}{0.692560in}}%
\pgfpathlineto{\pgfqpoint{5.400906in}{0.711040in}}%
\pgfpathlineto{\pgfqpoint{5.401729in}{0.692560in}}%
\pgfpathlineto{\pgfqpoint{5.405842in}{0.667920in}}%
\pgfpathlineto{\pgfqpoint{5.407487in}{0.674080in}}%
\pgfpathlineto{\pgfqpoint{5.409955in}{0.661760in}}%
\pgfpathlineto{\pgfqpoint{5.411600in}{0.667920in}}%
\pgfpathlineto{\pgfqpoint{5.413245in}{0.637120in}}%
\pgfpathlineto{\pgfqpoint{5.415713in}{0.618640in}}%
\pgfpathlineto{\pgfqpoint{5.416535in}{0.630960in}}%
\pgfpathlineto{\pgfqpoint{5.417358in}{0.618640in}}%
\pgfpathlineto{\pgfqpoint{5.418180in}{0.630960in}}%
\pgfpathlineto{\pgfqpoint{5.419003in}{0.624800in}}%
\pgfpathlineto{\pgfqpoint{5.422293in}{0.581680in}}%
\pgfpathlineto{\pgfqpoint{5.423116in}{0.600160in}}%
\pgfpathlineto{\pgfqpoint{5.423939in}{0.594000in}}%
\pgfpathlineto{\pgfqpoint{5.427229in}{0.575520in}}%
\pgfpathlineto{\pgfqpoint{5.428874in}{0.581680in}}%
\pgfpathlineto{\pgfqpoint{5.430519in}{0.569360in}}%
\pgfpathlineto{\pgfqpoint{5.433810in}{0.563200in}}%
\pgfpathlineto{\pgfqpoint{5.434632in}{0.550880in}}%
\pgfpathlineto{\pgfqpoint{5.435455in}{0.563200in}}%
\pgfpathlineto{\pgfqpoint{5.436277in}{0.557040in}}%
\pgfpathlineto{\pgfqpoint{5.438745in}{0.563200in}}%
\pgfpathlineto{\pgfqpoint{5.439568in}{0.550880in}}%
\pgfpathlineto{\pgfqpoint{5.442036in}{0.581680in}}%
\pgfpathlineto{\pgfqpoint{5.444503in}{0.587840in}}%
\pgfpathlineto{\pgfqpoint{5.446971in}{0.544720in}}%
\pgfpathlineto{\pgfqpoint{5.447794in}{0.563200in}}%
\pgfpathlineto{\pgfqpoint{5.450262in}{0.563200in}}%
\pgfpathlineto{\pgfqpoint{5.451084in}{0.557040in}}%
\pgfpathlineto{\pgfqpoint{5.453552in}{0.594000in}}%
\pgfpathlineto{\pgfqpoint{5.456020in}{0.575520in}}%
\pgfpathlineto{\pgfqpoint{5.456842in}{0.581680in}}%
\pgfpathlineto{\pgfqpoint{5.458487in}{0.563200in}}%
\pgfpathlineto{\pgfqpoint{5.461778in}{0.563200in}}%
\pgfpathlineto{\pgfqpoint{5.464246in}{0.575520in}}%
\pgfpathlineto{\pgfqpoint{5.465068in}{0.569360in}}%
\pgfpathlineto{\pgfqpoint{5.467536in}{0.569360in}}%
\pgfpathlineto{\pgfqpoint{5.468359in}{0.575520in}}%
\pgfpathlineto{\pgfqpoint{5.469181in}{0.563200in}}%
\pgfpathlineto{\pgfqpoint{5.470004in}{0.569360in}}%
\pgfpathlineto{\pgfqpoint{5.474117in}{0.550880in}}%
\pgfpathlineto{\pgfqpoint{5.475762in}{0.563200in}}%
\pgfpathlineto{\pgfqpoint{5.476585in}{0.569360in}}%
\pgfpathlineto{\pgfqpoint{5.479052in}{0.557040in}}%
\pgfpathlineto{\pgfqpoint{5.480697in}{0.563200in}}%
\pgfpathlineto{\pgfqpoint{5.481520in}{0.538560in}}%
\pgfpathlineto{\pgfqpoint{5.482343in}{0.550880in}}%
\pgfpathlineto{\pgfqpoint{5.485633in}{0.550880in}}%
\pgfpathlineto{\pgfqpoint{5.488101in}{0.532400in}}%
\pgfpathlineto{\pgfqpoint{5.491391in}{0.538560in}}%
\pgfpathlineto{\pgfqpoint{5.492214in}{0.538560in}}%
\pgfpathlineto{\pgfqpoint{5.493859in}{0.550880in}}%
\pgfpathlineto{\pgfqpoint{5.497149in}{0.550880in}}%
\pgfpathlineto{\pgfqpoint{5.499617in}{0.575520in}}%
\pgfpathlineto{\pgfqpoint{5.504553in}{0.557040in}}%
\pgfpathlineto{\pgfqpoint{5.505375in}{0.563200in}}%
\pgfpathlineto{\pgfqpoint{5.509488in}{0.544720in}}%
\pgfpathlineto{\pgfqpoint{5.510311in}{0.544720in}}%
\pgfpathlineto{\pgfqpoint{5.511133in}{0.575520in}}%
\pgfpathlineto{\pgfqpoint{5.513601in}{0.563200in}}%
\pgfpathlineto{\pgfqpoint{5.516069in}{0.606320in}}%
\pgfpathlineto{\pgfqpoint{5.516892in}{0.594000in}}%
\pgfpathlineto{\pgfqpoint{5.519359in}{0.587840in}}%
\pgfpathlineto{\pgfqpoint{5.520182in}{0.575520in}}%
\pgfpathlineto{\pgfqpoint{5.521005in}{0.600160in}}%
\pgfpathlineto{\pgfqpoint{5.521827in}{0.594000in}}%
\pgfpathlineto{\pgfqpoint{5.522650in}{0.612480in}}%
\pgfpathlineto{\pgfqpoint{5.527585in}{0.581680in}}%
\pgfpathlineto{\pgfqpoint{5.528408in}{0.587840in}}%
\pgfpathlineto{\pgfqpoint{5.532521in}{0.520080in}}%
\pgfpathlineto{\pgfqpoint{5.537456in}{0.520080in}}%
\pgfpathlineto{\pgfqpoint{5.538279in}{0.557040in}}%
\pgfpathlineto{\pgfqpoint{5.539102in}{0.550880in}}%
\pgfpathlineto{\pgfqpoint{5.539924in}{0.575520in}}%
\pgfpathlineto{\pgfqpoint{5.542392in}{0.557040in}}%
\pgfpathlineto{\pgfqpoint{5.544037in}{0.569360in}}%
\pgfpathlineto{\pgfqpoint{5.544860in}{0.606320in}}%
\pgfpathlineto{\pgfqpoint{5.545682in}{0.587840in}}%
\pgfpathlineto{\pgfqpoint{5.548150in}{0.587840in}}%
\pgfpathlineto{\pgfqpoint{5.548973in}{0.600160in}}%
\pgfpathlineto{\pgfqpoint{5.551440in}{0.557040in}}%
\pgfpathlineto{\pgfqpoint{5.553908in}{0.557040in}}%
\pgfpathlineto{\pgfqpoint{5.556376in}{0.600160in}}%
\pgfpathlineto{\pgfqpoint{5.557199in}{0.575520in}}%
\pgfpathlineto{\pgfqpoint{5.559666in}{0.569360in}}%
\pgfpathlineto{\pgfqpoint{5.560489in}{0.550880in}}%
\pgfpathlineto{\pgfqpoint{5.562957in}{0.575520in}}%
\pgfpathlineto{\pgfqpoint{5.565425in}{0.575520in}}%
\pgfpathlineto{\pgfqpoint{5.566247in}{0.563200in}}%
\pgfpathlineto{\pgfqpoint{5.568715in}{0.630960in}}%
\pgfpathlineto{\pgfqpoint{5.573650in}{0.544720in}}%
\pgfpathlineto{\pgfqpoint{5.574473in}{0.569360in}}%
\pgfpathlineto{\pgfqpoint{5.578586in}{0.544720in}}%
\pgfpathlineto{\pgfqpoint{5.580231in}{0.557040in}}%
\pgfpathlineto{\pgfqpoint{5.583522in}{0.557040in}}%
\pgfpathlineto{\pgfqpoint{5.585167in}{0.544720in}}%
\pgfpathlineto{\pgfqpoint{5.585989in}{0.569360in}}%
\pgfpathlineto{\pgfqpoint{5.589280in}{0.575520in}}%
\pgfpathlineto{\pgfqpoint{5.590925in}{0.569360in}}%
\pgfpathlineto{\pgfqpoint{5.591748in}{0.575520in}}%
\pgfpathlineto{\pgfqpoint{5.594215in}{0.563200in}}%
\pgfpathlineto{\pgfqpoint{5.595038in}{0.569360in}}%
\pgfpathlineto{\pgfqpoint{5.596683in}{0.606320in}}%
\pgfpathlineto{\pgfqpoint{5.597506in}{0.600160in}}%
\pgfpathlineto{\pgfqpoint{5.599973in}{0.569360in}}%
\pgfpathlineto{\pgfqpoint{5.601619in}{0.594000in}}%
\pgfpathlineto{\pgfqpoint{5.603264in}{0.649440in}}%
\pgfpathlineto{\pgfqpoint{5.605732in}{0.594000in}}%
\pgfpathlineto{\pgfqpoint{5.608199in}{0.667920in}}%
\pgfpathlineto{\pgfqpoint{5.611490in}{0.624800in}}%
\pgfpathlineto{\pgfqpoint{5.612312in}{0.643280in}}%
\pgfpathlineto{\pgfqpoint{5.613135in}{0.637120in}}%
\pgfpathlineto{\pgfqpoint{5.614780in}{0.655600in}}%
\pgfpathlineto{\pgfqpoint{5.618071in}{0.661760in}}%
\pgfpathlineto{\pgfqpoint{5.618893in}{0.667920in}}%
\pgfpathlineto{\pgfqpoint{5.620538in}{0.637120in}}%
\pgfpathlineto{\pgfqpoint{5.623006in}{0.618640in}}%
\pgfpathlineto{\pgfqpoint{5.624651in}{0.637120in}}%
\pgfpathlineto{\pgfqpoint{5.626296in}{0.624800in}}%
\pgfpathlineto{\pgfqpoint{5.629587in}{0.643280in}}%
\pgfpathlineto{\pgfqpoint{5.630409in}{0.637120in}}%
\pgfpathlineto{\pgfqpoint{5.632055in}{0.686400in}}%
\pgfpathlineto{\pgfqpoint{5.634522in}{0.692560in}}%
\pgfpathlineto{\pgfqpoint{5.635345in}{0.661760in}}%
\pgfpathlineto{\pgfqpoint{5.636168in}{0.667920in}}%
\pgfpathlineto{\pgfqpoint{5.636990in}{0.667920in}}%
\pgfpathlineto{\pgfqpoint{5.637813in}{0.674080in}}%
\pgfpathlineto{\pgfqpoint{5.641103in}{0.655600in}}%
\pgfpathlineto{\pgfqpoint{5.641926in}{0.667920in}}%
\pgfpathlineto{\pgfqpoint{5.643571in}{0.661760in}}%
\pgfpathlineto{\pgfqpoint{5.646039in}{0.661760in}}%
\pgfpathlineto{\pgfqpoint{5.646861in}{0.655600in}}%
\pgfpathlineto{\pgfqpoint{5.649329in}{0.667920in}}%
\pgfpathlineto{\pgfqpoint{5.653442in}{0.637120in}}%
\pgfpathlineto{\pgfqpoint{5.655087in}{0.655600in}}%
\pgfpathlineto{\pgfqpoint{5.657555in}{0.643280in}}%
\pgfpathlineto{\pgfqpoint{5.660845in}{0.680240in}}%
\pgfpathlineto{\pgfqpoint{5.663313in}{0.674080in}}%
\pgfpathlineto{\pgfqpoint{5.664958in}{0.655600in}}%
\pgfpathlineto{\pgfqpoint{5.666603in}{0.661760in}}%
\pgfpathlineto{\pgfqpoint{5.669071in}{0.661760in}}%
\pgfpathlineto{\pgfqpoint{5.671539in}{0.680240in}}%
\pgfpathlineto{\pgfqpoint{5.672362in}{0.674080in}}%
\pgfpathlineto{\pgfqpoint{5.674829in}{0.680240in}}%
\pgfpathlineto{\pgfqpoint{5.675652in}{0.692560in}}%
\pgfpathlineto{\pgfqpoint{5.678120in}{0.686400in}}%
\pgfpathlineto{\pgfqpoint{5.680588in}{0.667920in}}%
\pgfpathlineto{\pgfqpoint{5.683055in}{0.680240in}}%
\pgfpathlineto{\pgfqpoint{5.686346in}{0.649440in}}%
\pgfpathlineto{\pgfqpoint{5.687168in}{0.667920in}}%
\pgfpathlineto{\pgfqpoint{5.687991in}{0.637120in}}%
\pgfpathlineto{\pgfqpoint{5.688813in}{0.649440in}}%
\pgfpathlineto{\pgfqpoint{5.692104in}{0.618640in}}%
\pgfpathlineto{\pgfqpoint{5.692926in}{0.624800in}}%
\pgfpathlineto{\pgfqpoint{5.693749in}{0.624800in}}%
\pgfpathlineto{\pgfqpoint{5.694572in}{0.630960in}}%
\pgfpathlineto{\pgfqpoint{5.695394in}{0.618640in}}%
\pgfpathlineto{\pgfqpoint{5.699507in}{0.612480in}}%
\pgfpathlineto{\pgfqpoint{5.700330in}{0.606320in}}%
\pgfpathlineto{\pgfqpoint{5.701152in}{0.618640in}}%
\pgfpathlineto{\pgfqpoint{5.704443in}{0.581680in}}%
\pgfpathlineto{\pgfqpoint{5.705265in}{0.594000in}}%
\pgfpathlineto{\pgfqpoint{5.706088in}{0.575520in}}%
\pgfpathlineto{\pgfqpoint{5.706911in}{0.581680in}}%
\pgfpathlineto{\pgfqpoint{5.710201in}{0.563200in}}%
\pgfpathlineto{\pgfqpoint{5.711024in}{0.563200in}}%
\pgfpathlineto{\pgfqpoint{5.711846in}{0.557040in}}%
\pgfpathlineto{\pgfqpoint{5.712669in}{0.563200in}}%
\pgfpathlineto{\pgfqpoint{5.715136in}{0.538560in}}%
\pgfpathlineto{\pgfqpoint{5.715959in}{0.544720in}}%
\pgfpathlineto{\pgfqpoint{5.717604in}{0.520080in}}%
\pgfpathlineto{\pgfqpoint{5.718427in}{0.526240in}}%
\pgfpathlineto{\pgfqpoint{5.721717in}{0.526240in}}%
\pgfpathlineto{\pgfqpoint{5.722540in}{0.532400in}}%
\pgfpathlineto{\pgfqpoint{5.723362in}{0.520080in}}%
\pgfpathlineto{\pgfqpoint{5.724185in}{0.532400in}}%
\pgfpathlineto{\pgfqpoint{5.726653in}{0.532400in}}%
\pgfpathlineto{\pgfqpoint{5.727475in}{0.526240in}}%
\pgfpathlineto{\pgfqpoint{5.728298in}{0.507760in}}%
\pgfpathlineto{\pgfqpoint{5.729943in}{0.569360in}}%
\pgfpathlineto{\pgfqpoint{5.732411in}{0.532400in}}%
\pgfpathlineto{\pgfqpoint{5.733234in}{0.674080in}}%
\pgfpathlineto{\pgfqpoint{5.734056in}{0.618640in}}%
\pgfpathlineto{\pgfqpoint{5.735701in}{0.717200in}}%
\pgfpathlineto{\pgfqpoint{5.738169in}{0.717200in}}%
\pgfpathlineto{\pgfqpoint{5.738992in}{0.655600in}}%
\pgfpathlineto{\pgfqpoint{5.739814in}{0.661760in}}%
\pgfpathlineto{\pgfqpoint{5.740637in}{0.692560in}}%
\pgfpathlineto{\pgfqpoint{5.741459in}{0.791120in}}%
\pgfpathlineto{\pgfqpoint{5.743927in}{0.840400in}}%
\pgfpathlineto{\pgfqpoint{5.746395in}{1.154560in}}%
\pgfpathlineto{\pgfqpoint{5.747218in}{1.148400in}}%
\pgfpathlineto{\pgfqpoint{5.750508in}{1.191520in}}%
\pgfpathlineto{\pgfqpoint{5.752976in}{1.314720in}}%
\pgfpathlineto{\pgfqpoint{5.755444in}{1.246960in}}%
\pgfpathlineto{\pgfqpoint{5.757089in}{1.271600in}}%
\pgfpathlineto{\pgfqpoint{5.757911in}{1.228480in}}%
\pgfpathlineto{\pgfqpoint{5.758734in}{1.234640in}}%
\pgfpathlineto{\pgfqpoint{5.762024in}{1.166880in}}%
\pgfpathlineto{\pgfqpoint{5.763669in}{1.037520in}}%
\pgfpathlineto{\pgfqpoint{5.767782in}{1.043680in}}%
\pgfpathlineto{\pgfqpoint{5.769428in}{1.056000in}}%
\pgfpathlineto{\pgfqpoint{5.770250in}{1.049840in}}%
\pgfpathlineto{\pgfqpoint{5.772718in}{1.043680in}}%
\pgfpathlineto{\pgfqpoint{5.776008in}{0.914320in}}%
\pgfpathlineto{\pgfqpoint{5.778476in}{0.883520in}}%
\pgfpathlineto{\pgfqpoint{5.781766in}{0.698720in}}%
\pgfpathlineto{\pgfqpoint{5.786702in}{0.637120in}}%
\pgfpathlineto{\pgfqpoint{5.789992in}{0.630960in}}%
\pgfpathlineto{\pgfqpoint{5.791638in}{0.600160in}}%
\pgfpathlineto{\pgfqpoint{5.792460in}{0.606320in}}%
\pgfpathlineto{\pgfqpoint{5.793283in}{0.600160in}}%
\pgfpathlineto{\pgfqpoint{5.795751in}{0.594000in}}%
\pgfpathlineto{\pgfqpoint{5.797396in}{0.587840in}}%
\pgfpathlineto{\pgfqpoint{5.798218in}{0.587840in}}%
\pgfpathlineto{\pgfqpoint{5.799041in}{0.594000in}}%
\pgfpathlineto{\pgfqpoint{5.802331in}{0.581680in}}%
\pgfpathlineto{\pgfqpoint{5.803976in}{0.563200in}}%
\pgfpathlineto{\pgfqpoint{5.807267in}{0.563200in}}%
\pgfpathlineto{\pgfqpoint{5.808912in}{0.544720in}}%
\pgfpathlineto{\pgfqpoint{5.809735in}{0.544720in}}%
\pgfpathlineto{\pgfqpoint{5.813848in}{0.513920in}}%
\pgfpathlineto{\pgfqpoint{5.814670in}{0.532400in}}%
\pgfpathlineto{\pgfqpoint{5.815493in}{0.526240in}}%
\pgfpathlineto{\pgfqpoint{5.816315in}{0.538560in}}%
\pgfpathlineto{\pgfqpoint{5.818783in}{0.513920in}}%
\pgfpathlineto{\pgfqpoint{5.820428in}{0.532400in}}%
\pgfpathlineto{\pgfqpoint{5.821251in}{0.532400in}}%
\pgfpathlineto{\pgfqpoint{5.822074in}{0.538560in}}%
\pgfpathlineto{\pgfqpoint{5.826187in}{0.520080in}}%
\pgfpathlineto{\pgfqpoint{5.827832in}{0.544720in}}%
\pgfpathlineto{\pgfqpoint{5.830299in}{0.538560in}}%
\pgfpathlineto{\pgfqpoint{5.831122in}{0.526240in}}%
\pgfpathlineto{\pgfqpoint{5.832767in}{0.538560in}}%
\pgfpathlineto{\pgfqpoint{5.837703in}{0.513920in}}%
\pgfpathlineto{\pgfqpoint{5.839348in}{0.526240in}}%
\pgfpathlineto{\pgfqpoint{5.841816in}{0.526240in}}%
\pgfpathlineto{\pgfqpoint{5.843461in}{0.507760in}}%
\pgfpathlineto{\pgfqpoint{5.845106in}{0.538560in}}%
\pgfpathlineto{\pgfqpoint{5.848397in}{0.520080in}}%
\pgfpathlineto{\pgfqpoint{5.849219in}{0.520080in}}%
\pgfpathlineto{\pgfqpoint{5.850042in}{0.513920in}}%
\pgfpathlineto{\pgfqpoint{5.850864in}{0.526240in}}%
\pgfpathlineto{\pgfqpoint{5.854155in}{0.538560in}}%
\pgfpathlineto{\pgfqpoint{5.854977in}{0.532400in}}%
\pgfpathlineto{\pgfqpoint{5.856622in}{0.538560in}}%
\pgfpathlineto{\pgfqpoint{5.859090in}{0.532400in}}%
\pgfpathlineto{\pgfqpoint{5.859913in}{0.538560in}}%
\pgfpathlineto{\pgfqpoint{5.861558in}{0.526240in}}%
\pgfpathlineto{\pgfqpoint{5.862381in}{0.532400in}}%
\pgfpathlineto{\pgfqpoint{5.864848in}{0.532400in}}%
\pgfpathlineto{\pgfqpoint{5.865671in}{0.526240in}}%
\pgfpathlineto{\pgfqpoint{5.866494in}{0.532400in}}%
\pgfpathlineto{\pgfqpoint{5.867316in}{0.550880in}}%
\pgfpathlineto{\pgfqpoint{5.868139in}{0.544720in}}%
\pgfpathlineto{\pgfqpoint{5.870607in}{0.544720in}}%
\pgfpathlineto{\pgfqpoint{5.872252in}{0.526240in}}%
\pgfpathlineto{\pgfqpoint{5.873074in}{0.532400in}}%
\pgfpathlineto{\pgfqpoint{5.873897in}{0.532400in}}%
\pgfpathlineto{\pgfqpoint{5.876365in}{0.507760in}}%
\pgfpathlineto{\pgfqpoint{5.878010in}{0.532400in}}%
\pgfpathlineto{\pgfqpoint{5.879655in}{0.526240in}}%
\pgfpathlineto{\pgfqpoint{5.883768in}{0.520080in}}%
\pgfpathlineto{\pgfqpoint{5.885413in}{0.526240in}}%
\pgfpathlineto{\pgfqpoint{5.888704in}{0.513920in}}%
\pgfpathlineto{\pgfqpoint{5.891171in}{0.526240in}}%
\pgfpathlineto{\pgfqpoint{5.893639in}{0.520080in}}%
\pgfpathlineto{\pgfqpoint{5.894462in}{0.526240in}}%
\pgfpathlineto{\pgfqpoint{5.895284in}{0.507760in}}%
\pgfpathlineto{\pgfqpoint{5.896107in}{0.526240in}}%
\pgfpathlineto{\pgfqpoint{5.896929in}{0.520080in}}%
\pgfpathlineto{\pgfqpoint{5.900220in}{0.513920in}}%
\pgfpathlineto{\pgfqpoint{5.901042in}{0.513920in}}%
\pgfpathlineto{\pgfqpoint{5.901865in}{0.520080in}}%
\pgfpathlineto{\pgfqpoint{5.902688in}{0.513920in}}%
\pgfpathlineto{\pgfqpoint{5.905155in}{0.507760in}}%
\pgfpathlineto{\pgfqpoint{5.905978in}{0.526240in}}%
\pgfpathlineto{\pgfqpoint{5.906801in}{0.520080in}}%
\pgfpathlineto{\pgfqpoint{5.908446in}{0.526240in}}%
\pgfpathlineto{\pgfqpoint{5.910914in}{0.513920in}}%
\pgfpathlineto{\pgfqpoint{5.912559in}{0.520080in}}%
\pgfpathlineto{\pgfqpoint{5.913381in}{0.520080in}}%
\pgfpathlineto{\pgfqpoint{5.914204in}{0.513920in}}%
\pgfpathlineto{\pgfqpoint{5.917494in}{0.520080in}}%
\pgfpathlineto{\pgfqpoint{5.919139in}{0.507760in}}%
\pgfpathlineto{\pgfqpoint{5.919962in}{0.507760in}}%
\pgfpathlineto{\pgfqpoint{5.922430in}{0.501600in}}%
\pgfpathlineto{\pgfqpoint{5.923252in}{0.513920in}}%
\pgfpathlineto{\pgfqpoint{5.924075in}{0.507760in}}%
\pgfpathlineto{\pgfqpoint{5.925720in}{0.513920in}}%
\pgfpathlineto{\pgfqpoint{5.929011in}{0.507760in}}%
\pgfpathlineto{\pgfqpoint{5.929833in}{0.507760in}}%
\pgfpathlineto{\pgfqpoint{5.931478in}{0.520080in}}%
\pgfpathlineto{\pgfqpoint{5.933946in}{0.520080in}}%
\pgfpathlineto{\pgfqpoint{5.934769in}{0.513920in}}%
\pgfpathlineto{\pgfqpoint{5.935591in}{0.520080in}}%
\pgfpathlineto{\pgfqpoint{5.937237in}{0.507760in}}%
\pgfpathlineto{\pgfqpoint{5.939704in}{0.501600in}}%
\pgfpathlineto{\pgfqpoint{5.942995in}{0.520080in}}%
\pgfpathlineto{\pgfqpoint{5.945462in}{0.520080in}}%
\pgfpathlineto{\pgfqpoint{5.946285in}{0.526240in}}%
\pgfpathlineto{\pgfqpoint{5.947108in}{0.520080in}}%
\pgfpathlineto{\pgfqpoint{5.947930in}{0.526240in}}%
\pgfpathlineto{\pgfqpoint{5.948753in}{0.520080in}}%
\pgfpathlineto{\pgfqpoint{5.951221in}{0.520080in}}%
\pgfpathlineto{\pgfqpoint{5.952866in}{0.526240in}}%
\pgfpathlineto{\pgfqpoint{5.954511in}{0.526240in}}%
\pgfpathlineto{\pgfqpoint{5.956979in}{0.532400in}}%
\pgfpathlineto{\pgfqpoint{5.958624in}{0.526240in}}%
\pgfpathlineto{\pgfqpoint{5.959447in}{0.532400in}}%
\pgfpathlineto{\pgfqpoint{5.960269in}{0.526240in}}%
\pgfpathlineto{\pgfqpoint{5.962737in}{0.532400in}}%
\pgfpathlineto{\pgfqpoint{5.964382in}{0.526240in}}%
\pgfpathlineto{\pgfqpoint{5.966027in}{0.526240in}}%
\pgfpathlineto{\pgfqpoint{5.968495in}{0.520080in}}%
\pgfpathlineto{\pgfqpoint{5.970140in}{0.532400in}}%
\pgfpathlineto{\pgfqpoint{5.971785in}{0.538560in}}%
\pgfpathlineto{\pgfqpoint{5.975076in}{0.532400in}}%
\pgfpathlineto{\pgfqpoint{5.975898in}{0.538560in}}%
\pgfpathlineto{\pgfqpoint{5.976721in}{0.532400in}}%
\pgfpathlineto{\pgfqpoint{5.980834in}{0.532400in}}%
\pgfpathlineto{\pgfqpoint{5.981657in}{0.538560in}}%
\pgfpathlineto{\pgfqpoint{5.982479in}{0.532400in}}%
\pgfpathlineto{\pgfqpoint{5.986592in}{0.532400in}}%
\pgfpathlineto{\pgfqpoint{5.988237in}{0.520080in}}%
\pgfpathlineto{\pgfqpoint{5.989060in}{0.526240in}}%
\pgfpathlineto{\pgfqpoint{5.992350in}{0.526240in}}%
\pgfpathlineto{\pgfqpoint{5.993173in}{0.532400in}}%
\pgfpathlineto{\pgfqpoint{5.994818in}{0.520080in}}%
\pgfpathlineto{\pgfqpoint{5.998108in}{0.520080in}}%
\pgfpathlineto{\pgfqpoint{5.998931in}{0.526240in}}%
\pgfpathlineto{\pgfqpoint{5.999754in}{0.520080in}}%
\pgfpathlineto{\pgfqpoint{6.000576in}{0.526240in}}%
\pgfpathlineto{\pgfqpoint{6.003044in}{0.513920in}}%
\pgfpathlineto{\pgfqpoint{6.003867in}{0.532400in}}%
\pgfpathlineto{\pgfqpoint{6.004689in}{0.520080in}}%
\pgfpathlineto{\pgfqpoint{6.006334in}{0.526240in}}%
\pgfpathlineto{\pgfqpoint{6.008802in}{0.520080in}}%
\pgfpathlineto{\pgfqpoint{6.009625in}{0.513920in}}%
\pgfpathlineto{\pgfqpoint{6.010447in}{0.538560in}}%
\pgfpathlineto{\pgfqpoint{6.011270in}{0.532400in}}%
\pgfpathlineto{\pgfqpoint{6.012092in}{0.538560in}}%
\pgfpathlineto{\pgfqpoint{6.014560in}{0.532400in}}%
\pgfpathlineto{\pgfqpoint{6.015383in}{0.538560in}}%
\pgfpathlineto{\pgfqpoint{6.017028in}{0.532400in}}%
\pgfpathlineto{\pgfqpoint{6.021141in}{0.532400in}}%
\pgfpathlineto{\pgfqpoint{6.021964in}{0.526240in}}%
\pgfpathlineto{\pgfqpoint{6.022786in}{0.532400in}}%
\pgfpathlineto{\pgfqpoint{6.023609in}{0.526240in}}%
\pgfpathlineto{\pgfqpoint{6.027722in}{0.538560in}}%
\pgfpathlineto{\pgfqpoint{6.029367in}{0.532400in}}%
\pgfpathlineto{\pgfqpoint{6.031835in}{0.520080in}}%
\pgfpathlineto{\pgfqpoint{6.033480in}{0.526240in}}%
\pgfpathlineto{\pgfqpoint{6.034303in}{0.526240in}}%
\pgfpathlineto{\pgfqpoint{6.035125in}{0.532400in}}%
\pgfpathlineto{\pgfqpoint{6.038415in}{0.520080in}}%
\pgfpathlineto{\pgfqpoint{6.040883in}{0.532400in}}%
\pgfpathlineto{\pgfqpoint{6.043351in}{0.526240in}}%
\pgfpathlineto{\pgfqpoint{6.044996in}{0.544720in}}%
\pgfpathlineto{\pgfqpoint{6.046641in}{0.557040in}}%
\pgfpathlineto{\pgfqpoint{6.049109in}{0.538560in}}%
\pgfpathlineto{\pgfqpoint{6.049932in}{0.557040in}}%
\pgfpathlineto{\pgfqpoint{6.050754in}{0.550880in}}%
\pgfpathlineto{\pgfqpoint{6.051577in}{0.544720in}}%
\pgfpathlineto{\pgfqpoint{6.052400in}{0.550880in}}%
\pgfpathlineto{\pgfqpoint{6.054867in}{0.544720in}}%
\pgfpathlineto{\pgfqpoint{6.055690in}{0.550880in}}%
\pgfpathlineto{\pgfqpoint{6.056513in}{0.538560in}}%
\pgfpathlineto{\pgfqpoint{6.057335in}{0.550880in}}%
\pgfpathlineto{\pgfqpoint{6.061448in}{0.550880in}}%
\pgfpathlineto{\pgfqpoint{6.062271in}{0.544720in}}%
\pgfpathlineto{\pgfqpoint{6.063093in}{0.550880in}}%
\pgfpathlineto{\pgfqpoint{6.063916in}{0.544720in}}%
\pgfpathlineto{\pgfqpoint{6.066384in}{0.544720in}}%
\pgfpathlineto{\pgfqpoint{6.067206in}{0.532400in}}%
\pgfpathlineto{\pgfqpoint{6.069674in}{0.544720in}}%
\pgfpathlineto{\pgfqpoint{6.072142in}{0.544720in}}%
\pgfpathlineto{\pgfqpoint{6.073787in}{0.526240in}}%
\pgfpathlineto{\pgfqpoint{6.075432in}{0.532400in}}%
\pgfpathlineto{\pgfqpoint{6.077900in}{0.532400in}}%
\pgfpathlineto{\pgfqpoint{6.079545in}{0.550880in}}%
\pgfpathlineto{\pgfqpoint{6.081190in}{0.544720in}}%
\pgfpathlineto{\pgfqpoint{6.084481in}{0.538560in}}%
\pgfpathlineto{\pgfqpoint{6.086948in}{0.526240in}}%
\pgfpathlineto{\pgfqpoint{6.090239in}{0.532400in}}%
\pgfpathlineto{\pgfqpoint{6.091061in}{0.520080in}}%
\pgfpathlineto{\pgfqpoint{6.092707in}{0.532400in}}%
\pgfpathlineto{\pgfqpoint{6.095174in}{0.520080in}}%
\pgfpathlineto{\pgfqpoint{6.096820in}{0.526240in}}%
\pgfpathlineto{\pgfqpoint{6.098465in}{0.526240in}}%
\pgfpathlineto{\pgfqpoint{6.101755in}{0.513920in}}%
\pgfpathlineto{\pgfqpoint{6.102578in}{0.513920in}}%
\pgfpathlineto{\pgfqpoint{6.103400in}{0.507760in}}%
\pgfpathlineto{\pgfqpoint{6.104223in}{0.513920in}}%
\pgfpathlineto{\pgfqpoint{6.108336in}{0.507760in}}%
\pgfpathlineto{\pgfqpoint{6.109981in}{0.507760in}}%
\pgfpathlineto{\pgfqpoint{6.112449in}{0.501600in}}%
\pgfpathlineto{\pgfqpoint{6.113271in}{0.507760in}}%
\pgfpathlineto{\pgfqpoint{6.114917in}{0.495440in}}%
\pgfpathlineto{\pgfqpoint{6.119030in}{0.495440in}}%
\pgfpathlineto{\pgfqpoint{6.119852in}{0.489280in}}%
\pgfpathlineto{\pgfqpoint{6.120675in}{0.495440in}}%
\pgfpathlineto{\pgfqpoint{6.121497in}{0.489280in}}%
\pgfpathlineto{\pgfqpoint{6.126433in}{0.501600in}}%
\pgfpathlineto{\pgfqpoint{6.127255in}{0.495440in}}%
\pgfpathlineto{\pgfqpoint{6.130546in}{0.501600in}}%
\pgfpathlineto{\pgfqpoint{6.131368in}{0.501600in}}%
\pgfpathlineto{\pgfqpoint{6.133014in}{0.495440in}}%
\pgfpathlineto{\pgfqpoint{6.136304in}{0.489280in}}%
\pgfpathlineto{\pgfqpoint{6.137949in}{0.476960in}}%
\pgfpathlineto{\pgfqpoint{6.138772in}{0.483120in}}%
\pgfpathlineto{\pgfqpoint{6.142062in}{0.489280in}}%
\pgfpathlineto{\pgfqpoint{6.142885in}{0.483120in}}%
\pgfpathlineto{\pgfqpoint{6.144530in}{0.489280in}}%
\pgfpathlineto{\pgfqpoint{6.146998in}{0.489280in}}%
\pgfpathlineto{\pgfqpoint{6.148643in}{0.495440in}}%
\pgfpathlineto{\pgfqpoint{6.150288in}{0.489280in}}%
\pgfpathlineto{\pgfqpoint{6.154401in}{0.489280in}}%
\pgfpathlineto{\pgfqpoint{6.156046in}{0.476960in}}%
\pgfpathlineto{\pgfqpoint{6.159337in}{0.483120in}}%
\pgfpathlineto{\pgfqpoint{6.160159in}{0.483120in}}%
\pgfpathlineto{\pgfqpoint{6.160982in}{0.489280in}}%
\pgfpathlineto{\pgfqpoint{6.161804in}{0.483120in}}%
\pgfpathlineto{\pgfqpoint{6.165917in}{0.483120in}}%
\pgfpathlineto{\pgfqpoint{6.167563in}{0.476960in}}%
\pgfpathlineto{\pgfqpoint{6.171676in}{0.476960in}}%
\pgfpathlineto{\pgfqpoint{6.173321in}{0.489280in}}%
\pgfpathlineto{\pgfqpoint{6.175788in}{0.489280in}}%
\pgfpathlineto{\pgfqpoint{6.177434in}{0.476960in}}%
\pgfpathlineto{\pgfqpoint{6.179079in}{0.483120in}}%
\pgfpathlineto{\pgfqpoint{6.182369in}{0.489280in}}%
\pgfpathlineto{\pgfqpoint{6.184014in}{0.483120in}}%
\pgfpathlineto{\pgfqpoint{6.190595in}{0.483120in}}%
\pgfpathlineto{\pgfqpoint{6.193063in}{0.476960in}}%
\pgfpathlineto{\pgfqpoint{6.194708in}{0.489280in}}%
\pgfpathlineto{\pgfqpoint{6.196353in}{0.489280in}}%
\pgfpathlineto{\pgfqpoint{6.198821in}{0.495440in}}%
\pgfpathlineto{\pgfqpoint{6.200466in}{0.501600in}}%
\pgfpathlineto{\pgfqpoint{6.202111in}{0.501600in}}%
\pgfpathlineto{\pgfqpoint{6.205402in}{0.495440in}}%
\pgfpathlineto{\pgfqpoint{6.210337in}{0.495440in}}%
\pgfpathlineto{\pgfqpoint{6.212805in}{0.483120in}}%
\pgfpathlineto{\pgfqpoint{6.221854in}{0.483120in}}%
\pgfpathlineto{\pgfqpoint{6.223499in}{0.489280in}}%
\pgfpathlineto{\pgfqpoint{6.225144in}{0.476960in}}%
\pgfpathlineto{\pgfqpoint{6.227612in}{0.483120in}}%
\pgfpathlineto{\pgfqpoint{6.228434in}{0.489280in}}%
\pgfpathlineto{\pgfqpoint{6.230080in}{0.483120in}}%
\pgfpathlineto{\pgfqpoint{6.230902in}{0.489280in}}%
\pgfpathlineto{\pgfqpoint{6.233370in}{0.495440in}}%
\pgfpathlineto{\pgfqpoint{6.234193in}{0.501600in}}%
\pgfpathlineto{\pgfqpoint{6.235015in}{0.495440in}}%
\pgfpathlineto{\pgfqpoint{6.235838in}{0.501600in}}%
\pgfpathlineto{\pgfqpoint{6.236660in}{0.495440in}}%
\pgfpathlineto{\pgfqpoint{6.239128in}{0.495440in}}%
\pgfpathlineto{\pgfqpoint{6.239951in}{0.507760in}}%
\pgfpathlineto{\pgfqpoint{6.240773in}{0.501600in}}%
\pgfpathlineto{\pgfqpoint{6.242418in}{0.507760in}}%
\pgfpathlineto{\pgfqpoint{6.248177in}{0.507760in}}%
\pgfpathlineto{\pgfqpoint{6.251467in}{0.513920in}}%
\pgfpathlineto{\pgfqpoint{6.253935in}{0.513920in}}%
\pgfpathlineto{\pgfqpoint{6.256403in}{0.507760in}}%
\pgfpathlineto{\pgfqpoint{6.257225in}{0.513920in}}%
\pgfpathlineto{\pgfqpoint{6.258048in}{0.507760in}}%
\pgfpathlineto{\pgfqpoint{6.259693in}{0.520080in}}%
\pgfpathlineto{\pgfqpoint{6.262161in}{0.520080in}}%
\pgfpathlineto{\pgfqpoint{6.262983in}{0.526240in}}%
\pgfpathlineto{\pgfqpoint{6.263806in}{0.520080in}}%
\pgfpathlineto{\pgfqpoint{6.265451in}{0.526240in}}%
\pgfpathlineto{\pgfqpoint{6.267919in}{0.532400in}}%
\pgfpathlineto{\pgfqpoint{6.269564in}{0.538560in}}%
\pgfpathlineto{\pgfqpoint{6.271209in}{0.538560in}}%
\pgfpathlineto{\pgfqpoint{6.273677in}{0.526240in}}%
\pgfpathlineto{\pgfqpoint{6.274500in}{0.532400in}}%
\pgfpathlineto{\pgfqpoint{6.275322in}{0.520080in}}%
\pgfpathlineto{\pgfqpoint{6.276145in}{0.526240in}}%
\pgfpathlineto{\pgfqpoint{6.281080in}{0.544720in}}%
\pgfpathlineto{\pgfqpoint{6.282726in}{0.532400in}}%
\pgfpathlineto{\pgfqpoint{6.286839in}{0.526240in}}%
\pgfpathlineto{\pgfqpoint{6.287661in}{0.520080in}}%
\pgfpathlineto{\pgfqpoint{6.288484in}{0.526240in}}%
\pgfpathlineto{\pgfqpoint{6.290951in}{0.507760in}}%
\pgfpathlineto{\pgfqpoint{6.292597in}{0.513920in}}%
\pgfpathlineto{\pgfqpoint{6.293419in}{0.513920in}}%
\pgfpathlineto{\pgfqpoint{6.297532in}{0.489280in}}%
\pgfpathlineto{\pgfqpoint{6.298355in}{0.495440in}}%
\pgfpathlineto{\pgfqpoint{6.299177in}{0.489280in}}%
\pgfpathlineto{\pgfqpoint{6.300000in}{0.495440in}}%
\pgfpathlineto{\pgfqpoint{6.300000in}{0.495440in}}%
\pgfusepath{stroke}%
\end{pgfscope}%
\begin{pgfscope}%
\pgfsetrectcap%
\pgfsetmiterjoin%
\pgfsetlinewidth{1.003750pt}%
\definecolor{currentstroke}{rgb}{1.000000,1.000000,1.000000}%
\pgfsetstrokecolor{currentstroke}%
\pgfsetdash{}{0pt}%
\pgfpathmoveto{\pgfqpoint{0.875000in}{0.440000in}}%
\pgfpathlineto{\pgfqpoint{0.875000in}{3.520000in}}%
\pgfusepath{stroke}%
\end{pgfscope}%
\begin{pgfscope}%
\pgfsetrectcap%
\pgfsetmiterjoin%
\pgfsetlinewidth{1.003750pt}%
\definecolor{currentstroke}{rgb}{1.000000,1.000000,1.000000}%
\pgfsetstrokecolor{currentstroke}%
\pgfsetdash{}{0pt}%
\pgfpathmoveto{\pgfqpoint{6.300000in}{0.440000in}}%
\pgfpathlineto{\pgfqpoint{6.300000in}{3.520000in}}%
\pgfusepath{stroke}%
\end{pgfscope}%
\begin{pgfscope}%
\pgfsetrectcap%
\pgfsetmiterjoin%
\pgfsetlinewidth{1.003750pt}%
\definecolor{currentstroke}{rgb}{1.000000,1.000000,1.000000}%
\pgfsetstrokecolor{currentstroke}%
\pgfsetdash{}{0pt}%
\pgfpathmoveto{\pgfqpoint{0.875000in}{0.440000in}}%
\pgfpathlineto{\pgfqpoint{6.300000in}{0.440000in}}%
\pgfusepath{stroke}%
\end{pgfscope}%
\begin{pgfscope}%
\pgfsetrectcap%
\pgfsetmiterjoin%
\pgfsetlinewidth{1.003750pt}%
\definecolor{currentstroke}{rgb}{1.000000,1.000000,1.000000}%
\pgfsetstrokecolor{currentstroke}%
\pgfsetdash{}{0pt}%
\pgfpathmoveto{\pgfqpoint{0.875000in}{3.520000in}}%
\pgfpathlineto{\pgfqpoint{6.300000in}{3.520000in}}%
\pgfusepath{stroke}%
\end{pgfscope}%
\begin{pgfscope}%
\definecolor{textcolor}{rgb}{0.000000,0.000000,0.000000}%
\pgfsetstrokecolor{textcolor}%
\pgfsetfillcolor{textcolor}%
\pgftext[x=3.587500in,y=3.603333in,,base]{\color{textcolor}\rmfamily\fontsize{14.400000}{17.280000}\selectfont TED Spread from 2004 to 2022}%
\end{pgfscope}%
\end{pgfpicture}%
\makeatother%
\endgroup%

        }
        \caption{
            The spread between 3-Month LIBOR in USD and 3-Month Treasury bill.
            The spread is usually relatively low, 
            but blows out during recessions, 
            such as the financial crisis in 2007 to 2009 or
            the Covid-19 crisis in March 2020.
        }
        \label{fig:ted-spread}
    \end{figure}

    In the beginning of the financial crisis, the xIBORs rose significantly
    which widened spreads such as LIBOR-\OIS/ or LIBOR against the T-bill.
    The latter spread is shown in \cref{fig:ted-spread} which clearly 
    displays how the LIBOR against a very safe government bond blows out of proportions
    during recessions in the economy. 
    While not directly comparing LIBOR and \OIS/, 
    the TED spread conveys the conclusion very well nonetheless;
    the xIBOR contains other premiums than simply the risk-free rate.

    When the aforementioned spreads widened, 
    it had significant influence on the banks' pricing and valuation of derivatives.
    The arguments for xIBOR discounting previously mentioned
    were disproven as the spreads reached very high levels.
    The market was to find another curve to use for the risk-free rate.
    In this paper \OIS/ will be used as the risk-free rate;
    the following paragraph presents arguments for that decision.

    There are two sources of credit risk in an \OIS/.
    The first is the credit risk in the overnight borrowing of federal funds, which is very small. 
    The second is the risk of one of the swap counterparties defaulting.
    \textcite{HullWhiteOISvsLIBOR}
    argue that this second risk is negligible in collateralized transactions,
    which standard swaps like \OIS/ are due to legislation.
    Since credit risk is of inconsiderable size it can be concluded 
    that the \OIS/ rate is a good proxy for the risk-free rate.

\end{document}

% !TEX root = ./sub-main.tex
\documentclass[main.tex]{subfiles}

\begin{document}
    \subsection{Overnight Index Swap Rates}
    
    \subsubsection{The LIBOR-OIS spread}
    An index swap is a contract that exchanges a fixed cash flow, determined at inception,
    for a floating cash flow tied to some price index.
    As implied by the name, an Overnight Index Swap, OIS, is a special case of an index swap
    that uses an overnight rate index to calculate the floating leg.
    The floating payments of an OIS is the daily compounded overnight rate 
    over the floating coupon period,
    while the fixed rate is a rate referred to as the OIS rate.
    The price index that specifies the reference rate is typically the overnight 
    lending rate between banks published by the central bank, e.g. the Federal Funds Rate.
    At maturity of the OIS, the parties exchange the difference between the interest accrued 
    at the fixed rate and the interest accrued at the compounded floating index rate.
    There is no exchange of principal why an OIS generally carries very little credit risk. 
    Conceptually, the OIS rate can be thought of as representing a given country's
    central bank rate throughout a period set by the OIS' term. 

    The London Inter-Bank Offered Rate, LIBOR, represents the average rate 
    at which major banks charge each other for short-term unsecured borrowing.

    A common measure of banking system's health is the difference between the OIS rate
    and the LIBOR, known as the LIBOR-OIS spread.
    This spread can be used as an indication of how bank's perceive the creditworthiness
    of other financial institutions. 
    The LIBOR-OIS spread should be a better indicator of credit risk in the interbank lending market
    than the LIBOR itself. 
    The LIBOR is influenced both by the rates set by central banks and 
    the general credit risk in the interbank lending market,
    while the OIS rate is only based on the rates set by central banks.
    Therefore, subtracting the OIS rate from LIBOR should isolate the credit premium.
    A higher spread could be interpreted as a low willingness to lend by major banks
    and therefore low liquidity in the money market.

\end{document}
        
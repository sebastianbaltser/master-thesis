% !TEX root = ./sub-main.tex
\documentclass[../main.tex]{subfiles}

\begin{document}
    \subsubsection{Funding by equity issuance}
        In addition to obtaining debt, new projects can be financed by issuing additional shares
        that will be a source of equity capital.
        Share issuances dilutes the firm's stocks, which can reduce the value of the existing shareholders' stocks,
        which will be the source of the conflict of interest with this type of funding.
        It is assumed that the market for newly issued equity is sufficiently competitive,
        such that new shareholders break even when purchasing the newly issued shares;
        in other words, buying the firm's newly issued stock is a zero net present value investment.

        With equity financing, the legacy shareholders still receives the difference between assets and debt claims,
        however they now have to split the bounty with the new shareholders, 
        who takes an amount of shares worth $U(q)$. 
        The marginal increase in the value of the firms equity per unit investment is therefore:
            \begin{align}
                G_{\text{equity}} &=
                    \left.
                    \frac{
                        \partial 
                    }{
                        \partial 
                        q
                    }
                    \mathbb{E}^{\mathbb{Q}}\left[
                        \discountfactor 
                        \left(
                            A + qY - L
                        \right)^{+}
                    \right]
                    -
                    U(q) 
                    \right\rvert_{q=0} 
                    \nonumber
            \end{align}
        This derivative is very much like the derivative encountered under debt issuance.
        Therefore, the reformulation in this section will be less rigorous,
        as most details have already been covered in the previous section.
        Expressing first the derivative as a difference quotient:
            \begin{align}
                G_{\text{equity}} &=
                    \lim_{q\rightarrow0}
                    \discountfactor
                    \frac{
                        \mathbb{E}^{\mathbb{Q}}\left[
                        \left(
                            A + qY - L
                        \right)^{+}
                        \right] 
                        -
                        U(q)R
                        -
                        \mathbb{E}^{\mathbb{Q}}\left[
                        \left(
                            A - L
                        \right)^{+}  
                        \right] 
                    }{
                        q
                    }
                    \nonumber
            \intertext{Replacing max-functions by default indicators:}
                &=
                    \lim_{q\rightarrow0}
                    \discountfactor
                    \frac{
                        \mathbb{E}^{\mathbb{Q}}\left[
                        \mathbbm{1}_{\mathcal{D}^{c}(q)} 
                        \left(
                            A + qY - L
                        \right)
                        \right] 
                        -
                        U(q)R
                        -
                        \mathbb{E}^{\mathbb{Q}}\left[
                        \mathbbm{1}_{D^{c}} 
                        \left(
                            A - L
                        \right)  
                        \right] 
                    }{
                        q
                    } 
                    \nonumber \\
                &= 
                    \lim_{q\rightarrow0}
                    \discountfactor
                    \frac{
                        \mathbb{E}^{\mathbb{Q}}\left[
                        \mathbbm{1}_{\mathcal{D}^{c}(q)} 
                            qY
                        \right] 
                        -
                        U(q)R
                        +
                        \mathbb{E}^{\mathbb{Q}}\left[
                        \left(
                            \mathbbm{1}_{\mathcal{D}^{c}(q)}
                            -
                            \mathbbm{1}_{D^{c}} 
                        \right)
                        \left(
                            A - L
                        \right)  
                        \right] 
                    }{
                        q
                    }
                    \nonumber 
            \intertext{
                The limit of each term exists, and they are all known from and accounted for in the previous derivation. 
                Applying the limit yields:
            }
                &=
                    \discountfactor
                    \mathbb{E}^{\mathbb{Q}}\left[
                    \mathbbm{1}_{D^{c}} 
                        Y
                    \right]  
                    -
                    u
                    \nonumber
            \end{align}
        The reformulation of $\mathbb{E}^{\mathbb{Q}}\left[\mathbbm{1}_{D^{c}} Y\right]$ is substituted, 
        and the final result is therefore:
            \begin{align}
                G_{\text{equity}} &=
                    \discountfactor
                    \mathbb{E}^{\mathbb{Q}}\left[\mathbbm{1}_{D^{c}}\right]
                    \mathbb{E}^{\mathbb{Q}}\left[Y\right]
                    -
                    \discountfactor
                    \text{Cov}^{\mathbb{Q}}\left(
                        \mathbbm{1}_{D},
                        Y
                    \right)
                    -
                    u
                    \nonumber \\
                &= 
                    p^{\mathbb{Q}}
                    \left(
                        \pi
                        +
                        u
                    \right)
                    -
                    \discountfactor
                    \text{Cov}^{\mathbb{Q}}\left(
                        \mathbbm{1}_{D},
                        Y
                    \right)
                    -
                    u 
                    \nonumber \\
                &= 
                    p^{\mathbb{Q}}
                    \pi
                    -
                    \discountfactor
                    \text{Cov}^{\mathbb{Q}}\left(
                        \mathbbm{1}_{D},
                        Y
                    \right)
                    -
                    \left(
                        1
                        -
                        p^{\mathbb{Q}}
                    \right)
                    u 
                \label{eqn:marginal-shareholder-value-equity-financing}
            \end{align}
        The two first terms are recognized from the previous section in 
        \cref{eqn:marginal-shareholder-value-debt-financing}
        and their interpretation are unchanged. 
        The third term reflects the fact that, in the event of default, with probability $1-p^{\mathbb{Q}}$, 
        legacy shareholders must give up the entire valuation of the initial investment, $u$.
        This manifests the principle that legacy shareholders should be discouraged 
        from entering into new projects if the survival probability of the firm is too low,
        as they only have benefit potential in no-default states.
        Comparing $(1-p^{\mathbb{Q}})u$ to the quantity $\Phi$ from
        \cref{eqn:marginal-shareholder-value-debt-financing} can provide further insights into this principle.
        Under debt financing, the shareholders bear additional costs 
        due to the credit spread charged by the creditors, 
        but these costs are only to be payed in no-default states,
        where the firm can actually maintain solvency.

\end{document}
% !TEX root = sub-main.tex
\documentclass[main.tex]{subfiles}

\begin{document}
    \subsubsection{Funding by Equity Issuance}
        In addition to obtaining debt, new projects can be financed by issuing additional shares
        that will provide a source of equity capital.
        Share issuances dilute the firm's stocks, which can reduce the value of the existing shareholders' stocks.
        This dilution will be the cause of the conflict of interest with this type of funding.
        It is assumed that the market for newly issued equity is sufficiently competitive,
        such that new shareholders break even when purchasing the newly issued shares;
        in other words, buying the firm's newly issued stock is a zero net present value investment.

        With equity financing, the legacy shareholders still receive the difference between assets and debt claims,
        however they now have to split the bounty with the new shareholders, 
        whom, at time 0, take an amount of shares worth $U(q)$. 
        The new shareholders do not influence whether the firm defaults, so the default event is:
            \begin{equation*}
                \mathcal{D}(q)
                = 
                \{
                    A + qY < L
                \}
            \end{equation*}
        Considering the newly issued shares of value $U(q)$,
        the marginal increase in the value of equity owned by legacy shareholders per unit investment is:
            \begin{align}
                G_{\text{equity}} &=
                    \left.
                    \frac{
                        \partial 
                    }{
                        \partial 
                        q
                    }
                    \mathbb{E}^{\mathbb{Q}}\left[
                        \discountfactor 
                        \left(
                            A + qY - L
                        \right)^{+}
                    \right]
                    -
                    U(q) 
                    \right\rvert_{q=0} 
                    \nonumber
            \end{align}
        This derivative is very much like the derivative encountered under debt issuance.
        Therefore, the reformulation in this section will be less rigorous,
        as most details have already been covered in the previous section.
        Expressing first the derivative as a difference quotient:
            \begin{align}
                G_{\text{equity}} &=
                    \lim_{q\rightarrow0}
                    \discountfactor
                    \frac{
                        \mathbb{E}^{\mathbb{Q}}\left[
                        \left(
                            A + qY - L
                        \right)^{+}
                        \right] 
                        -
                        U(q)R
                        -
                        \mathbb{E}^{\mathbb{Q}}\left[
                        \left(
                            A - L
                        \right)^{+}  
                        \right] 
                    }{
                        q
                    }
                    \nonumber
            \intertext{Replacing max-functions by default indicators:}
                &=
                    \lim_{q\rightarrow0}
                    \discountfactor
                    \frac{
                        \mathbb{E}^{\mathbb{Q}}\left[
                        \mathbbm{1}_{\mathcal{D}^{c}(q)} 
                        \left(
                            A + qY - L
                        \right)
                        \right] 
                        -
                        U(q)R
                        -
                        \mathbb{E}^{\mathbb{Q}}\left[
                        \mathbbm{1}_{\mathcal{D}^{c}} 
                        \left(
                            A - L
                        \right)  
                        \right] 
                    }{
                        q
                    } 
                    \nonumber \\
                &= 
                    \lim_{q\rightarrow0}
                    \discountfactor
                    \frac{
                        \mathbb{E}^{\mathbb{Q}}\left[
                        \mathbbm{1}_{\mathcal{D}^{c}(q)} 
                            qY
                        \right] 
                        -
                        U(q)R
                        +
                        \mathbb{E}^{\mathbb{Q}}\left[
                        \left(
                            \mathbbm{1}_{\mathcal{D}^{c}(q)}
                            -
                            \mathbbm{1}_{\mathcal{D}^{c}} 
                        \right)
                        \left(
                            A - L
                        \right)  
                        \right] 
                    }{
                        q
                    }
                    \nonumber 
            \intertext{
                The limit of each term exists, and they are all known from and accounted for in the previous derivation. 
                Applying the limit yields:
            }
                &=
                    \discountfactor
                    \mathbb{E}^{\mathbb{Q}}\left[
                    \mathbbm{1}_{\mathcal{D}^{c}} 
                        Y
                    \right]  
                    -
                    u
                    \nonumber
            \end{align}
        The reformulation of $\mathbb{E}^{\mathbb{Q}}\left[\mathbbm{1}_{\mathcal{D}^{c}} Y\right]$ is substituted, 
        and the final result is therefore:
            \begin{align}
                G_{\text{equity}} &=
                    \discountfactor
                    \mathbb{E}^{\mathbb{Q}}\left[\mathbbm{1}_{\mathcal{D}^{c}}\right]
                    \mathbb{E}^{\mathbb{Q}}\left[Y\right]
                    -
                    \discountfactor
                    \text{Cov}^{\mathbb{Q}}\left(
                        \mathbbm{1}_{\mathcal{D}},
                        Y
                    \right)
                    -
                    u
                    \nonumber \\
                &= 
                    p^{\mathbb{Q}}
                    \left(
                        \mu
                        +
                        u
                    \right)
                    -
                    \discountfactor
                    \text{Cov}^{\mathbb{Q}}\left(
                        \mathbbm{1}_{\mathcal{D}},
                        Y
                    \right)
                    -
                    u 
                    \label{eqn:G-equity-step-1} \\
                &= 
                    p^{\mathbb{Q}}
                    \mu
                    -
                    \discountfactor
                    \text{Cov}^{\mathbb{Q}}\left(
                        \mathbbm{1}_{\mathcal{D}},
                        Y
                    \right)
                    -
                    \left(
                        1
                        -
                        p^{\mathbb{Q}}
                    \right)
                    u 
                \label{eqn:marginal-shareholder-value-equity-financing}
            \end{align}
        The first two terms are recognized from the previous section in 
        \cref{eqn:marginal-shareholder-value-debt-financing}
        and their interpretation are unchanged. 

        While \cref{eqn:marginal-shareholder-value-equity-financing} is clearly the easiest to compare to 
        \cref{eqn:marginal-shareholder-value-debt-financing},
        \cref{eqn:G-equity-step-1} is easier to interpret, so that will be the focus.
        The first term represents the added value to the legacy shareholders of 
        the firm's asset value increasing with the profit, $\mu$,
        and the cash injection from the new shareholders amounting to $u$.
        These two cash flows are only valuable to the shareholders if the firm can actually maintain solvency,
        so they are discounted by the survival probability, $p^{\mathbb{Q}}$.
        In addition, the legacy shareholders do not actually receive 
        the payment from the new shareholders, due to the stock dilution.
        This is represented by the third term $-u$.
        The new shareholders receive a share of equity worth $u$, 
        and this is clearly a loss to the legacy shareholders.
        The interpretation of the second term has been covered previously.

\end{document}
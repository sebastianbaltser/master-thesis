% !TEX root = ./sub-main.tex
\documentclass[../main.tex]{subfiles}

\begin{document}
    \subsubsection{Funding by existing cash}
        Instead of relying on external resources to fund the project,
        the firm might be in an excess of cash that can be used for financing,
        thus eliminating the need for external funding.
        If the firm uses cash from its balance sheet, the equity increases
        by the present value of the risk-neutral expectation to the project payoff,
        offset by the upfront cost and the opportunity cost of tying up capital in the project;
        capital that could otherwise have been invested in bonds earning the risk-free rate.
        The marginal increase in the equity valuation per unit investment is therefore:
            \begin{equation*}
                G_{\text{cash}} =
                    \left.
                    \frac{\partial }{\partial q} 
                    \discountfactor
                    \frac{
                        \mathbb{E}^{\mathbb{Q}}\left[
                            \left(
                                A - U(q)R + qY - L
                            \right)^{+}
                        \right] 
                    }{
                        q
                    } 
                    \right\rvert_{q=0}
            \end{equation*}
        This derivative is, again, very similar to the derivatives already examined,
        so, for brevity, the calculations in this section will be omitted.
        The derivative can be reformulated to the following:
            \begin{align}
                G_{\text{cash}} &=
                    \discountfactor
                    \mathbb{E}^{\mathbb{Q}}\left[
                        \mathbbm{1}_{D^{c}} Y
                    \right]
                    -
                    u
                    \mathbb{E}\left[\mathbbm{1}_{D^{c}} \right]
                    \nonumber \\
                &=
                    \discountfactor
                    \mathbb{E}^{\mathbb{Q}}\left[\mathbbm{1}_{D^{c}}\right] 
                    \mathbb{E}^{\mathbb{Q}}\left[Y\right] 
                    + 
                    \discountfactor
                    \text{Cov}^{\mathbb{Q}}\left(\mathbbm{1}_{D^{c}}, Y\right) 
                    -
                    u
                    \mathbb{E}\left[\mathbbm{1}_{D^{c}} \right]
                    \nonumber \\
                &= 
                    p^{\mathbb{Q}}
                    \left(
                        \pi + u
                    \right)
                    +
                    \discountfactor
                    \text{Cov}^{\mathbb{Q}}\left(\mathbbm{1}_{D^{c}}, Y\right) 
                    -
                    p^{\mathbb{Q}}
                    u
                    \nonumber \\
                &=
                    p^{\mathbb{Q}}
                    \pi
                    -
                    \discountfactor
                    \text{Cov}^{\mathbb{Q}}\left(\mathbbm{1}_{D}, Y\right) 
                \label{eqn:marginal-shareholder-value-cash-financing}
            \end{align}
        The derived equation only contains terms involving expectations of future cash flows,
        namely the payoff from the project.
        This reflects the fact that at time 0 there is no immediate value lost or gained
        by the shareholders, since, as the project is fairly priced, there is no difference
        in owning cash worth $u$ or a project worth $u$.

        Having derived the impact on equity valuation for these different funding types,
        a natural question to ask is which type of funding is preferred from shareholders perspective.
        This result can be obtained by comparing the three derived equations,
        and it is referred to by \textcite{ADS2016} as Proposition A1: A Pecking Order of Financing Preferences.
        The following section will derive and prove this proposition.
\end{document}
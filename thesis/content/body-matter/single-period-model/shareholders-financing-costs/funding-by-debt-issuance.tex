% !TEX root = ../main.tex
\documentclass[../main.tex]{subfiles}

\begin{document}
    \subsubsection{Funding by debt issuance}
        The firm can fund the upfront cost of the project by issuing new debt.
        This debt will be subject to a rate higher than the risk free rate, 
        however, in order to limit the complexity of the model, 
        it is assumed that a yield spread on debt issuance can only occur due to credit risk. 
        This is ensured by assuming that the market where the firm obtains its debt is fully efficient, 
        such that creditors break even by offering the market value of the firms' debt claims.
        The debt obtained to finance the upfront cost, $U(q)$, is assumed to rank pari passu with existing debt, 
        such that all creditors experience the same loss rate in case of a default.

        % TODO: Proof of the following
        The new creditors receive a credit spread denoted by $s(q)$,
        which generally depend on the share of the payoff being secured and the share being unsecured.
        The limiting spread, $\lim_{q\downarrow0}s(q)$,
        is however invariant to this decomposition and is given by:
            \begin{equation}
                S = \frac{
                    \mathbb{E}^{\mathbb{Q}}\left[\phi\right] 
                    \grossrfrate
                }{
                    1 - \mathbb{E}^{\mathbb{Q}}\left[\phi\right]
                }
            \end{equation}
        with $\phi$ being the loss rate to creditors given by:
            \begin{equation}
                \phi = \frac{
                        D_{FV} - \kappa A
                    }{
                        D_{FV}
                    } 
                \mathbbm{1}_{\mathcal{D}}
            \end{equation}
        The face value of the new debt is the upfront, which was the borrowed amount, 
        including the interest rates paid, namely 
        $U(q)(\grossrfrate + s(q))$. 
        The shareholders receive the residual of the assets after debt claims have been paid
        unless the firm defaults, in which case the shareholders receive nothing. 
        The marginal increase in the value of the firms equity per unit investment is therefore:
            \begin{align}
                & G_{\text{debt}} = 
                \left.
                \frac{
                    \partial 
                }{
                    \partial 
                    q
                }
                \mathbb{E}^{\mathbb{Q}}\left[
                    \discountfactor 
                    \left(
                        A + qY - L - U(q)(\grossrfrate + s(q))
                    \right)^{+}
                \right] 
                \right\rvert_{q=0} 
                \nonumber
            \intertext{Expressing this derivative as a difference quotient, using $U(0)=0$:}
                &=
                \lim_{q\rightarrow0} 
                \discountfactor
                \frac{
                    \mathbb{E}^{\mathbb{Q}}\left[ 
                    \left(
                        A + qY - L - U(q)(\grossrfrate + s(q))
                    \right)^{+}
                    \right]
                    -
                    \mathbb{E}^{\mathbb{Q}}\left[ 
                    \left(
                        A - L
                    \right)^{+}
                \right] 
                }{
                    q
                } 
                \nonumber
            \intertext{
                The argument in the first max-function can be recognized as the function 
                governing the firm's default event after entering the project. 
                Therefore it is positive exactly when the firm does not default and negative otherwise. 
                Applying the max-function, yields zero when the firm defaults
                and the shareholders' payoff when the firm does not default.
                The same idea applies for the argument in the second max-function,
                but this argument instead governs the firm default \textit{before} entering the project.
                Both max-functions can therefore be replaced by using indicators for the default events:
            }
                &=
                \lim_{q\rightarrow0} 
                \discountfactor
                \frac{
                    \mathbb{E}^{\mathbb{Q}}\left[ 
                    \mathbbm{1}_{\mathcal{D}^{c}(q)} 
                    \left(
                        A + qY - L - U(q)(\grossrfrate + s(q))
                    \right)
                    \right]
                    -
                    \mathbb{E}^{\mathbb{Q}}\left[
                    \mathbbm{1}_{D^{c}}  
                    \left(
                        A - L
                    \right)
                \right] 
                }{
                    q
                } \nonumber
            \intertext{Using the linearety of the expectation operator to collect terms involving $A-L$:}
                &=
                \lim_{q\rightarrow0} 
                \discountfactor
                \frac{
                    \mathbb{E}^{\mathbb{Q}}\left[ 
                    \mathbbm{1}_{\mathcal{D}^{c}(q)} 
                    \left(
                        qY - U(q)(\grossrfrate + s(q))
                    \right)
                    \right]
                    +
                    \mathbb{E}^{\mathbb{Q}}\left[
                    \left(
                        \mathbbm{1}_{\mathcal{D}^{c}(q)} 
                        - \mathbbm{1}_{D^{c}}
                    \right)
                    \left(
                        A - L
                    \right)
                \right] 
                }{
                    q
                } 
                \label{eqn:G-debt-step-1}
            \end{align}
        If the limit of the two addends in the numerator is finite, 
        the algebraic limit theorem ensures that the limit of the summation operation,
        can be expressed as the summation of the addends' limits.
        Deriving the limit of the first addend:
            \begin{align}
                &\quad
                \lim_{q\rightarrow 0}
                \discountfactor
                \frac{
                    \mathbb{E}^{\mathbb{Q}}\left[ 
                    \mathbbm{1}_{\mathcal{D}^{c}(q)} 
                    \left(
                        qY - U(q)(\grossrfrate + s(q))
                    \right)
                    \right]
                }{
                    q
                } \nonumber
            \intertext{Using again the linearity of the expectation operator:}
                &= 
                \lim_{q\rightarrow 0}
                \discountfactor
                \mathbb{E}^{\mathbb{Q}}\left[ 
                \mathbbm{1}_{\mathcal{D}^{c}(q)} 
                \left(
                    Y - \frac{U(q)}{q} (\grossrfrate + s(q))
                \right)
                \right]
                \nonumber
            \intertext{
                $A$, $L$, and $Y$ are assumed to have finite expectations, 
                which allows for interchanging limits and integrals, e.g. the expectation.
                The limit of $U(q)/q$ can be recognized as the marginal investment cost, $u$,
                while the limit of $s(q)$ is the limiting spread, $S$.
                The limit of $\mathcal{D}^{c}(q)$ is $D^{c}$,
                since the post-project default event with an infinitesimal investment in the project
                corresponds to the pre-project default event.
                The reformulation again uses the algebraic limit theorem,
                which applies since the limits of the factors exists:
            }
                &=
                \discountfactor
                \mathbb{E}^{\mathbb{Q}}\left[ 
                    \mathbbm{1}_{D^{c}} 
                    \left(
                        Y - u (\grossrfrate + S)
                    \right)
                \right] 
                \label{eqn:G-debt-step-2}
            \end{align}
        Turning to the limit of the second term in \cref{eqn:G-debt-step-1}.
        Using again the assumption that $A$, $L$, and $Y$ have finite expectations,
        and that the limit of $\mathbbm{1}_{\mathcal{D}^{c}(q)} - \mathbbm{1}_{D^{c}}$ equals zero:
            \begin{equation}
                \lim_{q\rightarrow0} 
                \discountfactor
                \frac{
                    \mathbb{E}^{\mathbb{Q}}\left[
                    \left(
                        \mathbbm{1}_{\mathcal{D}^{c}(q)} 
                        - \mathbbm{1}_{D^{c}}
                    \right)
                    \left(
                        A - L
                    \right)
                \right] 
                }{
                    q
                }
                = 0
                \label{eqn:G-debt-step-3}
            \end{equation}
        Applying the results of \cref{eqn:G-debt-step-2} and \cref{eqn:G-debt-step-3}
        to \cref{eqn:G-debt-step-1} yields:
            \begin{align}
                G_{\text{debt}} &=
                \discountfactor
                \mathbb{E}^{\mathbb{Q}}\left[
                    \mathbbm{1}_{D^{c}} 
                    \left(
                        Y - u (\grossrfrate + S)
                    \right)
                \right]
            \intertext{
                Using linearity of expectations and rearranging terms:
            }
                &=
                \discountfactor\left(
                    \mathbb{E}^{\mathbb{Q}}\left[
                        \mathbbm{1}_{D^{c}}Y
                    \right]
                    - 
                    \mathbb{E}^{\mathbb{Q}}\left[
                        \mathbbm{1}_{D^{c}} u (\grossrfrate + S)
                    \right]
                \right)
                \nonumber \\
                &=
                \discountfactor
                \mathbb{E}^{\mathbb{Q}}\left[
                    \mathbbm{1}_{D^{c}}Y
                \right]
                - 
                \mathbb{E}^{\mathbb{Q}}\left[
                    \mathbbm{1}_{D^{c}}
                \right] u
                - 
                \mathbb{E}^{\mathbb{Q}}\left[
                    \mathbbm{1}_{D^{c}}
                \right] \discountfactor u S
                \label{eqn:G-debt-step-4}
            \end{align}
        The expected value of two dependent terms in the first term, 
        $\mathbb{E}^{\mathbb{Q}}\left[\mathbbm{1}_{D^{c}}Y\right]$,
        can be reformulated by applying the definition of the covariance operator.
        Using the linearity of the expectation operator,
        the definition of covariance between two random variables $W$ and $Z$ can be rewritten:
            \begin{align}
                \text{Cov}\left(W, Z\right) &= 
                \mathbb{E}\left[
                    (W - \mathbb{E}\left[W\right])
                    (Z - \mathbb{E}\left[Z\right])
                \right] \nonumber\\
                &=
                \mathbb{E}\left[WZ\right]
                - \mathbb{E}\left[W\right] \mathbb{E}\left[Z\right] \nonumber
            \end{align}
        Using this result, the aforementioned expectation can be rewritten as follows:
            \begin{align}
                \mathbb{E}^{\mathbb{Q}}\left[
                    \mathbbm{1}_{D^{c}}Y
                \right]
                &=
                \mathbb{E}^{\mathbb{Q}}\left[\mathbbm{1}_{D^{c}}\right] 
                \mathbb{E}^{\mathbb{Q}}\left[Y\right] 
                + 
                \text{Cov}^{\mathbb{Q}}\left(\mathbbm{1}_{D^{c}}, Y\right) 
                \nonumber\\
                &=
                \mathbb{E}^{\mathbb{Q}}\left[\mathbbm{1}_{D^{c}}\right] 
                \mathbb{E}^{\mathbb{Q}}\left[Y\right] 
                - 
                \text{Cov}^{\mathbb{Q}}\left(\mathbbm{1}_{D}, Y\right) 
            \end{align}
        Where the last equality uses the definition of $D^{C}$ combined with a property of the covariance operator.
        Substituting into \cref{eqn:G-debt-step-4} finally yields the marginal value to shareholders of debt financing:
            \begin{align}
                G_{\text{debt}} &=
                \mathbb{E}^{\mathbb{Q}}\left[\mathbbm{1}_{D^{c}}\right] 
                \left(
                    \discountfactor
                    \mathbb{E}^{\mathbb{Q}}\left[Y\right] 
                    - u
                \right)
                -
                \discountfactor
                \text{Cov}^{\mathbb{Q}}\left(\mathbbm{1}_{D}, Y\right) 
                - 
                \mathbb{E}^{\mathbb{Q}}\left[
                    \mathbbm{1}_{D^{c}}
                \right] \discountfactor u S
                \nonumber \\
                &= 
                p^{\mathbb{Q}} \pi 
                - \discountfactor 
                \text{Cov}^{\mathbb{Q}}\left(
                    \mathbbm{1}_{D}, 
                    Y
                \right) 
                - \Phi
                \label{eqn:marginal-shareholder-value-debt-financing}
            \end{align}
        Where $p^{\mathbb{Q}} = 1 - \mathbb{P}^{\mathbb{Q}}\left(D\right)$ is the risk neutral probability of the firm not defaulting. 
        $\pi = \discountfactor \mathbb{E}^{\mathbb{Q}}\left[Y\right] - u$ 
        is the difference between the present value of the expected payoff and the upfront price, 
        i.e. $\pi$ is the promised marginal profit on the new project. 
        $\Phi = p^{\mathbb{Q}} \discountfactor u S$
        is the present value of the marginal excess return on the upfront price, 
        discounted by the probability of the firm not defaulting.
        $\Phi$ can be interpreted as the present value to the shareholders of their share of the financing costs, $uS$,
        which they pay iff the firm does not default. 
        If the firm's default event is positively correlated with the payoff of the project,
        the shareholders will miss out on an even higher firm value when the firm defaults.
        The opposite is the case when the default event is negatively correlated with the payoff.
        This effect is captured by the term 
        $\text{Cov}^{\mathbb{Q}}\left(\mathbbm{1}_{D}, Y\right)$.

\end{document}
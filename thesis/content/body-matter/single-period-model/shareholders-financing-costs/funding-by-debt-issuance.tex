% !TEX root = ./sub-main.tex
\documentclass[../main.tex]{subfiles}

\begin{document}
    \subsubsection{Funding by Debt Issuance}
        \label{sec:marginal-valuation-debt-issuance}
        The firm can fund the upfront cost of the project by issuing new debt.
        This debt will be subject to a rate higher than the risk-free rate, 
        however, in order to limit the complexity of the model, 
        it is assumed that a yield spread on debt issuance can only occur due to credit risk. 
        This is ensured by assuming that the market where the firm obtains its debt is fully efficient, 
        such that creditors break even by offering the market value of the firms' debt claims.
        The debt obtained to finance the upfront cost, $U(q)$, is assumed to rank pari passu with existing debt.

        The new creditors receive a credit spread denoted by $s(q)$,
        which generally depends on the share of the payoff being secured and the share being unsecured.
        The face value of the newly issued debt is therefore $U(q)(R + s(q))$.
        The firm defaults if it is not able to pay both the legacy liabilities and the new debt,
        therefore the default event is:
            \begin{equation*}
                \mathcal{D}(q)
                = 
                \{
                    A + qY < L + U(q)(R + s(q))
                \}
            \end{equation*}
        Since creditors, by assumption, break even on the new debt issued to finance the cost $U(q)$,
        the credit spread, $s(q)$, must solve the following:
            \begin{align}
                    U(q) 
                &=
                    \discountfactor
                    \mathbb{E}^{\rnmeasure}\left[
                        \mathbbm{1}_{\mathcal{D}^{c}(q)}
                        U(q)
                        (\grossrfrate + s(q))
                    \right]
                \nonumber \\
                &\quad+
                    \discountfactor
                    \mathbb{E}^{\rnmeasure}\left[
                        \mathbbm{1}_{\mathcal{D}(q)}
                        \frac{
                            \kappa 
                            \left(
                                A + q Y_{1} + q Y_{2}^{+}
                            \right)
                        }{
                            L 
                            +
                            U(q)
                            (\grossrfrate + s(q))
                            +
                            q Y_{2}^{-} 
                        } 
                        U(q)
                        (\grossrfrate + s(q))    
                    \right] 
                \label{eqn:new-creditors-breakeven}
            \end{align}
        Hence, the present value of the expected payoff to the new creditors
        must equal the amount needed to fund the new project, $U(q)$. 
        The expected payoff can be thought of as consisting of two legs;
        one leg that governs the payoff when the firm does not default, 
        and another that governs the payoff when the firm does default. 
        The first term in this equation is the contribution to the expected value from the promised payoff,
        the face value, which is paid when the firm does not default. 
        The face value equals the borrowed amount, $U(q)$, 
        forward appreciated by the gross risk-free rate and the credit spread.
        
        The second term is the contribution from the payoff when the firm defaults. 
        The asset base consists of the legacy assets, $A$, the secured payoff, $qY_{1}$, 
        and the unsecured payoff if it is in the firm's favour, $qY_{2}^{+}$.
        The secured payoff might be in the firm's favour, in which case the counterparty must pay it. 
        Alternatively, it is a liability in which case it is posted as collateral before the default,
        and therefore not part of the estate and not available for the creditors.
        If the unsecured payoff is a receivable for the firm, 
        the counterparty should pay it, like it should pay the secured payoff.
        If it is a liability, the counterparty ranks pari passu with the other claimants,
        and therefore experiences the loss rate. 
        The asset base is reduced by the distress costs, $1-\kappa$ so the estate available for claimants
        is a share, $\kappa$, of the assets. 
        This is the quantity found in the numerator of the fraction.
        
        The denominator represents the total amount owed at time 1, 
        which is the sum of the pre-project liabilities, $L$, the face value of the new debt, 
        and the liabilities due to the unsecured payoff of the project.
        The secured liabilities, if any, are already posted as collateral.
        The amount available for paying claimants divided by the amount owed to claimants, is the loss rate.

        Since the credit spread depends on the decomposition of the payoff into a secured and unsecured part,
        it is easier to consider the limiting spread, $\lim_{q\rightarrow0}s(q)$. 
        This quantity is derived in the following paragraphs.
        Dividing first both sides of \cref{eqn:new-creditors-breakeven} by the face value $U(q)(\grossrfrate+s(q))$ 
        and using the linearity of the expected value operator:
            \begin{align}
                    \frac{1}{\grossrfrate + s(q)} 
                &=
                    \discountfactor
                    \mathbb{E}^{\rnmeasure}\left[
                        \mathbbm{1}_{\mathcal{D}^{c}(q)}
                        +
                        \mathbbm{1}_{\mathcal{D}(q)}
                        \frac{
                            \kappa 
                            \left(
                                A + q Y_{1} + q Y_{2}^{+}
                            \right)
                        }{
                            L 
                            +
                            U(q)
                            (\grossrfrate + s(q))
                            +
                            q Y_{2}^{-} 
                        } 
                    \right]
                \nonumber
                \intertext{
                    Since the events $\mathcal{D}^{c}(q)$ and $\mathcal{D}(q)$ are complements,
                    the indicator of the no-default event, $\mathbbm{1}_{\mathcal{D}^{c}(q)}$,
                    can be rewritten to $1 - \mathbbm{1}_{\mathcal{D}(q)}$:
                }
                        \frac{1}{\grossrfrate + s(q)} 
                    &=
                        \discountfactor
                        \mathbb{E}^{\rnmeasure}\left[
                            1
                            +
                            \mathbbm{1}_{\mathcal{D}(q)}
                            \left(
                                \frac{
                                    \kappa 
                                    \left(
                                        A + q Y_{1} + q Y_{2}^{+}
                                    \right)
                                }{
                                    L 
                                    +
                                    U(q)
                                    (\grossrfrate + s(q))
                                    +
                                    q Y_{2}^{-} 
                                }
                                -
                                1
                            \right) 
                        \right]
                    \nonumber
            \end{align}
        If the firm defaults, the loss rate is sure to be greater than zero,
        since these two conditions are equivalent. 
        Hence, the right hand side is strictly positive, as well as non-zero,
        and the multiplicative inverse transformation can be applied to both sides:
            \begin{align}
                    s(q)
                &=
                    \grossrfrate
                    \left(
                        \mathbb{E}^{\rnmeasure}\left[
                            1
                            +
                            \mathbbm{1}_{\mathcal{D}(q)}
                            \left(
                                \frac{
                                    \kappa 
                                    \left(
                                        A + q Y_{1} + q Y_{2}^{+}
                                    \right)
                                }{
                                    L 
                                    +
                                    U(q)
                                    (\grossrfrate + s(q))
                                    +
                                    q Y_{2}^{-} 
                                }
                                -
                                1
                            \right) 
                        \right]^{-1}
                        - 
                        1
                    \right)
                \nonumber
            \end{align}
        Let $S$ denote the limiting spread which is defined as the credit spread on debt 
        charged for obtaining an infinitesimal quantity of the project, i.e.
        $S \equiv \lim_{q\rightarrow 0} s(q)$.
        If the limits of two addends in a sum are finite, 
        the algebraic limit theorem ensures that the limit of the summation operation,
        can be expressed as the summation of the addends' limits.
        This also applies to two factors in a product.
        Since $A$, $L$, and $Y$ have finite expectations,
        limits and integrals, e.g. the expectation in the above equation, can be interchanged.

        Applying the limit of $q$ approaching $0$:
            \begin{align}
                    S
                &=
                    \grossrfrate
                    \left(
                        \mathbb{E}^{\rnmeasure}\left[
                            1
                            +
                            \lim_{q\rightarrow 0}
                            \mathbbm{1}_{\mathcal{D}(q)}
                            \left(
                                \frac{
                                    \kappa 
                                    \left(
                                        A + q Y_{1} + q Y_{2}^{+}
                                    \right)
                                }{
                                    L 
                                    +
                                    U(q)
                                    (\grossrfrate + s(q))
                                    +
                                    q Y_{2}^{-} 
                                }
                                -
                                1
                            \right) 
                        \right]^{-1}
                        - 
                        1
                    \right)
                \nonumber
            \end{align}
        The limit of $\mathcal{D}(q)$ is $\mathcal{D}$,
        since the post-project default event with an infinitesimal investment in the project
        corresponds to the pre-project default event.
        In addition, the upfront price of the project, $U(q)$,
        approaches zero as the quantity of the project approaches zero.
        Applying these results yields:
            \begin{align}
                    S
                &=
                    \grossrfrate
                    \left(
                        \mathbb{E}^{\rnmeasure}\left[
                            1
                            +
                            \mathbbm{1}_{\mathcal{D}}
                            \left(
                                \frac{
                                    \kappa 
                                    A
                                }{
                                    L 
                                }
                                -
                                1
                            \right) 
                        \right]^{-1}
                        - 
                        1
                    \right)
                \nonumber \\
                &=
                    \grossrfrate
                    \left(
                        \mathbb{E}^{\rnmeasure}\left[
                            1
                            +
                            \mathbbm{1}_{\mathcal{D}}
                                \frac{
                                    \kappa 
                                    A
                                    -
                                    L
                                }{
                                    L 
                                }
                        \right]^{-1}
                        - 
                        1
                    \right)
                \label{eqn:limiting-credit-spread-step-1}
            \end{align}
        For ease of notation and interpretation the loss rate can be defined as:
            \begin{equation}
                \phi = \frac{
                        L - \kappa A
                    }{
                        L
                    } 
                \mathbbm{1}_{\mathcal{D}}
                \label{eqn:creditor-loss-rate}
            \end{equation}
        \Cref{eqn:limiting-credit-spread-step-1} can then be reformulated and finally result in the limiting spread:
            \begin{align}
                    S
                &=
                    \grossrfrate
                    \left(
                        \frac{
                            1   
                        }{
                            1
                            -
                            \mathbb{E}^{\rnmeasure}\left[
                                \phi
                            \right]
                        }
                        - 
                        1
                    \right)
                \nonumber \\
                &=
                    \frac{
                        \mathbb{E}^{\rnmeasure}\left[\phi\right]
                        \grossrfrate
                    }{
                        1
                        -
                        \mathbb{E}^{\rnmeasure}\left[\phi\right]
                    }
                \label{eqn:limiting-spread}
            \end{align}
        As anticipated, the limiting spread is invariant to the decomposition of the payoff into $Y_{1}$ and $Y_{2}$,
        as well as being independent to the quantity $q$. 
        These properties will make the limiting spread very useful in the following derivation
        of the marginal shareholder valuations.

        Having derived the spread on new debt for financing an infinitesimal project,
        the shareholders valuation of obtaining this new project can be derived.
        This derivation starts of by defining the payoff to the shareholders 
        of obtaining a project of size $q$,
        and then taking the derivative of that with respect to $q$.
        
        Again, the face value of the new debt is the upfront including the interest rates paid;
        namely, $U(q)(\grossrfrate + s(q))$. 
        The shareholders receive the residual of the assets after debt claims have been paid
        unless the firm defaults, in which case the shareholders receive nothing. 
        The marginal increase in the value of the firms equity per unit investment is therefore:
            \begin{equation*}
                G_{\text{debt}} 
                = 
                \left.
                \frac{
                    \partial 
                }{
                    \partial 
                    q
                }
                \mathbb{E}^{\mathbb{Q}}\left[
                    \discountfactor 
                    \left(
                        A + qY - L - U(q)(\grossrfrate + s(q))
                    \right)^{+}
                \right] 
                \right\rvert_{q=0} 
            \end{equation*}
        Expressing this derivative as a difference quotient, using $U(0)=0$:
            \begin{equation*}
                G_{\text{debt}} 
                =
                \lim_{q\rightarrow0} 
                \discountfactor
                \frac{
                    \mathbb{E}^{\mathbb{Q}}\left[ 
                    \left(
                        A + qY - L - U(q)(\grossrfrate + s(q))
                    \right)^{+}
                    \right]
                    -
                    \mathbb{E}^{\mathbb{Q}}\left[ 
                    \left(
                        A - L
                    \right)^{+}
                \right] 
                }{
                    q
                } 
            \end{equation*}
        The argument in the first max-function can be recognized as the function 
        governing the firm's default event after entering the project. 
        Therefore it is positive exactly when the firm does not default and negative otherwise. 
        Applying the max-function, yields zero when the firm defaults
        and the shareholders' payoff when the firm does not default.
        The same idea applies for the argument in the second max-function,
        but this argument instead governs the firm default \textit{before} entering the project.
        Both max-functions can therefore be replaced by using indicators for the default events:
            \begin{align}                
                G_{\text{debt}}
                &=
                \lim_{q\rightarrow0} 
                \discountfactor
                \frac{
                    \mathbb{E}^{\mathbb{Q}}\left[ 
                    \mathbbm{1}_{\mathcal{D}^{c}(q)} 
                    \left(
                        A + qY - L - U(q)(\grossrfrate + s(q))
                    \right)
                    \right]
                    -
                    \mathbb{E}^{\mathbb{Q}}\left[
                    \mathbbm{1}_{\mathcal{D}^{c}}  
                    \left(
                        A - L
                    \right)
                \right] 
                }{
                    q
                } \nonumber
            \intertext{Using the linearity of the expectation operator to collect terms involving $A-L$:}
                &=
                \lim_{q\rightarrow0} 
                \discountfactor
                \frac{
                    \mathbb{E}^{\mathbb{Q}}\left[ 
                    \mathbbm{1}_{\mathcal{D}^{c}(q)} 
                    \left(
                        qY - U(q)(\grossrfrate + s(q))
                    \right)
                    \right]
                    +
                    \mathbb{E}^{\mathbb{Q}}\left[
                    \left(
                        \mathbbm{1}_{\mathcal{D}^{c}(q)} 
                        - \mathbbm{1}_{\mathcal{D}^{c}}
                    \right)
                    \left(
                        A - L
                    \right)
                \right] 
                }{
                    q
                } 
                \label{eqn:G-debt-step-1}
            \end{align}
        Using the algebraic limit theorem and deriving the limit of the first addend:
            \begin{align}
                &\quad
                \lim_{q\rightarrow 0}
                \discountfactor
                \frac{
                    \mathbb{E}^{\mathbb{Q}}\left[ 
                    \mathbbm{1}_{\mathcal{D}^{c}(q)} 
                    \left(
                        qY - U(q)(\grossrfrate + s(q))
                    \right)
                    \right]
                }{
                    q
                } \nonumber
            \intertext{Using again the linearity of the expectation operator:}
                &= 
                \lim_{q\rightarrow 0}
                \discountfactor
                \mathbb{E}^{\mathbb{Q}}\left[ 
                \mathbbm{1}_{\mathcal{D}^{c}(q)} 
                \left(
                    Y - \frac{U(q)}{q} (\grossrfrate + s(q))
                \right)
                \right]
                \nonumber
            \intertext{
                The assumption of finite expectations of $A$, $L$, and $Y$ allows, again,
                for interchanging the limit and the expectation.
                The limit of $U(q)/q$ can be recognized as the marginal investment cost, $u$,
                while the limit of $s(q)$ is the limiting spread, $S$.
                The limit of the post-project no-default event, $\mathcal{D}^{c}(q)$, 
                is the pre-project no-default event $\mathcal{D}^{c}$.
                The reformulation again uses the algebraic limit theorem,
                which applies since the limits of the factors exist:
            }
                &=
                \discountfactor
                \mathbb{E}^{\mathbb{Q}}\left[ 
                    \mathbbm{1}_{\mathcal{D}^{c}} 
                    \left(
                        Y - u (\grossrfrate + S)
                    \right)
                \right] 
                \label{eqn:G-debt-step-2}
            \end{align}
        Turning to the limit of the second term in \cref{eqn:G-debt-step-1}.
        Using again the assumption that $A$, $L$, and $Y$ have finite expectations,
        the following expression is obtained:
            \begin{equation*}
                \discountfactor
                \mathbb{E}^{\mathbb{Q}}\left[
                    \lim_{q\rightarrow0} 
                    \frac{
                        \mathbbm{1}_{\mathcal{D}^{c}(q)} 
                        - \mathbbm{1}_{\mathcal{D}^{c}}
                    }{
                        q
                    }
                    \left(
                        A - L
                    \right)
                \right] 
            \end{equation*}
        Using the algebraic limit theorem on 
        $\lim_{q\rightarrow0}
        \left(
            (\mathbbm{1}_{\mathcal{D}^{c}(q)} 
            - \mathbbm{1}_{\mathcal{D}^{c}})
        / q
        \right) 
        $
        yields an indeterminate form, "$0/0$", since both the numerator and the denominator 
        approach zero as $q$ approaches zero.
        This limit can therefore be evaluated by applying 
        L'H\^{o}pital's  % cspell: disable-line
        rule.
        The derivative of the denominator, $q$, with respect to $q$, is $1$,
        which is different from $0$ as required. 
        \\
        Turning to the numerator, $\mathbbm{1}_{\mathcal{D}^{c}(q)} - \mathbbm{1}_{\mathcal{D}^{c}}$.
        The pre-project no-default event, $\mathbbm{1}_{\mathcal{D}^{c}}$ is constant in $q$, 
        and its derivative is zero. 
        For an infinitesimal investment the post-project no-default indicator, 
        $\mathbbm{1}_{\mathcal{D}^{c}(q)}$, can only change its value at the point $A = L$. 
        However, the assumption $\mathbb{P}\left(A = L\right) = 0$ ensures that this does not occur;
        the derivative of the post-project no-default event
        is therefore surely zero in an interval around $q=0$.
        Hence, it can be concluded that:
        \begin{equation}
            \lim_{q\rightarrow0} 
            \discountfactor
            \frac{
            \mathbb{E}^{\mathbb{Q}}\left[
                \left(
                    \mathbbm{1}_{\mathcal{D}^{c}(q)} 
                    - \mathbbm{1}_{\mathcal{D}^{c}}
                \right)
                \left(
                    A - L
                \right)
            \right] 
            }{
                q
            }
            = 0
            \label{eqn:G-debt-step-3}
        \end{equation}
        Applying the results of \cref{eqn:G-debt-step-2} and \cref{eqn:G-debt-step-3}
        to \cref{eqn:G-debt-step-1} yields:
            \begin{align}
                G_{\text{debt}} &=
                \discountfactor
                \mathbb{E}^{\mathbb{Q}}\left[
                    \mathbbm{1}_{\mathcal{D}^{c}} 
                    \left(
                        Y - u (\grossrfrate + S)
                    \right)
                \right]
                \nonumber
            \intertext{
                Using linearity of expectations and rearranging terms:
            }
                &=
                \discountfactor\left(
                    \mathbb{E}^{\mathbb{Q}}\left[
                        \mathbbm{1}_{\mathcal{D}^{c}}Y
                    \right]
                    - 
                    \mathbb{E}^{\mathbb{Q}}\left[
                        \mathbbm{1}_{\mathcal{D}^{c}} u (\grossrfrate + S)
                    \right]
                \right)
                \nonumber \\
                &=
                \discountfactor
                \mathbb{E}^{\mathbb{Q}}\left[
                    \mathbbm{1}_{\mathcal{D}^{c}}Y
                \right]
                - 
                \mathbb{E}^{\mathbb{Q}}\left[
                    \mathbbm{1}_{\mathcal{D}^{c}}
                \right] u
                - 
                \mathbb{E}^{\mathbb{Q}}\left[
                    \mathbbm{1}_{\mathcal{D}^{c}}
                \right] \discountfactor u S
                \label{eqn:G-debt-step-4}
            \end{align}
        The expected value of two dependent terms in the first term, 
        $\mathbb{E}^{\mathbb{Q}}\left[\mathbbm{1}_{\mathcal{D}^{c}}Y\right]$,
        can be reformulated by applying the definition of the covariance operator.
        This reformulation will later make interpretation easier.
        Using the linearity of the expectation operator,
        the definition of covariance between two random variables $W$ and $Z$ can be rewritten:
            \begin{align}
                \text{Cov}\left(W, Z\right) &= 
                \mathbb{E}\left[
                    (W - \mathbb{E}\left[W\right])
                    (Z - \mathbb{E}\left[Z\right])
                \right] \nonumber\\
                &=
                \mathbb{E}\left[WZ\right]
                - \mathbb{E}\left[W\right] \mathbb{E}\left[Z\right] \nonumber
            \end{align}
        According to this result, the aforementioned expectation can be rewritten as follows:
            \begin{align}
                \mathbb{E}^{\mathbb{Q}}\left[
                    \mathbbm{1}_{\mathcal{D}^{c}}Y
                \right]
                &=
                \mathbb{E}^{\mathbb{Q}}\left[\mathbbm{1}_{\mathcal{D}^{c}}\right] 
                \mathbb{E}^{\mathbb{Q}}\left[Y\right] 
                + 
                \text{Cov}^{\mathbb{Q}}\left(\mathbbm{1}_{\mathcal{D}^{c}}, Y\right) 
                \nonumber\\
                &=
                \mathbb{E}^{\mathbb{Q}}\left[\mathbbm{1}_{\mathcal{D}^{c}}\right] 
                \mathbb{E}^{\mathbb{Q}}\left[Y\right] 
                - 
                \text{Cov}^{\mathbb{Q}}\left(\mathbbm{1}_{\mathcal{D}}, Y\right) 
            \end{align}
        Where the last equality uses $\mathbbm{1}_{\mathcal{D}^{c}} = 1 - \mathbbm{1}_{\mathcal{D}}$ 
        combined with the standard properties of the covariance operator.
        Substituting into \cref{eqn:G-debt-step-4} finally yields the marginal value to shareholders of debt financing:
            \begin{align}
                G_{\text{debt}} &=
                \mathbb{E}^{\mathbb{Q}}\left[\mathbbm{1}_{\mathcal{D}^{c}}\right] 
                \left(
                    \discountfactor
                    \mathbb{E}^{\mathbb{Q}}\left[Y\right] 
                    - u
                \right)
                -
                \discountfactor
                \text{Cov}^{\mathbb{Q}}\left(\mathbbm{1}_{\mathcal{D}}, Y\right) 
                - 
                \mathbb{E}^{\mathbb{Q}}\left[
                    \mathbbm{1}_{\mathcal{D}^{c}}
                \right] \discountfactor u S
                \nonumber \\
                &= 
                p^{\mathbb{Q}} \pi 
                - \discountfactor 
                \text{Cov}^{\mathbb{Q}}\left(
                    \mathbbm{1}_{\mathcal{D}}, 
                    Y
                \right) 
                - \Phi
                \label{eqn:marginal-shareholder-value-debt-financing}
            \end{align}
        Where $p^{\mathbb{Q}} = 1 - \mathbb{P}^{\mathbb{Q}}\left(\mathcal{D}\right)$ 
        is the risk-neutral probability of the firm not defaulting. 
        $\pi = \discountfactor \mathbb{E}^{\mathbb{Q}}\left[Y\right] - u$ 
        is the difference between the present value of the expected payoff and the upfront price, 
        i.e. $\pi$ is the promised marginal profit on the new project. 
        $\Phi = p^{\mathbb{Q}} \discountfactor u S$
        is the present value of the marginal excess return on the upfront price, 
        discounted by the probability of the firm not defaulting.
        $\Phi$ can be interpreted as the present value to the shareholders of their share of the financing costs, $uS$,
        which they pay iff the firm does not default. 
        If the firm's default event is positively correlated with the payoff of the project,
        the shareholders will miss out on an even higher firm value when the firm defaults.
        The opposite is the case when the default event is negatively correlated with the payoff.
        This effect is captured by the term 
        $\text{Cov}^{\mathbb{Q}}\left(\mathbbm{1}_{\mathcal{D}}, Y\right)$.

\end{document}
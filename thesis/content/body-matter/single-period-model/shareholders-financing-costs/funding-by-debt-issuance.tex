% !TEX root = ../main.tex
\documentclass[../main.tex]{subfiles}

\begin{document}
    \subsubsection{Funding by debt issuance}
        The firm can fund the upfront cost of the project by issuing new debt.
        This debt will be subject to a rate higher than the risk free rate, 
        however, in order to limit the complexity of the model, 
        it is assumed that a yield spread on debt issuance can only occur due to credit risk. 
        This is ensured by assuming that the market where the firm obtains its debt is fully efficient, 
        such that creditors break even by offering the market value of the firms' debt claims.
        The debt obtained to finance the upfront cost, $U(q)$, is assumed to rank pari passu with existing debt, 
        such that all creditors experience the same loss rate in case of a default.

        % TODO: Proof of the following
        The new creditors receive a credit spread denoted by $s(q)$,
        which generally depend on the share of the payoff being secured and the share being unsecured.
        The limiting spread, $\lim_{q\downarrow0}s(q)$,
        is however invariant to this decomposition and is given by:
            \begin{equation}
                S = \frac{
                    \mathbb{E}^{\mathbb{Q}}\left[\phi\right] 
                    \grossrfrate
                }{
                    1 - \mathbb{E}^{\mathbb{Q}}\left[\phi\right]
                }
            \end{equation}
        with $\phi$ being the loss rate to creditors given by:
            \begin{equation}
                \phi = \frac{
                        D_{FV} - \kappa A
                    }{
                        D_{FV}
                    } 
                \mathbbm{1}_{\mathcal{D}}
            \end{equation}
        The face value of the new debt is the upfront, which was the borrowed amount, 
        including the interest rates paid, namely 
        $U(q)(\grossrfrate + s(q))$. 
        The shareholders receive the residual of the assets after debt claims have been paid
        unless the firm defaults, in which case the shareholders receive nothing. 
        The marginal increase in the value of the firms equity per unit investment is therefore:
            \begin{equation}
                G_{\text{debt}} = 
                \left.
                \frac{
                    \partial 
                }{
                    \partial 
                    q
                }
                \mathbb{E}^{\mathbb{Q}}\left[
                    \discountfactor 
                    \left(
                        A + qY - L - U(q)(\grossrfrate + s(q))
                        A + qY - L - U(q)(1 + \rfrate + s(q))
                        A + qY - D_{FV} - U(q)(1 + \rfrate + s(q))
                    \right)^{+}
                \right] 
                \right\rvert_{q=0} 
            \end{equation}
        
        % TODO: Proof

        Deriving this derivative yields the marginal value to shareholders of debt financing:

            \begin{equation}
                G_{\text{debt}} = 
                    p^{\mathbb{Q}} \pi 
                    - \discountfactor 
                    \text{Cov}^{\mathbb{Q}}\left(
                        \mathbbm{1}_{\mathcal{D}}, 
                        Y
                    \right) 
                    - \Phi
                    \label{eqn:marginal-shareholder-value-debt-financing}
            \end{equation}
        Where $p^{\mathbb{Q}} = 1 - \mathbb{P}^{\mathbb{Q}}\left(\mathcal{D}\right)$ is the risk neutral probability of the firm not defaulting. 
        $\pi = \discountfactor \mathbb{E}^{\mathbb{Q}}\left[Y\right] - u$ 
        is the difference between the present value of the expected payoff and the upfront price, 
        i.e. $\pi$ is the promised marginal profit on the new project. 
        $\Phi = p^{\mathbb{Q}} \discountfactor u S$
        is the present value of the marginal excess return on the upfront price, 
        discounted by the probability of the firm not defaulting.
        $\Phi$ can be interpreted as the present value to the shareholders of their share of the financing costs, $uS$,
        which they pay iff the firm does not default. 
        If the firm's default event is positively correlated with the payoff of the project,
        the shareholders will miss out on an even higher firm value when the firm defaults.
        The opposite is the case when the default event is negatively correlated with the payoff.
        This effect is captured by the term 
        $\text{Cov}^{\mathbb{Q}}\left(\mathbbm{1}_{\mathcal{D}}, Y\right)$.

\end{document}
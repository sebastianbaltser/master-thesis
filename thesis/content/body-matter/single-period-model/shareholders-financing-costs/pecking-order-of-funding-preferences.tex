% !TEX root = ./sub-main.tex
\documentclass[../main.tex]{subfiles}

\begin{document}
    \subsubsection{A pecking order of funding preferences}
        The result of this section will be the inequality relations between the marginal shareholder valuations
        derived in the previous sections, namely $G_{\text{debt}}$, $G_{\text{equity}}$, and $G_{\text{cash}}$.
        This will imply a ranking of the funding methods according to their utility to the shareholders.
        The direction of the inequality will be determined by the sign of the marginal investment cost, $u$,
        i.e. whether the project initially provides or claims capital. 
        First, the project is assumed to claim capital, such that $u$ is strictly positive,
        and relations between marginal shareholder valuations for other signs of $u$ will follow directly after.

        Comparing first the marginal shareholder valuations between debt and equity funding:
            \begin{align}
                G_{\text{debt}}
                -
                G_{\text{equity}}
                &=
                    - \Phi 
                    - 
                    (
                        - (1 - p^{\rnmeasure})
                        u
                    )
                    \nonumber \\
                &= 
                    - p^{\rnmeasure}
                    \discountfactor
                    u
                    S
                    +
                    (1 - p^{\rnmeasure})
                    u
                    \nonumber \\
                &=
                    u \left(
                        1
                        -
                        p^{\rnmeasure}
                        \discountfactor
                        \left(
                            R
                            +
                            S
                        \right)
                    \right)
                \label{eqn:G-debt-vs-G-equity-step-1}
            \end{align}
        Recalling the expression for the loss rate, $\phi$, and the limiting spread, $S$:
            \begin{equation*}
                \phi 
                = 
                    \frac{
                        L - \kappa A
                    }{
                        L
                    } 
                    \mathbbm{1}_{\mathcal{D}}
                \qquad
                S 
                = 
                    \frac{
                        \grossrfrate
                        \mathbb{E}^{\rnmeasure}\left[\phi\right]
                    }{
                        1
                        -
                        \mathbb{E}^{\rnmeasure}\left[\phi\right]  
                    } 
            \end{equation*}
        Since $(L-\kappa A) / L \leq 1$ the following holds:
            \begin{align}
                    \mathbb{E}^{\rnmeasure}\left[\phi\right]
                &\leq
                    \mathbb{E}^{\rnmeasure}\left[\mathbbm{1}_{\mathcal{D}} \right]
                \nonumber \\
                \Leftrightarrow 
                \qquad
                    \mathbb{E}^{\rnmeasure}\left[\phi\right]
                &\leq
                    1 - p^{\rnmeasure}
                \label{G-debt-vs-G-equity-inequality-1}
                \\
                \Leftrightarrow 
                \qquad
                p^{\rnmeasure}
                &\leq 
                1 - \mathbb{E}^{\rnmeasure}\left[\phi\right]
                \label{G-debt-vs-G-equity-inequality-2}
            \end{align}
        Returning to \cref{eqn:G-debt-vs-G-equity-step-1} and substituting the expression for $S$:
            \begin{align}
                G_{\text{debt}}
                -
                G_{\text{equity}}
                &=
                    u \left(
                        1
                        -
                        p^{\rnmeasure}
                        -
                        \frac{
                            p^{\rnmeasure}
                        }{
                            1
                            -
                            \mathbb{E}^{\rnmeasure}\left[\phi\right]
                        } 
                        \mathbb{E}^{\rnmeasure}\left[\phi\right]
                    \right)
            \end{align}
        Due to \cref{G-debt-vs-G-equity-inequality-2} it holds that
        $\frac{p^{\rnmeasure}}{1-\mathbb{E}^{\rnmeasure}\left[\phi\right]} \leq 1$,
        which, combined with \cref{G-debt-vs-G-equity-inequality-1}, ensures that:
            \begin{equation}
                G_{\text{debt}}
                \geq 
                G_{\text{equity}}
                \label{G-debt-vs-G-equity}
            \end{equation}
        Hence, shareholders' weakly prefer debt funding over equity funding in terms of their value lost.
        While shareholders will bear the cost of interest payments on loans obtained under debt funding,
        the dilution from share issuance is evidently less beneficial to them. 
        If the probability of default is non-zero, all inequalities are strict,
        implying strong preference of debt funding over equity funding, $G_{\text{debt}} > G_{\text{equity}}$.
        
\end{document}
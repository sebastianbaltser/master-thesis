% !TEX root = ./sub-main.tex
\documentclass[../main.tex]{subfiles}

\begin{document}
    \subsubsection{The pecking order of funding preferences}
    \label{sec:pecking-order-theory}
        The result of this section will be the inequality relations between the marginal shareholder valuations
        derived in the previous sections, namely $G_{\text{debt}}$, $G_{\text{equity}}$, and $G_{\text{cash}}$.
        This will imply a ranking of the funding methods according to their utility to the shareholders.
        The direction of the inequalities will be determined by the sign of the marginal investment cost, $u$,
        i.e. whether the project initially provides or claims capital. 
        Initially, the project is assumed to claim capital, such that $u$ is strictly positive;
        relations between marginal shareholder valuations for other signs of $u$ will follow directly after.
        Comparing first the marginal shareholder valuations between debt and equity funding:
            \begin{align}
                G_{\text{debt}}
                -
                G_{\text{equity}}
                &=
                    - \Phi 
                    - 
                    (
                        - (1 - p^{\rnmeasure})
                        u
                    )
                    \nonumber \\
                &= 
                    - p^{\rnmeasure}
                    \discountfactor
                    u
                    S
                    +
                    (1 - p^{\rnmeasure})
                    u
                    \nonumber \\
                &=
                    u \left(
                        1
                        -
                        p^{\rnmeasure}
                        \discountfactor
                        \left(
                            R
                            +
                            S
                        \right)
                    \right)
                \label{eqn:G-debt-vs-G-equity-step-1}
            \end{align}
        Recalling the expression for the loss rate, $\phi$, and the limiting spread, $S$:
            \begin{equation*}
                \phi 
                = 
                    \frac{
                        L - \kappa A
                    }{
                        L
                    } 
                    \mathbbm{1}_{\mathcal{D}}
                \qquad
                S 
                = 
                    \frac{
                        \grossrfrate
                        \mathbb{E}^{\rnmeasure}\left[\phi\right]
                    }{
                        1
                        -
                        \mathbb{E}^{\rnmeasure}\left[\phi\right]  
                    } 
            \end{equation*}
        Since $(L-\kappa A) / L \leq 1$ the following holds:
            \begin{align}
                    \mathbb{E}^{\rnmeasure}\left[\phi\right]
                &\leq
                    \mathbb{E}^{\rnmeasure}\left[\mathbbm{1}_{\mathcal{D}} \right]
                \nonumber \\
                \Leftrightarrow 
                \qquad
                    \mathbb{E}^{\rnmeasure}\left[\phi\right]
                &\leq
                    1 - p^{\rnmeasure}
                \label{eqn:G-debt-vs-G-equity-inequality-1}
                \\
                \Leftrightarrow 
                \qquad
                p^{\rnmeasure}
                &\leq 
                1 - \mathbb{E}^{\rnmeasure}\left[\phi\right]
                \label{eqn:G-debt-vs-G-equity-inequality-2}
            \end{align}
        Returning to \cref{eqn:G-debt-vs-G-equity-step-1} and substituting the expression for $S$:
            \begin{align}
                G_{\text{debt}}
                -
                G_{\text{equity}}
                &=
                    u \left(
                        1
                        -
                        p^{\rnmeasure}
                        -
                        \frac{
                            p^{\rnmeasure}
                        }{
                            1
                            -
                            \mathbb{E}^{\rnmeasure}\left[\phi\right]
                        } 
                        \mathbb{E}^{\rnmeasure}\left[\phi\right]
                    \right)
                \nonumber
            \end{align}
        Due to \cref{eqn:G-debt-vs-G-equity-inequality-2} it holds that
        $\frac{p^{\rnmeasure}}{1-\mathbb{E}^{\rnmeasure}\left[\phi\right]} \leq 1$,
        which, combined with \cref{eqn:G-debt-vs-G-equity-inequality-1}, ensures that:
            \begin{equation}
                \Rightarrow 
                \qquad
                G_{\text{debt}}
                \geq 
                G_{\text{equity}}
                \label{eqn:G-debt-vs-G-equity}
            \end{equation}
        Hence, shareholders weakly prefer debt funding over equity funding in terms of their perceived value lost.
        While shareholders will bear the cost of interest payments on loans obtained under debt funding,
        the dilution from share issuance is evidently less beneficial to them. 
        If the probability of default is non-zero, all inequalities are strict,
        implying strong preference of debt funding over equity funding, $G_{\text{debt}} > G_{\text{equity}}$.
        
        Turning now to the relation between the marginal shareholder valuations of cash and debt funding:
            \begin{align}
                G_{\text{debt}}
                -
                G_{\text{cash}}
                &=
                    - p^{\rnmeasure}
                    \discountfactor
                    u
                    S
            \intertext{
                All factors in the product on the right hand side are non-negative, implying the relation:
            }
                \Rightarrow
                \qquad
                    G_{\text{debt}}
                &\leq
                    G_{\text{cash}}
                \label{eqn:G-debt-vs-G-cash}
            \end{align}
        Again, if the probability of default is non-zero, the limiting spread is positive
        and the preference is strict, $G_{\text{debt}} < G_{\text{cash}}$.
        As shown, shareholders prefer funding with cash over funding with debt.
        While thoroughly derived here, it does not come as much of a surprise that this relation holds.
        The mechanism of using cash from the balance sheet are essentially identical to using cash obtained from debt, 
        except for the fact that the funding cost of debt is higher than that of balance sheet cash.
        Clearly, shareholders will rather want to suffer the opportunity cost of the risk-free interest rate
        than both the risk-free interest rate \textit{and} an additional spread due to the firms credit risk.

        \Cref{eqn:G-debt-vs-G-equity} combined with \cref{eqn:G-debt-vs-G-cash},
        the assumption that the marginal investment cost is strictly positive,
        and the assumption that the firm's probability of default is non-zero yields the following relation:
            \begin{equation}
                G_{\text{equity}}
                <
                G_{\text{debt}}
                <
                G_{\text{cash}}
                \label{eqn:pecking-order-of-financing-preferences}
            \end{equation}
        This result is referred to as the pecking order of financing preferences 
        and by \textcite{ADS2019} as Proposition A1.
        If the investment cost, $u$, is negative and the project provides capital, 
        the inequalities are reversed such that equity buyback is preferred over debt retiring
        that is preferred over buying bonds earning the risk-free rate.
        If there is no upfront payment to finance, $u=0$, or if the default probability is zero,
        the marginal shareholder valuations are zero, regardless of funding method, such that
        $G_{\text{equity}} = G_{\text{debt}} = G_{\text{cash}} = 0$.

        The equations describing the marginal valuations of projects to shareholders, 
        derived in this section, will be essential to the definition of \FVA/ in the next section.
        In addition, they will be helpful for understanding the impact of projects to shareholders 
        through examples in subsequent sections.

\end{document}
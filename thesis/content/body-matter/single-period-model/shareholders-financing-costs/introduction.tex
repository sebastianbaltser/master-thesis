% !TEX root = ./sub-main.tex
\documentclass[../main.tex]{subfiles}

\begin{document}
    This section will develop the theoretical framework for quantifying the impact of financing costs on the stakeholders of the firm.
    The theoretical results derived are mainly from \textcite{ADS2019} and will be followed by practical examples where they are applied.

    % TODO: The following section does not currently touch upon the section in ADS2019:
    %       "The Marginal Valuation of Corporate Assets, Liabilities, and Other Claims"
    %       though it probably should.

    The analysis will still regard the firm considering an investment in a project.
    For generality, the firm considers investing in $q$ units of the project; 
    the payoff is therefore $qY$ while the upfront cost is a function of $q$ and denoted $U(q)$. 
    The marginal investment cost per unit invested is given as 
    $u = \lim_{q\downarrow 0} U(q) / q$.
    Throughout it is assumed that the asset value, $A$, equals the face value of debt, $L$, 
    with probability zero, i.e. $\mathbb{P}\left(A = L\right) = 0$.
    This assumption is necessary to avoid singularities when calculating derivatives
    and it will be used in a proof later.

    The following subsections will derive the marginal shareholder value from entering into the project
    under three different funding assumptions, specifically debt, equity and cash funding.
\end{document}
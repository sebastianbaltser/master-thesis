% !TEX root = sub-main.tex
\documentclass[main.tex]{subfiles}

\begin{document}
    When the firm invests in the new project it has implications for its stakeholders,
    specifically its shareholders and creditors.
    The upfront price requires some type of financing, 
    which the firm will have to decide on.
    Obtaining this financing will have an influence on the value of the shareholders' claims;
    this section will derive the theoretical equations defining
    the impact of financing costs on the firm's shareholders.
    The theoretical results derived are mainly from \textcite{ADS2019} 
    and will be followed by practical examples where they are applied.

    For generality, the firm considers investing in $q$ units of the project; 
    the payoff is therefore $qY$ while the upfront cost is a function of $q$ and denoted $U(q)$. 
    The marginal investment cost per unit invested is given as 
    $u = \lim_{q\rightarrow 0} U(q) / q$.
    Throughout it is assumed that the asset value, $A$, equals the face value of debt, $L$, 
    with probability zero, i.e. $\mathbb{P}\left(A = L\right) = 0$.
    This assumption is necessary to avoid singularities when calculating derivatives
    and it will be used in a proof later.

    The following subsections will derive the marginal shareholder value from entering into the project
    under three different funding assumptions, specifically debt, equity and cash funding.
\end{document}
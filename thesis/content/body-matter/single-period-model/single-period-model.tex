% !TEX root = sub-main.tex
\documentclass[main.tex]{subfiles}

\begin{document}
    \section{Theorizing Funding Costs}
    \label{sec:single-period-model}
        The section will start off by defining the single-period model,
        followed by introducing a structural model of a firm that operates in such an economy.
        The firm will then be faced with a potential new project.
        The section will conclude with an mathematical assessment of 
        how obtaining the new project affects the wealth of the firm's shareholders.

    \subsection{The Single-Period Model}
        The single-period framework is modelling what will happen in an economy as it transitions from the present, time 0, to one time period ahead, time 1.
        At time 1 the economy can materialize in one of $N$ possible states, $\omega_{1}, \dots, \omega_{N}$, and each state occur with a strictly positive probability.
        A state is defined by the price of an Arrow-Debreu security paying one unit of numeraire if the state is achieved and zero otherwise.
        The price of this security is referred to as the state price and denoted $\psi_{i}$.
        The vector of state prices, $\psi \equiv \left(\psi_{1}, \dots, \psi_{N}\right)'$, can be used to price claims in the economy, since a state price defines the value of receiving a cash flow in a particular state.
        For simplicity, all investors are assumed to be in possession of the same market information.

        To characterize the market valuation of financial derivatives represented on a firm's balance sheet, the finite set, $\mathcal{L}$, of payoffs at time 1 is assigned a fair value at time 0 by some "fair market value"-function $V\, \colon \mathcal{L} \rightarrow \mathbb{R}$.
        Two required assumptions are imposed on the market valuation assignment: 
        (i) $V(\cdot)$ is linear, such that the value of a portfolio of cash flows is the sum of the values of the elements of the portfolio, and 
        (ii) $V(\cdot)$ is increasing in payoffs, such that if payoff $X$ is greater than or equal to payoff $Y$ in all states, and if $X>Y$ in some states, then $V(X)>V(Y)$.
        These assumptions imply that $V$ is strictly positive, such that any $X>0$ yields $V(X) > 0$.
        
        The price of a payoff $d=\left(d_{1}, \dots, d_{N}\right)'$ is given as:
            \begin{equation}
                \pi(d) = d'\psi = \sum_{i=1}^{N} d_{i}\psi_{i}
            \end{equation}
        Especially, a zero coupon bond paying one unit of numeraire in every state has the price:
            \begin{equation}
                d_{0} = \sum_{i=1}^{N} \psi_{i}
            \end{equation}
        from which the risk-free rate can be defined as:
            \begin{equation}
                \rfrate = \frac{1}{d_{0}} - 1
            \end{equation}
        The gross risk-free rate will be denoted by $\grossrfrate \equiv 1 + \rfrate$.
        The product between the gross risk-free rate and the state price vector produces the risk-neutral probability distribution, $\mathbb{Q}$, i.e. $q_{i} = \grossrfrate\psi_{i}$.

        In an OTC market, the market valuations of a derivative need not coincide with the price at which dealers are trading this derivative.
        At almost the same point in time, the same asset can be traded at several different prices reflecting distinct bids and asks of different dealers.
        The associated deviation in prices is partly explained by search costs, differences in dealer-client relationships, and differences in the dealers' capital structure.

        Having defined how valuation is achieved in the single-period model,
        the following section will introduce a firm operating in such an economy.

    \subsection{Firm capital structure}
        This section will describe the mathematical structure of a firm in a single-period model.
        This precedes describing how the firm's structure changes after it has obtained a new investment,
        
        Consider a firm operating in a single-period economy.
        The firm can invest in assets with risk but needs financing for supporting its investments.
        If the firm has no cash, financing can only be obtained at two different sources;
        namely, equity funding through distribution of the firm's own stocks or issuance of debt.
        The value of the firm's assets at time $1$ is given by the random variable $A$.
        At time 1 the firm's creditors are promised to receive some cash flow denoted by $L$.
        This cash flow might be fixed, i.e. a face value, 
        or random, i.e. depending on the realized state.
        In the event that the value of the firm, $A$, is not sufficient to pay the face value in its entirety, 
        the firm defaults, and the creditors receive the remaining estate.
        \\
        The firm's default event will be denoted by $\mathcal{D}$ and defined as $\{A < L\}$.
        The complement to the firm's default, its survival event, 
        will be denoted by $\mathcal{D}^{c}$.
        In a default event, the firm can suffer distress costs due to for example liquidation of its assets.
        The remaining estate after a default depends on the recovery parameter $\kappa \in (0;1]$, 
        such that the remaining estate is $\kappa A$.
        
        Given the possibility of a default, the creditors' payoff is random; 
        it is equal to $L$ in states where the firm does not default
        and equal to the remaining assets, $\kappa A$ in states where the firm does default.
        \\
        The time 1 payoff to the creditors will be denoted by $D$, short for debt,
        and is mathematically given by:
            \begin{equation}
                D
                =
                \mathbbm{1}_{\mathcal{D}} \kappa A
                +
                \mathbbm{1}_{\mathcal{D}^{c}} L
            \end{equation}
        
        The firm is owned by its shareholders who have a limited liability claim on the firm.
        Shareholders receive the remainder of the firm value after the debt has been paid;
        in particular, they receive nothing when the firm defaults.
        Hence, the shareholders' payoff is also random;
        equal to $A - L$ in no-default states and equal to zero in default states.
        \\
        The time 1 payoff to the shareholders will be denoted by $E$, short for equity,
        and is given by:
            \begin{equation}
                E
                =
                \mathbbm{1}_{\mathcal{D}^{c}} 
                (A - L)
            \end{equation}
        
        The next section will describe this same firm structure,
        but after the firm has obtained some new investment. 

\end{document}
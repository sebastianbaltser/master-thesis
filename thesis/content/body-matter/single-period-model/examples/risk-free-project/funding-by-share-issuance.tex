% !TEX root = sub-main.tex
\documentclass[main.tex]{subfiles}

\begin{document}
    \subsubsection{Increasing Funding Costs With Equity Issuance}
    \label{sec:example-equity-issuance}
        This section will consider how funding with equity issuance
        compares to results of the previous section, where the project was financed with debt issuance.
        Funding by share issuance will create a dilution, and the value of the already existing shares, owned by the legacy shareholders, will be reduced. 
        To compare debt and equity funding, 
        the project under consideration will be the same as in the previous section.
        \\
        A simple illustration of the new setup is depicted in \cref{fig:equity-issuance-setup}.

        The random payoff of the new shareholders will be denoted by $\tilde{E}$.
        The new shareholders will subscribe to the newly issued shares 
        only to the extent that the investment has a net present value of zero.
        To be able to finance the project, the value of the new equity
        should therefore equal the upfront price of the project:
        \begin{equation*}
            u = \pi(\tilde{E})
        \end{equation*}
        The new equity will lead to an increase in the total asset value of the firm,
        reducing the loss rate of the creditors in the default state.
        Therefore, the creditors should see an increase in the value of their claim.\\
        The increase in wealth for the creditors is, however, at the expense of the legacy shareholders,
        since, like under debt financing, the shareholders bear the funding costs.
        As shareholders have an even higher required rate of return compared to creditors,
        the wealth transfer from the shareholders in this section should also be higher.
        \begin{figure}[t]
            \centering
            \resizebox{\textwidth}{!}{%
                \begin{tikzpicture}
                    \import{\graphicsfolder/numerical-examples/}{equity-issuance.tex}
                \end{tikzpicture}
            }
            \caption{Illustration of funding a bond by issuing new equity.}
            \label{fig:equity-issuance-setup}
        \end{figure}

        The payoff of the legacy shareholders is computed slightly differently than in the case of debt issuance. 
        It is now essential to know the share of the equity the new shareholders are entitled to.
        This share will be denoted by $\alpha$.
        The new shareholders' share equals the size of their investment,
        relative to the present value of total equity after the capital inflow:
        \begin{equation*}
            \alpha = 
                \frac{
                    u
                }{
                    \discountfactor
                    \mathbb{E}^{\rnmeasure}\left[
                        (A + Y - L)^{+}
                    \right]
                }
        \end{equation*}
        The expected value of the equity is 
        $\mathbb{E}^{\rnmeasure}\left[(A + Y - L)^{+}\right] = \num{20.676768}$,
        and with a project price of $u = 0.99$, the new shareholders are entitled to a share of
        $\alpha = \pct{0.048363}$ of the total equity.
        The new shareholders then receive their share of the equity in each state, 
        and the payoff to the legacy shareholders is the residual share of the total equity.
        The random variables denoting the shareholders' payoffs are therefore given by:
            \begin{align*}
                E 
                &= 
                (1 - \alpha)
                (A + Y - L)^{+} 
                \\
                \tilde{E}
                &=
                \alpha
                (A + Y - L)^{+}
            \end{align*}
        
        After the transaction, the firm's balance sheet has increased with an amount equal to the premium of the derivative. 
        The firm has a new asset on the asset side, the derivative receivable worth $\num{0.989994}$, which funded by $\num{0.989994}$ worth of new equity.
        The resulting payoffs of the example are shown in \cref{tbl:equity-funding-payoff},
        where the realised asset values include the expected payoff of the project.
        \begin{table}[H]
            \centering\begin{tabular}{l|rrrrr||r}
                $i$ & 1 & 2 & 3 & 4 & 5 & Present value \\
                \hline
                \rule{0pt}{1.1em}
                $\tilde{A}(\omega_{i})$ & $\num{121}$ & $\num{111}$ & $\num{101}$ & $\num{96}$ & $\num{61}$ & $\num{97.39}$ \\
                $D(\omega_{i})$ & $\num{80}$ & $\num{80}$ & $\num{80}$ & $\num{80}$ & $\num{61}$ & $\num{76.92}$ \\
                $E(\omega_{i})$ & $\num{39.017098}$ & $\num{29.500733}$ & $\num{19.984367}$ & $\num{15.226185}$ & $\num{0}$ & $\num{19.48}$ \\
                $\tilde{E}(\omega_{i})$ & $\num{1.982902}$ & $\num{1.499267}$ & $\num{1.015633}$ & $\num{0.773815}$ & $\num{0}$ & $\num{0.989994}$ \\
            \end{tabular}
            \caption{}
            \label{tbl:equity-funding-payoff}
        \end{table}

        As anticipated,
        the project is a negative net present value investment for the legacy shareholders. 
        The decrease in the value of their claim is calculated using
        \cref{eqn:marginal-shareholder-value-equity-financing}:
        \begin{equation*}
                G_{\text{equity}}
            =
                - (\num{1} - \pct{0.878787}) * \num{0.989994}
            = 
                -\num{0.120001}
        \end{equation*}
        This amount corresponds to the transfer of wealth 
        from the legacy shareholders to the creditors,
        since the project is a zero net present value investment.
        Evidently, in accordance with the pecking order theory in 
        \cref{eqn:pecking-order-of-financing-preferences},
        the value lost by the legacy shareholders 
        is even higher for equity issuance than for debt issuance.

        With the wealth transfer from the shareholders,
        the project is a positive net present value investment for the creditors;
        their wealth increase by:
        \begin{equation*}\label{eqn:wealth-transfer-equality}
            -G_{\text{equity}} = \num{76.92} - \num{76.80} = \num{0.120001}
        \end{equation*}
        The value of the debt increases, 
        since the creditors get the entire derivative's receivables in the default state. 
        Following this, the loss rate as well as the credit spread decreases significantly 
        compared to the pre-project metrics. 
        The new loss rate in the default state is 
        $\phi(\omega_{5}) = (\num{80}-\num{61})/\num{80} = \pct{0.2375}$, 
        and the new credit spread is $\num{80}/\num{76.92} - \grossrfrate = \pct{0.029941}$.

        When trading decisions are made, the firm's preferences are assumed to be determined by the shareholders.
        It is clear that the legacy shareholders' position deteriorates, 
        as they bear the funding costs from financing the project.
        So, to get the shareholders on board would require a donation by the counterparty, 
        via a reduction in the upfront price of the project.

        The breakeven price necessary to account for funding costs under equity funding is calculated using
        \cref{eqn:shareholders-breakeven-equity-financing}:
            \begin{equation*}
                    u^{\ast}_{\text{equity}} 
                =
                    \frac{1}{1 + \pct{0.010101}} 
                    *
                    \pct{0.878787}
                    *
                    \left(
                        \num{1}
                        -
                        \num{0}
                    \right)
                =
                    \num{0.869999}
            \end{equation*}
        meaning 
        \begin{equation*}
            \FVA/ = u^{\ast}_{\text{equity}} - u =
            \num{0.989994}-\num{0.869999} =
            \num{0.12}
        \end{equation*}
        Notice that this amount is the same as the wealth transfer.
        This is because the donation from the counterparty does not affect the total asset value of the firm.
        Hence, the loss rate of the creditors remains unchanged from before the donation, and therefore so does the creditors' present value.

        \begin{figure}
            \centering
            \resizebox{\textwidth}{!}{%
                %% Creator: Matplotlib, PGF backend
%%
%% To include the figure in your LaTeX document, write
%%   \input{<filename>.pgf}
%%
%% Make sure the required packages are loaded in your preamble
%%   \usepackage{pgf}
%%
%% Also ensure that all the required font packages are loaded; for instance,
%% the lmodern package is sometimes necessary when using math font.
%%   \usepackage{lmodern}
%%
%% Figures using additional raster images can only be included by \input if
%% they are in the same directory as the main LaTeX file. For loading figures
%% from other directories you can use the `import` package
%%   \usepackage{import}
%%
%% and then include the figures with
%%   \import{<path to file>}{<filename>.pgf}
%%
%% Matplotlib used the following preamble
%%
\begingroup%
\makeatletter%
\begin{pgfpicture}%
\pgfpathrectangle{\pgfpointorigin}{\pgfqpoint{7.000000in}{4.000000in}}%
\pgfusepath{use as bounding box, clip}%
\begin{pgfscope}%
\pgfsetbuttcap%
\pgfsetmiterjoin%
\definecolor{currentfill}{rgb}{1.000000,1.000000,1.000000}%
\pgfsetfillcolor{currentfill}%
\pgfsetlinewidth{0.000000pt}%
\definecolor{currentstroke}{rgb}{0.500000,0.500000,0.500000}%
\pgfsetstrokecolor{currentstroke}%
\pgfsetdash{}{0pt}%
\pgfpathmoveto{\pgfqpoint{0.000000in}{0.000000in}}%
\pgfpathlineto{\pgfqpoint{7.000000in}{0.000000in}}%
\pgfpathlineto{\pgfqpoint{7.000000in}{4.000000in}}%
\pgfpathlineto{\pgfqpoint{0.000000in}{4.000000in}}%
\pgfpathlineto{\pgfqpoint{0.000000in}{0.000000in}}%
\pgfpathclose%
\pgfusepath{fill}%
\end{pgfscope}%
\begin{pgfscope}%
\pgfsetbuttcap%
\pgfsetmiterjoin%
\definecolor{currentfill}{rgb}{0.898039,0.898039,0.898039}%
\pgfsetfillcolor{currentfill}%
\pgfsetlinewidth{0.000000pt}%
\definecolor{currentstroke}{rgb}{0.000000,0.000000,0.000000}%
\pgfsetstrokecolor{currentstroke}%
\pgfsetstrokeopacity{0.000000}%
\pgfsetdash{}{0pt}%
\pgfpathmoveto{\pgfqpoint{0.875000in}{0.440000in}}%
\pgfpathlineto{\pgfqpoint{6.300000in}{0.440000in}}%
\pgfpathlineto{\pgfqpoint{6.300000in}{3.520000in}}%
\pgfpathlineto{\pgfqpoint{0.875000in}{3.520000in}}%
\pgfpathlineto{\pgfqpoint{0.875000in}{0.440000in}}%
\pgfpathclose%
\pgfusepath{fill}%
\end{pgfscope}%
\begin{pgfscope}%
\pgfpathrectangle{\pgfqpoint{0.875000in}{0.440000in}}{\pgfqpoint{5.425000in}{3.080000in}}%
\pgfusepath{clip}%
\pgfsetrectcap%
\pgfsetroundjoin%
\pgfsetlinewidth{0.803000pt}%
\definecolor{currentstroke}{rgb}{1.000000,1.000000,1.000000}%
\pgfsetstrokecolor{currentstroke}%
\pgfsetdash{}{0pt}%
\pgfpathmoveto{\pgfqpoint{0.875000in}{0.440000in}}%
\pgfpathlineto{\pgfqpoint{0.875000in}{3.520000in}}%
\pgfusepath{stroke}%
\end{pgfscope}%
\begin{pgfscope}%
\pgfsetbuttcap%
\pgfsetroundjoin%
\definecolor{currentfill}{rgb}{0.333333,0.333333,0.333333}%
\pgfsetfillcolor{currentfill}%
\pgfsetlinewidth{0.803000pt}%
\definecolor{currentstroke}{rgb}{0.333333,0.333333,0.333333}%
\pgfsetstrokecolor{currentstroke}%
\pgfsetdash{}{0pt}%
\pgfsys@defobject{currentmarker}{\pgfqpoint{0.000000in}{-0.048611in}}{\pgfqpoint{0.000000in}{0.000000in}}{%
\pgfpathmoveto{\pgfqpoint{0.000000in}{0.000000in}}%
\pgfpathlineto{\pgfqpoint{0.000000in}{-0.048611in}}%
\pgfusepath{stroke,fill}%
}%
\begin{pgfscope}%
\pgfsys@transformshift{0.875000in}{0.440000in}%
\pgfsys@useobject{currentmarker}{}%
\end{pgfscope}%
\end{pgfscope}%
\begin{pgfscope}%
\definecolor{textcolor}{rgb}{0.333333,0.333333,0.333333}%
\pgfsetstrokecolor{textcolor}%
\pgfsetfillcolor{textcolor}%
\pgftext[x=0.875000in,y=0.342778in,,top]{\color{textcolor}\rmfamily\fontsize{10.000000}{12.000000}\selectfont \(\displaystyle {0.70}\)}%
\end{pgfscope}%
\begin{pgfscope}%
\pgfpathrectangle{\pgfqpoint{0.875000in}{0.440000in}}{\pgfqpoint{5.425000in}{3.080000in}}%
\pgfusepath{clip}%
\pgfsetrectcap%
\pgfsetroundjoin%
\pgfsetlinewidth{0.803000pt}%
\definecolor{currentstroke}{rgb}{1.000000,1.000000,1.000000}%
\pgfsetstrokecolor{currentstroke}%
\pgfsetdash{}{0pt}%
\pgfpathmoveto{\pgfqpoint{1.650000in}{0.440000in}}%
\pgfpathlineto{\pgfqpoint{1.650000in}{3.520000in}}%
\pgfusepath{stroke}%
\end{pgfscope}%
\begin{pgfscope}%
\pgfsetbuttcap%
\pgfsetroundjoin%
\definecolor{currentfill}{rgb}{0.333333,0.333333,0.333333}%
\pgfsetfillcolor{currentfill}%
\pgfsetlinewidth{0.803000pt}%
\definecolor{currentstroke}{rgb}{0.333333,0.333333,0.333333}%
\pgfsetstrokecolor{currentstroke}%
\pgfsetdash{}{0pt}%
\pgfsys@defobject{currentmarker}{\pgfqpoint{0.000000in}{-0.048611in}}{\pgfqpoint{0.000000in}{0.000000in}}{%
\pgfpathmoveto{\pgfqpoint{0.000000in}{0.000000in}}%
\pgfpathlineto{\pgfqpoint{0.000000in}{-0.048611in}}%
\pgfusepath{stroke,fill}%
}%
\begin{pgfscope}%
\pgfsys@transformshift{1.650000in}{0.440000in}%
\pgfsys@useobject{currentmarker}{}%
\end{pgfscope}%
\end{pgfscope}%
\begin{pgfscope}%
\definecolor{textcolor}{rgb}{0.333333,0.333333,0.333333}%
\pgfsetstrokecolor{textcolor}%
\pgfsetfillcolor{textcolor}%
\pgftext[x=1.650000in,y=0.342778in,,top]{\color{textcolor}\rmfamily\fontsize{10.000000}{12.000000}\selectfont \(\displaystyle {0.75}\)}%
\end{pgfscope}%
\begin{pgfscope}%
\pgfpathrectangle{\pgfqpoint{0.875000in}{0.440000in}}{\pgfqpoint{5.425000in}{3.080000in}}%
\pgfusepath{clip}%
\pgfsetrectcap%
\pgfsetroundjoin%
\pgfsetlinewidth{0.803000pt}%
\definecolor{currentstroke}{rgb}{1.000000,1.000000,1.000000}%
\pgfsetstrokecolor{currentstroke}%
\pgfsetdash{}{0pt}%
\pgfpathmoveto{\pgfqpoint{2.425000in}{0.440000in}}%
\pgfpathlineto{\pgfqpoint{2.425000in}{3.520000in}}%
\pgfusepath{stroke}%
\end{pgfscope}%
\begin{pgfscope}%
\pgfsetbuttcap%
\pgfsetroundjoin%
\definecolor{currentfill}{rgb}{0.333333,0.333333,0.333333}%
\pgfsetfillcolor{currentfill}%
\pgfsetlinewidth{0.803000pt}%
\definecolor{currentstroke}{rgb}{0.333333,0.333333,0.333333}%
\pgfsetstrokecolor{currentstroke}%
\pgfsetdash{}{0pt}%
\pgfsys@defobject{currentmarker}{\pgfqpoint{0.000000in}{-0.048611in}}{\pgfqpoint{0.000000in}{0.000000in}}{%
\pgfpathmoveto{\pgfqpoint{0.000000in}{0.000000in}}%
\pgfpathlineto{\pgfqpoint{0.000000in}{-0.048611in}}%
\pgfusepath{stroke,fill}%
}%
\begin{pgfscope}%
\pgfsys@transformshift{2.425000in}{0.440000in}%
\pgfsys@useobject{currentmarker}{}%
\end{pgfscope}%
\end{pgfscope}%
\begin{pgfscope}%
\definecolor{textcolor}{rgb}{0.333333,0.333333,0.333333}%
\pgfsetstrokecolor{textcolor}%
\pgfsetfillcolor{textcolor}%
\pgftext[x=2.425000in,y=0.342778in,,top]{\color{textcolor}\rmfamily\fontsize{10.000000}{12.000000}\selectfont \(\displaystyle {0.80}\)}%
\end{pgfscope}%
\begin{pgfscope}%
\pgfpathrectangle{\pgfqpoint{0.875000in}{0.440000in}}{\pgfqpoint{5.425000in}{3.080000in}}%
\pgfusepath{clip}%
\pgfsetrectcap%
\pgfsetroundjoin%
\pgfsetlinewidth{0.803000pt}%
\definecolor{currentstroke}{rgb}{1.000000,1.000000,1.000000}%
\pgfsetstrokecolor{currentstroke}%
\pgfsetdash{}{0pt}%
\pgfpathmoveto{\pgfqpoint{3.200000in}{0.440000in}}%
\pgfpathlineto{\pgfqpoint{3.200000in}{3.520000in}}%
\pgfusepath{stroke}%
\end{pgfscope}%
\begin{pgfscope}%
\pgfsetbuttcap%
\pgfsetroundjoin%
\definecolor{currentfill}{rgb}{0.333333,0.333333,0.333333}%
\pgfsetfillcolor{currentfill}%
\pgfsetlinewidth{0.803000pt}%
\definecolor{currentstroke}{rgb}{0.333333,0.333333,0.333333}%
\pgfsetstrokecolor{currentstroke}%
\pgfsetdash{}{0pt}%
\pgfsys@defobject{currentmarker}{\pgfqpoint{0.000000in}{-0.048611in}}{\pgfqpoint{0.000000in}{0.000000in}}{%
\pgfpathmoveto{\pgfqpoint{0.000000in}{0.000000in}}%
\pgfpathlineto{\pgfqpoint{0.000000in}{-0.048611in}}%
\pgfusepath{stroke,fill}%
}%
\begin{pgfscope}%
\pgfsys@transformshift{3.200000in}{0.440000in}%
\pgfsys@useobject{currentmarker}{}%
\end{pgfscope}%
\end{pgfscope}%
\begin{pgfscope}%
\definecolor{textcolor}{rgb}{0.333333,0.333333,0.333333}%
\pgfsetstrokecolor{textcolor}%
\pgfsetfillcolor{textcolor}%
\pgftext[x=3.200000in,y=0.342778in,,top]{\color{textcolor}\rmfamily\fontsize{10.000000}{12.000000}\selectfont \(\displaystyle {0.85}\)}%
\end{pgfscope}%
\begin{pgfscope}%
\pgfpathrectangle{\pgfqpoint{0.875000in}{0.440000in}}{\pgfqpoint{5.425000in}{3.080000in}}%
\pgfusepath{clip}%
\pgfsetrectcap%
\pgfsetroundjoin%
\pgfsetlinewidth{0.803000pt}%
\definecolor{currentstroke}{rgb}{1.000000,1.000000,1.000000}%
\pgfsetstrokecolor{currentstroke}%
\pgfsetdash{}{0pt}%
\pgfpathmoveto{\pgfqpoint{3.975000in}{0.440000in}}%
\pgfpathlineto{\pgfqpoint{3.975000in}{3.520000in}}%
\pgfusepath{stroke}%
\end{pgfscope}%
\begin{pgfscope}%
\pgfsetbuttcap%
\pgfsetroundjoin%
\definecolor{currentfill}{rgb}{0.333333,0.333333,0.333333}%
\pgfsetfillcolor{currentfill}%
\pgfsetlinewidth{0.803000pt}%
\definecolor{currentstroke}{rgb}{0.333333,0.333333,0.333333}%
\pgfsetstrokecolor{currentstroke}%
\pgfsetdash{}{0pt}%
\pgfsys@defobject{currentmarker}{\pgfqpoint{0.000000in}{-0.048611in}}{\pgfqpoint{0.000000in}{0.000000in}}{%
\pgfpathmoveto{\pgfqpoint{0.000000in}{0.000000in}}%
\pgfpathlineto{\pgfqpoint{0.000000in}{-0.048611in}}%
\pgfusepath{stroke,fill}%
}%
\begin{pgfscope}%
\pgfsys@transformshift{3.975000in}{0.440000in}%
\pgfsys@useobject{currentmarker}{}%
\end{pgfscope}%
\end{pgfscope}%
\begin{pgfscope}%
\definecolor{textcolor}{rgb}{0.333333,0.333333,0.333333}%
\pgfsetstrokecolor{textcolor}%
\pgfsetfillcolor{textcolor}%
\pgftext[x=3.975000in,y=0.342778in,,top]{\color{textcolor}\rmfamily\fontsize{10.000000}{12.000000}\selectfont \(\displaystyle {0.90}\)}%
\end{pgfscope}%
\begin{pgfscope}%
\pgfpathrectangle{\pgfqpoint{0.875000in}{0.440000in}}{\pgfqpoint{5.425000in}{3.080000in}}%
\pgfusepath{clip}%
\pgfsetrectcap%
\pgfsetroundjoin%
\pgfsetlinewidth{0.803000pt}%
\definecolor{currentstroke}{rgb}{1.000000,1.000000,1.000000}%
\pgfsetstrokecolor{currentstroke}%
\pgfsetdash{}{0pt}%
\pgfpathmoveto{\pgfqpoint{4.750000in}{0.440000in}}%
\pgfpathlineto{\pgfqpoint{4.750000in}{3.520000in}}%
\pgfusepath{stroke}%
\end{pgfscope}%
\begin{pgfscope}%
\pgfsetbuttcap%
\pgfsetroundjoin%
\definecolor{currentfill}{rgb}{0.333333,0.333333,0.333333}%
\pgfsetfillcolor{currentfill}%
\pgfsetlinewidth{0.803000pt}%
\definecolor{currentstroke}{rgb}{0.333333,0.333333,0.333333}%
\pgfsetstrokecolor{currentstroke}%
\pgfsetdash{}{0pt}%
\pgfsys@defobject{currentmarker}{\pgfqpoint{0.000000in}{-0.048611in}}{\pgfqpoint{0.000000in}{0.000000in}}{%
\pgfpathmoveto{\pgfqpoint{0.000000in}{0.000000in}}%
\pgfpathlineto{\pgfqpoint{0.000000in}{-0.048611in}}%
\pgfusepath{stroke,fill}%
}%
\begin{pgfscope}%
\pgfsys@transformshift{4.750000in}{0.440000in}%
\pgfsys@useobject{currentmarker}{}%
\end{pgfscope}%
\end{pgfscope}%
\begin{pgfscope}%
\definecolor{textcolor}{rgb}{0.333333,0.333333,0.333333}%
\pgfsetstrokecolor{textcolor}%
\pgfsetfillcolor{textcolor}%
\pgftext[x=4.750000in,y=0.342778in,,top]{\color{textcolor}\rmfamily\fontsize{10.000000}{12.000000}\selectfont \(\displaystyle {0.95}\)}%
\end{pgfscope}%
\begin{pgfscope}%
\pgfpathrectangle{\pgfqpoint{0.875000in}{0.440000in}}{\pgfqpoint{5.425000in}{3.080000in}}%
\pgfusepath{clip}%
\pgfsetrectcap%
\pgfsetroundjoin%
\pgfsetlinewidth{0.803000pt}%
\definecolor{currentstroke}{rgb}{1.000000,1.000000,1.000000}%
\pgfsetstrokecolor{currentstroke}%
\pgfsetdash{}{0pt}%
\pgfpathmoveto{\pgfqpoint{5.525000in}{0.440000in}}%
\pgfpathlineto{\pgfqpoint{5.525000in}{3.520000in}}%
\pgfusepath{stroke}%
\end{pgfscope}%
\begin{pgfscope}%
\pgfsetbuttcap%
\pgfsetroundjoin%
\definecolor{currentfill}{rgb}{0.333333,0.333333,0.333333}%
\pgfsetfillcolor{currentfill}%
\pgfsetlinewidth{0.803000pt}%
\definecolor{currentstroke}{rgb}{0.333333,0.333333,0.333333}%
\pgfsetstrokecolor{currentstroke}%
\pgfsetdash{}{0pt}%
\pgfsys@defobject{currentmarker}{\pgfqpoint{0.000000in}{-0.048611in}}{\pgfqpoint{0.000000in}{0.000000in}}{%
\pgfpathmoveto{\pgfqpoint{0.000000in}{0.000000in}}%
\pgfpathlineto{\pgfqpoint{0.000000in}{-0.048611in}}%
\pgfusepath{stroke,fill}%
}%
\begin{pgfscope}%
\pgfsys@transformshift{5.525000in}{0.440000in}%
\pgfsys@useobject{currentmarker}{}%
\end{pgfscope}%
\end{pgfscope}%
\begin{pgfscope}%
\definecolor{textcolor}{rgb}{0.333333,0.333333,0.333333}%
\pgfsetstrokecolor{textcolor}%
\pgfsetfillcolor{textcolor}%
\pgftext[x=5.525000in,y=0.342778in,,top]{\color{textcolor}\rmfamily\fontsize{10.000000}{12.000000}\selectfont \(\displaystyle {1.00}\)}%
\end{pgfscope}%
\begin{pgfscope}%
\pgfpathrectangle{\pgfqpoint{0.875000in}{0.440000in}}{\pgfqpoint{5.425000in}{3.080000in}}%
\pgfusepath{clip}%
\pgfsetrectcap%
\pgfsetroundjoin%
\pgfsetlinewidth{0.803000pt}%
\definecolor{currentstroke}{rgb}{1.000000,1.000000,1.000000}%
\pgfsetstrokecolor{currentstroke}%
\pgfsetdash{}{0pt}%
\pgfpathmoveto{\pgfqpoint{6.300000in}{0.440000in}}%
\pgfpathlineto{\pgfqpoint{6.300000in}{3.520000in}}%
\pgfusepath{stroke}%
\end{pgfscope}%
\begin{pgfscope}%
\pgfsetbuttcap%
\pgfsetroundjoin%
\definecolor{currentfill}{rgb}{0.333333,0.333333,0.333333}%
\pgfsetfillcolor{currentfill}%
\pgfsetlinewidth{0.803000pt}%
\definecolor{currentstroke}{rgb}{0.333333,0.333333,0.333333}%
\pgfsetstrokecolor{currentstroke}%
\pgfsetdash{}{0pt}%
\pgfsys@defobject{currentmarker}{\pgfqpoint{0.000000in}{-0.048611in}}{\pgfqpoint{0.000000in}{0.000000in}}{%
\pgfpathmoveto{\pgfqpoint{0.000000in}{0.000000in}}%
\pgfpathlineto{\pgfqpoint{0.000000in}{-0.048611in}}%
\pgfusepath{stroke,fill}%
}%
\begin{pgfscope}%
\pgfsys@transformshift{6.300000in}{0.440000in}%
\pgfsys@useobject{currentmarker}{}%
\end{pgfscope}%
\end{pgfscope}%
\begin{pgfscope}%
\definecolor{textcolor}{rgb}{0.333333,0.333333,0.333333}%
\pgfsetstrokecolor{textcolor}%
\pgfsetfillcolor{textcolor}%
\pgftext[x=6.300000in,y=0.342778in,,top]{\color{textcolor}\rmfamily\fontsize{10.000000}{12.000000}\selectfont \(\displaystyle {1.05}\)}%
\end{pgfscope}%
\begin{pgfscope}%
\definecolor{textcolor}{rgb}{0.333333,0.333333,0.333333}%
\pgfsetstrokecolor{textcolor}%
\pgfsetfillcolor{textcolor}%
\pgftext[x=3.587500in,y=0.163766in,,top]{\color{textcolor}\rmfamily\fontsize{12.000000}{14.400000}\selectfont Option Price}%
\end{pgfscope}%
\begin{pgfscope}%
\pgfpathrectangle{\pgfqpoint{0.875000in}{0.440000in}}{\pgfqpoint{5.425000in}{3.080000in}}%
\pgfusepath{clip}%
\pgfsetrectcap%
\pgfsetroundjoin%
\pgfsetlinewidth{0.803000pt}%
\definecolor{currentstroke}{rgb}{1.000000,1.000000,1.000000}%
\pgfsetstrokecolor{currentstroke}%
\pgfsetdash{}{0pt}%
\pgfpathmoveto{\pgfqpoint{0.875000in}{0.696667in}}%
\pgfpathlineto{\pgfqpoint{6.300000in}{0.696667in}}%
\pgfusepath{stroke}%
\end{pgfscope}%
\begin{pgfscope}%
\pgfsetbuttcap%
\pgfsetroundjoin%
\definecolor{currentfill}{rgb}{0.333333,0.333333,0.333333}%
\pgfsetfillcolor{currentfill}%
\pgfsetlinewidth{0.803000pt}%
\definecolor{currentstroke}{rgb}{0.333333,0.333333,0.333333}%
\pgfsetstrokecolor{currentstroke}%
\pgfsetdash{}{0pt}%
\pgfsys@defobject{currentmarker}{\pgfqpoint{-0.048611in}{0.000000in}}{\pgfqpoint{-0.000000in}{0.000000in}}{%
\pgfpathmoveto{\pgfqpoint{-0.000000in}{0.000000in}}%
\pgfpathlineto{\pgfqpoint{-0.048611in}{0.000000in}}%
\pgfusepath{stroke,fill}%
}%
\begin{pgfscope}%
\pgfsys@transformshift{0.875000in}{0.696667in}%
\pgfsys@useobject{currentmarker}{}%
\end{pgfscope}%
\end{pgfscope}%
\begin{pgfscope}%
\definecolor{textcolor}{rgb}{0.333333,0.333333,0.333333}%
\pgfsetstrokecolor{textcolor}%
\pgfsetfillcolor{textcolor}%
\pgftext[x=0.422838in, y=0.648441in, left, base]{\color{textcolor}\rmfamily\fontsize{10.000000}{12.000000}\selectfont \(\displaystyle {\ensuremath{-}0.15}\)}%
\end{pgfscope}%
\begin{pgfscope}%
\pgfpathrectangle{\pgfqpoint{0.875000in}{0.440000in}}{\pgfqpoint{5.425000in}{3.080000in}}%
\pgfusepath{clip}%
\pgfsetrectcap%
\pgfsetroundjoin%
\pgfsetlinewidth{0.803000pt}%
\definecolor{currentstroke}{rgb}{1.000000,1.000000,1.000000}%
\pgfsetstrokecolor{currentstroke}%
\pgfsetdash{}{0pt}%
\pgfpathmoveto{\pgfqpoint{0.875000in}{1.124444in}}%
\pgfpathlineto{\pgfqpoint{6.300000in}{1.124444in}}%
\pgfusepath{stroke}%
\end{pgfscope}%
\begin{pgfscope}%
\pgfsetbuttcap%
\pgfsetroundjoin%
\definecolor{currentfill}{rgb}{0.333333,0.333333,0.333333}%
\pgfsetfillcolor{currentfill}%
\pgfsetlinewidth{0.803000pt}%
\definecolor{currentstroke}{rgb}{0.333333,0.333333,0.333333}%
\pgfsetstrokecolor{currentstroke}%
\pgfsetdash{}{0pt}%
\pgfsys@defobject{currentmarker}{\pgfqpoint{-0.048611in}{0.000000in}}{\pgfqpoint{-0.000000in}{0.000000in}}{%
\pgfpathmoveto{\pgfqpoint{-0.000000in}{0.000000in}}%
\pgfpathlineto{\pgfqpoint{-0.048611in}{0.000000in}}%
\pgfusepath{stroke,fill}%
}%
\begin{pgfscope}%
\pgfsys@transformshift{0.875000in}{1.124444in}%
\pgfsys@useobject{currentmarker}{}%
\end{pgfscope}%
\end{pgfscope}%
\begin{pgfscope}%
\definecolor{textcolor}{rgb}{0.333333,0.333333,0.333333}%
\pgfsetstrokecolor{textcolor}%
\pgfsetfillcolor{textcolor}%
\pgftext[x=0.422838in, y=1.076219in, left, base]{\color{textcolor}\rmfamily\fontsize{10.000000}{12.000000}\selectfont \(\displaystyle {\ensuremath{-}0.10}\)}%
\end{pgfscope}%
\begin{pgfscope}%
\pgfpathrectangle{\pgfqpoint{0.875000in}{0.440000in}}{\pgfqpoint{5.425000in}{3.080000in}}%
\pgfusepath{clip}%
\pgfsetrectcap%
\pgfsetroundjoin%
\pgfsetlinewidth{0.803000pt}%
\definecolor{currentstroke}{rgb}{1.000000,1.000000,1.000000}%
\pgfsetstrokecolor{currentstroke}%
\pgfsetdash{}{0pt}%
\pgfpathmoveto{\pgfqpoint{0.875000in}{1.552222in}}%
\pgfpathlineto{\pgfqpoint{6.300000in}{1.552222in}}%
\pgfusepath{stroke}%
\end{pgfscope}%
\begin{pgfscope}%
\pgfsetbuttcap%
\pgfsetroundjoin%
\definecolor{currentfill}{rgb}{0.333333,0.333333,0.333333}%
\pgfsetfillcolor{currentfill}%
\pgfsetlinewidth{0.803000pt}%
\definecolor{currentstroke}{rgb}{0.333333,0.333333,0.333333}%
\pgfsetstrokecolor{currentstroke}%
\pgfsetdash{}{0pt}%
\pgfsys@defobject{currentmarker}{\pgfqpoint{-0.048611in}{0.000000in}}{\pgfqpoint{-0.000000in}{0.000000in}}{%
\pgfpathmoveto{\pgfqpoint{-0.000000in}{0.000000in}}%
\pgfpathlineto{\pgfqpoint{-0.048611in}{0.000000in}}%
\pgfusepath{stroke,fill}%
}%
\begin{pgfscope}%
\pgfsys@transformshift{0.875000in}{1.552222in}%
\pgfsys@useobject{currentmarker}{}%
\end{pgfscope}%
\end{pgfscope}%
\begin{pgfscope}%
\definecolor{textcolor}{rgb}{0.333333,0.333333,0.333333}%
\pgfsetstrokecolor{textcolor}%
\pgfsetfillcolor{textcolor}%
\pgftext[x=0.422838in, y=1.503997in, left, base]{\color{textcolor}\rmfamily\fontsize{10.000000}{12.000000}\selectfont \(\displaystyle {\ensuremath{-}0.05}\)}%
\end{pgfscope}%
\begin{pgfscope}%
\pgfpathrectangle{\pgfqpoint{0.875000in}{0.440000in}}{\pgfqpoint{5.425000in}{3.080000in}}%
\pgfusepath{clip}%
\pgfsetrectcap%
\pgfsetroundjoin%
\pgfsetlinewidth{0.803000pt}%
\definecolor{currentstroke}{rgb}{1.000000,1.000000,1.000000}%
\pgfsetstrokecolor{currentstroke}%
\pgfsetdash{}{0pt}%
\pgfpathmoveto{\pgfqpoint{0.875000in}{1.980000in}}%
\pgfpathlineto{\pgfqpoint{6.300000in}{1.980000in}}%
\pgfusepath{stroke}%
\end{pgfscope}%
\begin{pgfscope}%
\pgfsetbuttcap%
\pgfsetroundjoin%
\definecolor{currentfill}{rgb}{0.333333,0.333333,0.333333}%
\pgfsetfillcolor{currentfill}%
\pgfsetlinewidth{0.803000pt}%
\definecolor{currentstroke}{rgb}{0.333333,0.333333,0.333333}%
\pgfsetstrokecolor{currentstroke}%
\pgfsetdash{}{0pt}%
\pgfsys@defobject{currentmarker}{\pgfqpoint{-0.048611in}{0.000000in}}{\pgfqpoint{-0.000000in}{0.000000in}}{%
\pgfpathmoveto{\pgfqpoint{-0.000000in}{0.000000in}}%
\pgfpathlineto{\pgfqpoint{-0.048611in}{0.000000in}}%
\pgfusepath{stroke,fill}%
}%
\begin{pgfscope}%
\pgfsys@transformshift{0.875000in}{1.980000in}%
\pgfsys@useobject{currentmarker}{}%
\end{pgfscope}%
\end{pgfscope}%
\begin{pgfscope}%
\definecolor{textcolor}{rgb}{0.333333,0.333333,0.333333}%
\pgfsetstrokecolor{textcolor}%
\pgfsetfillcolor{textcolor}%
\pgftext[x=0.530863in, y=1.931775in, left, base]{\color{textcolor}\rmfamily\fontsize{10.000000}{12.000000}\selectfont \(\displaystyle {0.00}\)}%
\end{pgfscope}%
\begin{pgfscope}%
\pgfpathrectangle{\pgfqpoint{0.875000in}{0.440000in}}{\pgfqpoint{5.425000in}{3.080000in}}%
\pgfusepath{clip}%
\pgfsetrectcap%
\pgfsetroundjoin%
\pgfsetlinewidth{0.803000pt}%
\definecolor{currentstroke}{rgb}{1.000000,1.000000,1.000000}%
\pgfsetstrokecolor{currentstroke}%
\pgfsetdash{}{0pt}%
\pgfpathmoveto{\pgfqpoint{0.875000in}{2.407778in}}%
\pgfpathlineto{\pgfqpoint{6.300000in}{2.407778in}}%
\pgfusepath{stroke}%
\end{pgfscope}%
\begin{pgfscope}%
\pgfsetbuttcap%
\pgfsetroundjoin%
\definecolor{currentfill}{rgb}{0.333333,0.333333,0.333333}%
\pgfsetfillcolor{currentfill}%
\pgfsetlinewidth{0.803000pt}%
\definecolor{currentstroke}{rgb}{0.333333,0.333333,0.333333}%
\pgfsetstrokecolor{currentstroke}%
\pgfsetdash{}{0pt}%
\pgfsys@defobject{currentmarker}{\pgfqpoint{-0.048611in}{0.000000in}}{\pgfqpoint{-0.000000in}{0.000000in}}{%
\pgfpathmoveto{\pgfqpoint{-0.000000in}{0.000000in}}%
\pgfpathlineto{\pgfqpoint{-0.048611in}{0.000000in}}%
\pgfusepath{stroke,fill}%
}%
\begin{pgfscope}%
\pgfsys@transformshift{0.875000in}{2.407778in}%
\pgfsys@useobject{currentmarker}{}%
\end{pgfscope}%
\end{pgfscope}%
\begin{pgfscope}%
\definecolor{textcolor}{rgb}{0.333333,0.333333,0.333333}%
\pgfsetstrokecolor{textcolor}%
\pgfsetfillcolor{textcolor}%
\pgftext[x=0.530863in, y=2.359552in, left, base]{\color{textcolor}\rmfamily\fontsize{10.000000}{12.000000}\selectfont \(\displaystyle {0.05}\)}%
\end{pgfscope}%
\begin{pgfscope}%
\pgfpathrectangle{\pgfqpoint{0.875000in}{0.440000in}}{\pgfqpoint{5.425000in}{3.080000in}}%
\pgfusepath{clip}%
\pgfsetrectcap%
\pgfsetroundjoin%
\pgfsetlinewidth{0.803000pt}%
\definecolor{currentstroke}{rgb}{1.000000,1.000000,1.000000}%
\pgfsetstrokecolor{currentstroke}%
\pgfsetdash{}{0pt}%
\pgfpathmoveto{\pgfqpoint{0.875000in}{2.835556in}}%
\pgfpathlineto{\pgfqpoint{6.300000in}{2.835556in}}%
\pgfusepath{stroke}%
\end{pgfscope}%
\begin{pgfscope}%
\pgfsetbuttcap%
\pgfsetroundjoin%
\definecolor{currentfill}{rgb}{0.333333,0.333333,0.333333}%
\pgfsetfillcolor{currentfill}%
\pgfsetlinewidth{0.803000pt}%
\definecolor{currentstroke}{rgb}{0.333333,0.333333,0.333333}%
\pgfsetstrokecolor{currentstroke}%
\pgfsetdash{}{0pt}%
\pgfsys@defobject{currentmarker}{\pgfqpoint{-0.048611in}{0.000000in}}{\pgfqpoint{-0.000000in}{0.000000in}}{%
\pgfpathmoveto{\pgfqpoint{-0.000000in}{0.000000in}}%
\pgfpathlineto{\pgfqpoint{-0.048611in}{0.000000in}}%
\pgfusepath{stroke,fill}%
}%
\begin{pgfscope}%
\pgfsys@transformshift{0.875000in}{2.835556in}%
\pgfsys@useobject{currentmarker}{}%
\end{pgfscope}%
\end{pgfscope}%
\begin{pgfscope}%
\definecolor{textcolor}{rgb}{0.333333,0.333333,0.333333}%
\pgfsetstrokecolor{textcolor}%
\pgfsetfillcolor{textcolor}%
\pgftext[x=0.530863in, y=2.787330in, left, base]{\color{textcolor}\rmfamily\fontsize{10.000000}{12.000000}\selectfont \(\displaystyle {0.10}\)}%
\end{pgfscope}%
\begin{pgfscope}%
\pgfpathrectangle{\pgfqpoint{0.875000in}{0.440000in}}{\pgfqpoint{5.425000in}{3.080000in}}%
\pgfusepath{clip}%
\pgfsetrectcap%
\pgfsetroundjoin%
\pgfsetlinewidth{0.803000pt}%
\definecolor{currentstroke}{rgb}{1.000000,1.000000,1.000000}%
\pgfsetstrokecolor{currentstroke}%
\pgfsetdash{}{0pt}%
\pgfpathmoveto{\pgfqpoint{0.875000in}{3.263333in}}%
\pgfpathlineto{\pgfqpoint{6.300000in}{3.263333in}}%
\pgfusepath{stroke}%
\end{pgfscope}%
\begin{pgfscope}%
\pgfsetbuttcap%
\pgfsetroundjoin%
\definecolor{currentfill}{rgb}{0.333333,0.333333,0.333333}%
\pgfsetfillcolor{currentfill}%
\pgfsetlinewidth{0.803000pt}%
\definecolor{currentstroke}{rgb}{0.333333,0.333333,0.333333}%
\pgfsetstrokecolor{currentstroke}%
\pgfsetdash{}{0pt}%
\pgfsys@defobject{currentmarker}{\pgfqpoint{-0.048611in}{0.000000in}}{\pgfqpoint{-0.000000in}{0.000000in}}{%
\pgfpathmoveto{\pgfqpoint{-0.000000in}{0.000000in}}%
\pgfpathlineto{\pgfqpoint{-0.048611in}{0.000000in}}%
\pgfusepath{stroke,fill}%
}%
\begin{pgfscope}%
\pgfsys@transformshift{0.875000in}{3.263333in}%
\pgfsys@useobject{currentmarker}{}%
\end{pgfscope}%
\end{pgfscope}%
\begin{pgfscope}%
\definecolor{textcolor}{rgb}{0.333333,0.333333,0.333333}%
\pgfsetstrokecolor{textcolor}%
\pgfsetfillcolor{textcolor}%
\pgftext[x=0.530863in, y=3.215108in, left, base]{\color{textcolor}\rmfamily\fontsize{10.000000}{12.000000}\selectfont \(\displaystyle {0.15}\)}%
\end{pgfscope}%
\begin{pgfscope}%
\definecolor{textcolor}{rgb}{0.333333,0.333333,0.333333}%
\pgfsetstrokecolor{textcolor}%
\pgfsetfillcolor{textcolor}%
\pgftext[x=0.367283in,y=1.980000in,,bottom,rotate=90.000000]{\color{textcolor}\rmfamily\fontsize{12.000000}{14.400000}\selectfont Marginal Value}%
\end{pgfscope}%
\begin{pgfscope}%
\pgfpathrectangle{\pgfqpoint{0.875000in}{0.440000in}}{\pgfqpoint{5.425000in}{3.080000in}}%
\pgfusepath{clip}%
\pgfsetrectcap%
\pgfsetroundjoin%
\pgfsetlinewidth{1.505625pt}%
\definecolor{currentstroke}{rgb}{0.886275,0.290196,0.200000}%
\pgfsetstrokecolor{currentstroke}%
\pgfsetdash{}{0pt}%
\pgfpathmoveto{\pgfqpoint{0.875000in}{3.434444in}}%
\pgfpathlineto{\pgfqpoint{1.030000in}{3.348889in}}%
\pgfpathlineto{\pgfqpoint{1.185000in}{3.263333in}}%
\pgfpathlineto{\pgfqpoint{1.340000in}{3.177778in}}%
\pgfpathlineto{\pgfqpoint{1.495000in}{3.092222in}}%
\pgfpathlineto{\pgfqpoint{1.650000in}{3.006667in}}%
\pgfpathlineto{\pgfqpoint{1.805000in}{2.921111in}}%
\pgfpathlineto{\pgfqpoint{1.960000in}{2.835556in}}%
\pgfpathlineto{\pgfqpoint{2.115000in}{2.750000in}}%
\pgfpathlineto{\pgfqpoint{2.270000in}{2.664444in}}%
\pgfpathlineto{\pgfqpoint{2.425000in}{2.578889in}}%
\pgfpathlineto{\pgfqpoint{2.580000in}{2.493333in}}%
\pgfpathlineto{\pgfqpoint{2.735000in}{2.407778in}}%
\pgfpathlineto{\pgfqpoint{2.890000in}{2.322222in}}%
\pgfpathlineto{\pgfqpoint{3.045000in}{2.236667in}}%
\pgfpathlineto{\pgfqpoint{3.200000in}{2.151111in}}%
\pgfpathlineto{\pgfqpoint{3.355000in}{2.065556in}}%
\pgfpathlineto{\pgfqpoint{3.510000in}{1.980000in}}%
\pgfpathlineto{\pgfqpoint{3.665000in}{1.894444in}}%
\pgfpathlineto{\pgfqpoint{3.820000in}{1.808889in}}%
\pgfpathlineto{\pgfqpoint{3.975000in}{1.723333in}}%
\pgfpathlineto{\pgfqpoint{4.130000in}{1.637778in}}%
\pgfpathlineto{\pgfqpoint{4.285000in}{1.552222in}}%
\pgfpathlineto{\pgfqpoint{4.440000in}{1.466667in}}%
\pgfpathlineto{\pgfqpoint{4.595000in}{1.381111in}}%
\pgfpathlineto{\pgfqpoint{4.750000in}{1.295556in}}%
\pgfpathlineto{\pgfqpoint{4.905000in}{1.210000in}}%
\pgfpathlineto{\pgfqpoint{5.060000in}{1.124444in}}%
\pgfpathlineto{\pgfqpoint{5.215000in}{1.038889in}}%
\pgfpathlineto{\pgfqpoint{5.370000in}{0.953333in}}%
\pgfpathlineto{\pgfqpoint{5.525000in}{0.867778in}}%
\pgfpathlineto{\pgfqpoint{5.680000in}{0.782222in}}%
\pgfpathlineto{\pgfqpoint{5.835000in}{0.696667in}}%
\pgfpathlineto{\pgfqpoint{5.990000in}{0.611111in}}%
\pgfpathlineto{\pgfqpoint{6.145000in}{0.525556in}}%
\pgfpathlineto{\pgfqpoint{6.300000in}{0.440000in}}%
\pgfusepath{stroke}%
\end{pgfscope}%
\begin{pgfscope}%
\pgfpathrectangle{\pgfqpoint{0.875000in}{0.440000in}}{\pgfqpoint{5.425000in}{3.080000in}}%
\pgfusepath{clip}%
\pgfsetrectcap%
\pgfsetroundjoin%
\pgfsetlinewidth{1.505625pt}%
\definecolor{currentstroke}{rgb}{0.203922,0.541176,0.741176}%
\pgfsetstrokecolor{currentstroke}%
\pgfsetdash{}{0pt}%
\pgfpathmoveto{\pgfqpoint{0.875000in}{3.006667in}}%
\pgfpathlineto{\pgfqpoint{1.030000in}{3.006667in}}%
\pgfpathlineto{\pgfqpoint{1.185000in}{3.006667in}}%
\pgfpathlineto{\pgfqpoint{1.340000in}{3.006667in}}%
\pgfpathlineto{\pgfqpoint{1.495000in}{3.006667in}}%
\pgfpathlineto{\pgfqpoint{1.650000in}{3.006667in}}%
\pgfpathlineto{\pgfqpoint{1.805000in}{3.006667in}}%
\pgfpathlineto{\pgfqpoint{1.960000in}{3.006667in}}%
\pgfpathlineto{\pgfqpoint{2.115000in}{3.006667in}}%
\pgfpathlineto{\pgfqpoint{2.270000in}{3.006667in}}%
\pgfpathlineto{\pgfqpoint{2.425000in}{3.006667in}}%
\pgfpathlineto{\pgfqpoint{2.580000in}{3.006667in}}%
\pgfpathlineto{\pgfqpoint{2.735000in}{3.006667in}}%
\pgfpathlineto{\pgfqpoint{2.890000in}{3.006667in}}%
\pgfpathlineto{\pgfqpoint{3.045000in}{3.006667in}}%
\pgfpathlineto{\pgfqpoint{3.200000in}{3.006667in}}%
\pgfpathlineto{\pgfqpoint{3.355000in}{3.006667in}}%
\pgfpathlineto{\pgfqpoint{3.510000in}{3.006667in}}%
\pgfpathlineto{\pgfqpoint{3.665000in}{3.006667in}}%
\pgfpathlineto{\pgfqpoint{3.820000in}{3.006667in}}%
\pgfpathlineto{\pgfqpoint{3.975000in}{3.006667in}}%
\pgfpathlineto{\pgfqpoint{4.130000in}{3.006667in}}%
\pgfpathlineto{\pgfqpoint{4.285000in}{3.006667in}}%
\pgfpathlineto{\pgfqpoint{4.440000in}{3.006667in}}%
\pgfpathlineto{\pgfqpoint{4.595000in}{3.006667in}}%
\pgfpathlineto{\pgfqpoint{4.750000in}{3.006667in}}%
\pgfpathlineto{\pgfqpoint{4.905000in}{3.006667in}}%
\pgfpathlineto{\pgfqpoint{5.060000in}{3.006667in}}%
\pgfpathlineto{\pgfqpoint{5.215000in}{3.006667in}}%
\pgfpathlineto{\pgfqpoint{5.370000in}{3.006667in}}%
\pgfpathlineto{\pgfqpoint{5.525000in}{3.006667in}}%
\pgfpathlineto{\pgfqpoint{5.680000in}{3.006667in}}%
\pgfpathlineto{\pgfqpoint{5.835000in}{3.006667in}}%
\pgfpathlineto{\pgfqpoint{5.990000in}{3.006667in}}%
\pgfpathlineto{\pgfqpoint{6.145000in}{3.006667in}}%
\pgfpathlineto{\pgfqpoint{6.300000in}{3.006667in}}%
\pgfusepath{stroke}%
\end{pgfscope}%
\begin{pgfscope}%
\pgfpathrectangle{\pgfqpoint{0.875000in}{0.440000in}}{\pgfqpoint{5.425000in}{3.080000in}}%
\pgfusepath{clip}%
\pgfsetbuttcap%
\pgfsetroundjoin%
\pgfsetlinewidth{0.501875pt}%
\definecolor{currentstroke}{rgb}{0.501961,0.501961,0.501961}%
\pgfsetstrokecolor{currentstroke}%
\pgfsetdash{{3.200000pt}{0.800000pt}{0.500000pt}{0.800000pt}}{0.000000pt}%
\pgfpathmoveto{\pgfqpoint{0.875000in}{1.980000in}}%
\pgfpathlineto{\pgfqpoint{6.300000in}{1.980000in}}%
\pgfusepath{stroke}%
\end{pgfscope}%
\begin{pgfscope}%
\pgfpathrectangle{\pgfqpoint{0.875000in}{0.440000in}}{\pgfqpoint{5.425000in}{3.080000in}}%
\pgfusepath{clip}%
\pgfsetbuttcap%
\pgfsetroundjoin%
\pgfsetlinewidth{0.501875pt}%
\definecolor{currentstroke}{rgb}{0.000000,0.000000,0.000000}%
\pgfsetstrokecolor{currentstroke}%
\pgfsetdash{{1.850000pt}{0.800000pt}}{0.000000pt}%
\pgfpathmoveto{\pgfqpoint{5.370000in}{0.594000in}}%
\pgfpathlineto{\pgfqpoint{5.370000in}{3.366000in}}%
\pgfusepath{stroke}%
\end{pgfscope}%
\begin{pgfscope}%
\pgfpathrectangle{\pgfqpoint{0.875000in}{0.440000in}}{\pgfqpoint{5.425000in}{3.080000in}}%
\pgfusepath{clip}%
\pgfsetbuttcap%
\pgfsetroundjoin%
\pgfsetlinewidth{0.501875pt}%
\definecolor{currentstroke}{rgb}{0.000000,0.000000,0.000000}%
\pgfsetstrokecolor{currentstroke}%
\pgfsetdash{{1.850000pt}{0.800000pt}}{0.000000pt}%
\pgfpathmoveto{\pgfqpoint{3.510000in}{0.594000in}}%
\pgfpathlineto{\pgfqpoint{3.510000in}{3.366000in}}%
\pgfusepath{stroke}%
\end{pgfscope}%
\begin{pgfscope}%
\pgfsetrectcap%
\pgfsetmiterjoin%
\pgfsetlinewidth{1.003750pt}%
\definecolor{currentstroke}{rgb}{1.000000,1.000000,1.000000}%
\pgfsetstrokecolor{currentstroke}%
\pgfsetdash{}{0pt}%
\pgfpathmoveto{\pgfqpoint{0.875000in}{0.440000in}}%
\pgfpathlineto{\pgfqpoint{0.875000in}{3.520000in}}%
\pgfusepath{stroke}%
\end{pgfscope}%
\begin{pgfscope}%
\pgfsetrectcap%
\pgfsetmiterjoin%
\pgfsetlinewidth{1.003750pt}%
\definecolor{currentstroke}{rgb}{1.000000,1.000000,1.000000}%
\pgfsetstrokecolor{currentstroke}%
\pgfsetdash{}{0pt}%
\pgfpathmoveto{\pgfqpoint{6.300000in}{0.440000in}}%
\pgfpathlineto{\pgfqpoint{6.300000in}{3.520000in}}%
\pgfusepath{stroke}%
\end{pgfscope}%
\begin{pgfscope}%
\pgfsetrectcap%
\pgfsetmiterjoin%
\pgfsetlinewidth{1.003750pt}%
\definecolor{currentstroke}{rgb}{1.000000,1.000000,1.000000}%
\pgfsetstrokecolor{currentstroke}%
\pgfsetdash{}{0pt}%
\pgfpathmoveto{\pgfqpoint{0.875000in}{0.440000in}}%
\pgfpathlineto{\pgfqpoint{6.300000in}{0.440000in}}%
\pgfusepath{stroke}%
\end{pgfscope}%
\begin{pgfscope}%
\pgfsetrectcap%
\pgfsetmiterjoin%
\pgfsetlinewidth{1.003750pt}%
\definecolor{currentstroke}{rgb}{1.000000,1.000000,1.000000}%
\pgfsetstrokecolor{currentstroke}%
\pgfsetdash{}{0pt}%
\pgfpathmoveto{\pgfqpoint{0.875000in}{3.520000in}}%
\pgfpathlineto{\pgfqpoint{6.300000in}{3.520000in}}%
\pgfusepath{stroke}%
\end{pgfscope}%
\begin{pgfscope}%
\definecolor{textcolor}{rgb}{0.000000,0.000000,0.000000}%
\pgfsetstrokecolor{textcolor}%
\pgfsetfillcolor{textcolor}%
\pgftext[x=5.425556in,y=1.008889in,left,base]{\color{textcolor}\rmfamily\fontsize{8.000000}{9.600000}\selectfont \(\displaystyle -0.1200\)}%
\end{pgfscope}%
\begin{pgfscope}%
\definecolor{textcolor}{rgb}{0.000000,0.000000,0.000000}%
\pgfsetstrokecolor{textcolor}%
\pgfsetfillcolor{textcolor}%
\pgftext[x=5.425556in,y=3.062222in,left,base]{\color{textcolor}\rmfamily\fontsize{8.000000}{9.600000}\selectfont \(\displaystyle 0.1200\)}%
\end{pgfscope}%
\begin{pgfscope}%
\definecolor{textcolor}{rgb}{0.000000,0.000000,0.000000}%
\pgfsetstrokecolor{textcolor}%
\pgfsetfillcolor{textcolor}%
\pgftext[x=3.565556in,y=2.035556in,left,base]{\color{textcolor}\rmfamily\fontsize{8.000000}{9.600000}\selectfont \(\displaystyle 0.0000\)}%
\end{pgfscope}%
\begin{pgfscope}%
\definecolor{textcolor}{rgb}{0.000000,0.000000,0.000000}%
\pgfsetstrokecolor{textcolor}%
\pgfsetfillcolor{textcolor}%
\pgftext[x=3.565556in,y=3.062222in,left,base]{\color{textcolor}\rmfamily\fontsize{8.000000}{9.600000}\selectfont \(\displaystyle 0.1200\)}%
\end{pgfscope}%
\begin{pgfscope}%
\definecolor{textcolor}{rgb}{0.000000,0.000000,0.000000}%
\pgfsetstrokecolor{textcolor}%
\pgfsetfillcolor{textcolor}%
\pgftext[x=3.587500in,y=3.603333in,,base]{\color{textcolor}\rmfamily\fontsize{14.400000}{17.280000}\selectfont Marginal shareholder and creditor value for different option prices}%
\end{pgfscope}%
\begin{pgfscope}%
\pgfsetbuttcap%
\pgfsetmiterjoin%
\definecolor{currentfill}{rgb}{0.898039,0.898039,0.898039}%
\pgfsetfillcolor{currentfill}%
\pgfsetfillopacity{0.800000}%
\pgfsetlinewidth{0.501875pt}%
\definecolor{currentstroke}{rgb}{0.800000,0.800000,0.800000}%
\pgfsetstrokecolor{currentstroke}%
\pgfsetstrokeopacity{0.800000}%
\pgfsetdash{}{0pt}%
\pgfpathmoveto{\pgfqpoint{0.972222in}{0.509444in}}%
\pgfpathlineto{\pgfqpoint{2.189816in}{0.509444in}}%
\pgfpathquadraticcurveto{\pgfqpoint{2.217594in}{0.509444in}}{\pgfqpoint{2.217594in}{0.537222in}}%
\pgfpathlineto{\pgfqpoint{2.217594in}{0.910679in}}%
\pgfpathquadraticcurveto{\pgfqpoint{2.217594in}{0.938457in}}{\pgfqpoint{2.189816in}{0.938457in}}%
\pgfpathlineto{\pgfqpoint{0.972222in}{0.938457in}}%
\pgfpathquadraticcurveto{\pgfqpoint{0.944444in}{0.938457in}}{\pgfqpoint{0.944444in}{0.910679in}}%
\pgfpathlineto{\pgfqpoint{0.944444in}{0.537222in}}%
\pgfpathquadraticcurveto{\pgfqpoint{0.944444in}{0.509444in}}{\pgfqpoint{0.972222in}{0.509444in}}%
\pgfpathlineto{\pgfqpoint{0.972222in}{0.509444in}}%
\pgfpathclose%
\pgfusepath{stroke,fill}%
\end{pgfscope}%
\begin{pgfscope}%
\pgfsetrectcap%
\pgfsetroundjoin%
\pgfsetlinewidth{1.505625pt}%
\definecolor{currentstroke}{rgb}{0.886275,0.290196,0.200000}%
\pgfsetstrokecolor{currentstroke}%
\pgfsetdash{}{0pt}%
\pgfpathmoveto{\pgfqpoint{1.000000in}{0.834290in}}%
\pgfpathlineto{\pgfqpoint{1.138889in}{0.834290in}}%
\pgfpathlineto{\pgfqpoint{1.277778in}{0.834290in}}%
\pgfusepath{stroke}%
\end{pgfscope}%
\begin{pgfscope}%
\definecolor{textcolor}{rgb}{0.000000,0.000000,0.000000}%
\pgfsetstrokecolor{textcolor}%
\pgfsetfillcolor{textcolor}%
\pgftext[x=1.388889in,y=0.785679in,left,base]{\color{textcolor}\rmfamily\fontsize{10.000000}{12.000000}\selectfont Shareholders}%
\end{pgfscope}%
\begin{pgfscope}%
\pgfsetrectcap%
\pgfsetroundjoin%
\pgfsetlinewidth{1.505625pt}%
\definecolor{currentstroke}{rgb}{0.203922,0.541176,0.741176}%
\pgfsetstrokecolor{currentstroke}%
\pgfsetdash{}{0pt}%
\pgfpathmoveto{\pgfqpoint{1.000000in}{0.640617in}}%
\pgfpathlineto{\pgfqpoint{1.138889in}{0.640617in}}%
\pgfpathlineto{\pgfqpoint{1.277778in}{0.640617in}}%
\pgfusepath{stroke}%
\end{pgfscope}%
\begin{pgfscope}%
\definecolor{textcolor}{rgb}{0.000000,0.000000,0.000000}%
\pgfsetstrokecolor{textcolor}%
\pgfsetfillcolor{textcolor}%
\pgftext[x=1.388889in,y=0.592006in,left,base]{\color{textcolor}\rmfamily\fontsize{10.000000}{12.000000}\selectfont Creditors}%
\end{pgfscope}%
\begin{pgfscope}%
\pgfpathrectangle{\pgfqpoint{0.875000in}{0.440000in}}{\pgfqpoint{5.425000in}{3.080000in}}%
\pgfusepath{clip}%
\pgfsetbuttcap%
\pgfsetroundjoin%
\definecolor{currentfill}{rgb}{0.000000,0.000000,0.000000}%
\pgfsetfillcolor{currentfill}%
\pgfsetlinewidth{0.501875pt}%
\definecolor{currentstroke}{rgb}{0.000000,0.000000,0.000000}%
\pgfsetstrokecolor{currentstroke}%
\pgfsetdash{}{0pt}%
\pgfsys@defobject{currentmarker}{\pgfqpoint{-0.026896in}{-0.026896in}}{\pgfqpoint{0.026896in}{0.026896in}}{%
\pgfpathmoveto{\pgfqpoint{0.000000in}{-0.026896in}}%
\pgfpathcurveto{\pgfqpoint{0.007133in}{-0.026896in}}{\pgfqpoint{0.013974in}{-0.024062in}}{\pgfqpoint{0.019018in}{-0.019018in}}%
\pgfpathcurveto{\pgfqpoint{0.024062in}{-0.013974in}}{\pgfqpoint{0.026896in}{-0.007133in}}{\pgfqpoint{0.026896in}{0.000000in}}%
\pgfpathcurveto{\pgfqpoint{0.026896in}{0.007133in}}{\pgfqpoint{0.024062in}{0.013974in}}{\pgfqpoint{0.019018in}{0.019018in}}%
\pgfpathcurveto{\pgfqpoint{0.013974in}{0.024062in}}{\pgfqpoint{0.007133in}{0.026896in}}{\pgfqpoint{0.000000in}{0.026896in}}%
\pgfpathcurveto{\pgfqpoint{-0.007133in}{0.026896in}}{\pgfqpoint{-0.013974in}{0.024062in}}{\pgfqpoint{-0.019018in}{0.019018in}}%
\pgfpathcurveto{\pgfqpoint{-0.024062in}{0.013974in}}{\pgfqpoint{-0.026896in}{0.007133in}}{\pgfqpoint{-0.026896in}{0.000000in}}%
\pgfpathcurveto{\pgfqpoint{-0.026896in}{-0.007133in}}{\pgfqpoint{-0.024062in}{-0.013974in}}{\pgfqpoint{-0.019018in}{-0.019018in}}%
\pgfpathcurveto{\pgfqpoint{-0.013974in}{-0.024062in}}{\pgfqpoint{-0.007133in}{-0.026896in}}{\pgfqpoint{0.000000in}{-0.026896in}}%
\pgfpathlineto{\pgfqpoint{0.000000in}{-0.026896in}}%
\pgfpathclose%
\pgfusepath{stroke,fill}%
}%
\begin{pgfscope}%
\pgfsys@transformshift{5.370000in}{0.953333in}%
\pgfsys@useobject{currentmarker}{}%
\end{pgfscope}%
\begin{pgfscope}%
\pgfsys@transformshift{5.370000in}{3.006667in}%
\pgfsys@useobject{currentmarker}{}%
\end{pgfscope}%
\end{pgfscope}%
\begin{pgfscope}%
\pgfpathrectangle{\pgfqpoint{0.875000in}{0.440000in}}{\pgfqpoint{5.425000in}{3.080000in}}%
\pgfusepath{clip}%
\pgfsetbuttcap%
\pgfsetroundjoin%
\definecolor{currentfill}{rgb}{0.000000,0.000000,0.000000}%
\pgfsetfillcolor{currentfill}%
\pgfsetlinewidth{0.501875pt}%
\definecolor{currentstroke}{rgb}{0.000000,0.000000,0.000000}%
\pgfsetstrokecolor{currentstroke}%
\pgfsetdash{}{0pt}%
\pgfsys@defobject{currentmarker}{\pgfqpoint{-0.026896in}{-0.026896in}}{\pgfqpoint{0.026896in}{0.026896in}}{%
\pgfpathmoveto{\pgfqpoint{0.000000in}{-0.026896in}}%
\pgfpathcurveto{\pgfqpoint{0.007133in}{-0.026896in}}{\pgfqpoint{0.013974in}{-0.024062in}}{\pgfqpoint{0.019018in}{-0.019018in}}%
\pgfpathcurveto{\pgfqpoint{0.024062in}{-0.013974in}}{\pgfqpoint{0.026896in}{-0.007133in}}{\pgfqpoint{0.026896in}{0.000000in}}%
\pgfpathcurveto{\pgfqpoint{0.026896in}{0.007133in}}{\pgfqpoint{0.024062in}{0.013974in}}{\pgfqpoint{0.019018in}{0.019018in}}%
\pgfpathcurveto{\pgfqpoint{0.013974in}{0.024062in}}{\pgfqpoint{0.007133in}{0.026896in}}{\pgfqpoint{0.000000in}{0.026896in}}%
\pgfpathcurveto{\pgfqpoint{-0.007133in}{0.026896in}}{\pgfqpoint{-0.013974in}{0.024062in}}{\pgfqpoint{-0.019018in}{0.019018in}}%
\pgfpathcurveto{\pgfqpoint{-0.024062in}{0.013974in}}{\pgfqpoint{-0.026896in}{0.007133in}}{\pgfqpoint{-0.026896in}{0.000000in}}%
\pgfpathcurveto{\pgfqpoint{-0.026896in}{-0.007133in}}{\pgfqpoint{-0.024062in}{-0.013974in}}{\pgfqpoint{-0.019018in}{-0.019018in}}%
\pgfpathcurveto{\pgfqpoint{-0.013974in}{-0.024062in}}{\pgfqpoint{-0.007133in}{-0.026896in}}{\pgfqpoint{0.000000in}{-0.026896in}}%
\pgfpathlineto{\pgfqpoint{0.000000in}{-0.026896in}}%
\pgfpathclose%
\pgfusepath{stroke,fill}%
}%
\begin{pgfscope}%
\pgfsys@transformshift{3.510000in}{1.980000in}%
\pgfsys@useobject{currentmarker}{}%
\end{pgfscope}%
\begin{pgfscope}%
\pgfsys@transformshift{3.510000in}{3.006667in}%
\pgfsys@useobject{currentmarker}{}%
\end{pgfscope}%
\end{pgfscope}%
\end{pgfpicture}%
\makeatother%
\endgroup%

            }
            \caption{
                Marginal valuation of equity financing 
                assuming that new shareholders break even.
            }
            \label{fig:marginal-value-equity-financing}
        \end{figure}

        This is confirmed by \cref{fig:marginal-value-equity-financing} showing
        the marginal value for creditors and shareholders for different upfront prices.
        The creditors' marginal value is constant in the donation from the counterparty,
        but they benefit from a fixed wealth increase since the project increases the asset value.
        Therefore, under equity financing, 
        the legacy shareholders receive the entire counterparty donation.
        As the donation from the counterparty increases,
        the new shareholders will finance a still cheaper project, 
        and, since it is still assumed to be a zero net present value investment for them, 
        their share of the firm's remaining capital is reduced.
        On the contrary, the legacy shareholders' share increases,
        and so does the value of their claim.

        Compared to debt funding, 
        issuing new equity is significantly worse for the legacy shareholders.
        The new shareholders have a much higher required rate of return than the new creditors,
        and as the legacy shareholders bear the funding cost, it is more expensive for them.
        A firm maximising its shareholders' wealth should therefore rather fund with debt issuance.

        However, donations from the counterparty are more effective
        when the project is funded by equity issuance.
        When funded by debt, a share of the donation from the counterparty goes to the creditors;
        therefore, that share are not earned by the shareholders.
        As already covered, this is evident from \cref{fig:marginal-value-debt-financing}
        by the slope on the line showing the creditors' marginal value.
        Under equity financing the entire donation goes to the legacy shareholders;
        hence, the steepness of the line showing the shareholders' valuation in 
        \cref{fig:marginal-value-equity-financing} is greater than the corresponding line in 
        \cref{fig:marginal-value-debt-financing}.

\end{document}
% !TEX root = sub-main.tex
\documentclass[main.tex]{subfiles}

\begin{document}
    \subsubsection{Funding by share issuance}
    \label{sec:example-equity-issuance}
        Assume that the premium paid by the firm to the counterparty is funded with new equity instead of with debt from a new creditor. 
        In this section the impact on the firm's capital structure of issuing new shares is discussed.
        This includes a study of the effects on the shareholders' and creditors' wealth. 
        Funding by share issuance will create a dilution, and the value of the already existing shares, owned by the legacy shareholders, will be reduced. 
        The counterparty of the new project which the firm enters is still credit risk-free, 
        and the payoff at time 1, $Y$, is known with certainty. 
        To compare debt and equity funding, 
        the project under consideration will be the same as in the previous section.

        The random payoff of the new shareholders will be denoted by $\tilde{E}$.
        The new shareholders will subscribe to the newly issued shares 
        only to the extent that the investment has a net present value of zero.
        To be able to finance the project, the value of the new equity,
        should therefore equal the upfront price of the project:
        \begin{equation}\label{eqn:derivative-zero-npv}
            u = \pi(\tilde{E})
        \end{equation}
        The new equity will increase the total asset value of the firm,
        reducing the loss rate of the creditors in the default state.
        Therefore, the creditors should see an increase in the value of their claim.\\
        The increase in wealth for the creditors is, however, at the expense of the legacy shareholders,
        since, as under debt financing, the shareholders bear the funding costs.
        As shareholders have an even higher required rate of return compared to creditors,
        the wealth transfer from the shareholders in this section should also be higher.

        While the payoff of the legacy creditors, with a face value of $\num{80}$, 
        only change in the default state, 
        the payoff of the shareholders is computed slightly differently than in the case of debt issuance. 
        It's now essential to know the share of the equity, $\alpha$, 
        which the new shareholders are entitled to.
        The new shareholders are entitled to a share equal to the size of their investment,
        relative to the discounted expected value of the total payoff to shareholders:
        \begin{equation}
            \alpha = 
                \frac{
                    u
                }{
                    \discountfactor
                    \mathbb{E}^{\rnmeasure}\left[
                        A - D
                    \right]
                }
        \end{equation}
        The expected value of the equity is 
        $\mathbb{E}^{\rnmeasure}\left[A - D\right] = \num{20.47}$,
        and with a project price of $u = 0.99$, the new shareholders are entitled to a share
        $\alpha = \pct{0.048363}$ of the equity.
        The new shareholders then receive their share of the equity in each state, 
        and the payoff to the legacy shareholders is the residual share of the equity.
        The random variables denoting the shareholders' payoffs are therefore given by:
            \begin{align}
                E 
                &= 
                (1 - \alpha)
                (A - D) 
                \\
                \tilde{E}
                &=
                \alpha
                (A - D)
            \end{align}
        The resulting payoffs of the example are shown in \cref{tbl:equity-funding-payoff}.
        \begin{table}[H]
            \centering\begin{tabular}{l|rrrrr||r}
                $i$ & 1 & 2 & 3 & 4 & 5 & Present value \\
                \hline
                $A_{i}$ & $\num{121}$ & $\num{111}$ & $\num{101}$ & $\num{96}$ & $\num{61}$ & $\num{97.39}$ \\
                $D_{i}$ & $\num{80}$ & $\num{80}$ & $\num{80}$ & $\num{80}$ & $\num{61}$ & $\num{76.92}$ \\
                $E_{i}$ & $\num{39.017098}$ & $\num{29.500733}$ & $\num{19.984367}$ & $\num{15.226185}$ & $\num{0}$ & $\num{19.48}$ \\
                $\tilde{E}_{i}$ & $\num{1.982902}$ & $\num{1.499267}$ & $\num{1.015633}$ & $\num{0.773815}$ & $\num{0}$ & $\num{0.989994}$ \\
            \end{tabular}
            \caption{}
            \label{tbl:equity-funding-payoff}
        \end{table}

        After the transaction, the firm's balance sheet has increased with an amount equal to the premium of the derivative. 
        The firm has a new asset on the asset side, the derivative receivable worth $\num{0.989994}$, and that is funded by $\num{0.989994}$ worth of new equity.

        The project is a negative net present value investment for the legacy shareholders. 
        The decrease in the equity value is calculated using
        \cref{eqn:marginal-shareholder-value-equity-financing}:
        \begin{equation}
                \Delta \pi(E) 
            =  
                G_{\text{equity}}
            =
                - (\num{1} - \pct{0.878787}) * \num{0.989994}
            = 
                -\num{0.120001}
        \end{equation}
        This amount is also what is referred to as the transfer of wealth, $\Delta W$, when no excess yield or donation is received on the derivative. 
        And as mentioned above, the marginal value of the project to the shareholders is even lower when the project is financed by share issuance,
        compared to the same quantity when the firm uses debt funding.
        This is justified by the pecking order of financing choices established by
        \cref{eqn:pecking-order-of-financing-preferences}, 
        which holds since the default probability of the firm is non-zero.

        The project is a positive net present value investment for the creditors. 
        Their increase in wealth is computed as:
        \begin{equation}\label{eqn:wealth-transfer-equality}
            \Delta \pi(D) = -\Delta \pi(E) = \num{76.92} - \num{76.80} = \num{0.120001}
        \end{equation}
        The value of the debt increases, since the creditors get the entire derivative's receivables in the default state. Following this, the loss rate as well as the credit spread decreases significantly compared to the pre-project metrics. The new loss rate is $\phi = (\num{80}-\num{61})/\num{80} = \pct{0.2375}$, and the new credit spread is $S = \num{80}/\num{76.92} - \grossrfrate = \pct{0.029941}$.
        \\
        A possible donation by the counterparty would also have to be significantly higher than in the case of debt funding for the legacy shareholders' equity value to be left unaffected by the investment. 

        The marginal impact on the firm's balance sheet of entering the project is given by:
        \begin{equation}
            \pi(A) + \pi(Y) = (\pi(E) + \Delta \pi(E) + \pi(\tilde{E})) + (\pi(D) + \Delta \pi(D))
        \end{equation}
        where $\pi(\tilde{E})$ is the market value of the newly issued equity. 
        \Cref{eqn:wealth-transfer-equality} implies that:
        \begin{equation}
            \pi(A) + \pi(Y) = (\pi(E) + \pi(\tilde{E})) + \pi(D)
        \end{equation}
        As mentioned in \cref{eqn:derivative-zero-npv}, the firm assumes to raise an amount of new equity equal to the premium of the derivative. 
        This means that the Modigliani-Miller invariance proposition holds, and also shows, that regardless of the strategy of funding a financial project, the funding costs has no impact on the valuation of the derivative nor the asset base of the firm itself.

        When trading decisions are made, the firm's preferences are assumed to be determined by the shareholders.
        However, in the case of funding by share issuance, it's clear that the legacy shareholders' position deteriorates, as they bear the funding costs.\\
        An adjustment to the value of the derivative would require a donation by the counterparty, such that the value of derivatives' receivables would be worth more than the premium.
        This should only go to an extent so that the legacy shareholders become indifferent of whether the the firm enters the project or not.
        But whether the counterparty has incentives to do so requires more information about their capital structure, and will remain indeterminate.\\
        These conclusions hold for both types of funding, but the amount of wealth transfer does depend on the instrument used. Shareholders are worse off with share issuance than with debt issuance.
        The availability of cash in the firm's deposits would solve the issue faced by the shareholders:
        Suppose the firm uses their own cash to purchase the derivative from the counterparty at a market fair value.
        This is a zero net present value investment for the firm, where, in addition, no wealth is transferred from the shareholders, and neither does the creditors become better off. 

        The breakeven price necessary to account for funding costs under equity funding is calculated using
        \cref{eqn:shareholders-breakeven-equity-financing}:
            \begin{equation}
                    u^{\ast}_{\text{equity}} 
                =
                    \frac{1}{1 + \pct{0.010101}} 
                    *
                    \pct{0.878787}
                    *
                    \left(
                        \num{1}
                        -
                        \num{0}
                    \right)
                =
                    \num{0.869999}
            \end{equation}
        meaning 
        \begin{equation}
            \FVA/ = u^{\ast}_{\text{equity}} - u =
            \num{0.989994}-\num{0.869999} =
            \num{0.12}
        \end{equation}
        Notice that this amount is the same as the wealth transfer.
        This is because the donation from the counterparty does not affect the total asset value of the firm.
        Hence, the loss rate of the creditors remains unchanged from before the donation, and therefore so does the creditors' payoff's present value.
        The creditors will then still receives the wealth transfer, $\Delta \pi(D)$,
        due to the project being entered.
        The new shareholders will however finance a cheaper project as a result of the \FVA/, and since it is still assumed to be a zero net present value investment for them, their share of the firm's remaining capital, $\alpha$, is reduced.
        This leads to a direct increase in the legacy shareholders' wealth from before the valuation adjustment by an amount of exactly the donation, as they are now the only ones to benefit from the price reduction. 

\end{document}
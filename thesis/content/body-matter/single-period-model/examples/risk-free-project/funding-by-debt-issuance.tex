% !TEX root = sub-main.tex
\documentclass[main.tex]{subfiles}

\begin{document}
    \subsubsection{Funding by debt issuance}\label{sec:example-risk-free-project-debt-issuance}
        In order to finance the upfront of the project, $u$, the firm issues debt to new creditors, 
        such that the price of the debt, denoted $\pi(\tilde{D})$, equals the price of the project.
        As assumed previously, the new debt ranks pari passu with the legacy debt, 
        such that all creditors experience the same loss rate in states where the firm defaults.
        A claim with face value $\tilde{D}_{FV}$ which ranks pari passu to another claim 
        with face value $D_{FV}$ has the random loss rate given by \cref{eqn:creditor-loss-rate}:
            \begin{align}
                \phi
                &=
                    \frac{
                        \tilde{D}_{FV} + D_{FV} - A
                    }{
                        \tilde{D}_{FV} + D_{FV}
                    }
                    \mathbbm{1}_{\mathcal{D}}
                \nonumber \\
                &=
                    \frac{
                        80.00 + \tilde{D}_{FV} - A
                    }{
                        80.00 + \tilde{D}_{FV}
                    }
                    \mathbbm{1}_{\{A + 10.00 < 80.00 + \tilde{D}_{FV}\}} 
            \end{align}
        If the asset value is larger than the total face value to be repaid,
        the firm does not default and the loss rate is $0$.
        If the firm defaults, the loss rate is equal to the share of the total face value missing from the asset value.
        The random payoff of the new pari passu debt is denoted by $\tilde{D}$ and given as:
            \begin{align}
                \tilde{D}
                    = (1 - \phi)\tilde{D}_{FV}
            \end{align}

        Due to the credit risk of the firm, the new creditors will require a credit spread on the debt 
        in addition to the risk free return.
        This additional interest has implications for the firm's perceived value of the project, 
        which raises the problem for debate about whether the price should be reduced 
        to accommodate the firm's assessment of the value.

        In order for the firm to attract new creditors they must offer a large enough interest rate on the debt, 
        such that buying the debt is a zero net present value investment.
        The new debt should be able to cover the investment cost of the new project,
        why the present value of the new debt must equal, $u$.
        Therefore the face value must be chosen to solve the following equation:
            \begin{align}
                u &= \pi(\tilde{D}) \\
                \Leftrightarrow  \qquad
                9.90 &= \mathbb{E}^{\rnmeasure}\left[1 - \phi\right] \tilde{D}_{FV}
            \end{align}
        While seemingly simple, this equation proves very difficult to solve analytically for $\tilde{D}_{FV}$.
        Substituting the loss rate, it is clear that the face value is quite entangled in the equation;
        both in the default event and in the fraction:
            \begin{align}
                9.90
                &= 
                \mathbb{E}^{\rnmeasure}\left[
                    1 
                    - 
                    \frac{
                        80.00 + \tilde{D}_{FV} - A
                    }{
                        80.00 + \tilde{D}_{FV}
                    }
                    \mathbbm{1}_{\{A + 10.00 < 80.00 + \tilde{D}_{FV}\}} 
                \right] 
                \tilde{D}_{FV} 
            \end{align}
        However, by studying this expanded equation, it is clear that the right hand side 
        is increasing in $\tilde{D}_{FV}$, as that holds true for all the terms that constitutes it.
        Hence, the equation $u - \pi(\tilde{D})$ is very much suited for any simple numerical optimization procedure
        for root finding, as long as the evaluation of the function itself is not too computationally heavy.
        In a single period framework, especially one with as few states as the current,
        the evaluation of the expectation is by no means problematic.
        Since it is effective and simple, the equation will be solved by numerical procedures, 
        which yields the following solution:
            \begin{align}
                \Leftrightarrow  \qquad
                \tilde{D}_{FV} &= 10.28
            \end{align}
        This face value results in an expected default loss to creditors of 
        $\mathbb{E}^{\rnmeasure}\left[\phi\right] = 2.7228\%$,
        and a limiting spread of:
        \begin{equation}
            S 
            = \frac{
                2.7228\% * (1 + \rfrate)
            }{
                1 - 2.7228\%
            } 
            = 2.8273\%
        \end{equation}
        Entering into the project and issuing new debt with face value $10.28$ 
        alters the payoffs associated with the firm in the following way:
        \begin{table}[H]
            \centering
            \begin{tabular}{l|rrrrr||r}
                $i$ & 1 & 2 & 3 & 4 & 5 & Present value \\
                \hline
                $A(\omega_{i})$ & $130$ & $120$ & $110$ & $105$ & $70$ & $106.30$ \\
                $D(\omega_{i})$ & $80$ & $80$ & $80$ & $80$ & $62.03$ & $77.04$ \\
                $S(\omega_{i})$ & $39.72$ & $29.72$ & $19.72$ & $14.72$ & $0$ & $19.36$ \\
                $\tilde{D}(\omega_{i})$ & $10.28$ & $10.28$ & $10.28$ & $10.28$ & $7.97$ & $9.90$ \\
            \end{tabular}
            \caption{}
        \end{table}

        Investing in the project increases the present value of the firm's assets by $106.30 - 96.40 = 9.90$, 
        which, not surprisingly, is the value of the the project.
        More interesting is the impact to the shareholder's claim that decreases by an amount $19.36 - 19.60 = -0.24$, 
        why investing in the project is of negative value to the shareholders.
        When the firm defaults, the shareholders still receive nothing, 
        but in all other states where the firm does not default, 
        the shareholders pay the present value of the interest owed to the new creditors 
        due to the credit risk in the firm.
        The legacy creditors observe their debt claim with face value $80$ increase by an amount of
        $77.04 - 76.80 = 0.24$.

        Evidently, while the project is a negative net present value investment for the shareholders,
        the investment is a positive net present value investment for the legacy creditors.
        When the firm does not default, the creditors still receive their promised payoff 
        corresponding to the face value of their debt.
        The new project has increased the asset base of the firm, and when the firm defaults, 
        the legacy creditors will share part of the payoff from the option with the new creditors.
        Hence, the legacy creditors receive a larger payoff when the firm defaults and their claim increases in value. 

        The Modigliani-Miller invariance proposition assures that making a zero net present value investment 
        does not increase the value of the firm.
        That is to say that obtaining the new project should neither create nor destroy value in aggregate terms,
        which explains why the amount lost by shareholders is the amount gained by legacy creditors.
        The asset base also increases by the value of the project, 
        but decreases by the value of the debt claim used to finance it; 
        both are priced the same and therefore offset each other.
        In this example, the wealth lost by the shareholders is entirely transferred to the legacy creditors,
        since the project is a zero net present value both to the new creditors 
        and to the counterparty offering the project.
        \textcolor{red}{
            In subsequent sections, the consequences of obtaining projects that are non-zero net present value investments
            for the new creditors and the counterparty are covered.
        }

        The wealth transfer from shareholders to legacy creditors due to the new project 
        suggests quantifying the \FVA/ of the project.
        For completeness, this first example will calculate all three possible definitions of \FVA/,
        as described in \cref{sec:defining-fva}, and compare them.

        The first possible definition of \FVA/ as the promised excess funding cost can be determined by
        \cref{eqn:fva-as-promised-excess-funding-cost}:
            \begin{align}
                0.99 * 9.90 * 2.8273\% = 0.2771
            \end{align}
        This definition suggests adjusting the project value by $0.2771$ to account for the funding costs.
        As discussed earlier, this definition ignores the fact that shareholders are exempt from paying
        interest rates in the default state; they have limited liability and, since they lose everything
        in the event of default, the marginal value to them in default states is zero.
        
        Calculating instead the expected excess funding cost, i.e. $\Phi$:
            \begin{equation}
                \Phi 
                = 87.88\% * 0.99  * 9.90 * 2.8273\% 
                = 0.2435
            \end{equation}
        This quantity is also the value lost by the shareholders and gained by the legacy creditors,
        since the net profit of the project is zero, and the payoff is risk free.
        This can be thoroughly calculated by using the marginal shareholder value under debt funding in 
        \cref{eqn:marginal-shareholder-value-debt-financing}
        which will result in the equation $G_{\debt} = -\Phi$. 
        
        Turning instead to the definition of \FVA/ as the adjustment needed for shareholders to breakeven.
        Using \cref{eqn:shareholders-breakeven-debt-financing}, the price needed for breakeven is:
            \begin{align}
                    u^{\ast}_{\text{debt}} 
                &=
                    \frac{
                        1
                    }{
                        1 + 1.0101\% + 2.8273\%
                    }
                    \left(
                        10
                        -
                        0
                    \right)
                \nonumber \\
                &=
                    9.6333
            \end{align}
        And, according to \cref{eqn:fva-debt-financing}, the \FVA/ is therefore:
            \begin{equation}
                    \FVA/ 
                =
                    9.90 - 9.6333
                =
                    0.2667
            \end{equation}
        Any prices above the breakeven price, $u^{\ast}_{\text{debt}}$, 
        will make the project a negative net present value investment for the shareholders, while any prices below will make it positive. 
        In this regard, the promised excess funding cost of $0.2771$ would be an overadjustment,
        and, conversely, the expected excess funding cost, $\Phi$, of $0.2435$ would be an underadjustment.

\end{document}
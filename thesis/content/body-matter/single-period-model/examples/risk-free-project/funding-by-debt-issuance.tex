% !TEX root = sub-main.tex
\documentclass[main.tex]{subfiles}

\begin{document}
    \subsubsection{Funding by debt issuance}\label{sec:example-risk-free-project-debt-issuance}
        In order to finance the upfront of the project, $u$, the firm issues debt to new creditors, 
        such that the price of the debt, denoted $\pi(\tilde{D})$, equals the price of the project.
        As assumed previously, the new debt ranks pari passu with the legacy debt, 
        such that all creditors experience the same loss rate in states where the firm defaults.
        A claim with face value $\tilde{D}_{FV}$ which ranks pari passu to another claim 
        with face value $D_{FV}$ has the random loss rate given by \cref{eqn:creditor-loss-rate}:
            \begin{align}
                \phi
                &=
                    \frac{
                        \tilde{D}_{FV} + D_{FV} - A
                    }{
                        \tilde{D}_{FV} + D_{FV}
                    }
                    \mathbbm{1}_{\mathcal{D}}
                \nonumber \\
                &=
                    \frac{
                        80.00 + \tilde{D}_{FV} - A
                    }{
                        80.00 + \tilde{D}_{FV}
                    }
                    \mathbbm{1}_{\{A + 10.00 < 80.00 + \tilde{D}_{FV}\}} 
            \end{align}
        If the asset value is larger than the total face value to be repaid
        the firm does not default and the loss rate is $0$.
        If the firm defaults, the loss rate is equal to the share of the total face value missing from the asset value.
        The random payoff of the new pari passu debt is denoted by $\tilde{D}$ and given as:
            \begin{align}
                \tilde{D}
                    = (1 - \phi)\tilde{D}_{FV}
            \end{align}

        Due to the credit risk of the firm, the new creditors will require a credit spread on the debt 
        in addition to the risk free return.
        This additional interest has implications for the firm's perceived value of the project, 
        which raises the problem for debate about whether the price should be reduced 
        to accommodate the firm's assessment of the value.

        In order for the firm to attract new creditors they must offer a large enough interest rate on the debt, 
        such that buying the debt is a zero net present value investment.
        The new debt should be able to cover the investment cost of the new project,
        why the present value of the new debt must equal, $u$.
        Therefore the face value must be chosen to solve the following equation:
            \begin{align}
                \pi(Y) &= \pi(\tilde{D}) \\
                \Leftrightarrow  \qquad
                9.90 &= \mathbb{E}^{\rnmeasure}\left[1 - \phi\right] \tilde{D}_{FV} \\
                \Leftrightarrow  \qquad
                \tilde{D}_{FV} &= 10.28
            \end{align}
        This face value implies an interest rate of $10.28 / 9.90 - 1 = 3.84\%$
        and therefore a credit spread of $\tilde{s} = 3.84\% - \rfrate = 2.83\%$.
        Entering into the project and issuing new debt with face value $10.28$ 
        alters the payoffs associated with the firm in the following way:
        \begin{table}[H]
            \centering
            \begin{tabular}{l|rrrrr||r}
                $i$ & 1 & 2 & 3 & 4 & 5 & Present value \\
                \hline
                $A(\omega_{i})$ & $130$ & $120$ & $110$ & $105$ & $70$ & $106.30$ \\
                $D(\omega_{i})$ & $80$ & $80$ & $80$ & $80$ & $62.03$ & $77.04$ \\
                $S(\omega_{i})$ & $19.72$ & $29.72$ & $14.72$ & $39.72$ & $0$ & $19.36$ \\
                $\tilde{D}(\omega_{i})$ & $10.28$ & $10.28$ & $10.28$ & $10.28$ & $7.97$ & $9.90$ \\
            \end{tabular}
            \caption{}
        \end{table}

        Investing in the project increases the present value of the firm's assets by $106.30 - 96.40 = 9.90$, 
        which, not surprisingly, is the value of the the project.
        More interesting is the impact to the shareholder's claim that decreases by an amount $19.36 - 19.60 = -0.24$, 
        why investing in the project is of negative value to the shareholders.
        When the firm defaults the shareholders still receive nothing, 
        but in all other states where the firm does not default, 
        the shareholders pay the present value of the interest owed to the new creditors 
        due to the credit risk in the firm.
        The observed decrease in shareholder value aligns well with 
        \cref{eqn:marginal-shareholder-value-debt-financing}. 
        The risk neutral default probability is 
        $\mathbb{P}^{\rnmeasure}\left(\mathcal{D}\right) 
            = \mathbb{P}^{\rnmeasure}\left(\{\omega_{5}\}\right) 
            = 12.12\%$,
        why $p^{\rnmeasure} = 87.88\%$.
        The promised marginal profit of the new project is:
        $\pi = 0.99 * 10 - 9.90 = 0$.
        The payoff is constant, so the default event and the payoff have zero covariance. 
        The expected default loss to creditors is 
        $\mathbb{E}^{\rnmeasure}\left[\phi\right] = 2.72\%$,
        and the limiting spread is:
        \begin{equation}
            S 
            = \frac{
                2.72\% * (1 + \rfrate)
            }{
                1 - 2.72\%
            } 
            = 2.83\%
        \end{equation}
        The marginal valuation to the legacy creditors can then be calculated as:
        \begin{equation}
            \Phi 
            = 87.88\% * \frac{1}{1+\rfrate}  * 9.90 * 2.83\% 
            = 0.2435
        \end{equation}

        The loss of the shareholders can then be calculated as: 
            \begin{align}
                G = 87.88\%*0 - \frac{1}{1+\rfrate} * 0 - 0.2435 = -0.2435
            \end{align}
        This is also the fair price, with opposite sign, 
        of a security that pays the promised return of the new debt in every no default state and zero otherwise. 
        While the project is a negative net present value investment for the shareholders,
        the investment is a positive net present value investment for the legacy creditors.
        When the firm does not default they still receive their promised payoff 
        corresponding to the face value of their debt.
        The new project has increased the asset base of the firm, and when the firm defaults, 
        the legacy creditors will share part of the payoff from the option with the new creditors.
        Hence, the legacy creditors receive a larger payoff when the firm defaults. 

        The Modigliani-Miller invariance proposition assures that making a zero net present value investment 
        does not increase the value of the firm.
        Therefore, the wealth lost by the shareholders must be entirely transferred to the legacy creditors 
        increasing the present value of their debt claim by $-G = 0.2435$. 
        This is also the observed effect when comparing the new present value with the old 
        $77.04 - 76.80 = 0.24$.

        The wealth transfer from shareholders to legacy creditors due to the new project 
        suggests defining and quantifying the \FVA/ of the project. 

        Using \cref{eqn:fva-as-promised-excess-funding-cost}, the present value of the funding costs 
        in excess of the risk free rate paid by the shareholders is: 
        \begin{align}
            0.99 * 9.90 * 2.83\% = 0.2771
        \end{align}

        The wealth transfer from the shareholders to the legacy creditors, 
        calculated as $\Phi$ in \cref{eqn:marginal-shareholder-value-debt-financing}, is
        \begin{equation}
            \Phi = 0.2435
        \end{equation}
        The two quantities differ by the no default probability, such that 
        $0.2771 * (1-\mathbb{P}\left(\mathcal{D}\right)) = 0.2435$.

\end{document}
% !TEX root = sub-main.tex
\documentclass[main.tex]{subfiles}

\begin{document}
    The derivations in previous sections, 
    which described limiting spreads and marginal shareholder valuations, 
    were rooted in the assumption of the firm obtaining an infinitesimal project.
    This section will use the results in a practical application,
    where a larger than infinitesimal project will be obtained by the firm.
    Therefore, the results obtained from the equations will, to some extent, be inaccurate.
    The magnitude of the inaccuracy will depend on how much the new project alters
    the creditors' loss rate, since that controls the credit spread.
    The scale of the impact on the creditors' loss rate 
    depends on the the actual size of the project compared to value of the firm's assets.
    The direction is determined by the payoff structure of the project
    compared to the structure of the firm's assets.

    Fortunately, in the discrete framework used here, 
    the actual marginal shareholder valuations can be obtained with numerical methods,
    such that the results of the equations can be verified and compared to the correct results.
    For that reason, the first project considered will be of a size that
    results in a slight inaccuracy of the equations such that
    it gives a sense of the problem's magnitude.
    
    Assume that the firm faces a new risk-free project in which it can invest
    and that the project has a promised return of $\num{1}$ at time 1.
    The project is risk-free in the sense that the payoff is known with certainty at time 0, so:
        \begin{table}[H]
            \centering
            \begin{tabular}{l|rrrrr}
                $i$ & 1 & 2 & 3 & 4 & 5 \\
                \hline
                $Y(\omega_{i})$ & $\num{1}$ & $\num{1}$ & $\num{1}$ & $\num{1}$ & $\num{1}$
            \end{tabular}
            \caption{}
            \label{tbl:single-period-simple-derivative-payoff}
        \end{table}
    which implies a present value of 
    $\pi(Y) = \discountfactor \mathbb{E}^{\rnmeasure}\left[Y\right] = \num{0.99}$, 
    corresponding to the price of 1 zero coupon bond, 
    as the project is merely an investment in the risk-free asset.
    Obtaining the payoff requires an upfront cost equal to the fair value of the project.
    With a slightly redefined notation, since the project is not infinitesimal,
    the marginal investment cost of the project will be denoted by $u$.
    Assume that the counterparty offering the project makes a zero net present value investment, 
    such that the price of the project equals the value, i.e. $u = \pi(Y)$.
    As discussed in previous sections, the firm has different means of obtaining funding
    for paying the investment cost of the project.
    Financing through debt issuance is examined in the following section 
    and financing by share issuance in the section after that.

    \subparagraph{Digression on the limiting spread}
    As was calculated above, the limiting spread is equal to the credit spread of the existing debt.
    This is no coincidence, and this digression will explain why it is so,
    and what the implications are.
    As it turns out, 
    the framework presented here provides a helpful interpretation of the limiting spread.
    Recall the definition of the limiting spread, repeated here for convenience:
        \begin{equation*}
            S
            =
            \frac{
                \mathbb{E}^{\rnmeasure}\left[\phi\right]
                \grossrfrate
            }{
                1
                -
                \mathbb{E}^{\rnmeasure}\left[\phi\right]
            }
        \end{equation*}
    For a constant face value,
    the credit spread can be calculated by dividing the face value of the debt
    with the present value of the debt claim and subtracting the gross risk free rate.
    Denote the face value by $F$. 
    The present value of the debt claim is the discounted face value, $\discountfactor F$,
    adjusted for the credit risk. 
    The credit value adjustment, as usual, corresponds to the discounted expected value of
    the positive exposure, $F$, multiplied by the loss rate given default, $\phi(\omega_{5})$.
    Hence the credit spread of existing debt is given by:
        \begin{equation*}
            \frac{
                F
            }{
                \discountfactor F
                -
                \discountfactor
                \mathbb{E}^{\rnmeasure}\left[
                    F \phi
                \right] 
            } 
            -
            \grossrfrate
            =
            R \left(
            \frac{
                1
            }{
                    1
                    -
                    \mathbb{E}^{\rnmeasure}\left[
                        \phi
                    \right] 
            } 
            - 
            1
            \right)
            =
            \frac{
                \mathbb{E}^{\rnmeasure}\left[\phi\right]
                \grossrfrate
            }{
                1
                -
                \mathbb{E}^{\rnmeasure}\left[\phi\right]
            }
        \end{equation*}
        Which can be recognized as the limiting spread. 
        Hence, in the current setup, 
        the definitions of the marginal shareholder valuations are based on
        the assumption that the post-project credit spread 
        will equal the pre-project credit spread.
        However, these definitions are derived using infinitesimal projects. 
        In practical applications,
        it would therefore be more appropriate to refer to the limiting spread as an approximation 
        of the credit spread after obtaining a larger-than-infinitesimal project.
        The mathematics have been accounted for, 
        but the approximation does seem to be financial legitimate as well.
        First of all, the post-project credit spread is based on the risk of the entire firm,
        which most likely encompasses many projects.
        Each new project is therefore going to be of relatively small size compared
        to the entire firm, and it will only have a small effect on the overall riskiness.
        Second, this effect is enhanced if the firm mostly invests in projects that match
        its overall riskiness in the first place.
        Third and last, as was mentioned earlier, according to \textcite{Castagna2012FVA},
        the cost of capital will only gradually update, 
        such that obtaining a new project only have a very marginal effect on the credit spread.

        In conclusion, the limiting spread is not merely a mathematical result
        but also a financially sound approximation of the incremental cost of obtaining new projects.
        With this interpretation in mind, the example can continue.

\end{document}
% !TEX root = sub-main.tex
\documentclass[main.tex]{subfiles}

\begin{document}
    This section will provide an example of funding costs in a single period model
    using a risk free asset as the project under the firm's consideration.
    Consider the economy defined by a single period model with $N=5$ states, and the following properties:
        \begin{table}[H]
            \centering
            \begin{tabular}{l|rrrrr}
                $i$ & 1 & 2 & 3 & 4 & 5 \\
                \hline
                $\psi_{i}$ & $0.06$ & $0.24$ & $0.29$ & $0.28$ & $0.12$ \\
            \end{tabular}
            \caption{}
            \label{tbl:example-firm-structure}
        \end{table}
    implying a discount factor of $d_{0} = 0.99$ and a risk free interest rate of $r_{f} = 1.0101\%$.
    A firm operates in this economy and invests in risky assets 
    which are funded by equity and debt deposits and returns payoff specified shortly.
    The firm has been funded by debt such that the face value of debt is $D_{FV} = 80$, 
    which gives rise to the following payoff structure:
    \begin{table}[H]
        \centering
        \begin{tabular}{l|rrrrr}
            $i$ & 1 & 2 & 3 & 4 & 5 \\
            \hline
            $A(\omega_{i})$ & 120 & 110 & 100 & 95 & 60 \\
            $D(\omega_{i})$ & 80 & 80 & 80 & 80 & 60 \\
            $S(\omega_{i})$ & 40 & 30 & 20 & 15 & 0
        \end{tabular}
        \caption{}
    \end{table}
    The associated present values of the payoffs, $A$, $D$ and $S$, 
    are discounted expected value of the payoff with respect to the risk neutral probability measure:
        \begin{gather}
            \pi(A) = \frac{1}{1+r_{f}} \mathbb{E}^{\mathbb{Q}}\left[A\right] = 96.40 \\
            \pi(D) = \frac{1}{1+r_{f}} \mathbb{E}^{\mathbb{Q}}\left[D\right] = 76.80
            \qquad \pi(S) = \frac{1}{1+r_{f}} \mathbb{E}^{\mathbb{Q}}\left[S\right] = 19.60
        \end{gather}

    Assume that the firm faces a new risk free project in which it can invest.
    The project is risk free in the sense that the payoff is known with certainty at time 0, so:
        \begin{table}[H]
            \centering
            \begin{tabular}{l|rrrrr}
                $i$ & 1 & 2 & 3 & 4 & 5 \\
                \hline
                $Y(\omega_{i})$ & $10$ & $10$ & $10$ & $10$ & $10$
            \end{tabular}
            \caption{}
        \end{table}
    which implies a present value of $Y_{0} = 15d_{0} = 9.90$, 
    since the project is merely an investment in the risk free asset.
    Obtaining the payoff requires an upfront cost equal to the fair value of the project, $\pi(Y) = Y_{0}$.
    The firm can either obtain the funding required by issuing debt or by selling equity.
    The former case is examined in the following section and the latter in the section after that.

\end{document}
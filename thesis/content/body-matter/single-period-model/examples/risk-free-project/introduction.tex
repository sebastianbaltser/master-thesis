% !TEX root = sub-main.tex
\documentclass[main.tex]{subfiles}

\begin{document}
    Assume that the firm faces a new risk-free project in which it can invest,
    and that the project has a promised return of $\num{1}$ at time 1 summarised in \cref{tbl:single-period-simple-derivative-payoff}.
    The project is risk-free in the sense that there is no market risk,
    and the payoff is known with certainty at time 0.
    Additionally, the counterparty of the project is assumed to be credit risk-free.
        \begin{table}[H]
            \centering
            \begin{tabular}{l|rrrrr||r}
                $i$ & 1 & 2 & 3 & 4 & 5 & Present value\\
                \hline
                $Y(\omega_{i})$ & 
                    \num{1} & \num{1} & \num{1} & \num{1} & \num{1} & \num{0.99}
            \end{tabular}
            \caption{}
            \label{tbl:single-period-simple-derivative-payoff}
        \end{table}
    Obtaining the payoff requires an upfront cost of the project.
    With a slightly redefined notation, since the project is not infinitesimal,
    the marginal investment cost of the project will be denoted by $u$.
    Assume that the counterparty offering the project makes a zero net present value investment, 
    such that the price of the project equals the expected discounted value, i.e. $u = \pi(Y)$.
    When obtaining financing the total asset value of the firm increases correspondingly.
    The time 1 value of the assets after obtaining the project will be denoted by $\tilde{A}$.
    \\
    As discussed in previous sections, the firm has different means of obtaining funding
    for paying the investment cost of the project.
    Financing through debt issuance is examined in the following section 
    and financing by equity issuance in the section after that.

\end{document}
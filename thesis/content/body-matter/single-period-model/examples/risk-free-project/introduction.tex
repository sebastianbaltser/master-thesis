% !TEX root = sub-main.tex
\documentclass[main.tex]{subfiles}

\begin{document}
    This section will provide an example of funding costs in a single period model
    using a risk-free asset as the project under the firm's consideration.
    Consider the economy defined by a single period model with $N=5$ states, 
    and the following associated Arrow-Debreu prices:
        \begin{table}[H]
            \centering
            \begin{tabular}{l|rrrrr}
                $i$ & 1 & 2 & 3 & 4 & 5 \\
                \hline
                $\psi_{i}$ & $0.06$ & $0.24$ & $0.29$ & $0.28$ & $0.12$ \\
            \end{tabular}
            \caption{}
            \label{tbl:example-firm-structure}
        \end{table}
    implying a discount factor of $\discountfactor = 0.99$ and a risk free interest rate of $\rfrate = 1.0101\%$.
    A firm operates in this economy and invests in risky assets 
    which are funded by equity and debt deposits and returns payoff specified shortly.
    In the event of default, the firm is assumed to face no distress cost, i.e. $\kappa = 1$, 
    such that the bankruptcy estate is distributed entirely to the creditors.
    The firm has been funded by debt such that the face value of debt is $80$, 
    which gives rise to the following payoff structure:
    \begin{table}[H]
        \centering
        \begin{tabular}{l|rrrrr}
            $i$ & 1 & 2 & 3 & 4 & 5 \\
            \hline
            $A(\omega_{i})$ & 120 & 110 & 100 & 95 & 60 \\
            $D(\omega_{i})$ & 80 & 80 & 80 & 80 & 60 \\
            $S(\omega_{i})$ & 40 & 30 & 20 & 15 & 0
        \end{tabular}
        \caption{}
        \label{tbl:example-pre-project-capital-structure}
    \end{table}
    The associated present values of the payoffs, $A$, $D$ and $S$, 
    are discounted expected value of the payoff with respect to the risk neutral probability measure:
        \begin{gather}
            \pi(A) = \discountfactor \mathbb{E}^{\rnmeasure}\left[A\right] = 96.40 \\
            \pi(D) = \discountfactor \mathbb{E}^{\rnmeasure}\left[D\right] = 76.80
            \qquad \pi(S) = \discountfactor \mathbb{E}^{\rnmeasure}\left[S\right] = 19.60
        \end{gather}
    The face value of the debt includes a credit spread, S, that the firm must pay in order to account for its own credit risk.
    As seen in \cref{tbl:example-pre-project-capital-structure},
    the creditors do not receive their entire face value in the default event, $\omega_5$.
    For the creditors to make sure they enter a non-negative net present value deal with the firm,
    they add a credit spread to the interest rate.
    This credit spread is computed as
    $S = 80/76.80 - \grossrfrate = 3.157\%$.
    \\
    The loss rate of the creditors in the default event is computed using \cref{eqn:creditor-loss-rate}: $\phi = (80-60)/80 = 25.0\%$.
    
    Assume that the firm faces a new risk-free project in which it can invest
    and that the project has a promised return of $10$ at time 1.
    The project is risk free in the sense that the payoff is known with certainty at time 0, so:
        \begin{table}[H]
            \centering
            \begin{tabular}{l|rrrrr}
                $i$ & 1 & 2 & 3 & 4 & 5 \\
                \hline
                $Y(\omega_{i})$ & $10$ & $10$ & $10$ & $10$ & $10$
            \end{tabular}
            \caption{}
        \end{table}
    which implies a present value of $\pi(Y) = \discountfactor \mathbb{E}^{\rnmeasure}\left[Y\right] = 9.90$, 
    corresponding to the price of 10 zero coupon bonds, as the project is merely an investment in the risk free asset.
    Obtaining the payoff requires an upfront cost equal to the fair value of the project, 
    i.e. the marginal investment cost is $u = \pi(Y)$.
    As discussed in previous sections the firm have different means of obtaining funding
    for paying the investment cost of the project.
    Financing through debt issuance is examined in the following section 
    and financing by share issuance in the section after that.

\end{document}
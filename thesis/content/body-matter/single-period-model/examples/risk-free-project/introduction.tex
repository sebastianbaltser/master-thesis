% !TEX root = sub-main.tex
\documentclass[main.tex]{subfiles}

\begin{document}
    This section will provide an example of funding costs in a single-period model
    using a risk-free asset as the project under the firm's consideration.
    Consider the economy defined by a single-period model with $N=5$ states, 
    and the following associated Arrow-Debreu prices:
        \begin{table}[H]
            \centering
            \begin{tabular}{l|rrrrr}
                $i$ & 1 & 2 & 3 & 4 & 5 \\
                \hline
                $\psi_{i}$ & $\num{0.06}$ & $\num{0.24}$ & $\num{0.29}$ & $\num{0.28}$ & $\num{0.12}$ \\
            \end{tabular}
            \caption{}
            \label{tbl:example-firm-structure}
        \end{table}
    implying a discount factor of $\discountfactor = \num{0.99}$ and a risk-free interest rate of $\rfrate = \pct{0.010101}$.
    A firm operates in this economy and invests in risky assets 
    which are funded by equity and debt deposits and returns payoff specified shortly.
    In the event of default, the firm is assumed to face no distress cost, i.e. $\kappa = 1$, 
    such that the bankruptcy estate is distributed entirely to the creditors.
    The firm has been funded by debt such that the face value of debt is $\num{80}$, 
    which gives rise to the following payoff structure:
    \begin{table}[H]
        \centering
        \begin{tabular}{l|rrrrr}
            $i$ & 1 & 2 & 3 & 4 & 5 \\
            \hline
            $A(\omega_{i})$ & \num{120} & \num{110} & \num{100} & \num{95} & \num{60} \\
            $D(\omega_{i})$ & \num{80} & \num{80} & \num{80} & \num{80} & \num{60} \\
            $E(\omega_{i})$ & \num{40} & \num{30} & \num{20} & \num{15} & \num{0}
        \end{tabular}
        \caption{}
        \label{tbl:example-pre-project-capital-structure}
    \end{table}
    The associated present values of the payoffs, $A$, $D$ and $E$, 
    are discounted expected value of the payoff with respect to the risk-neutral probability measure:
        \begin{gather*}
            \pi(A) = \discountfactor \mathbb{E}^{\rnmeasure}\left[A\right] = \num{96.40} \\
            \pi(D) = \discountfactor \mathbb{E}^{\rnmeasure}\left[D\right] = \num{76.80}
            \qquad \pi(E) = \discountfactor \mathbb{E}^{\rnmeasure}\left[E\right] = \num{19.60}
        \end{gather*}
    The face value of the debt includes a credit spread that the firm must pay in order to account for its own credit risk.
    As seen in \cref{tbl:example-pre-project-capital-structure},
    the creditors do not receive their entire face value in the default event, $\omega_5$.
    The relative difference between the discounted payoff and the promised face value
    represents the credit spread of the creditors.
    If the creditors obtain the debt by paying the discounted payoff,
    the credit spread will ensure that the net present value of obtaining the debt is zero.
    In addition to calculating the shareholders' marginal valuation of a new project,
    it is interesting to also evaluate the impact on the creditors' valuation of their claim. 
    Therefore, it is useful to calculate the credit spread of the firm's debt 
    before entering any projects and the credit spread after entering a project.
    The same goes for the loss rate.

    The credit spread before entering into any projects is 
    $\num{80}/\num{76.80} - \grossrfrate = \pct{0.031566}$.
    Using \cref{eqn:creditor-loss-rate}, the loss rate of the creditors in the default event is:
    $\phi = (\num{80}-\num{60})/\num{80} = \pct{0.25}$
    while in all other states the loss rate is zero, as the firm does not default.
    Recall that the limiting spread is the spread on the firm's debt 
    after obtaining an infinitesimal amount of a project. 
    This is identical to the pre-project spread on debt, 
    but for completeness the loss rate can be substituted into \cref{eqn:limiting-spread} 
    and limiting spread can be calculated:
        \begin{equation}
            S
            =
            \frac{
                \mathbb{E}_{t}\left[\phi\right]
                R
            }{
                1 
                -
                \mathbb{E}_{t}\left[\phi\right] 
            } 
            =
            \pct{0.031566}
        \end{equation}
    The marginal shareholder valuations derived in previous sections, 
    were rooted in the assumption of the firm obtaining an infinitesimal project.
    This section will use the results in a practical application,
    where a larger than infinitesimal project will be obtained by the firm.
    Therefore, the results obtained from the equations will, to some extent, be inaccurate.
    The magnitude of the inaccuracy will depend on how much the new project alters
    the creditors' loss rate, since that controls the credit spread.
    The scale of the impact on the creditors' loss rate 
    depends on the the actual size of the project compared to value of the firm's assets.
    The direction is determined by the payoff structure of the project
    compared to the structure of the firm's assets.

    Fortunately, in the discrete framework used here, 
    the actual marginal shareholder valuations can be obtained with numerical methods,
    such that the results of the equations can be verified and compared to the correct results.
    For that reason, the first project considered will be of a size that
    results in a slight inaccuracy of the equations such that
    it gives a sense of the problem's magnitude.
    
    Assume that the firm faces a new risk-free project in which it can invest
    and that the project has a promised return of $\num{1}$ at time 1.
    The project is risk-free in the sense that the payoff is known with certainty at time 0, so:
        \begin{table}[H]
            \centering
            \begin{tabular}{l|rrrrr}
                $i$ & 1 & 2 & 3 & 4 & 5 \\
                \hline
                $Y(\omega_{i})$ & $\num{1}$ & $\num{1}$ & $\num{1}$ & $\num{1}$ & $\num{1}$
            \end{tabular}
            \caption{}
            \label{tbl:single-period-simple-derivative-payoff}
        \end{table}
    which implies a present value of 
    $\pi(Y) = \discountfactor \mathbb{E}^{\rnmeasure}\left[Y\right] = \num{0.99}$, 
    corresponding to the price of 1 zero coupon bond, 
    as the project is merely an investment in the risk-free asset.
    Obtaining the payoff requires an upfront cost equal to the fair value of the project.
    With a slightly redefined notation, since the project is not infinitesimal,
    the marginal investment cost of the project will be denoted by $u$.
    Assume that the counterparty offering the project makes a zero net present value investment, 
    such that the price of the project equals the value, i.e. $u = \pi(Y)$.
    As discussed in previous sections, the firm has different means of obtaining funding
    for paying the investment cost of the project.
    Financing through debt issuance is examined in the following section 
    and financing by share issuance in the section after that.

\end{document}
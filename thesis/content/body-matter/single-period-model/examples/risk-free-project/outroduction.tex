% !TEX root = sub-main.tex
\documentclass[main.tex]{subfiles}

\begin{document}
    These two examples have shown how funding costs have influenced 
    the values of shareholders' and creditors' claims.
    They clearly displayed the free rider problem that occurs
    since creditors reap the benefits of the project financed by the shareholders.
    This friction restricts which projects can be obtained by a firm;
    even positive net present value investments might deteriorate the shareholders' position,
    and will therefore not be obtained by the firm.
    The problem is even greater when financing with equity, 
    since the new shareholders' required rate of return is higher than new creditors'.
   
    The firm can use the breakeven price as a benchmark for projects. 
    If the price of the project is higher than the breakeven price, 
    the shareholders will lose wealth from obtaining it.
    Therefore, the breakeven price and an \FVA/ provide a mechanism for the firm
    to align its interest with its shareholders'.
    If the firm's dealers apply \FVA/s in their valuation,
    they will only obtain projects that, at a minimum, preserve the shareholders' wealth.
    
    The section has not covered the case of financing with existing cash
    since the main results of doing that can be summarised briefly here.
    If the project in question is a zero net present value investment, 
    there is no wealth transfer under cash funding,
    and all stakeholders' wealth are preserved.
    When receiving a donation from the counterparty, 
    some of the donation is obtained by the creditors, similar to the mechanism under debt funding.
    The same graphic for cash funding, as drawn for debt- and equity funding, can be found in Appendix
    \ref{apx:marginal-valuation-cash-funding}.
\end{document}
% !TEX root = sub-main.tex
\documentclass[main.tex]{subfiles}

\begin{document}
    These two examples have shown how funding costs have influenced 
    the value of shareholders' and creditors' claims.
    They clearly displayed the free rider problem that occurs
    since creditors reap the benefits of the project financed by the shareholders.
    This friction restricts which projects can be obtained by a firm;
    even positive net present value investments might deteriorate the shareholders' position,
    and will therefore not be obtained by the firm.
    The problem is even greater when financing with equity, 
    since the new shareholders' required rate of return is higher than new creditors'.
    
    The section have not covered the case of financing with existing cash
    since the main results of doing that can be summarized briefly here.
    If the project is a zero net present value investment, there is not wealth transfer,
    and all stakeholders' wealth are preserved.
    When receiving a donation from the counterparty, 
    some of the donation is obtained by the creditors, similar to the mechanism under debt funding.
    The same graphic for cash funding, as drawn for debt- and equity funding, can be found in 
    \cref{apx:marginal-valuation-cash-funding}.
\end{document}
% !TEX root = sub-main.tex
\documentclass[main.tex]{subfiles}

\begin{document}
    \subsection{Funding Secured Derivatives}
    \label{sec:example-secured-derivative}
        Consider again the risk-free contract that had a promised payoff of $1$ in every state.
        In this section, the firm sells the contract to a counterparty.
        Assume now that the firm and the counterparty have in place a CSA agreement
        that requires collateralization with no threshold or minimum transfer amount.
        Therefore, at time 0, when the counterparty pays the upfront price,
        the firm posts the same amount as collateral.

        In addition to the variation margin, 
        the CSA agreement requires an independent amount to be posted,
        corresponding to \qty[round-precision=0]{50}{\percent} of the purchase price at time 0.

        The independent amount will be a source of funding costs.
        Since the upfront price offsets the call for variation margin,
        neither the upfront price nor the variation margin will have any funding costs or benefits.
        The independent amount, on the other hand, has no offsetting cash flows, 
        and therefore requires financing.
        In this section it is assumed that the firm finances the independent amount by issuing new debt,
        and the issue raised by this setup is how the valuation of the derivative is affected.

        As previously, financing with debt issuance is obtained 
        such that the new creditors make a zero net present value investment.
        The new creditors' debt claim ranks pari passu with the existing debt.
        At time 0, the asset value increases by the upfront price,
        and by the debt obtained for the independent amount.
        At time 1 the asset value increases by these amounts forward discounted,
        since the quantities earn the risk-free rate when posted as collateral.

        The collateralization is more than enough to secure the payoff of the derivative;
        therefore, the counterparty cannot suffer losses on the contract.
        Assume then that the price of the derivative is given as the credit risk-free
        value, which is the discounted promised payoff:
            \begin{equation*}
                u = \num{-0.99}
            \end{equation*}
        Since the new debt is unsecured, 
        the firm must pay a credit spread to cover its own credit risk. 
        The deal is still a zero net present value investment for the new creditors.
        Hence, the present value of the new creditors claim, $\tilde{D}$, 
        must equal the borrowed amount, $-u \frac{1}{2}$,
        and the face value of the new debt must therefore solve:
            \begin{equation*}
                -
                u 
                \frac{1}{2} 
                = 
                \num{0.495}
                =
                \discountfactor
                \mathbb{E}^{\rnmeasure}[\tilde{D}]
            \end{equation*}
        Solving this numerically yields a face value of \num{0.515534}.
        The payoff structure at time 1 of the firm and its stakeholders
        is summarized in \cref{tbl:example-collateralized-derivative}.

        \begin{table}[H]
            \centering
            \begin{tabular}{l|rrrrr||r}
                $i$ & 1 & 2 & 3 & 4 & 5 & Present value \\
                \hline
                $A(\omega_{i})$ & 
                    \num{120.5} & \num{110.5} & \num{100.5} & \num{95.5} & \num{60.5} & \num{96.895} \\
                $D(\omega_{i})$ & 
                    \num{80} & \num{80} & \num{80} & \num{80} & \num{60.112624} & \num{76.813515} \\
                $E(\omega_{i})$ & 
                    \num{39.984466} & \num{29.984466} & \num{19.984466} & \num{14.984466} & \num{0} & \num{19.586485}\\
                $\tilde{D}(\omega_{i})$ & 
                    \num{0.515534} & \num{0.515534} & \num{0.515534} & \num{0.515534} & \num{0.387376} & \num{0.495}\\
                $Y_C(\omega_{i})$ & 
                    \num{-1} & \num{-1} & \num{-1} & \num{-1} & \num{-1} & \num{-0.99} \\
            \end{tabular}
            \caption{}
            \label{tbl:example-collateralized-derivative}
        \end{table}
        The loss rate given default and the new credit spread is respectively computed as:
            \begin{gather*}
                \phi(\omega_{5})
                = 
                    1 
                    - 
                    \frac{
                        \num{0.387376}+\num{60.112624}
                    }{
                        \num{0.515534}+\num{80}
                    }
                = 
                    \pct{0.248592}
                \\
                    \frac{
                        \num{80}    
                    }{
                        \num{76.813515} 
                    } 
                    - 
                    \grossrfrate 
                = 
                    \frac{
                        \num{0.515534}
                    }{        
                        \num{0.495}
                    } 
                    -
                    \grossrfrate 
                = 
                    \pct{0.031382} 
            \end{gather*}
        These have both clearly decreased compared to the pre-project values, 
        which suggests that the legacy creditors' position has improved.

        The collateralization has turned the funding implications around,
        such that selling the derivative no longer provides funding benefits.
        In the default state, all creditors share the independent amount,
        and the payoff of the legacy creditors increases. 
        Since the legacy creditors maintain their payoff in all other states,
        the total value of their claim increases.
        The shareholders are unable to capitalize on the upfront price,
        since that rests with the counterparty earning the risk-free rate.
        Instead, they bear the funding costs of obtaining unsecured funds 
        for financing the independent amount;
        hence, the value of the shareholders claim has decreased.
        The exact change of wealths for the legacy creditors as well as the shareholders are given by:
            \begin{align*}
                \num{76.813515} - \num{76.80} &= \num{0.013515}\\
                \num{19.586485} - \num{19.60} &= \num{-0.013515}
            \end{align*}
        The shareholders pay the funding costs of the project,
        and their welfare loss is transferred to the legacy creditors
        who enjoy the increase in asset value.

        These results were found numerically, 
        but can be verified by the marginal valuation equations
        if they are adapted slightly to the collateral postings.
        If the independent amount for an investment of size $q$ is denoted by $I(q)$,
        the marginal increase in the value of the firms equity per unit investment is given by:
            \begin{equation*}
                G_{\text{secured}} = 
                    \left.
                    \frac{
                        \partial 
                    }{
                        \partial 
                        q
                    }
                    \mathbb{E}^{\mathbb{Q}}\left[
                        \discountfactor 
                        \left(
                            A + qY - L - U(q)\grossrfrate - I(q) s(q))
                        \right)^{+}
                    \right] 
                    \right\rvert_{q=0} 
                    \nonumber
            \end{equation*}
        The upfront price $U(q)$ earns the risk free rate,
        while borrowing the independent amount costs $s(q)$.
        Using the same derivation as earlier this can be rewritten to:
            \begin{equation}
                G_{\text{secured}} 
                =
                    \mathbb{E}^{\mathbb{Q}}\left[\mathbbm{1}_{\mathcal{D}^{c}}\right] 
                    \left(
                        \discountfactor
                        \mathbb{E}^{\mathbb{Q}}\left[Y\right] 
                        - u
                    \right)
                    -
                    \discountfactor
                    \text{Cov}^{\mathbb{Q}}\left(\mathbbm{1}_{\mathcal{D}}, Y\right) 
                    - 
                    \mathbb{E}^{\mathbb{Q}}\left[
                        \mathbbm{1}_{\mathcal{D}^{c}}
                    \right] \discountfactor I S
                \label{eqn:marginal-shareholder-value-debt-financing-collateral}
            \end{equation}
        The last term might be referred to as the \textit{margin value adjustment} of the derivative;
        as the name implies, it accounts for the funding implications of the margin postings.
        In this paper it is simply considered a part of the \FVA/.

        If the independent amount is a share of the upfront price, 
        the wealth increase for shareholders in this setup is lower than
        if there was no collateralization. 
        If the independent amount is determined as $I=-\alpha u$ for some positive share $\alpha$, 
        then, comparing with $\Phi$ from \cref*{eqn:marginal-shareholder-value-debt-financing}:
        \begin{equation*}
            \mathbb{E}^{\mathbb{Q}}\left[
                \mathbbm{1}_{\mathcal{D}^{c}}
            \right] \discountfactor (-\alpha u) S
            >
            \Phi
        \end{equation*}
        if $u<0$.
        This leads to the, perhaps unsurprising, conclusion
        that shareholders are worse of when required to invest the upfront price at the risk-free rate
        and obtaining unsecured funding
        than if they could retire debt with the upfront price.

        Returning to the example at hand;
        \cref{eqn:marginal-shareholder-value-debt-financing-collateral} verifies the results
        by substituting in the survival probability, the independent amount, 
        and the actual credit spread after obtaining the project:
            \begin{equation*}
                G_{\text{secured}}
                =
                - 
                \discountfactor
                *
                \pct{0.878788}
                *
                \num{0.495}
                *
                \pct{0.031382}
                =
                -\num{0.013515}
            \end{equation*}
        In addition, the equation can help estimate the breakeven price.
        Substituting $I=-\frac{1}{2} u$ into 
        \cref{eqn:marginal-shareholder-value-debt-financing-collateral}, 
        setting equal to zero, and solving for $u$ yields:
            \begin{align*}
                0
                &=
                    \mathbb{E}^{\mathbb{Q}}\left[\mathbbm{1}_{\mathcal{D}^{c}}\right] 
                    \left(
                        \discountfactor
                        \mathbb{E}^{\mathbb{Q}}\left[Y\right] 
                        - u
                    \right)
                    -
                    \discountfactor
                    \text{Cov}^{\mathbb{Q}}\left(\mathbbm{1}_{\mathcal{D}}, Y\right) 
                    +
                    \mathbb{E}^{\mathbb{Q}}\left[
                        \mathbbm{1}_{\mathcal{D}^{c}}
                    \right] \discountfactor \frac{1}{2} u S
                    \nonumber \\
                u^{\ast}_{\text{secured}}
                &\equiv
                    \frac{
                        1
                    }{
                        \grossrfrate - \frac{1}{2} S
                    } 
                    \left(
                        \mathbb{E}^{\mathbb{Q}}\left[Y\right]
                        - \frac{
                            \text{Cov}^{\mathbb{Q}}\left(\mathbbm{1}_{\mathcal{D}}, Y\right)
                        }{
                            p^{\mathbb{Q}}  
                        } 
                    \right)
            \end{align*}
        Substituting the quantities from the example:
            \begin{equation*}
                u^{\ast}_{\text{secured}}
                =
                \num{-1.005622}
            \end{equation*}    
        So, for the shareholders to accept entering the project,
        the counterparty must be willing to donate a price increase of \num{0.015622}.
        
        Again, the shareholders are not willing to enter into the project 
        at the credit risk-free price.
        The shareholders bear the funding costs of financing the independent amount,
        and so they transfer wealth to the legacy creditors, 
        who reaps the benefits of entering the project.
        
        Even if the counterparty donated the amount lost by shareholders, \num{0.013515},
        the project would still be a losing trade for the shareholders.
        The donation to breakeven must be more than the amount lost,
        since the legacy creditors receive a share of the donation. 

        In conclusion, a strong CSA agreement requiring independent amounts, 
        might quickly turn a derivative with possible funding benefits into one with funding costs, 
        such that shareholders are no longer on board without price adjustments.
\end{document}
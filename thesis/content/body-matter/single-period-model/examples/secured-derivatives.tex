% !TEX root = sub-main.tex
\documentclass[../main.tex]{subfiles}

\begin{document}
    \subsection{Funding Secured Derivatives}
    \label{sec:example-secured-derivative}
        Now consider the same derivative contract as above which is sold by the firm to the counterparty.
        Assume now that the firm and the counterparty have in place a CSA agreement
        that requires collateralization with no threshold or minimum transfer amount.
        Therefore at time 0, when the counterparty pays the upfront price,
        the firm posts the same amount as collateral to the counterparty.

        In addition to the variation margin, 
        the CSA agreement requires an independent amount to be posted,
        corresponding to \qty[round-precision=0]{50}{\percent} of the purchase price at time 0.

        Since the upfront price equals offsets the call for variation margin,
        neither the upfront price nor the variation margin will have any funding costs or benefits.
        The independent amount, on the other hand, has no offsetting cash flows, 
        and therefore requires financing.
        In this section it is assumed that the firm finances the independent amount by issuing new debt,
        and the issue issue raised by this setup is how this affects the valuation of the derivative.
        
        % Hillion creates a portfolio where the option is bought by the counterparty and also sold to a hedger dealer. This will cause a wealth transfer that consist of both \MVA/ and \FCA/.

        As previously, financing for the independent amount is obtained 
        such that the new creditors make a zero net present value investment.
        The new creditors debt claim ranks pari passu with the existing debt.
        At time 0 the asset value increases by the upfront price,
        and by the debt obtained for the independent amount.
        At time 1 the asset value increases by these amounts forward discounted,
        since the quantities earn the risk-free rate when posted as collateral.

        The credit risk-free value of the derivative contract to the firm 
        is the discounted promised payoff and is given by:
        \begin{equation}
            \pi(Y) = \num{-2.345}
        \end{equation}
        The required collateralization there
        The premium of the derivative contract is computed as:

        To enter this project, the firm must post \qty[round-precision=0]{50}{\percent}
        of the price as initial margin as well as the total price in variation margin.
        The variation margin has higher priority than the existing debt, hence the derivative contract becomes risk-free.

        While the variation margin is funded by the premium of the derivative contract, the initial margin is funded by new debt. And since this debt is not secured, the firm must pay a credit spread to cover its own credit risk. The deal is still a zero net present value investment for the new creditors, meaning that the present value equals the initial margin paid at time 0:
        \begin{equation}
            \qty[round-precision=0]{50}{\percent} 
            \cdot 
            \pi(Y_C) = \mathbb{E}^{\rnmeasure}\left[\tilde{D}\right] = \num{1.1725}
        \end{equation}

        The payoff structure at time 1 of the firm and its stakeholders
        is summarized in \cref{tbl:example-collateralized-derivative}.

        \begin{table}[H]
            \centering
            \begin{tabular}{l|rrrrr||r}
                $i$ & 1 & 2 & 3 & 4 & 5 & Present value \\
                \hline
                $A(\omega_{i})$ & 
                    \num{123.553030} & \num{113.553030} & \num{103.553030} & \num{98.553030} & \num{63.553030} & \num{99.918} \\
                $D(\omega_{i})$ & 
                    \num{80} & \num{80} & \num{80} & \num{80} & \num{61.61472} & \num{76.993768} \\
                $E(\omega_{i})$ & 
                    \num{38.334747} & \num{28.834747} & \num{19.834747} & \num{15.834747} & \num{0} & \num{19.406232} \\
                $\tilde{D}(\omega_{i})$ & 
                    \num{1.218283} & \num{1.218283} & \num{1.218283} & \num{1.218283} & \num{0.938302} & \num{1.172502} \\
                $Y_C(\omega_{i})$ & 
                    \num{4} & \num{3.5} & \num{2.5} & \num{1.5} & \num{1} & \num{2.345} \\
            \end{tabular}
            \caption{}
            \label{tbl:example-collateralized-derivative}
        \end{table}

        As can be seen on \cref{tbl:example-collateralized-derivative}, the payoff of the legacy creditors at the default event is less than the pre-project value, while the payoff of the shareholders in the events of no default has also decreased. As the project is a zero net present value for the new creditors, there must be a wealth transfer from both the legacy creditors and the shareholders to the counterparty. Being the only party secured from the firm's credit risk, they receive the full derivative payable even in the event of default.

        The loss rate given default and the credit spread is respectively computed as:
            \begin{gather}
                \phi 
                = 
                    1 
                    - 
                    \frac{
                        \num{0.938302}+\num{61.61472}
                    }{
                        \num{1.218283}+\num{80}
                    }
                = 
                    \pct{0.229816}
                \\
                S 
                = 
                    \frac{
                        \num{80}    
                    }{
                        \num{76.993768} 
                    } 
                    - 
                    \grossrfrate 
                = 
                    \frac{
                        \num{1.218283}
                    }{        
                        \num{1.172502} 
                    } 
                    -
                    \grossrfrate 
                = 
                    \pct{0.028944} 
            \end{gather}
        These have both clearly decreased compared to the pre-project values, which suggest that the legacy creditors' position has improved.
        The change of wealth for the legacy creditors as well as the shareholders verify this and are given by:
        \begin{align}
            \Delta \pi(D) &= \num{76.993768} - \num{76.80} = \num{0.193768}\\
            \Delta \pi(E) &= \num{19.406232} - \num{19.60} = \num{-0.193768}
        \end{align}
        In case the variation margin is ignored from the setup, the counterparty would likewise suffer from the credit risk of the firm.
        The loss rate would then decrease,
        as the recovered total asset amount would have to be shared between the three parties on the liability side now ranking pari passu.
        The face value of the new debt would then decrease as well, as the credit spread would be reduced.

        To find the price that makes the shareholders indifferent to the project,
        a variable representing the donation is added to the price, i.e. the \MVA/.
        The donation is essentially the counterparty paying the firm more for the derivative contract than its market value.
        The problem is solved again including the valuation adjustment,
        such that the received premium on the derivative contract, $u$, is its present value plus an \MVA/:
        \begin{equation}
            u = \pi(Y_C) + \MVA/
        \end{equation}
        which is solved for the shareholder's point of breakeven:
        \begin{equation}
            \Delta \pi(E) = 0
        \end{equation}
        yielding $\MVA/ = \num{0.220082}$. So, for the shareholders to accept entering the project, the derivative premium must be at least $\num{2.345} + \num{0.220082} = \num{2.565082}$.
        The donation adds to the total asset value of the firm,
        which eventually makes it possible for the firm to give more back to the creditors in case of defaulting.
        This will give the firm a lower credit spread on the debt, as well as it will give the creditors a lower loss rate.

        \textcolor{red}{A comparison to the FVAs computed in the other sections could be interesting to discuss.}
        
\end{document}
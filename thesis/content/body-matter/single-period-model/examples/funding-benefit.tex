% !TEX root = sub-main.tex
\documentclass[../main.tex]{subfiles}

\begin{document}
    \subsection{Funding Benefits}
    \label{sec:example-funding-benefit}
        In the previous section, 
        the outgoing cash flows, due to the upfront cost of the derivative,
        demanded funding from the firm, which led to funding costs. 
        The firm would ultimately have to do a funding value adjustment,
        to account for the funding expenses that obtaining the derivative would lead to.
        On the contrary, when an instrument has cash flows going into the firm,
        it might have funding implications that are beneficial to the firm.
        If the firm is buying a derivative,
        funding value adjustments can both adjust the perceived value downwards due to funding costs,
        but just as well adjust the perceived value upwards due to funding benefits.
        Recall, from the introduction of \FVA/, the decomposition of \FVA/ into two elements,
        repeated here for convenience:
        \begin{equation}
            \FVA/ = \FCA/ + \FBA/
        \end{equation}
        In this section, the firm is assumed to enter a one-year derivative contract 
        with a counterparty that has a positive default probability, i.e. credit risk.
        Suppose the firm sells the simple derivative in \cref{tbl:single-period-simple-derivative-payoff} to the counterparty which implies a payoff at time 1 of $-Y$.
        The derivative is unsecured, and will now be treated as a new liability to the firm, 
        which ranks pari passu with the already existing debt.

        Since the counterparty buys the derivative from the credit-risky firm, 
        the counterparty must account for this exposure. 
        If the firm defaults before or at the derivative's expiry, the firm will not be able to meet the obligations, 
        and the potential profit from the unsecured derivative contract will not be fully paid out. 
        For the counterparty's incentives to be re-established, 
        the price of the derivative needs a valuation adjustment, i.e. $\CVA/_C$. 
        By symmetry, $\CVA/_C = \DVA/_F$, 
        where $\DVA/_F$ is the debit value adjustment made by the firm 
        in order to account for its own risk of default.\\
        For simplicity, it is assumed that the derivative contract 
        can only be of negative value to the firm, 
        such that its expected positive exposure is zero, from the firm's point of view.
        For this reason, no adjustment for the credit risk of the counterparty is performed and 
        $\CVA/_F = \DVA/_C = 0$.
        The theoretical market price of the derivative should therefore be given by:
        \begin{equation}\label{eqn:option-value-credit-risk}
            P = P_{rf} - \CVA/_C = P_{rf} - \DVA/_F
        \end{equation}
        where $P_{rf}$ is the price of the derivative without any risk considered.

        Now consider the funding issue. On the transaction date of the derivative, 
        the firm receives premium from the counterparty that must be invested, 
        which will deem beneficial in the funding frame of mind. 
        This section will study three possibilities of funding benefits, 
        as suggested by \textcite{Hillion2016}.
        (i): the cash is assumed to be used to retire the existing debt 
        that trades at a credit spread over the risk-free rate. 
        As mentioned, the derivative contract is considered a liability to the firm
        that ranks pari passu with other debt, 
        and the issue to determine is whether the firm is willing to lower the derivative premium 
        compensating for the funding benefits. 
        (ii): the cash from the derivative premium is assumed to buy back equity. 
        The case examines the impact of the derivative value as well as the firm's balance sheet. 
        (iii): the cash is assumed to be invested in riskless securities. 
        This is also called the asymmetric funding case where shortfalls of cash are funded by new debt, 
        and the surpluses of cash are invested at the risk-free rate.

        \subsubsection{Retiring Legacy Debt}
            Suppose that when the firm receives the bond premium at time 0, 
            they immediately invest their surplus of cash by retiring some of their already existing debt.
            
            This section will display how the described procedure will lead to the \DVA/
            accounting for all funding benefits and eliminating the need for an \FVA/.
            Before seeing this from calculations, it can be argued conceptually
            and the reasons for that can be highlighted.
            By the assumptions in these examples,
            the only source of the firm's credit spread is the firm's credit risk, 
            which implies that there is no difference between the CDS spread referencing the firm,
            and the rate that the firm pays for funding.
            The CDS spread determines the cost to the counterparty 
            of hedging its credit exposure to the firm.
            The funding rate determines the benefits to the firm of cash surpluses.
            Therefore there is a strong relation between the two adjustments.
            Having hinted the possible concerns in the current setup, 
            the example can continue and the issues can be quantified.

            Notice that when the upfront cost is used to purchase debt at a fair market value, 
            a new liability, the bond, is replacing a part of the existing liability, the legacy debt. 
            Remembering the pari passu assumption, this debt-for-debt substitution 
            leaves the total liability unchanged.

            The implications of debt buyback are the following.
            First, the net present value for the creditors who tender is zero. 
            Hence, the loss rate as well as the credit spread has to be left unchanged
            after partly retiring the debt, otherwise the creditors would not tender. 
            Second, the counterparty charges for the firm's credit risk 
            to the extend that it is a zero net present value for the counterparty. 
            The cash flow following the bond can be viewed as newly issued debt
            with a fair credit spread offered to the creditors.
            Third, the remaining creditors stay unaffected of the transaction, 
            as the loss rate and credit spread is left intact. 
            Since the bond is a zero net present value for the firm, 
            the counterparty, and the creditors, there is no wealth to transfer anywhere,
            and the contract must also have zero net present value to the shareholders. 
            Hence, the bond fair value must be given by \cref{eqn:option-value-credit-risk} 
            as any other value would influence the wealth of the parties involved.

            Assume now that the firm sells the bond a counterparty
            for an upfront price of \num{0.96},
            which is spent on retiring debt from the firm's creditors. 
            The bond payable is now a liability to the firm, 
            ranking pari passu with the existing debt.
            Recall that the pre-project face value of the debt were \num{80}. 
            The payoffs to the firm's stakeholders after selling the bond 
            are shown in \cref{tbl:example-debt-retiring},
            where the payoff to the counterparty is denoted by $Y_{C}$. 
            The loss rate in the default state remains unchanged 
            and is identical for the creditors and the counterparty
            $\phi(\omega_{5})=1-\num{60}/\num{80}=\pct{0.25}$. 
            Similarly, the credit spread remains the same as the pre-project value
            $\num{79}/\num{75.84}-R=\pct{0.031566}$.
            
            \begin{table}[H]
                \centering
                \begin{tabular}{l|rrrrr||r}
                    $i$ & 1 & 2 & 3 & 4 & 5 & Present value \\
                    \hline
                    $A(\omega_{i})$ & $\num{120}$ & $\num{110}$ & $\num{100}$ & $\num{95}$ & $\num{60}$ & $\num{96.40}$ \\
                    $D(\omega_{i})$ & $\num{79}$ & $\num{79}$ & $\num{79}$ & $\num{79}$ & $\num{59.25}$ & $\num{75.84}$ \\
                    $E(\omega_{i})$ & $\num{40}$ & $\num{30}$ & $\num{20}$ & $\num{15}$ & $\num{0}$ & $\num{19.60}$ \\
                    $Y_C(\omega_{i})$ & $\num{1}$ & $\num{1}$ & $\num{1}$ & $\num{1}$ & $\num{0.75}$ & $\num{0.96}$ \\
                \end{tabular}
                \caption{}
                \label{tbl:example-debt-retiring}
            \end{table}

            The liability side of the firm's balance sheet will then look like:

            \begin{table}[H]
                \centering
                \begin{tabular}{l|c|c|c}
                     & \textbf{Book Value} & \textbf{\DVA/} & \textbf{Market Value} \\
                    \hline
                    Equity & $\num{19.60}$ & $\num{0}$ & $\num{19.60}$\\
                    Debt & $\num{79.20}$ & $\num{3.36}$ & $\num{75.84}$\\
                    Derivative Payable & $\num{0.99}$ & $\num{0.03}$ & $\num{0.96}$\\
                    \hline
                    Total & $\num{99.79}$ & $\num{3.39}$ & $\num{96.40}$
                \end{tabular}
            \end{table}

            The transaction is a zero net present value investment for the shareholders. 
            As can be seen in \cref{tbl:example-debt-retiring}, 
            the shareholders receive a payoff at time 1 equal to the pre-project value.
            The counterparty's payoff coming from the bond, $Y_C$, has a market value of:
            \begin{equation*}
                \pi(Y_{C}) = \discountfactor \mathbb{E}^{\rnmeasure}\left[Y_{C}\right] = \num{0.96}
            \end{equation*}
            This concludes that the bond premium that the firm receives at time 0 is equal to $\num{0.96}$, 
            which is then the amount used for retiring debt. 
            On top of this, the amount can be recognized 
            as the price of the riskless bond adjusted for the firm's credit risk:
            \begin{equation*}
                V = \discountfactor * 1 - \DVA/_F = \num{0.99} - \num{0.03} = \num{0.96}
            \end{equation*}
            which validates \cref{eqn:option-value-credit-risk}. 

            All funding benefits have already been accounted for by the \DVA/;
            therefore, there is no \FVA/ to make.
            Take the first possible definition of \FVA/, 
            where it was defined as the promised excess funding cost.
            Calculating this quantity yields 
            $\discountfactor u S = \num{0.99} * \num{-0.96} * \pct{0.031566} = \num{-0.03}$,
            which is exactly equal to the \DVA/ with the sign flipped.
            Recall that \textcite{HullWhite2012FVA} use this exact argument to argue against \FVA/s.
            As hinted in the beginning, 
            this is not a coincidence as the CDS spread determining \DVA/
            is identical to the funding rate determining \FVA/.
            Adjusting the \DVA/-adjusted price with the first definition of \FVA/ would be double counting,
            and would lead to the firm undervaluing the bond.
            
            Consider also the last, and chosen, definition of \FVA/,
            where it was defined as the adjustment needed to the shareholders' breakeven.
            As already mentioned, the marginal shareholder value of the bond is zero,
            meaning that the price is already adjusted to the shareholders' breakeven.
            The fact is that the bond in this example needs no \FVA/,
            as the \DVA/ accounts for all funding benefits.
            Differences between \DVA/ and an adjustment for funding benefits can only occur 
            either because of a mismatch between the CDS spread and the funding spread
            or because of a difference in the portfolio on which the two quantities are calculated. 
            The former has been assumed non-existing.
            The latter could occur if the firm had offsetting trades in its portfolio,
            since the \FVA/ takes into account all sorts of netting benefits.
            In this example there is no offsetting trades, 
            which is why there is no need for an \FVA/, as the \DVA/ have already accounted for everything.

        \subsubsection{Equity buyback}
            Assume that the firm enters a project 
            by selling the same derivative contract with a known payoff at time 1 of 1. 
            The derivative premium received at time 0 is now used for buying back equity from the shareholders. 
            The example from above is used again to address the impacts of this issue.

            Instead of swapping some existing debt for a derivative payable, 
            the derivative value now adds to the aggregate liabilities of the firm, 
            and ranks pari passu with the existing creditors.
            This section studies the beneficial impacts on the firm's shareholders of doing this transaction.
            In the state of default, the firm has already at time 0 allocated some cash to the shareholders, 
            leaving a smaller amount to the creditors compared to the pre-derivative capital structure, 
            as the loss rate increases.

            The firm's balance sheet is left unchanged, meaning that the sum of the market values of debt, equity, 
            and derivative payable is the same after the transaction. 
            The values of the different parties are shown in \cref{tbl:example-equity-buyback}. 
            The loss rate increases from the pre-derivative value and is computed as:
            $\phi = 1 - \num{60}/\num{81} = \pct{0.25925926}$.
            Similarly, the credit spread increases to
            $\num{80}/\num{76.711111} - \grossrfrate = \pct{0.032772686}$.
            \begin{table}[H]
                \centering
                \begin{tabular}{l|rrrrr||r}
                    $i$ & 1 & 2 & 3 & 4 & 5 & Present value \\
                    \hline
                    $A(\omega_{i})$ & $\num{120}$ & $\num{110}$ & $\num{100}$ & $\num{95}$ & $\num{60}$ & $\num{96.4}$ \\
                    $D(\omega_{i})$ & $\num{80}$ & $\num{80}$ & $\num{80}$ & $\num{80}$ & $\num{59.259259}$ & $\num{76.711111}$ \\
                    $E(\omega_{i})$ & $\num{39}$ & $\num{29}$ & $\num{19}$ & $\num{14}$ & $\num{0}$ & $\num{18.73}$ \\
                    $Y_C(\omega_{i})$ & $\num{1}$ & $\num{1}$ & $\num{1}$ & $\num{1}$ & $\num{0.74074074}$ & $\num{0.95888889}$ \\
                \end{tabular}
                \caption{}
                \label{tbl:example-equity-buyback}
            \end{table}

            It is assumed that the firm receives the credit risk adjusted market price,
            i.e. $u=\num{-0.95888889}$, as the derivative premium at time 0, which is immediately spend on equity buyback.

            Buying back equity leaves two payoffs for the shareholders:
            (i) the immediate payoff corresponding to the derivative premium, and
            (ii) the present value of the expected payoff at time 1.
            To establish whether entering into this project is attractive for the shareholders,
            the expression for the marginal valuation of equity funding, \cref{eqn:marginal-shareholder-value-equity-financing}, is used.
            The payoff for the firm is here the negative derivative payoff, i.e. $Y$.
            The survival probability is in this constructed framework given by $p^{\rnmeasure} = \sum_{i=1}^{4} \psi_{i} * \grossrfrate = \pct{0.878788}$, and the covariance between the indicator of default states and $Y$ is 0 as the promised payoff is known with certainty.
            The change of wealth for the shareholders is then calculated as:
            \begin{equation}
                G_{\text{equity}} = \num{0.088888889}
            \end{equation}
            
            As the price of the derivative is assumed to be the credit risk adjusted market value,
            the derivative is a zero net present value for the counterparty.
            The positive marginal valuation for the shareholders then implies a wealth transfer from the creditors to the shareholders.
            Hence, the derivative contract is a positive net present value investment for the shareholders, 
            where the change of wealth is verified by the change in the present value of the expected payoff:
            \begin{equation}
                \Delta \pi(E) = (\num{18.73} + \num{0.95888889}) - \num{19.6} = \num{0.088888889}
            \end{equation}
            On the other hand, the transaction is a negative net present value investment for the creditors:
            \begin{equation}
                \Delta \pi(D) = \num{76.711111} - \num{76.8} = \num{-0.088888889}
            \end{equation}

            If the firm is only interested in maintaining its shareholders wealth,
            it will be willing to accept an even lower price than the credit risk adjusted market price.
            The lowest price the firm will accept, is the breakeven price,
            which, under equity financing, can be computed from 
            \cref{eqn:shareholders-breakeven-equity-financing}:
            \begin{equation}
                u_{\text{equity}}^{\ast} =
                - \discountfactor
                \pct{0.878788}
                = \num{-0.87}
            \end{equation}
            which suggest an \FVA/ given by:
            \begin{equation}
                \FVA/ =
                u_{\text{equity}}^{\ast} - u =
                \num{-0.87} - \left(\num{-0.95888889}\right) =
                \num{0.088888889}
            \end{equation}
            The positive valuation adjustment suggests an overall funding benefit.
            
            Similar to the case of buying the derivative by issuing new shares analysed in \cref{sec:example-equity-issuance},
            the \FVA/ amount is equal to the wealth transfer.
            Evidently, the price cut from the firm is translated directly to a wealth loss
            for the shareholders while the creditors wealth is maintained. 
            The donation from the firm to the counterparty does not affect the total asset value;
            therefore, the loss rate on the debt and the derivative payable remains unchanged,
            and so does the present value of the creditors' and counterparty's payoffs.
            \\
            The creditors still lose wealth worth $\Delta \pi(D)$ from the firm obtaining the project.
            The shareholders are however willing to receive a lower price of the project than $\num{0.95888889}$.
            For all prices down to $\num{0.87}$,
            the shareholders will still consider the transaction a non-negative net present value investment.

            In conclusion, when financing with equity issuance or buybacks,
            the price of the project is irrelevant to the creditors,
            as price adjustments are transferred directly to shareholders.
            The wealth of the creditors is solely dependent on the payoff structure of the project;
            Their wealth increases when the asset value increases and vice versa.

        \subsubsection{No funding buyback}
            Now suppose, instead of buying back debt or equity, 
            the firm spends the cash proceeds from the derivative premium in riskless assets. 
            If the derivative traded with the counterparty has a known payoff at time 1, this procedure is virtually the same as if the firm borrows money to buy the riskless asset 
            as described in \cref{sec:example-risk-free-project-debt-issuance}. 
            To deviate from the exact same conclusions, the derivative is now assumed to have a payoff structure at time 1 
            as described in \cref{tbl:risky-option-payoff}. 
            The derivative will still be fairly priced, i.e. the price is the discounted expected payoff for the counterparty. 
            The credit risk still needs to be accounted for in the price. 
            At time 1 the firm will pay the counterparty accordingly, hence the derivative payable, $Y_C$, 
            is still a contingent liability to the firm and ranks pari passu with the already existing debt.
            \\
            This section analyses whether this capital structure setup causes reasons for funding benefits.
            Intuitively, this could seem appropriate as the firm receives cash proceeds from the upfront price that can be invested beneficially.
            
            \begin{table}[H]
                \centering
                \begin{tabular}{l|rrrrr}
                    $i$ & 1 & 2 & 3 & 4 & 5 \\
                    \hline
                    $Y(\omega_{i})$ & $\num{-4}$ & $\num{-3.5}$ & $\num{-2.5}$ & $\num{-1.5}$ & $\num{-1}$ \\
                \end{tabular}
                \caption{}
                \label{tbl:risky-option-payoff}
            \end{table}

            After receiving the derivative premium in the form of cash, the firm immediately invests in the risk-free asset. 
            This implies that the firm's total asset value increases with an amount equal to the discounted derivative premium. 
            The capital structure, after this transaction, is described in \cref{tbl:example-no-funding-buyback}, 
            where the derivative theoretical fair value price and loss rate are found
            by solving the following equation system for $\pi(Y_C)$ and $\phi$:
            \begin{align}
                \pi(Y_C) &= \discountfactor\mathbb{E}^{\rnmeasure}\left[Y_{C}\right]
                \label{eqn:no-funding-buyback-derivative-fair-value}
                \\
                A(\omega_5) &= D(\omega_5) + Y_C(\omega_5)
                \label{eqn:no-funding-buyback-default-state}
            \end{align}
            with \cref{eqn:no-funding-buyback-default-state} making sure that the creditors and derivative payable
            takes the total asset value amount in case of default, i.e. state $\omega_5$.

            \begin{table}[H]
                \centering
                \begin{tabular}{l|rrrrr||r}
                    $i$ & 1 & 2 & 3 & 4 & 5 & Present value \\
                    \hline
                    $A(\omega_{i})$ & $\num{122.3408}$ & $\num{112.3408}$ & $\num{102.3408}$ & $\num{97.3408}$ & $\num{62.3408}$ & $\num{98.71736}$ \\
                    $D(\omega_{i})$ & $\num{80}$ & $\num{80}$ & $\num{80}$ & $\num{80}$ & $\num{61.5711}$ & $\num{76.988534}$ \\
                    $E(\omega_{i})$ & $38.3408$ & $\num{28.8408}$ & $\num{19.8408}$ & $\num{15.8408}$ & $\num{0}$ & $\num{19.411469}$ \\
                    $Y_C(\omega_{i})$ & $\num{4}$ & $\num{3.5}$ & $\num{2.5}$ & $\num{1.5}$ & $\num{0.769639}$ & $\num{2.3173567}$ \\
                \end{tabular}
                \caption{}
                \label{tbl:example-no-funding-buyback}
            \end{table}

            The loss rate obtained from solving the above problem is $\phi = \pct{0.230361}$, 
            and the new credit spread is then 
            $\num{80}/\num{76.988534} - \grossrfrate = \pct{0.0290148}$.
            Compare the new credit spread with the pre-project limiting spread of \pct{0.031566};
            by obtaining the new project, the credit spread has decreased quite significantly.
            The reduction can be explained by the project's payoff structure,
            as the states where the firm's asset value is the highest are
            also the states where the project's payoff is the highest and vice versa. 
            Therefore, by selling the project, the firm has reduced its riskiness
            as well as the loss rate and credit spread of the creditors.

            The market value of the derivative including adjustments for credit risk would deem a fair price of the project,
            as it would be a zero net present value investment for the counterparty.
            With such pricing, however, a transfer of wealth will occur between the shareholders and the creditors.
            The firm's incentives of not decreasing the shareholders' wealth
            makes an argument for a valuation adjustment of the derivative.
            Selling a derivative and using the sales proceeds for buying the risk-free asset,
            is equivalent, but with opposite signs,
            to buying a derivative and funding the upfront price with existing cash.
            In the latter case, the existing cash could have earned the risk-free rate
            and it has an opportunity cost,
            but in the former case the sales proceeds actually earn the risk-free rate.
            This is taken care of by \cref{eqn:marginal-shareholder-value-cash-financing},
            which defines the marginal change in the shareholders' wealth using existing cash financing.
            \\
            The marginal change in the equity valuation with $u=\num{-2.3173567}$ is therefore:
            \begin{equation}
                G_{\text{cash}} = p^{\rnmeasure}\pi
                -\discountfactor \text{Cov}^{\rnmeasure}\left(\mathbbm{1}_{\mathcal{D}},Y\right) = \num{-0.188534}
            \end{equation}

            The change in wealth for the shareholders can be verified by the difference between the new present value of the equity and the old present value of the equity:
            \begin{equation}
                \Delta \pi(E) = \num{19.411469} - \num{19.6} = \num{-0.188531}
            \end{equation}
            The change in wealth for the creditors is similarly verified by:
            \begin{equation}
                \Delta \pi(D) = \num{76.988534} - \num{76.8} = \num{0.188534}
            \end{equation}
            The increment happens as the risk-free asset, which the firm invested in with the derivative premium, 
            makes the asset value increase, and a larger total amount is recovered in case of default. 
            In the cases of no default, the creditors are not affected by the new project.
            \\
            Hence, this is a negative net present value for the shareholders 
            and a positive net present value for the creditors. 
            Due to the risk in the derivative, the shareholders do have a chance to gain from the project, 
            where in state $\omega_4$, the derivative payable is less than the return on the riskless asset.

            On the basis of the wealth transfer there seems to be some funding implications.
            The \FVA/ is, as always, the adjustment to the project's apparent value to the firm that makes the shareholders indifferent to taking on the project.
            As the firm is funding the risk-free project with the premium proceeds, no funding benefits actually occur, and the \FVA/ is constituted entirely by the funding costs from paying the derivative payable, i.e. the \FCA/.
            \\
            The valuation adjustment suggests a higher derivative premium than the market value.
            Specifically, the theoretical premium to be paid by the counterparty would be $u_{\text{cash}}^{\ast}$.
            However, whether the counterparty is willing to pay the adjusted price still remains indeterminate.
            \\
            Using \cref{eqn:shareholdes-breakeven-cash-financing}, which describes the shareholders' price of breakeven with cash financing, the adjusted price becomes:
            \begin{equation}
                u_{\text{cash}}^{\ast} = \discountfactor \left(
                    \mathbb{E}^{\rnmeasure}\left[Y\right] -
                    \frac{\text{Cov}^{\rnmeasure}\left(\mathbbm{1}_{\mathcal{D}},Y\right)}
                    {p^{\rnmeasure}}
                \right)
                = \num{2.5318966}
            \end{equation}
            yielding a valuation adjustment of:
            \begin{equation}
                \FVA/ = u_{\text{cash}}^{\ast} - u = \num{2.5318966} - \num{2.3173567} = \num{0.2145446}
            \end{equation}

            Alternatively, this value can be computed by solving the related equation system for \FCA/ as described above, where the value to the firm is the funding  cost adjusted market value:
            \begin{align}
                V &= \pi(Y_C) + \FVA/
                \label{eqn:no-funding-buyback-derivative-adjusted-value}
                \\
                A(\omega_5) &= D(\omega_5) + Y_C(\omega_5)
                \\
                \Delta \pi(E) &= 0
                \label{eqn:no-wealth-transfer-no-funding-buyback}
            \end{align}

            The adjusted derivative value to the firm, \cref{eqn:no-funding-buyback-derivative-adjusted-value},
            will replace the fair market value pricing assumption, \cref{eqn:no-funding-buyback-derivative-fair-value}, in the new setup.
            Furthermore, the condition of no wealth transfer from the shareholders, \cref{eqn:no-wealth-transfer-no-funding-buyback}, is added to the equation system.
            The new problem solved yields a loss rate of $\phi = \pct{0.22768554}$,
            which is slightly lower,
            and the donation needed from the counterparty to make the project a zero net present value for the shareholders is $\FVA/ = \num{0.2145446}$.
\end{document}
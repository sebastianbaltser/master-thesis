% !TEX root = sub-main.tex
\documentclass[../main.tex]{subfiles}

\begin{document}
    \subsection{Funding Benefits}
    \label{sec:example-funding-benefit}
        In some instances the value adjustment coming from funding can have beneficial outcomes for the firm. 
        Funding value adjustment consist of the both the aforementioned funding cost as well as funding benefits. 
        A portfolio can include instances of both cases, such that the \FVA/ for a portfolio is given by:
        \begin{equation}
            \text{\FVA/} = \text{\FVA/}_{\text{cost}} + \text{\FVA/}_{\text{benefit}}
        \end{equation}
        In this section, the firm is assumed to enter a one-year derivative contract 
        with different counterparty
        which now has credit risk.
        Suppose the firm sells the derivative to the counterparty. 
        The derivative is unsecured, and will now be treated as a new liability to the firm, 
        which ranks pari passu with the already existing debt.

        Since the counterparty buys the derivative from the credit-risky firm, 
        the counterparty must account for this exposure. 
        If the firm defaults before or at the derivative's expiry, the firm will not be able to meet the obligations, 
        and the potential profit from the unsecured derivative contract will not be fully paid out. 
        For the counterparty's incentives to be re-established, 
        the price of the derivative needs a valuation adjustment, i.e. $\CVA/_C$. 
        By symmetry, $\CVA/_C = \DVA/_F$, 
        where $\text{\DVA/}(F)$ is the debit value adjustment made by the firm 
        in order to account for its own risk of default.\\
        The profile of the constructed derivative contract is assumed to make the firm having no exposure to the counterparty,
        and, as it's a liability to the firm, 
        no adjustment for the credit risk of the counterparty is performed. 
        This also implies that $\text{\CVA/}(F) = \text{\DVA/}(C) = 0$, 
        and, by initially ignoring the funding issue, the value of the derivative contract, denoted by $V$, is given by:
        \begin{equation}\label{eqn:option-value-credit-risk}
            V = V_{rf} - \text{\CVA/}(C) = V_{rf} - \text{\DVA/}(F)
        \end{equation}
        where $V_{rf}$ is the value of the derivative without any risk considered.

        Now consider the funding issue. On the transaction date of the derivative, 
        the firm receives premium from the counterparty that must be invested, 
        which will deem beneficial in the funding frame of mind. 
        This section will study three possibilities of funding benefits, 
        as suggested by \textcite{Hillion2016}.
        (i): the cash is assumed to be used to retire the existing debt 
        that trades at a credit spread over the risk-free rate. 
        As mentioned, the derivative contract is considered a liability to the firm
        that ranks pari passu with other debt, 
        and the issue to determine is whether the firm is willing to lower the derivative premium 
        compensating for the funding benefits. 
        (ii): the cash from the derivative premium is assumed to buy back equity. 
        The case examines the impact of the derivative value as well as the firm's balance sheet. 
        (iii): the cash is assumed to be invested in riskless securities. 
        This is also called the asymmetric funding case where shortfalls of cash are funded by new debt, 
        and the surpluses of cash are invested at the risk-free rate.

        \subsubsection{Debt retiring}
            Suppose when the firm receives the derivative premium at time 0, 
            they immediately invest their surplus of cash by retiring some of their already existing debt. 
            The value to the firm will arguably be lower when including this funding benefit, 
            and one might rewrite \cref{eqn:option-value-credit-risk} as:
            \begin{equation}
                V = V_{rf} - \text{\CVA/}(C) - \text{\FVA/} = V_{rf} - \text{\CVA/}(C) - \text{\FVA/}_{\text{benefit}}
            \end{equation}
            where a riskless firm would have no benefit of \FVA/ as it would be able to borrow at the risk-free rate.
            
            The issue with this reformulation is that both adjustments are driven by the firm's credit risk. 
            This will cause an imbalanced valuation in the sense that the two values overlap, 
            and the Modigliani-Miller invariance Proposition will be violated. 
            Notice that when the cash raised by selling the OTC derivative contract 
            is used to purchase debt at a fair market value, a new liability is replacing a part of the existing liability. 
            Remembering the pari passu assumption, this debt-for-debt swap leaves the total liability unchanged.

            The implications of debt buyback are: 
            First, the net present value of the creditors who tender is zero. 
            This means that the loss rate as well as the credit spread has to be left unchanged
            after partly retiring the debt, otherwise the creditors would not tender. 
            Second, the counterparty charges for the firm's credit risk 
            to the extend that it's a zero net present value for the counterparty. 
            The cash flow following the derivative contract can be viewed as newly issued debt
            with a fair credit spread offered to the creditors.
            Third, the remaining creditors stay unaffected of the transaction, 
            as the loss rate and credit spread is left intact. 
            Since the derivative contract is a zero net present value for the firm, 
            and no wealth is transferred at the neither the counterparty nor any of the creditors, 
            so should it be for the shareholders. 
            So, the derivative fair value must be given by \cref{eqn:option-value-credit-risk} 
            as any other value would influence the wealth of the parties involved.

            Returning to the previous example of a risky firm with a capital structure
            as described in \cref{tbl:example-firm-structure}, 
            the firm now enters a derivative contract by selling a project to a counterparty
            with a credit value adjusted present value of 1. 
            The derivative premium is priced with no discount 
            and is immediately spent on retiring debt from its already existing creditors. 
            The derivative payable is now a liability to the firm, ranking pari passu with the existing debt, 
            which before the derivative contract had a face value of 80. 
            For simplicity, we assume, as in \cref{sec:risk-free-project},
            that the derivative payable is known with certainty, hence it's a risk-free project. 
            The firm's balance sheet isn't affected by the transaction. 
            The change happens on the liability side, where the debt is partly swapped with a derivative payable.\\
            The values of the different parties are shown in \cref{tbl:example-debt-retiring}. 
            The loss rate remains unchanged and is the same for the debt and the counterparty: $\phi=1-60/80=25\%$. 
            Similarly, the credit spread remains the same as the pre-project value: $S=79/75.84-r_{f}-1=3.157\%$.
            
            \begin{table}[H]
                \centering
                \begin{tabular}{l|rrrrr||r}
                    $i$ & 1 & 2 & 3 & 4 & 5 & Present value \\
                    \hline
                    $A(\omega_{i})$ & $120$ & $110$ & $100$ & $95$ & $60$ & $96.40$ \\
                    $D(\omega_{i})$ & $79$ & $79$ & $79$ & $79$ & $59.25$ & $75.84$ \\
                    $S(\omega_{i})$ & $40$ & $30$ & $20$ & $15$ & $0$ & $19.60$ \\
                    $Y_C(\omega_{i})$ & $1$ & $1$ & $1$ & $1$ & $0.75$ & $0.96$ \\
                \end{tabular}
                \caption{}
                \label{tbl:example-debt-retiring}
            \end{table}

            The liability side of the firm's balance sheet will then look like:

            \begin{table}[H]
                \centering
                \begin{tabular}{l|c|c|c}
                     & \textbf{Book Value} & \textbf{\DVA/} & \textbf{Market Value} \\
                    \hline
                    Equity & $19.60$ & $0$ & $19.60$\\
                    Debt & $79.20$ & $3.36$ & $75.84$\\
                    Derivative Payable & $0.99$ & $0.03$ & $0.96$\\
                    \hline
                    Total & 99.79 & 3.39 & 96.40
                \end{tabular}
            \end{table}

            The transaction is a zero net present value investment for the shareholders. 
            As can be seen in \cref{tbl:example-debt-retiring}, 
            the shareholders receive a payoff at time 1 equal to the pre-derivative value.

            The counterparty's payoff coming from the derivative, $Y_C$, has a market value of:
            \begin{equation}
                \pi(Y_{C}) = \sum_i \psi_i Y_{C,i} = 0.96
            \end{equation}
            This concludes that the derivative premium that the firm receives at time 0 is equal to 0.96, 
            which is then the amount used for retiring debt. 
            On top of this, the amount is recognized as the difference between the riskless project value and the \DVA/:
            \begin{equation}
                V = V_{rf} - \text{\DVA/}(F) = 0.99 - 0.03 = 0.96
            \end{equation}
            which validates \cref{eqn:option-value-credit-risk}. 
            Including an adjustment from the benefits of funding 
            would lead to a double counting effect due to two offsetting effects. 
            (i): By retiring debt with the cash coming from the derivative premium at time 0, 
            the firm will receive a funding benefit in the sense that it saves some interest expenses. 
            (ii): However, the debt retiring decreases the \DVA/ of the firm's existing debt 
            by exactly $\text{\FVA/}_{\text{benefit}}$ amount.

        \subsubsection{Equity buyback}
            Assume that the firm enters a project 
            by selling the same derivative contract with a known payoff at time 1 of 1. 
            The derivative premium at time 0 is now used for buying back equity from the shareholders. 
            The example from above is used again to address the impacts of this issue.

            Instead of swapping some existing debt for a derivative payable, 
            the derivative value now adds to the aggregate liabilities of the firm, 
            and ranks pari passu with the existing creditors. 
            In the state of default, the firm has already at time 0 allocated some cash to the shareholders, 
            leaving a smaller amount to the creditors compared to the pre-derivative capital structure, 
            as the loss rate increases.\\
            The firm's balance sheet is left unchanged, meaning that the sum of the market values of debt, equity, 
            and derivative payable is the same after the transaction. 
            The values of the different parties are shown in \cref{tbl:example-equity-buyback}. 
            The loss rate increases and is computed as: $\phi = 1 - 60/81 = 25.93\%$. 
            Similarly, the credit spread increases from the pre-project value of 3.157\%: $S = 80/76.711 - \grossrfrate = 3.28\%$.
            \begin{table}[H]
                \centering
                \begin{tabular}{l|rrrrr||r}
                    $i$ & 1 & 2 & 3 & 4 & 5 & Present value \\
                    \hline
                    $A(\omega_{i})$ & $120$ & $110$ & $100$ & $95$ & $60$ & $96.400$ \\
                    $D(\omega_{i})$ & $80$ & $80$ & $80$ & $80$ & $59.259$ & $76.711$ \\
                    $S(\omega_{i})$ & $39$ & $29$ & $19$ & $14$ & $0$ & $18.730$ \\
                    $Y_C(\omega_{i})$ & $1$ & $1$ & $1$ & $1$ & $0.741$ & $0.959$ \\
                \end{tabular}
                \caption{}
                \label{tbl:example-equity-buyback}
            \end{table}

            The firm receives 0.959 as the derivative premium at time 0, which is immediately spend on equity buyback. 
            The derivative contract is a negative net present value for the for the creditors, 
            where the change of wealth is computed by:
            \begin{equation}
                \Delta \pi(D) = 76.711 - 76.800 = -0.089
            \end{equation}
            On the other hand, the transaction is a positive net present value for the shareholders:
            \begin{equation}
                \Delta \pi(S) = (18.730 + 0.959) - 19.600 = 0.089
            \end{equation}
            meaning there is a wealth transfer from the creditors to the shareholders.

            The derivative is a zero net present value for the counterparty. 
            The price of the derivative is lower compared to the case of debt retiring, 
            which is explained by the higher loss rate in case of default. 
            The only valuation adjustment to the price is still the \CVA/ charged by the counterparty, 
            which again validates \cref{eqn:option-value-credit-risk}.

        \subsubsection{No funding buyback}
            Now suppose, instead of buying back debt or equity, 
            the firm spends the cash proceeds from the derivative premium in riskless assets. 
            If the derivative traded with the counterparty has a known payoff at time 1, this procedure is exactly the same as if the firm borrows money to buy the riskless asset 
            as described in \cref{sec:example-risk-free-project-debt-issuance}. 
            To deviate from the exact same conclusions, the derivative is now assumed to have a payoff structure at time 1 
            as described in \cref{tbl:risky-option-payoff}. 
            The derivative will still be fairly priced, i.e. the price is the discounted expected payoff for the counterparty. 
            The credit risk still needs to be accounted for in the price. 
            At time 1 the firm will pay the counterparty accordingly, hence the derivative payable, $Y_C$, 
            is still a liability to the firm and ranks pari passu with the already existing debt.
            
            \begin{table}[H]
                \centering
                \begin{tabular}{l|rrrrr}
                    $i$ & 1 & 2 & 3 & 4 & 5 \\
                    \hline
                    $Y(\omega_{i})$ & $4$ & $3.5$ & $2.5$ & $1.5$ & $1$ \\
                \end{tabular}
                \caption{}
                \label{tbl:risky-option-payoff}
            \end{table}

            \begin{table}[H]
                \centering
                \begin{tabular}{l|rrrrr||r}
                    $i$ & 1 & 2 & 3 & 4 & 5 & Present value \\
                    \hline
                    $A(\omega_{i})$ & $122.341$ & $112.341$ & $102.341$ & $97.341$ & $62.341$ & $98.717$ \\
                    $D(\omega_{i})$ & $80$ & $80$ & $80$ & $80$ & $61.571$ & $76.989$ \\
                    $S(\omega_{i})$ & $38.341$ & $28.841$ & $19.841$ & $15.841$ & $0$ & $19.412$ \\
                    $Y_C(\omega_{i})$ & $4.000$ & $3.500$ & $2.500$ & $1.500$ & $0.770$ & $2.317$ \\
                \end{tabular}
                \caption{}
                \label{tbl:example-no-funding-buyback}
            \end{table}

            After receiving the derivative premium in the form of cash, the firm immediately invests in the riskless asset. 
            This implies that the firm's total asset value increases with an amount equal to the discounted derivative premium. 
            The capital structure, after this transaction, is described in \cref{tbl:example-no-funding-buyback}, 
            where the derivative fair value price and loss rate are found
            by solving the following equation system for $\pi(Y_C)$ and $\phi$:
            \begin{align}
                \pi(Y_C) &= \mathbb{E}^{\rnmeasure}\left[Y_{C}\right]\\
                A(\omega_5) &= D(\omega_5) + Y_C(\omega_5)
                \label{eqn:no-funding-buyback-default}
            \end{align}
            with \cref{eqn:no-funding-buyback-default} making sure that the creditors and derivative payable
            takes the total asset value amount in case of default, i.e. state $\omega_5$.

            The change in wealth for the creditors is now computed as:
            \begin{equation}
                \Delta \pi(D) = 76.989 - 76.800 = 0.189
            \end{equation}
            The increment happens as the riskless asset, which the firm invested in with the derivative premium, 
            makes the asset value increase, and a larger total amount is recovered in case of default. 
            In the cases of no default, the creditors are not affected by the new project.\\
            The change in wealth for the shareholders is computed by:
            \begin{equation}
                \Delta \pi(S) = 19.412 - 19.600 = -0.188
            \end{equation}
            where the difference between $\Delta \pi(D)$ and $-\Delta \pi(S)$ is caused by rounding in the present values. 
            Therefore, this is a negative net present value for the shareholders 
            and a positive net present value for the creditors. 
            Due to the risk in the derivative, the shareholders do have a chance to gain from the project, 
            where in state $\omega_4$, the derivative payable is less than the return on the riskless asset.

\end{document}
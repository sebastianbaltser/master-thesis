% !TEX root = sub-main.tex
\documentclass[main.tex]{subfiles}

\begin{document}
    Having derived a framework in which funding costs are defined, 
    the thesis can move on to applying the results in a practical application.
    The current section will consider the implications of funding costs and funding value adjustments.
    As evident from the previous theoretical derivations, 
    obtaining projects might influence the value of the shareholders' and the creditors' claims.

    Before putting numbers to the matter, consider the following example,
    describing the approach taken.

    \begin{example}
    
    A financial institution financed by debt and equity operates in a single-period model.
    At time 1, two possible states can have realized;
    an up state where the value of assets are high
    or a down state where the value of the assets are low.
    In the up state the firm pays its debt to the creditors,
    and the remaining assets are liquidated and paid to the shareholders.
    In the down state, the assets are not sufficient to pay the firm's creditors, 
    and the firm defaults, leaving the creditors with a loss. 

    This situation is depicted in \cref{fig:funding-examples} 
    by the leftmost pair of blocks labelled "Pre-project. 
    The \textcolor{wtf-red}{red block} shows the value of the assets in each state.
    The \textcolor{wtf-blue}{blue block} shows the share of assets claimed by creditors,
    and the \textcolor{wtf-orange}{orange block} the share claimed by shareholders.

    The firm considers investing in a new project, with a high payoff in the up state,
    and a lower payoff in the down state.

    \newcommand{\striped}[2]{%
        \raisebox{-3pt}{%
        \tikz{%
            \draw%
                node[inner sep=0, outer sep=0] {\textcolor{#1}{#2}}%
                node[
                    fill, 
                    pattern=north east lines, 
                    pattern color=white,
                    inner sep=0, outer sep=0,
                ] {\phantom{#2}};}%
        }%
    }

    Consider the middle pair of blocks.
    Here, the firm finances the project by issuing debt to new creditors
    promising them a face value corresponding to the striped blue block in the up state.
    Since the firm might default, the new creditors are offered a credit spread,
    and therefore the new creditors claims more than the project pays off in the up state.
    Hence, the shareholders payoff are lower. 
    In the down state the legacy creditors gets a share of the projects payoff.

    In the rightmost pair of blocks,
    the firm finances the project by issuing equity to new shareholders.
    Since the new creditors are not compensated in the down state,
    they require an even higher return in the up state.
    The legacy shareholders payoff are therefore reduced even more.
    In the down state the legacy creditors receive the entire payoff of the project.
    

    \end{example}

    \begin{figure}
        \centering
        \resizebox{\textwidth}{!}{%
        \begin{tikzpicture}
            % !TEX root = ./test-graphics.tex

\pgfdeclarelayer{background}
\pgfsetlayers{background,main}

\tikzmath{
    \stateSeperator=5px;
    \firmSeperator=1.5cm;
    \valueMultiplier=2/3 px;
}

\tikzset{
    baseblock/.style n args = {1}{
        minimum width=0.55cm,
        inner sep=0,
        outer xsep=0,
        outer ysep=0,
        minimum height=#1*\valueMultiplier,
    },
    asset/.style n args = {1}{
        baseblock={#1},
        fill=wtf-red,
        anchor=south west,
    },
    new-asset/.style n args = {1}{
        asset={#1},
        anchor=south,
        postaction={pattern=north east lines, pattern color=white},
    },
    debt/.style n args = {1}{
        baseblock={#1},
        fill=wtf-blue,
        anchor=south west,
    },
    new-debt/.style n args = {1}{
        debt={#1},
        anchor=south,
        postaction={pattern=north east lines, pattern color=white},
    },
    lost-debt/.style n args = {1}{
        debt={#1},
        anchor=south,
        opacity=0.15,
    },
    equity/.style n args = {1}{
        baseblock={#1},
        fill=wtf-orange,
        anchor=south
    },
    new-equity/.style n args = {1}{
        equity={#1},
        anchor=south,
        postaction={pattern=north east lines, pattern color=white},
    },
    firm/.style n args = {1}{
        inner sep=0,
        label=below:\footnotesize #1,
    }
};

\tikzmath{
    \assetUp = 80;
    \assetDown = 40;
    \debtUp = 60;
    \debtDown = \assetDown;
    \equityUp = \assetUp - \debtUp;
    \equityDown = 0;
    \projectUp = 15;
    \projectDown = 10;
    \legacyDebtDownShare=0.5;
}

\coordinate (cursor) at (0, 0);

%%%% First block:
\coordinate (start) at (cursor);
\coordinate (block-south-west) at (cursor);
\node (asset) [asset={(\debtUp+\equityUp)}] at (cursor) {};
\coordinate (block-north-west) at (asset.north west);
\node (debt) [debt={\debtUp}] at (asset.south east) {};
\node (equity) [equity={\equityUp}] at (debt.north) {};
\coordinate (cursor) at ([xshift=\stateSeperator]debt.south east);

\node (label) at ([yshift=1em]asset.north east) {\scriptsize$\strut$Up};

\node (asset) [asset={(\debtDown)}] at (cursor) {};
\node (debt) [debt={\debtDown}] at (asset.south east) {};
\node [lost-debt={(\debtUp-\debtDown)}] at (debt.north) {};
\coordinate (block-south-east) at (debt.south east);

\node at (asset.north east |- label) {\scriptsize$\strut$Down};

\node[firm=Pre-project,
    fit=(block-south-west) (block-south-east) (block-north-west)] (firm-base) {};

%%%% Second block:

\coordinate (cursor) at ([xshift=\firmSeperator]firm-base.south east);
\coordinate (block-south-west) at (cursor);
\node (asset) [asset={\assetUp}] at (cursor) {};
\node (new-asset) [new-asset={\projectUp}] at (asset.north) {};
\coordinate (block-north-west) at (new-asset.north west);
\node (debt) [debt={\debtUp}] at (asset.south east) {};
\node (equity) [equity={(\equityUp-5)}] at (debt.north) {};
\node (new-debt) [new-debt={(\projectUp+5)}] at (equity.north) {};
\coordinate (cursor) at ([xshift=\stateSeperator]debt.south east);

\node (asset) [asset={\assetDown}] at (cursor) {};
\node (new-asset) [new-asset={\projectDown}] at (asset.north) {};
\node (debt) [debt={(\debtDown + \projectDown*\legacyDebtDownShare)}] at (asset.south east) {};
\coordinate (block-south-east) at (debt.south east);
\node (new-debt) [new-debt={(\projectDown*(1-\legacyDebtDownShare))}] at (debt.north) {};
\node [lost-debt={(\debtUp-\debtDown+\projectUp+5-\projectDown)}] at (new-debt.north) {};

\node[firm=Debt funding,
    fit=(block-south-west) (block-south-east) (block-north-west)] (firm-debt) {};


%%%% Third block:

\coordinate (cursor) at ([xshift=\firmSeperator]firm-debt.south east);
\coordinate (block-south-west) at (cursor);
\node (asset) [asset={\assetUp}] at (cursor) {};
\node (new-asset) [new-asset={\projectUp}] at (asset.north) {};
\coordinate (block-north-west) at (new-asset.north west);
\node (debt) [debt={\debtUp}] at (asset.south east) {};
\node (equity) [equity={(\equityUp-8)}] at (debt.north) {};
\node (new-equity) [new-equity={(\projectUp+8)}] at (equity.north) {};
\coordinate (cursor) at ([xshift=\stateSeperator]debt.south east);

\node (asset) [asset={\assetDown}] at (cursor) {};
\node (new-asset) [new-asset={\projectDown}] at (asset.north) {};
\node (debt) [debt={(\debtDown + \projectDown)}] at (asset.south east) {};
\node [lost-debt={(\debtUp-\debtDown-\projectDown)}] at (debt.north) {};
\coordinate (block-south-east) at (debt.south east);

\node[firm=Equity funding,
    fit=(block-south-west) (block-south-east) (block-north-west)] (firm-equity) {};


\coordinate (origin) at ([xshift=-1cm]firm-base.south west);
\coordinate (x-limit) at ([xshift=1cm]firm-equity.south east);
\begin{pgfonlayer}{background}
    \draw [thick, dotted] 
        (origin) -- 
        (x-limit)
    ;
    \draw [thick, dotted]
        (origin) --
        ++(0, 2.6cm)
    ;
    \draw [dotted] 
        ([yshift=\assetUp*\valueMultiplier]origin) -- 
        ([yshift=\assetUp*\valueMultiplier]x-limit)
    ;
    \draw [dotted] 
        ([yshift=\assetDown*\valueMultiplier]origin) -- 
        ([yshift=\assetDown*\valueMultiplier]x-limit)
    ;
\end{pgfonlayer}

\node[fit=(firm-base) (firm-debt) (firm-equity)] (main) {};

\tikzset{
    legend/.style n args = {1}{
        minimum width=0.4cm,
        minimum height=0.4cm,
        inner sep=0,
        outer xsep=0,
        outer ysep=0,
        label=right:\scriptsize#1,
        anchor=center,
    },
};

\node [
    matrix, 
    column sep = 1cm,
    row sep=0,
    inner sep=3px,
    nodes={minimum height=0.5cm},
] (labels) at ([yshift=-1.5cm]main.south) {
    \node[asset=0, legend=Legacy Assets] {}; &
    \node[debt=0, legend=Legacy Debt] {}; &
    \node[equity=0, legend=Legacy Equity] {}; \\
    \node[new-asset=0, legend=New Project] {}; &
    \node[new-debt=0, legend=New Debt] {}; &
    \node[new-equity=0, legend=New Equity] {}; \\
    & \node[lost-debt=0, legend=Lost Face Value] {};\\
};

        \end{tikzpicture}        
        }   
        \caption{
            Distribution of assets to creditors and shareholders before entering a project,
            after entering a project financed by debt, 
            and after entering a project financed by equity.
        }
        \label{fig:funding-examples}
    \end{figure}

    Deliberately free of too many details, 
    this example should show the overall workings of the setup, 
    which the following sections will use.
    A firm considers a project, 
    and its choice of financing will affect the payoff of shareholders and creditors.
    The payoff of the project under consideration has a big influence on the funding implications
    and the project can have other important features such as collateralization requirements.


\end{document}
% !TEX root = sub-main.tex
\documentclass[main.tex]{subfiles}

\begin{document}
    \subsection{A Dealer in a Single-Period Economy}

    A derivatives dealer is operating in a single-period economy
    defined by $N=5$ states and the associated Arrow-Debreu prices given in \cref{tbl:example-firm-structure}.
        \begin{table}[H]
            \centering
            \begin{tabular}{l|rrrrr}
                $i$ & 1 & 2 & 3 & 4 & 5 \\
                \hline
                $\psi_{i}$ & $\num{0.06}$ & $\num{0.24}$ & $\num{0.29}$ & $\num{0.28}$ & $\num{0.12}$ \\
            \end{tabular}
            \caption{}
            \label{tbl:example-firm-structure}
        \end{table}
    This implies a discount factor of $\discountfactor = \num{0.99}$ and a risk-free interest rate of $\rfrate = \pct{0.010101}$.
    The dealer invests in risky assets 
    which are funded by equity and debt deposits and return payoff specified shortly.
    In the event of default, the dealer's firm faces no distress cost, i.e. $\kappa = 1$, 
    such that the bankruptcy estate is distributed entirely to the creditors.
    The firm has been funded by debt such that the face value of debt is its total liabilities $L=\num{80}$, 
    which gives rise to the following payoff structure:
    \begin{table}[H]
        \centering
        \begin{tabular}{l|rrrrr||r}
            $i$ & 1 & 2 & 3 & 4 & 5 & Present value \\
            \hline
            $A(\omega_{i})$ & \num{120} & \num{110} & \num{100} & \num{95} & \num{60} & \num{96.4} \\
            $D(\omega_{i})$ & \num{80} & \num{80} & \num{80} & \num{80} & \num{60} & \num{76.80}\\
            $E(\omega_{i})$ & \num{40} & \num{30} & \num{20} & \num{15} & \num{0} & \num{19.6}
        \end{tabular}
        \caption{}
        \label{tbl:example-pre-project-capital-structure}
    \end{table}
    The associated present values of the payoffs, $A$, $D$ and $E$, 
    are the discounted expected values
    with respect to the risk-neutral probability measure:
        \begin{gather*}
            \pi(A) = \discountfactor \mathbb{E}^{\rnmeasure}\left[A\right] = \num{96.40} \\
            \pi(D) = \discountfactor \mathbb{E}^{\rnmeasure}\left[D\right] = \num{76.80}
            \qquad \pi(E) = \discountfactor \mathbb{E}^{\rnmeasure}\left[E\right] = \num{19.60}
        \end{gather*}
    As seen in \cref{tbl:example-pre-project-capital-structure},
    the creditors do not receive their entire face value in the default event, $\omega_5$.
    The relative difference between the discounted payoff and the promised face value
    governs the credit spread, which compensates the creditors for this risk of default.

    In addition to calculating the shareholders' marginal valuation of a new project,
    it is interesting to also evaluate the impact on the creditors' valuation of their claim. 
    Therefore, it is useful to calculate the credit spread and loss rate of the firm's debt,
    since that can be used for comparison later.

    The credit spread before entering into any projects is 
    $\num{80}/\num{76.80} - \grossrfrate = \pct{0.031566}$.
    Using \cref{eqn:creditor-loss-rate}, the loss rate of the creditors in the default event is:
        \begin{equation*}
            \phi(\omega_{5}) 
            = 
                \frac{
                    \num{80}-\num{60}
                }{
                    \num{80}
                } 
            = 
                \pct{0.25}
        \end{equation*}
    In all other states the loss rate is zero, as the firm does not default.
    Recall that the limiting spread is the spread on the firm's debt 
    after obtaining an infinitesimal amount of a project. 
    The loss rate can be substituted into \cref{eqn:limiting-spread} 
    and the limiting spread can be calculated:
        \begin{equation*}
            S
            =
            \frac{
                \mathbb{E}_{t}\left[\phi\right]
                R
            }{
                1 
                -
                \mathbb{E}_{t}\left[\phi\right] 
            } 
            =
            \pct{0.031566}
        \end{equation*}

    The derivations in previous sections, 
    which described limiting spreads and marginal shareholder valuations, 
    were rooted in the assumption of the firm obtaining an infinitesimal project.
    This section will use the results in a practical application,
    where a larger than infinitesimal project will be obtained by the firm.
    Therefore, the results obtained from the equations will, to some extent, be inaccurate.
    This is solely due to the limiting spread;
    it is an estimate of the post-project credit spread,
    and therefore might be inaccurate to a degree that depends on the project.
    The magnitude of the inaccuracy will depend on how much the new project alters
    the creditors' loss rate, since that controls the credit spread.
    The scale of the impact on the creditors' loss rate 
    depends on the the actual size of the project compared to value of the firm's assets.
    The direction is determined by the payoff structure of the project
    compared to the structure of the firm's assets.

    Fortunately, in the discrete framework used here, 
    the actual marginal shareholder valuations can be obtained with numerical methods,
    such that the results of the equations can be verified and compared to the correct results.

    \subsubsection*{Digression on The Limiting Spread}

    As was calculated above, the limiting spread is equal to the credit spread of the existing debt.
    This is no coincidence, and this digression will explain why it is so,
    and what the implications are.
    As it turns out, 
    the framework presented here provides a helpful interpretation of the limiting spread.
    Recall the definition of the limiting spread:
        \begin{equation*}
            S
            =
            \frac{
                \mathbb{E}^{\rnmeasure}\left[\phi\right]
                \grossrfrate
            }{
                1
                -
                \mathbb{E}^{\rnmeasure}\left[\phi\right]
            }
        \end{equation*}
    For a constant face value,
    the actual credit spread can be calculated by dividing the face value of the debt
    with the present value of the debt claim and subtracting the gross risk free rate.
    Denote the face value by $F$. 
    The present value of the debt claim is the discounted face value, $\discountfactor F$,
    adjusted for the credit risk. 
    As usual, the \CVA/ corresponds to the discounted expected positive exposure
    multiplied by the loss rate given default, 
    i.e. $\CVA/ = \discountfactor\mathbb{E}^{\rnmeasure}\left[F \phi \right] $.
    Hence the credit spread of existing debt is given by:
        \begin{equation*}
            \frac{
                F
            }{
                \discountfactor F
                -
                \discountfactor
                \mathbb{E}^{\rnmeasure}\left[
                    F \phi
                \right] 
            } 
            -
            \grossrfrate
            =
            R \left(
            \frac{
                1
            }{
                    1
                    -
                    \mathbb{E}^{\rnmeasure}\left[
                        \phi
                    \right] 
            } 
            - 
            1
            \right)
            =
            \frac{
                \mathbb{E}^{\rnmeasure}\left[\phi\right]
                \grossrfrate
            }{
                1
                -
                \mathbb{E}^{\rnmeasure}\left[\phi\right]
            }
        \end{equation*}
        Which can be recognized as the limiting spread. 
        Hence, in the current setup, 
        the definitions of the marginal shareholder valuations are based on
        the assumption that the post-project credit spread 
        will equal the pre-project credit spread.
        However, these definitions are derived using infinitesimal projects. 
        In practical applications,
        it would be more appropriate to refer to the limiting spread as an approximation 
        of the credit spread after obtaining a larger-than-infinitesimal project.
        
        The mathematics have been accounted for, 
        but the approximation does seem to be financial legitimate as well.
        First, the post-project credit spread is based on the risk of the entire firm,
        which likely encompasses many projects.
        Each new project is therefore going to be of relatively small size compared
        to the entire firm, and it will only have a small effect on the overall riskiness.
        Second, this effect is enhanced if the firm mostly invests in projects that match
        its overall riskiness in the first place.
        These two points work, 
        since the credit spread does not change if the riskiness does not change. 

        Third and last, as was mentioned earlier, according to \textcite{Castagna2012FVA},
        the cost of capital will only gradually update, 
        such that obtaining a new project only have a very marginal effect on the credit spread.

        In conclusion, the limiting spread is not merely a mathematical result
        but also a financially sound approximation of the incremental cost of obtaining new projects.
        With this interpretation in mind, the example can continue.

\end{document}
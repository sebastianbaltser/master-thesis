% !TEX root = sub-main.tex
\documentclass[main.tex]{subfiles}

\begin{document}
    \subsection{A dealer in a single-period economy}

    A derivatives dealer is operating in a single-period economy
    defined by $N=5$ states and the following associated Arrow-Debreu prices:
        \begin{table}[H]
            \centering
            \begin{tabular}{l|rrrrr}
                $i$ & 1 & 2 & 3 & 4 & 5 \\
                \hline
                $\psi_{i}$ & $\num{0.06}$ & $\num{0.24}$ & $\num{0.29}$ & $\num{0.28}$ & $\num{0.12}$ \\
            \end{tabular}
            \caption{}
            \label{tbl:example-firm-structure}
        \end{table}
    implying a discount factor of $\discountfactor = \num{0.99}$ and a risk-free interest rate of $\rfrate = \pct{0.010101}$.
    The dealer invests in risky assets 
    which are funded by equity and debt deposits and returns payoff specified shortly.
    In the event of default, the dealer's firm faces no distress cost, i.e. $\kappa = 1$, 
    such that the bankruptcy estate is distributed entirely to the creditors.
    The firm has been funded by debt such that the face value of debt is $\num{80}$, 
    which gives rise to the following payoff structure:
    \begin{table}[H]
        \centering
        \begin{tabular}{l|rrrrr}
            $i$ & 1 & 2 & 3 & 4 & 5 \\
            \hline
            $A(\omega_{i})$ & \num{120} & \num{110} & \num{100} & \num{95} & \num{60} \\
            $D(\omega_{i})$ & \num{80} & \num{80} & \num{80} & \num{80} & \num{60} \\
            $E(\omega_{i})$ & \num{40} & \num{30} & \num{20} & \num{15} & \num{0}
        \end{tabular}
        \caption{}
        \label{tbl:example-pre-project-capital-structure}
    \end{table}
    The associated present values of the payoffs, $A$, $D$ and $E$, 
    are the discounted expected value of the payoff 
    with respect to the risk-neutral probability measure:
        \begin{gather*}
            \pi(A) = \discountfactor \mathbb{E}^{\rnmeasure}\left[A\right] = \num{96.40} \\
            \pi(D) = \discountfactor \mathbb{E}^{\rnmeasure}\left[D\right] = \num{76.80}
            \qquad \pi(E) = \discountfactor \mathbb{E}^{\rnmeasure}\left[E\right] = \num{19.60}
        \end{gather*}
    The face value of the debt includes a credit spread that the firm must pay in order to account for its own credit risk.
    As seen in \cref{tbl:example-pre-project-capital-structure},
    the creditors do not receive their entire face value in the default event, $\omega_5$.
    The relative difference between the discounted payoff and the promised face value
    governs the credit spread, which compensates the creditors for the risk of a default.

    In addition to calculating the shareholders' marginal valuation of a new project,
    it is interesting to also evaluate the impact on the creditors' valuation of their claim. 
    Therefore, it is useful to calculate the credit spread and loss rate of the firm's debt,
    since that can be used for comparison later.

    The credit spread before entering into any projects is 
    $\num{80}/\num{76.80} - \grossrfrate = \pct{0.031565}$.
    Using \cref{eqn:creditor-loss-rate}, the loss rate of the creditors in the default event is:
    $\phi(\omega_{5}) = (\num{80}-\num{60})/\num{80} = \pct{0.25}$
    while in all other states the loss rate is zero, as the firm does not default.

\end{document}
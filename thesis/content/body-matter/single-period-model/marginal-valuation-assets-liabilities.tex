% !TEX root = ./single-period-model.tex
\documentclass[../main.tex]{subfiles}

\begin{document}
    \subsection{Marginal valuation of corporate assets and liabilities}
        Consider a firm whose assets and liabilities have random payoffs at time 1 denoted respectively by $A$ and $L$.  
        The firm can obtain an amount $q$ of the project.
        Therefore, the cash flow to the firm at time 1 for a position of size $q$ is given by $qY$.
        By entering into the project, the firm will have altered its default event,
        since the cash flows at time 1 have changed.
        The default event for the firm after it has obtained a quantity $q$ of the project
        is denoted by $\mathcal{D}(q)$ and the precise definition varies,
        depending on how the firm chooses to finance obtaining the project.
        The pre-project default event is therefore essentially equal to $\mathcal{D}(0)$,
        but, to keep the notation simple, the argument is dropped leaving just $\mathcal{D}$.
        In order to gain further insights into the mechanics of this framework, 
        assume initially that $qY$ is the only default claim besides $L$. 
        In that case, the the definition of the firms post-project default event is given by 
        $\mathcal{D}(q) = \{A + qY < L\}$.
        Similarly, $\mathcal{D}^{c}(q) = \{A + qY \geq L\}$ defines the event of no default.
        The net asset value of the firm after defaulting, with the distress costs comprised, is $\kappa A$ for some recovery rate, $\kappa$, where the distress costs equal $1-\kappa$.
        With the project in its portfolio the payoffs to creditors and shareholders are given by:
            \begin{align}
                S 
                &= 
                \mathbbm{1}_{\mathcal{D}^{c}(q)}
                (A + qY - L)
                \\
                D 
                &= 
                \mathbbm{1}_{\mathcal{D}(q)}
                \kappa A 
                + 
                \mathbbm{1}_{\mathcal{D}^{c}(q)} 
                L
            \end{align}
        The payoff of the project, $Y$, can take on both positive and negative values
        depending on the state realized a time 1, 
        but of course also dependent on the specific type of project. 
        A swap contract would be an example of a project where the payoff can take both signs,
        while the payoff of a bought call option could only take non-negative values.
        The treatment of the payoff in default events will depend on the sign of the payoff,
        and therefore it will be useful to be able to separate the two cases;
        positive payoff where the firm is owed cash
        and negative payoff where the firm owes cash.

        The positive part of the payoff, $Y^{+}=\max\left(Y,0\right)$, is perceived as an asset of the firm, and is measured net of any potential losses due to credit risk of the counterparty.
        In addition, the negative part of the payoff, $Y^{-} = \max \left(-Y,0\right)$, is a contingent liability, and is also interpreted as the contractual amount that the firm is obliged to pay at time 1, before taking the firm's risk of default into consideration.
        If the contingent liability is fully secured, e.g. by collateralization, the total value to the firm of the financial instrument is then given by:
        \begin{equation}
            \pi(Y) = \discountfactor \mathbb{E}^{\mathbb{Q}} \left[Y\right]
        \end{equation}

        If, however, the contingent liability, $Y^{-}$, is not fully secured, a specification of how the associated counterparty recovers, in case of the firm defaulting, is needed.
        For simplicity, and also as described by \textcite{ADS2019}, it's assumed that the portion of unsecured contingent liabilities acquired by the firm ranks pari passu with each other, such that the various claimants' default recoveries are pro rata with their claim amounts.
        In practice, these liabilities more often rank pari passu with the firm's senior creditors as well.
        Let $\mathcal{C}$ be the firm's cash flows at time 1 related to the portion of a total of $N$ unsecured contracts with different claimants. The value to the firm of these cash flows is then given by:
        \begin{equation} % Do we need a proof for this equation?
            \pi(\mathcal{C}) = \discountfactor \mathbb{E}^{\mathbb{Q}}\left[
            \sum_{i=1}^{N} \left(
                \mathbbm{1}_{\mathcal{D}^{c}(q)} qY_{i}
                + \mathbbm{1}_{\mathcal{D}(q)} qY_{i}^{+}
                - \mathbbm{1}_{\mathcal{D}(q)} \rho_{i} \kappa A
            \right)\right]
        \end{equation}
        where $Y = \sum_{i} Y_{i}$ is the payoff of the portfolio, which the firm has undertaken, and $\rho_{i} = qY_{i}^{-}/(L + qY^{-})$ is the pro rata share of contingent liability $i$. The first term in the summation expresses the firm's payoff from the different projects in case of no default. The second term adds the positive parts of the different investments made by the firm in case of default, since these are assets to the firm. The third part subtracts each contingent liabilities which the firm is obliged to pay in case of default. Each contingent liability receives their own pro rata share of the recovered total asset value.

        Practically, it's possible for a firm to also have unsecured position in its portfolio.
        The analysis is therefore extended to consider swap positions that include both secured as well as unsecured components.
        For this motive, the payoff is divided in two, such that $Y=Y_1 + Y_2$, where $Y_1^{-}$ reflects the secured contingent liability, and $Y_2^{-}$ reflects the unsecured contingent liability.
        The financial position is still assumed to pay a payoff, $Y$, at time 1 which is unrelated to the credit risk of the firm, and the unsecured contingent liability related to claimant $i$ is still assumed to rank pari passu with all other unsecured creditor claims.\\
        For the contingent liability, $qY_1^{-}$, to be secured, $A+qY_1 > 0$ is a necessary condition, and therefore assumed.
        $\mathcal{C}$ now reflects the cash flows of the firm where each swap position can have both secured and unsecured positions, and the firm's valuation of the associated net time 1 cash flows is given by:
        \begin{equation}
            \pi(\mathcal{C}) = \discountfactor \mathbb{E}^{\mathbb{Q}}\left[
            \sum_{i=1}^{N} \left(
                q Y_{i,1}
                + \mathbbm{1}_{\mathcal{D}^{c}(q)} qY_{i,2}
                + \mathbbm{1}_{\mathcal{D}(q)} qY_{i,2}^{+}
                - \mathbbm{1}_{\mathcal{D}(q)} \kappa (A + qY_{i,1}) \rho_{i}
            \right)\right]
        \end{equation}
        where $\rho_{i} = qY_{i,2}^{-}/(L + qY_{2}^{-})$ is the pro rata share of contingent liability $i$. Notice that the secured positions are now paid at time 1 no matter if the firm defaults or not. Only the unsecured positions depend on the outcome of the total asset value.
        
\end{document}
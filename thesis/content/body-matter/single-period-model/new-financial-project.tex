% !TEX root = sub-main.tex
\documentclass[main.tex]{subfiles}

\begin{document}
    \subsection{Obtaining a New Financial Project}
        When a firm enters a new project, the new payoffs and other cash flows
        will alter the structure of the firm and
        change how much value shareholders and creditors assign to their own claim.
        This section will derive the mathematical details of the project,
        and the following sections will dive into the implications of obtaining the project.

        The firm has an opportunity to engage in a new project, e.g. buying or selling a security,
        with a, possibly random, payoff denoted by $Y$.
        The firm considers obtaining an amount $q$ of the project;
        therefore, the potential cash flow to the firm at time 1 will be $qY$.
        By entering into the project, the firm will have altered its default event,
        since assets and liabilities at time 1 have changed.
        The default event for the firm after it has obtained a quantity $q$ of the project
        is denoted by $\mathcal{D}(q)$.
        The precise definition of the default event varies, 
        depending on how the firm chooses to finance the project.
        The pre-project default event is essentially equal to $\mathcal{D}(0)$,
        but, to keep the notation simple, the argument is dropped leaving just 
        $\mathcal{D}\equiv\mathcal{D}(0)$.

        The purpose of the following derivations is to describe the cash flows
        to the firm of obtaining the project.
        For this purpose, assume that $qY$ is the only default claim besides $L$. 
        In that case, the definition of the firm's post-project default event is given by 
        $\mathcal{D}(q) = \{A + qY < L\}$.
        Similarly, $\mathcal{D}^{c}(q) = \{A + qY \geq L\}$ defines the event of no default.
        The net asset value of the firm after defaulting, comprising the distress costs, is $\kappa A$ for some recovery rate, $\kappa$, where the distress costs equal $1-\kappa$.

        The payoff of the project, $Y$, can take on both positive and negative values
        depending on the state realized a time 1, 
        but of course also depends on the specific type of project. 
        A swap contract would be an example of a project where the payoff can take both signs,
        while the payoff of a bought call option can only take non-negative values.
        The treatment of the payoff in default events will depend on the sign of the payoff,
        and therefore it will be useful to be able to separate the two cases;
        positive payoff where the firm is owed cash,
        and negative payoff where the firm owes cash.

        The positive part of the payoff, $Y^{+}=\max\left(Y,0\right)$, is perceived as an asset of the firm, and is measured net of any potential losses due to credit risk of the counterparty.
        In addition, the negative part of the payoff, $Y^{-} = \max \left(-Y,0\right)$, is a contingent liability, and is also interpreted as the contractual amount that the firm is obliged to pay at time 1, before taking the firm's risk of default into consideration.

        If the contingent liability, $Y^{-}$, is not fully secured, a specification of how the associated counterparty recovers, in case of the firm defaulting, is needed.
        According to \textcite{ADS2019}, in practice, a firm's swap-contingent liabilities 
        are normally pari passu with its unsecured debt claims.
        When liabilities rank pari passu, they experience the same loss rate in a default.
        In other words, they receive a share of the remaining assets 
        corresponding to their share of the total liabilities.
        \\
        Since it is common in practice, throughout this paper it is assumed that the firm's 
        unsecured liabilities rank pari passu with each other;
        therefore, the counterparty to the new project ranks pari passu with the creditors.

        If the firm defaults, the share of the bankruptcy estate paid to the counterparty with 
        the contingent liability $Y^{-}$ is:
            \begin{equation*}
                \rho = 
                \frac{
                    qY^{-}
                }{
                    L + qY^{-}
                } 
            \end{equation*}
        Then, at time 1, the random cash flow to the firm is given by:
        \begin{equation*}
            \mathcal{C} 
            = 
            \mathbbm{1}_{\mathcal{D}^{c}(q)} qY
            + \mathbbm{1}_{\mathcal{D}(q)} qY^{+}
            - \mathbbm{1}_{\mathcal{D}(q)} \rho \kappa A
        \end{equation*}
        The first term expresses the firm's payoff from the project in case it does not default. 
        The positive payoff to the firm, $Y^{+}$, is paid in full if the firm defaults,
        which is represented by the second term.
        If the firm defaults, it pays a share, $\rho$, of the remaining assets, $\kappa A$,
        to the counterparty of the project.
        This is represented by the third term.
        For clarification, note that $Y > 0$ implies $\rho=0$, 
        such that the firm does not pay anything if it does not owe anything.

        Practically, it is possible for the contingent liability of the position
        to have both a secured as well as an unsecured component.
        For this motive, the payoff is divided in two, such that $Y=Y_1 + Y_2$, where $Y_1^{-}$ reflects the secured contingent liability, and $Y_2^{-}$ reflects the unsecured contingent liability.
        The financial position is still assumed to pay a payoff, $Y$, at time 1 which is unrelated to the credit risk of the firm, and the unsecured contingent liability is still assumed to rank pari passu with all other unsecured creditor claims.\\
        For the contingent liability, $qY_1^{-}$, to be secured, $A+qY_1 > 0$ is a necessary condition, and therefore assumed.

        Then, the cash flow of the firm where the position 
        can have both secured and unsecured parts is given by:
        \begin{equation*}
            \mathcal{C} =
                q Y_{1}
                + \mathbbm{1}_{\mathcal{D}^{c}(q)} qY_{2}
                + \mathbbm{1}_{\mathcal{D}(q)} qY_{2}^{+}
                - \mathbbm{1}_{\mathcal{D}(q)} \rho\kappa (A + qY_{1})
        \end{equation*}
        where $\rho = qY_{2}^{-}/(L + qY_{2}^{-})$ is the pro rata share of the contingent liability. 
        Notice that the secured position is now paid at time 1 no matter if the firm defaults or not. 
        Only the unsecured position depends on the outcome of the total asset value.

        This section has defined the firm's default event when considering the possibility
        of obtaining a quantity $q$ of a project with payoff $Y$.
        The structure of the payoff has been elaborated on
        and it has been assumed that the payoff can be both a liability and an asset to the firm
        as well as having both secured and unsecured parts.
        With these definitions in order, the analysis can now move on to consider the
        funding implications of the firm entering into the project.
        Specifically, the next section considers the impact on shareholders' valuation 
        of obtaining the new project.
        
\end{document}
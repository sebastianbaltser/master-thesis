% !TEX root = ../single-period-model.tex
\documentclass[../main.tex]{subfiles}

\begin{document}
    \subsection{Funding Secured Derivatives}
        Now consider the same derivative contract as above which is sold by the firm to the counterparty.
        The firm spends the premium by posting collateral at a third party (possibly a custodian) where it earns the risk-free rate.
        The collateral posted is a variation margin that covers the entire amount of the derivative payable.
        On top of the variation margin, the firm also posts an initial margin worth 50\% of the purchase price at time 0.
        The initial margin is funded by new debt.
        The firm as still assumed to have credit risk, whereas the counterparty is without credit risk.

        % Hillion creates a portfolio where the option is bought by the counterparty and also sold to a hedger dealer. This will cause a wealth transfer that consist of both \MVA/ and \FCA/.

        This example suggest that, for a derivative contract as such,
        the variation margin will not cause any funding costs or benefits,
        as the option premium offsets the posted collateral.
        However, this is not the case for the initial margin which requires funding.
        The collateral must be posted in cash or assets, and cannot be rehypothecated by the third party.
        The issue raised by this setup, is whether a valuation adjustment to the derivative is needed due to the margin requirements.
        This is denoted as \MVA/.

        The financing costs from the initial margin do not differ much from the principles of funding costs discussed in \cref{sec:risk-free-project}.
        Issuing debt to fund the initial margin is a zero net present value for the new creditors.
        This leads to the conclusion that the shareholders are worse off as they pay the funding of the risk-free security held by the third party in the event of no default, i.e. the initial margin.
        The wealth transfer favors the legacy creditors,
        and the derivative contract seems as a subject to an \MVA/.
        The \MVA/ determine the difference between the market fair value and value that makes the shareholders indifferent of entering the new secured project.

        At time 1, the firm's asset value has increased with future value of the initial- and variation margin posted to the third party.
        By funding this initial margin the firm obtains a new liability in terms of a new debt which ranks pari passu with the existing debt.
        The capital structure of the firm at time 0 is summarized in \cref{tbl:example-collateralized-derivative}.

        \begin{table}[H]
            \centering
            \begin{tabular}{l|rrrrr||r}
                $i$ & 1 & 2 & 3 & 4 & 5 & Present value \\
                \hline
                $A(\omega_{i})$ & $123.553$ & $113.553$ & $103.553$ & $98.553$ & $63.553$ & $99.918$ \\
                $D(\omega_{i})$ & $80.000$ & $80.000$ & $80.000$ & $80.000$ & $61.615$ & $76.994$ \\
                $S(\omega_{i})$ & $38.335$ & $28.835$ & $19.835$ & $15.835$ & $0$ & $19.406$ \\
                $\tilde{D}(\omega_{i})$ & $1.218$ & $1.218$ & $1.218$ & $1.218$ & $0.938$ & $1.173$ \\
                $Y_C(\omega_{i})$ & $4.000$ & $3.500$ & $2.500$ & $1.500$ & $1.000$ & $2.345$ \\
            \end{tabular}
            \caption{}
            \label{tbl:example-collateralized-derivative}
        \end{table}

\end{document}
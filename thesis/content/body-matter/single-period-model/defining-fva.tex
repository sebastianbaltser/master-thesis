% !TEX root = ../main.tex
\documentclass[../main.tex]{subfiles}

\begin{document}
    \subsection{Defining FVA}
        Still remaining at this point is to properly define \FVA/, at least in a technical manner.
        According to \cite{ADS2019} there are multiple ways of calculating \FVA/ used in practice and theory,
        and a couple of viable definitions will be explained in this section.
        The definitions are due to \cite{Hillion2016}.

        \subsubsection*{FVA as the promised excess funding cost}
            Since the firm obtains funds for the project at a spread in excess of the risk free rate, 
            the shareholders pay an additional rate when the firm does not default. 
            This is a form of funding cost and argues in favor of defining the \FVA/ as
            the present value of the costs paid by the shareholders in excess of the risk free rate:
                \begin{align}
                    \FVA/ 
                    &= 
                        \discountfactor \left(
                            U(q) (1 + \rfrate + s(q))
                            - U (1 + \rfrate)
                        \right) 
                        \nonumber \\
                    &= 
                        \discountfactor U(q) s(q)
                    \label{eqn:fva-as-promised-excess-funding-cost}
                \end{align}
            Since the entire credit spread is compensation for credit risk,
            the \FVA/ defined here is the compensation provided to the creditors,
            for the possibility that the firm might default. 
            If the firm defaults, this amount is not to be paid; 
            the \FVA/ is therefore equal to the expected benefit to the firm from defaulting, also known as the DVA. 
            The result that the \FVA/ equals the DVA when credit spreads are merely compensation for credit risk
            is shown by \cite{HullWhiteFVA}.

            This definition of \FVA/ does not taking into account 
            that the shareholders are not paying the credit spread when the firm defaults, 
            as opposed to the next possible definition.

        \subsubsection*{FVA as the expected excess funding cost}
            The FVA could also be defined by the quantity $\Phi$ in \cref{eqn:marginal-shareholder-value-debt-financing}, 
            i.e. the marginal valuation of the new project to the firm's legacy creditors. 
            This value is also equal to the wealth transfer from the shareholders to the legacy creditors
            of obtaining the new project.
            Compared to the previous definition of FVA, 
            $\Phi$ captures the expected funding cost as opposed to the excess funding cost.
            Thus the the two quantities differ by a factor corresponding to the no default probability.

            This approach seems more sensible than the previous, 
            since it focuses more on the shareholders' actual costs of obtaining the project.
            Still, this adjustment might not be be the breakeven adjustment
            for the firm to enter the project from the viewpoint of shareholder value maximization.
            As value adjustments are generally made as compensation for some quantity,
            in such a way that the product being adjusted ends up as a zero net present value investment,
            it would be coherent to also define \FVA/ as such.
            
        \subsubsection*{FVA as the adjustment to shareholders' breakeven}
            This suggests defining FVA as the difference between the project's theoretical value
            and the price that makes the shareholders' indifferent to engaging in the project.
            Phrased differently, the FVA is the donation needed from the project counterparty in order
            to convince the shareholders to enter the project.

            Deriving the shareholder's breakeven price
            is a matter of setting equal to zero the marginal value of entering project 
            and solving for the marginal investment cost. 
            With three different funding types considered, this will result in three different breakeven prices,
            as well as three different definitions of \FVA/.
            These will be derived in the following paragraphs.
            
            \textit{Debt financing:} \\
            Under debt financing, the breakeven price is determined by setting
            \cref{eqn:marginal-shareholder-value-debt-financing} equal to zero
            and solving for the marginal investment cost:
                \begin{align}
                    0 &= G_{\text{debt}} 
                        \nonumber\\
                    &=
                        p^{\mathbb{Q}} \left(
                            \discountfactor
                            \mathbb{E}^{\mathbb{Q}}\left[Y\right]
                            - u
                        \right)
                        -
                        \discountfactor
                        \text{Cov}^{\mathbb{Q}}\left(\mathbbm{1}_{\mathcal{D}}, Y\right) 
                        - 
                        p^{\mathbb{Q}} \discountfactor u S 
                        \nonumber\\
                    &= 
                        \discountfactor
                        \mathbb{E}^{\mathbb{Q}}\left[Y\right]  
                        - u
                        - 
                        \discountfactor
                        \frac{
                            \text{Cov}^{\mathbb{Q}}\left(\mathbbm{1}_{\mathcal{D}}, Y\right) 
                        }{
                            p^{\mathbb{Q}}
                        }
                        - \discountfactor u S 
                        \nonumber\\
                    u^{\ast}_{\text{debt}}
                    &\equiv
                        \frac{
                            1
                        }{
                            \grossrfrate + S
                        } 
                        \left(
                            \mathbb{E}^{\mathbb{Q}}\left[Y\right]
                            - \frac{
                                \text{Cov}^{\mathbb{Q}}\left(\mathbbm{1}_{\mathcal{D}}, Y\right)
                            }{
                                p^{\mathbb{Q}}  
                            } 
                        \right)
                    \label{shareholders-breakeven-debt-financing}
                \end{align}
            A result named Proposition 3 by \cite{ADS2019}.
            The expected value of the cash flow, $\mathbb{E}^{\mathbb{Q}}\left[Y\right]$,
            includes the counterparty credit risk, since that is inherited in the cash flow, 
            but excludes the firm's own credit risk. 
            In other words, $\mathbb{E}^{\mathbb{Q}}\left[Y\right]$ is adjusted for \CVA/ but not \DVA/.
            The second term is the covariance between the default event and the cash flow,
            inflated by the probability of not defaulting. 
            The shareholders' breakeven value can then be seen as the discounted expected value of the cash flow
            adjusted for the inflated covariance.
            The discounting rate used is however not the risk free rate, but rather the funding rate, $\grossrfrate+S$.
            Having derived the shareholders' breakeven value, the definition of the \FVA/ under debt financing
            can be defined as the adjustment to the actual price needed to arrive at breakeven:
                \begin{equation}
                    \FVA/_{\text{debt}} 
                        = u^{\ast}_{\text{debt}} - u
                    \label{fva-debt-financing}
                \end{equation}

            \textit{Equity financing:} \\
            Likewise the breakeven price under equity financing can be found by solving for $u$
            when \cref{eqn:marginal-shareholder-value-equity-financing} equals zero:
                \begin{align}
                    0 &= G_{\text{equity}} 
                        \nonumber \\
                    &=
                        p^{\mathbb{Q}} \left(
                            \discountfactor
                            \mathbb{E}^{\mathbb{Q}}\left[Y\right]
                            - u
                        \right)
                        -
                        \discountfactor
                        \text{Cov}^{\mathbb{Q}}\left(\mathbbm{1}_{\mathcal{D}}, Y\right) 
                        - 
                        \left(
                            1
                            -
                            p^{\mathbb{Q}}
                        \right)
                        u 
                        \nonumber \\
                    u_{\text{equity}}^{\ast}
                    &\equiv
                        \discountfactor
                        \left(
                            p^{\mathbb{Q}}
                            \mathbb{E}^{\mathbb{Q}}\left[Y\right]
                            -
                            \text{Cov}^{\mathbb{Q}}\left(\mathbbm{1}_{\mathcal{D}}, Y\right) 
                        \right)
                        \nonumber \\
                    &= 
                        \discountfactor
                        p^{\mathbb{Q}}
                        \left(
                            \mathbb{E}^{\mathbb{Q}}\left[Y\right]
                            -
                            \frac{
                                \text{Cov}^{\mathbb{Q}}\left(\mathbbm{1}_{\mathcal{D}}, Y\right)
                            }{
                                p^{\mathbb{Q}}  
                            }  
                        \right)
                \end{align}
            Again, the shareholders' breakeven value can then be seen as the discounted expected value of the cash flow
            adjusted for the inflated covariance.
            However, the discount factor applied is different from the risk rate plus the credit spread as seen before.
            The discount factor in this case $\discountfactor p^{\mathbb{Q}}$ corresponds to 
            the price of an Arrow-Debreu security paying one unit of numeraire if the firm defaults and zero if not.
            The \FVA/ under equity funding is then:
                \begin{equation}
                    \FVA/_{\text{equity}} 
                        = u^{\ast}_{\text{equity}} - u
                    \label{fva-equity-financing}
                \end{equation}

            \textit{Cash funding:} \\
            The breakeven price under cash funding is derived using
            \cref{eqn:marginal-shareholder-value-cash-financing}:
                \begin{align}
                    0 &= G_{\text{cash}} 
                        \nonumber \\
                    &=
                        p^{\mathbb{Q}} \left(
                            \discountfactor
                            \mathbb{E}^{\mathbb{Q}}\left[Y\right]
                            - u
                        \right)
                        -
                        \discountfactor
                        \text{Cov}^{\mathbb{Q}}\left(\mathbbm{1}_{\mathcal{D}}, Y\right) 
                        \nonumber \\
                    u_{\text{cash}}^{\ast}
                    &\equiv
                        \discountfactor
                        \left(
                            \mathbb{E}^{\mathbb{Q}}\left[Y\right]
                            -
                            \frac{
                                \text{Cov}^{\mathbb{Q}}\left(\mathbbm{1}_{\mathcal{D}}, Y\right)
                            }{
                                p^{\mathbb{Q}}  
                            }  
                        \right)
                \end{align}
            Once more, the breakeven value is the expected value of the cash flow
            adjusted for the inflated covariance, both discounted. 
            Under cash funding the discount rate is the funding cost of using excess cash,
            i.e. the opportunity cost of tying up capital which equals the risk free rate.
            The \FVA/ under cash funding is:
                \begin{equation}
                    \FVA/_{\text{cash}} 
                        = u^{\ast}_{\text{cash}} - u
                    \label{fva-cash-financing}
                \end{equation}

            If the firm has as objective to maximize shareholder value, 
            it will not trade unless the upfront payment to the counterparty is at most the breakeven value,
            or, conversely, unless the upfront payment to the firm from the counterparty is at least the breakeven value.

\end{document}
% !TEX root = ../main.tex
\documentclass[../main.tex]{subfiles}

\begin{document}
    \subsection{Defining FVA}
        Still remains at this point to properly define \FVA/, at least in a technical manner.
        According to \cite{ADS2019} there are multiple ways of calculating \FVA/ used in practice and theory,
        and a couple of viable definitions will be explained in this section.

        \subsubsection{FVA as the promised excess funding cost}
            Since the firm borrows funds for the project at a credit spread in excess of the risk free rate, 
            the shareholders pays an additional rate when the firm does not default. 
            This is a form of funding cost and argues in favor of defining the \FVA/ as
            the present value of the funding cost in excess of the risk free rate paid by the shareholders:
            \begin{align}
                \FVA/ 
                &= \discountfactor \left(
                    U(q) (1 + \rfrate + s(q))
                    - U (1 + \rfrate)
                \right) 
                \nonumber \\
                &= \discountfactor U(q) s(q)
                \label{eqn:fva-as-promised-excess-funding-cost}
            \end{align}
            Since the entire credit spread is compensation for credit risk,
            the \FVA/ defined here is the compensation provided to the creditors,
            for the possibility that the firm might default. 
            If the firm defaults, this amount is not to be paid and the \FVA/ is therefore, in addition, 
            equal to the expected benefit to the firm from defaulting, also known as the DVA. 
            That the \FVA/ equals the DVA when credit spreads are merely compensation for credit risk, 
            is shown by \cite{HullWhiteFVA}.

\end{document}
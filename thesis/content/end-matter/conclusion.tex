% !TEX root = end-matter.tex
\documentclass[main.tex]{subfiles}

\begin{document}
    \part{Conclusion}

    As described by the title, this paper sat out to study the implications
    of funding costs in derivatives and the impact of accounting for them with \FVA/s.
    This included researching why funding costs appear in the first place,
    i.e. what the source of funding costs in financial derivatives are.
    
    \textbf{\researchQuestionFundingCosts}\\
    When managing a trade position, a derivatives dealer may need to obtain funding 
    to address the financing requirements of the position.
    Financial institutions cannot obtain external funding for OTC derivatives at risk-free rates;
    hence, funding comes at a cost due to the dealer's borrowing rate.
    This cost is \textit{the funding cost}.
    The financing needs vary, but the paper has especially considered
    the cash demands from the hedges of a trade position
    and from the calls for collateral of a trade position.
    \\
    Replications of OTC derivatives are often imperfect,
    and the cash flows from a trade and its hedge might not perfectly offset each other.
    When the incoming cash flow does not fully cover the outgoing cash flow,
    the dealer will have to supply additional funding.
    \\
    In addition to unaligned cash flows, 
    a trade and its hedge might also have asymmetrical collateral agreements,
    such that collateral calls from one trade cannot be covered by rehypothecated collateral from the other.
    \\
    In summary, funding costs exist in OTC derivatives
    because managing them requires financing 
    that has to be obtained at costs above the risk-free rates.

    \textbf{\researchQuestionFvaDebate}\\
    Increases in banks' borrowing rates after the financial crisis,
    have led some institutions to adopt valuation adjustments, \FVA/s, to account for funding costs.
    Financial institutions and their traders argue that \FVA/s are necessary,
    since the financing requirements of OTC derivatives provide a very tangible cost,
    which they will necessarily incur.
    If an \FVA/ is not applied to a derivative that require financing,
    its valuation will leave out actual costs, 
    which could otherwise have influenced the trading decision.
    \\
    On the opposite side of the debate is theoreticians.
    They argue that the valuation of a project should depend on its riskiness
    and not the riskiness of the institution that undertakes it.
    This argument implies that dealers should not be using their funding curve to establish 
    valuations, i.e. they should not be applying \FVA/s.

    The paper has not made an attempt to settle the dispute about applying \FVA/s,
    but has merely accounted for the arguments of each side.
    However, in support of this paper's relevance,
    it has been concluded that \FVA/s are a necessity 
    when determining a trade's \textit{value to the dealer}.
    This conclusion is not necessarily in conflict with either side of the debate,
    but points out a part of the topic where theoreticians and practitioners might not disagree.
    In addition, it motivates the final research question about applying \FVA/.

    \textbf{\researchQuestionFvaImplications}\\
    By leveraging the simple setup of a structural model,
    the paper was able to study how stakeholders were affected by a firm obtaining a new project.
    A new project creates different incentives for shareholders, creditors and the firm in general.

    If a firm obtains a project without considering funding costs, 
    i.e. without applying an \FVA/,
    the firm might transfer wealth between its shareholders and creditors.
    If the project needs financing, 
    the analysis showed that the funding costs 
    of obtaining the project are borne by the shareholders.
    The funding costs are the extra payoff that the financing entity requires
    for loaning unsecured to a firm that might default.
    Therefore, the shareholders are charged with the funding cost, when the firm does not default.
    \\
    The welfare lost by shareholders is transferred to the legacy creditors,
    who experience a lower loss rate when the firm defaults, 
    since the project has increased the asset base.
    Hence, obtaining projects can result in a free rider problem, 
    as shareholders pay the funding costs of financing the project,
    but do not enjoy the benefits from the project.
    Those benefits are instead enjoyed by the firm's creditors. 
    \\
    This mechanism proved to be a friction for the firm, 
    since it restricted what projects could be obtained if shareholders' wealth should be maintained.
    Even when a positive net present value investment was offered to the firm, 
    it was not necessarily able to preserve its shareholders' wealth.
    \\
    That is the implication for a firm of not using \FVA/s.

    Applying an \FVA/ to the valuation of a project is a way for the firm 
    to align its interests with its shareholders'.
    When a project has funding costs,
    the firm can use an \FVA/ to evaluate
    whether a given price is a losing trade for its shareholders or not.
    By doing so, the firm can ensure that it obtains the project at a high enough discount
    to offset the funding costs and maintain the shareholders' wealth.

    The magnitude of the shareholders' wealth loss depends both on the project's structure
    but also highly on the way it is financed. 
    The derived pecking order showed that shareholders prefer 
    financing with existing cash over financing with debt issuance, 
    and financing with debt issuance over financing with equity issuance.
    Also, collateralization agreements can greatly affect the funding costs,
    as the firm might also need to post collateral, which again needs financing.

    The opposite of funding costs, funding benefits, occur when projects provide cash surpluses.
    In that case, the funds can be used to cover financing needs elsewhere in the organization;
    then, potentially costly funding does not have to be obtained.
    \\
    Projects with funding benefits increase the shareholders' wealth,
    and their financing preferences are reversed, 
    such that they prefer buying back equity, over retiring debt, over investing in risk-less bonds.

    In conclusion, if the firm does not apply \FVA/s it risks putting the shareholders at a loss
    corresponding to the funding costs of the project.
    By using \FVA/s the firm can align its interest with its shareholders,
    such that it avoids trading decisions that deteriorates their wealth. 

\end{document}
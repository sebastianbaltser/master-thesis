% !TEX root = end-matter.tex
\documentclass[main.tex]{subfiles}

\begin{document}
    \section{Conclusion}

    As described by the title, this thesis sat out to study the implications
    of funding costs in derivatives and the impact of accounting for them with \FVA/s.
    This included researching why funding costs appeared in the first place,
    i.e. what the source of funding costs in financial derivatives were.
    
    When managing a trade position a derivatives dealer may need to to obtain funding 
    to address financing needs of maintaining the position.
    Financial institutions cannot obtain funding for OTC derivatives at risk-free rates;
    hence, funding comes at a cost due to the dealer's borrowing rate.
    This cost is \textit{the funding cost}.
    The financing needs vary, but the thesis have especially considered
    the cash demands from the hedges of a trade position
    and from the calls for collateral of a trade position.
    Replications of OTC derivatives are often imperfect,
    and the cash flows from a trade and its hedge might not perfectly offset each other.
    When the incoming cash flow does not fully cover the outgoing cash flow,
    the dealer will have to supply additional funding.
    In addition to unaligned cash flows, 
    a trade and its hedge might also have asymmetrical collateral agreements,
    such that collateral calls from one trade cannot be covered by rehypothecated collateral from the other.
    In summary, funding costs exists in OTC derivatives
    because managing them requires financing 
    that has to be obtained at costs above the risk-free rates.
\end{document}
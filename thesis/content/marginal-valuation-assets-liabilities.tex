% !TEX root = ../main.tex
\documentclass[../main.tex]{subfiles}

\begin{document}
    \subsection{Marginal valuation of corporate assets and liabilities}
        Consider a firm whose assets, $A$, and liabilities, $L$, have payoff at time 1 before any other additional trades or transactions.        
        The firm will default at event $\mathcal{D}_K=\{A + K<L\}$, i.e. when the firm no longer have enough assets to cover the liabilities, including the firm's debt, at which distress costs may occur.
        $K$ represents the project(s) that the firm has entered, which initially is assumed to be none.
        The distress costs refer to the excess expenses from the usual business costs, that the firm faces when it's unable to meet its financial obligations.
        The net asset value of the firm after defaulting, with the distress costs comprised, is $\kappa A$ for some recovery rate, $\kappa$.
        The market values of the firm's debt and equity are then computed respectively as:
        \begin{align}
            \pi(D) &= \discountfactor \mathbb{E}^{\mathbb{Q}}\left[\max\left(A-L,0\right)\right]\\
            \pi(S) &= \discountfactor \mathbb{E}^{\mathbb{Q}}\left[\mathbbm{1}_{\{\mathcal{D}_{K}\}}\kappa A + \mathbbm{1}_{\{\mathcal{D}_{K}^c\}}L\right]
        \end{align}
        where $\mathcal{D}_{K}^c = \{A + K \geq L\}$ is the event of no default.

        Now consider a new potential investment, that the firm has a possibility to undertake.
        The investment has a payoff, $Y$, which can both be negative and positive in different states in the economy (for instance a swap contract).
        The positive part of the payoff, $Y^{+}=\max\left(Y,0\right)$, is perceived as an asset of the firm, and is measured net of any potential losses due to credit risk of the counterparty.
        In addition, the negative part of the payoff, $Y^{-} = \max \left(-Y,0\right)$, is a contingent liability, and is also recognized as the contractual amount due, before taking the firm's credit risk into consideration.
        Supposing the contingent liability is fully secured, e.g. by collateralization, the total value to the firm of the financial instrument is then given by:
        \begin{equation}
            \pi(Y) = \discountfactor \mathbb{E}^{\mathbb{Q}} \left[Y\right]
        \end{equation}

        If, however, the contingent liability, $Y^{-}$, is not fully secured, a specification of how the associated counterparty recovers, in case of the firm defaulting, is needed.
        For simplicity it's assumed that the portion of unsecured contingent liabilities acquired by the firm ranks pari passu with each other, such that the various claimants' default recoveries are pro rata with their claim amounts.
        In practice these liabilities often rank pari passu with the firm's senior creditors as well.
        With $\mathcal{C}$ representing the firm's cash flows related to the portion of a total of $N$ unsecured contracts with different claimants, the value to the firm of these cash flows is given by:
        \begin{equation}
            \pi(\mathcal{C}) = \discountfactor \mathbb{E}^{\mathbb{Q}}\left[
            \sum_{i=1}^{N} \left(
                \mathbbm{1}_{\mathcal{D}_{Y}^{c}}Y_{i}
                + \mathbbm{1}_{\mathcal{D}_{Y}} Y_{i}^{+}
                - \mathbbm{1}_{\mathcal{D}_{Y}} \rho_{i} \kappa A
            \right)\right]
        \end{equation}
        where $Y = \sum_{i} Y_{i}$, and $\rho_{i} = Y_{i}^{-}/(L + Y_{i}^{-})$ is the pro rata share of contingent liability $i$.

        The analysis can be extended to consider swap positions that include both secured as well as unsecured components.
        For this motive the payoff is divided in two, such that $Y=Y_1 + Y_2$, where $Y_1^{-}$ reflects the secured contingent liability, and $Y_2^{-}$ reflects the unsecured contingent liability.
        The financial position is still assumed to pay payoff at time 1 before considering the credit risk of the firm, and the unsecured contingent liability related to claimant $i$ is still assumed to rank pari passu with all other unsecured creditor claims.\\
        For the contingent liability, $Y_1^{-}$, to be secured, $A+Y_1 > 0$ is a necessary condition, and therefore assumed.
        $\mathcal{C}$ now reflects the cash flows of the firm, where each swap position can have both secured and unsecured positions, and the firm's valuation of the associated net time 1 cash flows is given by:
        \begin{equation}
            \pi(\mathcal{C}) = \discountfactor \mathbb{E}^{\mathbb{Q}}\left[
            \sum_{i=1}^{N} \left(
                Y_{i,1}
                + \mathbbm{1}_{\mathcal{D}_{Y}^{c}}Y_{i,2}
                + \mathbbm{1}_{\mathcal{D}_{Y}} Y_{i,2}^{+}
                - \mathbbm{1}_{\mathcal{D}_{Y}} \kappa (A + Y_{i,1}) \rho_{i}
            \right)\right]
        \end{equation}
        where $\rho_{i} = Y_{i,2}^{-}/(L + Y_{i,2}^{-})$ is the pro rata share of contingent liability $i$.
        
\end{document}
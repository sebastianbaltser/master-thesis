% !TEX root = ../main.tex
\documentclass[../main.tex]{subfiles}

\begin{document}
    \subsection{Marginal valuation of corporate assets and liabilities}
        Consider a firm whose assets, $A$, and liabilities, $L$, have payoff at time 1 before any other additional trades or transactions.        
        The firm will default at event $\mathcal{D}_K=\{A + K<L\}$, i.e. when the firm no longer have enough assets to cover the liabilities, including the firm's debt, at which distress costs may occur.
        $K$ represents the project(s) that the firm has entered, which initially is assumed to be none.
        The distress costs refer to the excess expenses from the usual business costs, that the firm faces when it's unable to meet its financial obligations.
        The net asset value of the firm after defaulting, with the distress costs comprised, is $\kappa A$ for some recovery rate, $\kappa$.
        The market values of the firm's debt and equity are then computed respectively as:
        \begin{align}
            \pi(D) &= \discountfactor \mathbb{E}^{\mathbb{Q}}\left[\max\left(A-L,0\right)\right]\\
            \pi(S) &= \discountfactor \mathbb{E}^{\mathbb{Q}}\left[\mathbbm{1}_{\{\mathcal{D}_{K}\}}\kappa A + \mathbbm{1}_{\{\mathcal{D}_{K}^c\}}L\right]
        \end{align}
        where $\mathcal{D}_{K}^c = \{A + K \geq L\}$ is the event of no default.
        
        Now consider a new potential investment, that the firm has a possibility to undertake.
        The investment has a payoff, $Y$, which can both be negative and positive in different states in the economy (for instance a swap contract).
        The positive part of the payoff, $Y^{+}=\max\left(Y,0\right)$, is perceived as an asset of the firm, and is measured net of any potential losses due to credit risk of the counterparty.
        In addition, the negative part of the payoff, $Y^{-} = \max \left(-Y,0\right)$, is a contingent liability, and is also recognized as the contractual amount due, before taking the firm's credit risk into consideration.
        Supposing the contingent liability is fully secured, e.g. by collateralization, the total value to the firm of the financial instrument is then given by:
        \begin{equation}
            \pi(Y) = \discountfactor \mathbb{E}^{\mathbb{Q}} \left[Y\right]
        \end{equation}
\end{document}
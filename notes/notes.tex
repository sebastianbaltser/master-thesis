\documentclass[10pt,a4paper]{article}

\usepackage[utf8]{inputenc}
\usepackage[hidelinks]{hyperref}
\usepackage[english, danish]{babel}
\usepackage[left=3.00cm, right=3.00cm, top=3.00cm, bottom=3.00cm]{geometry}
\usepackage{parskip}

\usepackage{amsmath}
\usepackage{amsfonts}
\usepackage{amssymb}
\usepackage{float}
\usepackage{bbm}

\begin{document}
    \tableofcontents
    \section{Thesis Ideas}
        \begin{itemize}
            \item Define the framework from Andersen Duffie Song and go through the sections of Hillion but using an interest rate swap instead of an option.
            \item Has the debate on whether FVA should be accounted for in the prices or not been settled? If it is still an ongoing discussion we could open the dissertation with a that discussion, presenting arguments from each side. 
            \item How should FVA be defined and interpreted? What are the implications, pros/cons, etc. of the three different definitions defined by Hillion?
            \item Should prices be adjusted for FVA or should they not.
        \end{itemize}

        \subsection{Thesis outline}
            \begin{enumerate}
                \item Review and describe the FVA debate by presenting arguments from each side. Maybe end the chapter by reaching a conclusion that will be used throughout the remaining thesis. Should amount to approximately $10\%$ of the thesis.
                \item Define a framework that can account for FVA and define how it should be calculated. Should amount to approximately $50\%$ of the thesis.
                \item Use the previously defined theory by implementing a model for calculating FVA. Use the model to derive interesting properties of FVA and study it's behavior under different circumstances, i.e. by simulated examples. Should amount to the remaining thesis excluding final discussions.
            \end{enumerate}

    \section{Funding Value Adjustments}
        FVA is the cost to the dealer's shareholders for financing upfront counterparty cash payments, variation margin payments, and collateral requirements. This cost to shareholders is offset at least in part by a change in the value of dealer creditor claims. The sum of these value effects on shareholders and creditors is a change in the value of the dealer's frictional financial distress costs. 

    \section{Andersen Duffie Song Model}

        Andersen, Duffie and Song show that funding value adjustments are the costs to the shareholders for financing up-front counterparty cash payments and collateral requirements, and they derive the price that makes the shareholders indifferent to a new derivative trade.
        
        The firm considers buying a quantity $q$ of a new position and they want to know the impact of this investments on the firm's shareholders. The investments cost for $q$ units is $U(q)$. The payoff of the investment at time $1$ is given by $Y=Y_{1} + Y_{2}$, where $Y_{1}^{-}$ is secured and $Y_{2}^{-}$ is unsecured.

        To finance the cost $U(q)$ the firm must issue new debt, which has a credit spread of $s(q)$. If $U(q) <0$ the firm receives the cash and uses it to retire debt. The contractual new debt payback at time 1 is then $(R+s(q))U(q)$. Shareholders receive $A + qY - L - U(q)(R+s(q))$ unless this amount is negative, in which case the firm defaults and shareholders get nothing. The marginal increase in the value of the firm's equity per unit investment, is therefore:
            \begin{align}
                G &= \frac{\partial \mathbb{E}^{\mathbb{Q}}\left[\delta(A + qY - L - U(q)(R+s(q)))^{+}\right] }{\partial q} \\
                \intertext{And due to Proposition 1:}
                &= p^{\ast}\pi + \delta \text{Cov}\left(\mathbbm{1}_{D}, Y\right)  - \Phi
            \end{align}
        where:
            \begin{enumerate}
                \item $p^{\ast}=\mathbb{P}^{\mathbb{Q}}\left(D^{c}\right) $ is the risk-neutral survival probability of the bank (firm???).
                \item $\pi = \delta \mathbb{E}^{\mathbb{Q}}\left[Y\right] - u $ is the marginal profit on the trade for a hypothetical risk-free dealer.
                \item $\text{Cov}\left(\mathbbm{1}_{D}, Y\right)$ is the marginal cost of asset substitution to shareholders of investing in an asset whose payoff is positively correlated with the firm's default, given that shareholders give up all payoffs to creditors in the event of default. 
                \item $\Phi = p^{\ast}\delta u S$ is defined to be the FVA. This is the present value to shareholders of their share of the net financing costs, $uS$. Shareholders pay these financing costs iff the firm survives.
            \end{enumerate}

        \subsection{How Funding Costs Affect Swap Valuation}
            How does FVA impact the swap prices that a dealer would quote for its shareholders to break even. A swap is a contract that promises some underlying floating payment $X$ in exchange for some fixed payment $K$. The dealer pays fixed and receives floating and thus receives $X-K$ at time $1$, before considering counterparty default. 

            \subsubsection{Valuing Unsecured Swaps with Up-Front Payments}
                In this section the swap is assumed to be fully unsecured, i.e. not covered by collateral. $B$ is the client default event at which the dealer recovers $\beta$ of $(X-K)^{+}$.

                A position of size $q$ requires the dealer to make an initial payment of $U(q)$. Positive payments is preferably funded by excess balance sheet cash, and negative payments is preferably used to retire equity. This is not possible in practice due to constraints, therefore positive payments are financed by issuing debt. If the dealer does not default the payment to the dealer at time 1, per unit notional position is:
                    \begin{equation}
                        Y = y(K) \equiv X - K \gamma (X-K)^{+}
                    \end{equation}
                With $\gamma=(1-\beta)\mathbbm{1}_{B}$ being the fractional counterparty default loss. The effective payoff at time 1 of the swap to the dealer, including potential default of the dealer, is:
                    \begin{equation}
                        \mathcal{C}(q) 
                            = \underbrace{q(X-K)}_{\text{No defaults}}
                            - \underbrace{q\gamma(X-K)^{+}}_{\text{Client defaults}}
                            + \underbrace{(1-\kappa\rho(q))}_{\text{Pro rata}}
                            \underbrace{q(X-K)^{-}}_{\text{Dealer defaults}}
                            \mathbbm{1}_{\mathcal{D}(q)}
                    \end{equation}
                with $\mathcal{D}(q)$ is the dealers default event after considering the new position. The fair value of the swap is $\mathcal{V}(q) = \delta \mathbb{E}^{\mathbb{Q}}\left[\mathcal{C}(q)\right] $.

                Proposition $2$ shows that the marginal value $v=\frac{\partial \mathcal{V}(q)}{\partial q} |_{q=0}$ of the swap payoff at time 1, after paying $U(q)$, does not depend on the financing strategy. But in general the value $\mathcal{V}(q)$ depends on the financing method since the financing method influences the default distress costs, $\kappa$. 

                If $\kappa=1$, choosing $q$ such that $U(q)=v$ makes a total claimant on the dealer's balance sheet indifferent to entering the swap transaction. The dealer will try to maximize shareholder value. Under debt financing the upfront payment for the swap that is breakeven for dealer shareholders, i.e. $G=0$, is:
                    \begin{equation}
                        v^{\ast} = \frac{\mathbb{E}^{\mathbb{Q}}\left[Y\right]}{R+S} - \frac{\text{Cov}^{\mathbb{Q}}\left(\mathbbm{1}_{D}, Y\right)}{p^{\ast}\left(R+S\right) }
                    \end{equation}
                If the dealer's default indicator and the swap cash flow are uncorrelated under the risk free measure, then:
                    \begin{equation}
                        v^{\ast} = (v - \text{DVA})\frac{R}{R+S} 
                    \end{equation}
                meaning that the shareholder breakeven upfront price for entering the swap is an adjustment of the fair market value, $v$, that:
                    \begin{enumerate}
                        \item \textbf{Removes the DVA from $v$.} This reflects the lack of any shareholder benefit from paying the swap counterparty less than the contractually promised amount when the dealer defaults. Because shareholders receive nothing at default.
                        \item \textbf{Substitutes the dealer's unsecured discount rate $R+S$ for the risk-free rate $R$.} This reflects the funding cost to shareholders, who must pay the credit spread $S$ to the new creditors without gaining any marginal benefit ($G=0$) from the "right" to default on the new debt.
                    \end{enumerate}
                $v^{\ast} - v$ is the net shareholder cost of entering the swap. When $\kappa=1$ this amount is entirely transferred to the dealer's creditors. If $\kappa<1$ the cost is not entirely transferred to the dealer's creditors. 

            \subsubsection{Dealer Quotation and FVA for Unsecured Swaps}
                The dealer would not trade the swap unless the upfront payment to the dealer is at least $v^{\ast}$, in which case the firm as a whole would make a trading profit of $v-v^{\ast}$. DVA always decreases $v^{\ast}$ relative to $v$ but the funding component depends on DVA relative to $v$:
                    \begin{itemize}
                        \item $v < \text{DVA}$: The funding cost component increases $v^{\ast}$ relative to $v$. 
                        \item $v > \text{DVA}$: The funding cost component decreases $v^{\ast}$ relative to $v$.
                    \end{itemize}

        \subsection{Valuation Adjustments for Long-Term Swaps}
            The dealer defaults at time $\tau_{D}$ with conditional mean arrival rate $\lambda_{D}(t)$ at time $t$. The fractional loss to the creditor claim in this case is $\ell_{D}(t) \in [0, 1]$. The dealer's short term credit spread is $S_{t} = \lambda_{D}(t)\ell_{D}(t)$ at time $t$. The client swap counterparty has the same default risk characteristics but denoted with subscript $C$.

            The market value of the default-free version of the swap at time $t<T$ is:
                \begin{equation}
                    V_{t} = \mathbb{E}_{t}^{\mathbb{Q}}\left[\sum_{j=\eta(t)+1}^{N} \delta_{t,t_{j}}C_{j}\right] 
                \end{equation}
            The CVA and DVA are:
                \begin{align}
                    \text{CVA} &= \mathbb{E}^{\mathbb{Q}}\left[
                        \mathbbm{1}_{\{T > \tau_{C} ~\vee~ \tau_{D} > \tau_{C}\}} \delta_{0,\tau_{C}} \ell_{C} V(\tau_{C})^{+}
                    \right]  \\
                    \text{DVA} &= \mathbb{E}^{\mathbb{Q}}\left[
                        \mathbbm{1}_{\{T > \tau_{D} ~\vee~ \tau_{C} > \tau_{D}\}} \delta_{0,\tau_{D}} \ell_{D} V(\tau_{D})^{+}
                    \right] 
                \end{align}
            The market value of the swap, i.e. the total value of the swap cash flows to both equity and debt claimants, is:
                \begin{equation}
                    v \equiv V_{0} - \text{CVA} - \text{DVA}
                \end{equation}
            No FVA is assignable to this total swap market value. The market value is thus obtained by subtracting the CVA net of the DVA from the value $V_{0}$ of a default-free swap. The upfront payment $v^{\ast}$ that would leave shareholders indifferent to the swap transaction is approximated by:
                \begin{equation}
                    v^{\ast} \approx V_{0} - \text{CVA} - \Phi(V_{0}) - \Psi = v - \text{DVA} - \Phi(V_{0}) - \Psi
                \end{equation}
            $\Phi(V_{0})$ is the FVA and $\Psi$ is the MVA. The DVA is subtracted from the market value since it is of no value to shareholders. 
        
    \section{Hillion}
        The goal of this paper is to examine the issue of funding and derivatives valuation using a simple corporate finance approach. The paper argues that:
            \begin{enumerate}
                \item Funding costs and benefits should not be accounted for in the valuation of derivatives.
                \item The derivatives' funding costs and benefits leave the market value of the bank unaffected.
                \item The derivatives' funding costs and benefits generate wealth transfers between the shareholders and the creditors.
            \end{enumerate}

        \subsection{Funding Costs Adjustments}
            In this section, the bank, B, is assumed to purchase the option contract from the riskless counterparty, C. The new asset on the bank's balance sheet raises credit and funding issues.

            When the option's upfront is payed, there is no counterparty credit risk to neither B or C, since C is riskless and B does not have any outstanding debt to C. The option is a derivative receivable to the bank.

            The bank must decide to fund the purchase by issuing new debt or by issuing new equity.

            \subsubsection{Debt Funding}
                New debt has a cost above the risk-free rate. The issue is to determine whether the bank should pay less for the option contract by an amount equal to the funding cost. To the new creditors the option is a zero NPV investment, since the debt that they are buying must be fairly priced. To the legacy creditors, the option is a positive net-present-value investment. To the shareholders, the option is a negative NPV investment.

                Since projects with a NPV of zero do not create value and the Modigliani-Miller irrelevance Proposition suggests that the value of a firm is not affected by financing decisions, the increase in the legacy debt value must be exactly offset by an equal decrease in the equity value, if the derivative contract must have no impact on the bank value. Thus wealth is transferred from the shareholders to the legacy creditors. For this reason the price of the option payed by the bank cannot be less FVA, since that would create value for the bank at the expense of the counterparty.

                Three definitions and interpretations of FCA emerge:
                    \begin{enumerate}
                        \item Funding costs are the present value of the excess funding costs paid by the shareholders. These funding costs are borne by the shareholders as long as the bank does not default. These costs are offset by the gains earned by the legacy creditors when the bank defaults. 
                        \item Funding costs are the \textit{expected} loss experienced by the shareholders, that is, the decrease in the equity value at time 0, due to the derivative. This corresponds to the wealth transfer from the shareholders to the bondholders. This interpretation captures the expected funding costs, i.e., the funding costs times the probability that the bank does not default. \textit{This is the definition used by Andersen Duffie Song.}
                        \item Funding costs are the difference between the derivative's value and it's price, with the latter calculated to make the shareholders indifferent. This funding cost is paid by the counterparty as a donation to the bank. Like the shareholders, the new creditors are entering an investment with zero NPV, but the legacy creditors are better of by the donation. 
                    \end{enumerate}

            \subsubsection{Equity Funding}
                In the previous section the option premium is funded by issuing debt. The issue now is whether the previous results is affected by the funding instrument.

                Three conclusions emerge:
                    \begin{enumerate}
                        \item The value of the derivative contracts should not be adjusted for funding costs, regardless if the funding is from debt or equity.
                        \item The value of the bank is not affected by the funding costs required by derivative receivables. Derivative contracts make the legacy creditors better off at the expense of the shareholders, regardless if the funding is from debt or equity.
                        \item Shareholders are worse off with equity funding than with debt funding, because creditors does not have to share with other creditors in the default state, which increases the value of debt even further.
                    \end{enumerate}

     \section{Ruiz}
        Funding risk arises wherever a cashflow occurs in an institution. If this is an outflow it needs to be funded, and if it's an inflow it can either reduce the funding needed, or be lent out. Ruiz talks about two distinguishable main sources of funding risk:

        There are two discussions nested in the FVA debate: 1) Whether FVA can create arbitrage opportunities and 2) whether including DVA and FVA is double-counting.

        CVA is driven by how much it costs to hedge out the default risk from a counterpart. The CVA can be seen as the difference in cost between hedging default risk from a riskless counterparty and hedging default risk from the actual counterparty. The liability side of CVA, i.e. DVA, can be seen as the cost of ensuring that we do not default. In an ideal world this cost is going to be exactly the same as the cost to our counterparty for hedging their credit-exposure to us. 

        We have a derivatives portfolio with a risky counterparty that is worth $\$100$ in our favor. Suddenly the credit spread of the counterparty rises from $100bp$ to $1000bp$, and we decide that we want to close our position. The counterparty will agree to close the position by paying us $\$80$ and we agree to these terms, since the would rather lose $\$20$ now than risking losing it all at a later time. The counterparty has thus monetized it's part of it's DVA, which is possible since it was more expensive for us to buy credit protection for a company with a credit spread of $1000bp$.

        There can be two sources of FVA:
        \begin{description}
            \item[FVA from collateral asymmetry:] A dealer sells an OTC derivative to a counterparty with whom it has a CSA agreement such that collateral is posted each day/week/month. The dealer pays $OIS$ for the collateral posted. The dealer still faces the market risk from the derivative so it creates a market risk hedge at the exchange. The exchange requires full collateralization. Any collateral that needs to be posted to the exchange must be borrowed from the dealer's funding institution. The price of borrowing from the funding institution is $OIS + \text{funding spread}$, which is of course the dealers funding cost. The collateral posted at the exchange earns $OIS$. Some of the collateral to the exchange can be covered by the collateral posted by the counterparty, but there is a asymmetry between these collateral postings since the counterparty only marks to market every day/week/month. The difference between the two collateral postings is the amount needed to be borrowed from the funding institution and the dealer will therefore lose it's funding spread on the amount borrowed. 
            
            Therefore a dealer is going to have a funding cost due to the asymmetry between collateral needs between the OTC derivative and the hedging sides. Dealers need to account for this by putting the funding cost as an adjustment to the value of the OTC derivative.

            \item[FVA from derivative cashflow] The assumptions in the BS model states that an unlimited amount of cash can be bought and sold at the risk-free rate. However cashflows from derivatives must be funded at a non-risk rate. Buying options will require funding the option premium by borrowing from the funding institution at the rate $OIS + \text{funding spread}$. Receiving cashflows e.g. from a swap can lower the funding needs elsewhere in the instution and will therefore effectively earn a rate of $OIS + \text{funding spread}$ by reducing the amount needed to be borrowed from the funding institution.
        \end{description}

        For the reasons mentioned it is common for institutions to reduce the price of a deal with an FVA component such that:
            \begin{equation}
                V = P_{risk-free} - CVA - FVA
            \end{equation}
        However this could be viewed as double-counting, since the liability side of CVA (DVA) can be viewed as a funding component.
        
        \begin{description}
            \item[Double-counting - Ruiz]
            When accounting for both CVA and FVA, we do \textbf{not} double-count the default probability. The confusion may arise from the fact that the liability side of the CVA can be understood as the funding part of CVA. The liability side of the CVA, $V^{CVA_l}$, is the cost of ensuring that we do not default. It's used to account for the cost of hedging against default risk by the counterparty. This value can be approximated as sum across each expected negative exposure, $ENE$, on the CDS multiplied by the credit spread. That is, cumulated on each group of transactions with a single counterparty, how much the dealer on average can owe if they default multiplied by the premium that the protection buyer pays at each time period until maturity in order to insure against a credit event:
            \begin{equation}
                V^{CVA_l} \approx \sum_{\text{nettingset}} ENE_{\text{nettingset}} \cdot s^{CDS}
            \end{equation}
            On the other hand, the value of the FVA, $V^{FVA}$, is the computed as the sum of the expected positive exposure on the entire portfolio, $EPE$, and the initial margin requirement, $IM$, multiplied by the funding spread, $s^{fund}$:
            \begin{equation}
                V^{FVA} \approx (EPE_{\text{portfolio}}+IM)\cdot s^{fund}
            \end{equation}
            The $EPE$ represents how much on average the dealer can be owed, if the counterparty defaults. Ruiz argues, that these two calculation are completely different measurements.
            \item[Double-counting - Andrew Green]
            Consider a risky counterparty, $C$, to the dealer (who for simplicity reasons is considered riskfree). The counterparty borrows a premium, $P$, paid by the dealer today, and the dealer then receives a cashflow, $K$, at maturity time, $T$. Here, the counterparty obviously receives a DVA benefit on the bond from the risk that it might default between today and time $T$, $(\tau_C < T)$ while the dealer faces a CVA cost associated with the agreement, again due to the counterparty's risk of defaulting. The discount rate is assumed to comprise the riskfree rate as well as the funding spread, $s_{C} = \gamma_C + \lambda_C$, where $\gamma_C$ is the bond-CDS basis, and $\lambda_C$ is the default intensity for the countreparty. Also, the recovery rate is assumed to be zero. From the dealers perspective, the value of the transaction is:
            \begin{align}
                V_{L} &= \mathbb{E}\left[
                    e^{rT}K \mathbbm{1}_{\{\tau_C > T\}}
                \right] - P \\
                &= e^{-rT} K e^{-\lambda_{C}T} - P
            \end{align}
            The dealer discounts the cash flow at the risk free rate, $r$. Had the counterparty been risk free the value to the dealer would be:
                \begin{align}
                    V_{L}^{\ast} = e^{-rT}K - P
                \end{align}
            Yielding the CVA:
                \begin{align}
                    \text{CVA}_{L} 
                        &= V^{\ast}_{L} - V_{L} \\
                        & = e^{-rT}K(1-e^{-\lambda_{C}T})
                \end{align}
            The counterparty discounts the cash flow at the discounting rate and to that party the transaction has the value:
                \begin{align}
                    V_{C} 
                    &= - \left(
                        \mathbb{E}\left[
                            e^{-(r+s_{C})T}K \mathbbm{1}_{\{\tau_{C}>T\}}
                        \right] - P
                    \right) \\
                    &= - e^{-rT}Ke^{-2\lambda_{C}T} e^{-\gamma_{C}T} + P
                \end{align}
            The unadjusted value, i.e. the value without credit risk and funding spread, would be: 
                \begin{align}
                    V_{C}^{\ast} 
                    &= -e^{-rT}K + P
                \end{align}
            Why the total price adjustments, $\text{XVA} = \text{CVA} + \text{DVA} + \text{FVA}$ would be:
                \begin{align}
                    \text{XVA}_{C}
                    &= V_{C}^{\ast} - V_{C} \\
                    &= -e^{-rT}K(1 - e^{-2\lambda_{C}T}e^{-\gamma_{C}T})
                \end{align}
            Had the funding costs been ignored the value to the counterparty would be:
                \begin{align}
                    V_{C}' 
                    &= - \left(
                        \mathbb{E}\left[
                            e^{-rT}K \mathbbm{1}_{\{\tau_{C} > T\}} 
                        \right] - P
                    \right) \\
                    &= - e^{-rT}Ke^{-\lambda_{C}} + P
                \end{align}
            And the DVA could be obtained as:
                \begin{align}
                    \text{DVA}_{C} 
                    &= V_{C}' - V_{C}\ast \\
                    &= - e^{-rT}Ke^{-\lambda_{C}} + P
                    -(-e^{-rT}K + P) \\
                    &= -e^{rT}K (1 - e^{-\lambda_{C}})
                \end{align}
            which is exactly equal to $-\text{CVA}_{L}$. These two values being exactly the same with the signs flipped ensures that the valuation is symmetric between the two parties. When considering the funding costs the valuation is asymmetric, but the concerning fact is that the total price adjustment to $V_{C}$, $\text{XVA}_{C}$, contains 2 times the default intensity of the counterparty, $\lambda_{C}$. This must mean that the FVA is also accounting for the default intensity, implying that there is an overlap between FVA and DVA.

            Asymmetric valuation means that the two parties will never agree on the value of the transaction, even at trade inception.
            
            KPMG states that conceptually FVA can be thought of as consisting of three elements according to the following equation:
                \begin{equation}
                    \text{FVA} = \text{FVA}_{\text{cost}} + \text{FVA}_{\text{benefit}} + \text{FVA}_{\text{buffer}}
                \end{equation}
            For the purposes of this discussion $\text{FVA}_{\text{buffer}}$ can be ignored. $\text{FVA}_{\text{cost}}$ will reduce the value of the derivative as percieved by an entity, while $\text{FVA}_{\text{benefit}}$ will increase it. $\text{FVA}_{\text{cost}}$ is calculated on the expected positive exposure, EPE, and $\text{FVA}_{\text{benefit}}$ is calculated on the ENE. According to KMPG $\text{FVA}_{\text{benefit}}$ has an obvious potential overlap with $DVA$.
            
            It would seem like Ruiz is defining the $\text{FVA}_{\text{cost}}$ since he is using EPE to calculate it. His conclusion is that there is no overlap. In the example by Andrew Green, the funding costs $s_{C}$ are increasing the value of the derivative as percieved by $C$ and so the example seems to be concerning $\text{FVA}_{\text{benefit}}$. Green reaches the conclusion that there is an overlap. 

            If Ruiz and Green are actually referring to two different parts of $\text{FVA}$ then their two conclusions are not necessarily opposites. In addition to the equation from KPMG, the funding costs between Ruiz and Green also seem different. Green uses the funding rate to discount future cashflows, while Ruiz seems more like comparing different interest rates. 

        \end{description}

     \section{Castagna}
        The price and the value of a derivative is not necessarily the same.

        \begin{description}
            \item[Price:] The terms that both parties agree upon when closing the deal. When both parties have an even bargaining power, they have to acknowledge the other parties risks and costs, so that the final price includes the total net risks and costs borne by both parties.
            \item[Value:] The total of the production costs, which are the present value of the costs paid to attain the intermediate and final pay-off until the expiry, considering also the costs and the losses due to counterparty credit risk, funding and liquidity premium.
        \end{description}

        When both parties operate in a perfect and frictionless market, where there are no transaction costs and counterparty risks, the price and the value of a contract will be identical.

        The FVA is the discounted value of the spread paid by the bank over the risk-free interest rate to finance the net amount of cash needed for the collateral account and the underlying asset position in the dynamic replication strategy.

    \section{Hull and White}
    \subsection{Is FVA a Cost for Derivatives Desks}
        For risk-free projects the risk-free valuation paradigm requires the use of the risk-free rate as the discount-rate. It would make sense then to use the risk-free rate for discouting fully collateralized transactions, since these are risk-free through the collateral. Nontheless some banks use the OIS rate for discounting, with the argument that the transactions are funded at the OIS rate. 

        An argument for FVA is the use of hedging and the funding cost that hedging requires. However hedging is a zero net present value investment, so entering into a hedge should not affect valuation. The valuation of an investment should depend on the risk of the project, not on the riskiness of the firm that makes the investment. Since the funding cost is directly related to the riskiness of the firm this is to say that an FVA is not appropriate when pricing investments. 

        Suppose a company has a funding rate of $r_{f} + 200bp$ and meets an opportunity to take on a nearly risk-free investment with a return of $r_{f} + 80bp$. The discount rate for the stand-alone investment is $r_{f} + 30bp$. Hull and White argues that the investment should be made since $H*(r_{f} + 80bp) / (r_{f} + 30bp) - H > 0$, i.e. the funding rate should be ignored. The investment will create shareholder value by decreasing the overall riskiness of the company, since the new investment is less risky than the average riskiness of the company's existing investments. Decreasing the overall riskiness also subsequently decreases the funding rate of the company. 

        Castagna argues that the example above only works when the funding rate of the company is the fair rate of return on the investment, $r_{f} + 30bp$. If the company has a funding rate of $r_{f} + 200bp$ the decrease in the funding rate by taking on the investment is more than offset by the loss incurred by the company in funding the return of $r_{f} + 80bp$ at an even higher rate.
    
    \subsection{FVA and Fair Value}
        Different stakeholders have different opinions on how to account for the funding costs. Traders are determined that the derivatives desk are charged the funding costs, and to not make a loss on the traders P\&L. To account for this they make an FVA on the pricing of the derivatives. Accountants state that to determine the value of derivatives we have to look at the fair value of such. However, the fair value is measured entirely on the market, and not specified to each entity distributing the derivatives. Finally, the theoreticians state that in the risk-neutral world pricing a derivative should only be valued as the expected cash flows discounted by the risk free discount rate. The funding costs are "fixed" in the sense the they can vary from different entities (and derivatives?).

        "Finance theory argues that this practice makes risky projects seem relatively more attractive and projects with very little risk seem relatively less attractive."

        The same way that the counterparty's risk is gathered in the CVA value, the market participant's own risk of defaulting is gathered in the DVA value. The same setup is applicable when funding costs are included. We then divide DVA into DVA1 and DVA2, which refer to the risk of default due to derivatives obligations and funding costs respectively.
        \begin{equation}
            \text{Portfolio value} = NDV - CVA + DVA1 + DVA2 - FVA
        \end{equation}
        For transactions that are totally uncollateralized, both FVA and DVA2 additive across the transactions.

        The inclusion of FVA in the fair value can allow for arbitrage as an end user buying an option from a bank with high funding costs and then selling the same option to a bank with low funding costs can provide a profit for all three parties.

        \textbf{SB's notes:} \\
        Should a FVA be made or not when valuing derivatives? The participants in the discussion can roughly be split up into three groups, each having different arguments and opinions:

        \begin{description}
            \item[The trader] The trader is charged the funding cost by the funding desk and must therefore make FVA. If he doesn't it will show a loss on trades that require funding.
            
            \item[The accountant] "Fair value is a market-based measurement, not an entity-specific measurement". Since FVA is taking into account different funding costs for different entities, FVA should not be made. Otherwise different banks will price the same transaction differently.
            
            \item[The theoretican] The discount rate for a project should be determined by the risk of the project. The funding cost is irrelevant. Since the discount rate is used for calculating the derivative price, the funding cost is irrelevant to the derivative price, and an FVA should not be made.
        \end{description}

            Hull \& White definition of FVA: The FVA is the difference between the no-default-value (NDV) obtained when using the risk free rate and the NDV obtained when using discounting at the dealer's funding costs. Credit risk is taken into account by CVA and DVA, which yields the portfolio value:
            \begin{equation}
                \text{Portfolio value} = \text{NDV} - \text{CVA} + \text{DVA} - \text{FVA}
            \end{equation}

            Per the example: The valuation of a derivative depends on two interest rates. 1) The rate at which a position in the underlying asset can be funded and 2) the rate at which a position in the derivative can be funded. Hedging the derivative might require buying one unit of the underlying stock, which needs to be funded at some rate. Buying the derivative might require paying the counterparty, which also needs to be funded at some, potentially different, rate.
            
            \textbf{The European Call option example:} Why is the quoted price to another dealer higher than the quoted price to an end-user?
            
            When assessing the value of a risk-free project, the risk-free rate must be used for discounting, i.e. not the funding rate. Banks will enter into different projects of varying risk, but over time the average cost of funding should remain approximately the same. Entering into risk-free projects will enable the bank to enter into riskier-than-average projects.
            
            \textbf{The Fair Value argument:} The fair value us the price that would be received to sell an asset in an orderly transaction between market participants at the measurement date. While different entities might have different funding costs, that is not relevant for the fair value of a product. The funding cost influences only the private value of a product.
    
    \section{Pricing Debt and Equity as Options}
    \subsection{The single period model}
        We are currently at time $0$ and at time $1$ there are $S$ possible states, denoted:
            \begin{equation}
                \Omega = \{\omega_{1}, \dots, \omega_{S}\}
            \end{equation}
        The true probability of a state occuring is denoted $p_{i} = \mathbb{P}\left(\{\omega_{i}\}\right)$. An model asset has a random payoff at time $1$ denoted $d=(d_{1}, \dots, d_{S})$, where $d_{i}$ is the payoff if $\omega_{i}$ is realized. The price of the asset at time $0$ is $\pi(d)$. The Arrow-Debreu price is the price of a claim that pays 1 in a particular state, and is denoted $\psi_{i}$ for the claim that pays $1$ when $\omega_{i}$ is realized and zero otherwise. The state price vector is denoted $\psi = (\psi_{i}, \dots, \psi_{S})$.

        Any payoff vector can be priced using the state price vector, i.e. a security with the payoff $d$ has the price:
            \begin{equation}
                \pi(d) = d_{1}\psi_{1} + \cdots + d_{S}\psi_{S}
            \end{equation}
        A riskless bond that pays $1$ in each state has the price $d_{0} = \omega_{1} + \cdots + \omega_{S}$ and the riskless interest rate can then be defined as:
            \begin{equation}
                r = \frac{1}{d_{0}} - 1
            \end{equation}
        For a security with payoff $d$ and price $\pi(d)$ the vector of gross returns and rate of returns can be defined as:
            \begin{align}
                R(d) &= \left(\frac{d_{1}}{\pi(d)}, \dots, \frac{d_{S}}{\pi(d)} \right) \\
                r(d) &= \left(\frac{d_{1}}{\pi(d)} - 1, \dots, \frac{d_{S}}{\pi(d)} - 1\right)
            \end{align}
        Giving rise to the expected gross return and expected rate of return:
            \begin{align}
                ER(d) &= \frac{p_{1}d_{1} + \cdots + p_{S}d_{S}}{\pi(d)} \\
                Er(d) &= ER(d) - 1
            \end{align}
        The Arrow-Debreu prices and the riskless rate also defines the risk-neutral probability distribution $\mathbb{Q}$:
            \begin{equation}
                q_{i} = (1 + r)\psi_{i}
            \end{equation}
        The price of the security with payoff $d$ can be rewritten in terms of the riskless rate and the risk neutral probability:
            \begin{equation}
                \pi(d) = \frac{d_{1}q_{1} + \cdots + d_{S}q_{S}}{1+r} = \frac{\mathbb{E}^{\mathbb{Q}}\left[d\right]}{1+r} 
            \end{equation}
        The price of a security can also be represented using a state-price deflator, $m$:
            \begin{equation}
                m_{i} = \frac{q_{i}}{p_{i}(1+r)} 
            \end{equation}
        in the following way:
            \begin{equation}
                \pi(d) = p_{1}d_{1}m_{1} + \cdots + p_{S}d_{S}m_{S}
            \end{equation}

    \subsection{Modigliani-Miller Theorem Version 1}
        Consider an economy with states $(\omega_{1}, \dots, \omega_{S})$ and state prices $\psi = (\psi_{1}, \dots, \psi_{S})$, as well as a bank which can invest in risky assets but requires funding to do so. Funding can be obtained through equity and deposits. 

        In the first example owners are willing to invest an amount of equity $S_{0}$ in the bank and the bank can attract deposits worth $D_{0}$. After obtaining the deposits, the bank can invest in risky assets worth $A_{0}$, where:
            \begin{equation}
                A_{0} = S_{0} + D_{0}
            \end{equation}
        At time $1$ depositors receives the face value, $D_{0}$ plus an interest rate depend on the size of deposits, denoted $\rho_{D_{0}}$. The bank might default in which case the depositors will not receive the entire promised amount. 

        With $A$ being the vector of time $1$ payoffs from the assets in which the bank has invested at time $0$, the price of the assets are:
            \begin{equation}
                A_{0} = \psi_{1}A_{1} + \cdots + \psi_{S}A_{S}
            \end{equation}
        The payoff to depositors is denoted $D$ and when $\omega_{i}$ is realized it is:
            \begin{equation}
                D_{i} = \min\left(D_{0}(1+\rho(D_{0})),\; A_{i}\right)
            \end{equation}
        I.e. the payoff to depositors is the promised payoff unless the assets are worth less than that in which case the depositors take over the bank. If deposits are prices according to the market value, the bank must offer $\rho(D_{0})$ satisfying:
            \begin{equation}
                D_{0} = \psi_{1}D_{1} + \cdots + \psi_{S}D_{S}
            \end{equation}
        Assuming that this is the case the payoff to the bank equity in state $i$ at time $1$ is:
            \begin{equation}
                S_{i} = \left(A_{i} - D_{0}(1+\rho(D_{0}))\right)^{+}
            \end{equation}
        I.e. equity receives the remaining assets after paying deposits or receives nothing if the bank defaults. Using the relation $S_{i} + D_{i} = A_{i}$ it can be stated that:
            \begin{gather}
                \pi(S) + \pi(D) = \pi(A) \\
                \Leftrightarrow \qquad \pi(S)=\pi(A)-\pi(D) = A_{0} - D_{0} = S_{0}
            \end{gather}
        Hence, the value of bank equity, $\pi(S)$, is independent of the amount of deposits attracted, $D_{0}$, when the deposits are priced at market rates. Therefore equity holders have no incentive to lever up the bank, since it will not change the time $0$ value of equity. Note however that if the assets in which the bank invests are risky, the riskiness of equity will depend on the chosen level of deposit funding.

    \subsection{Modigliani-Miller Theorem Version 2}
        Consider one bank with the funding profile:
            \begin{equation}
                A_{0} = S_{0} + D_{0}
            \end{equation}
        and another bank with the same total asset value but a different funding profile:
            \begin{equation}
                A_{0} = S_{0}' + D_{0}'
            \end{equation}
        Assuming that the cash flow of assets at time $1$ is divided and split between the cash flow to depositors and equity, i.e.:
            \begin{equation}
                A_{i} = S_{i} + D_{i} = S_{i}'+D_{i}'
            \end{equation}
        then the total value of the bank payoffs at time $0$ is $A_{0}$ and:
            \begin{equation}
                A_{0} = \pi(S) + \pi(D) = \pi(S') + \pi(D')
            \end{equation}
        Thus the value of the bank does not depend on the funding profile. 

        Denoting the fraction of the bank's assets that is financed by equity as $w = S_{0} / A_{0}$, and the expected return on equity as $Er(S)$. Then the total funding rate for the banks can be written as the expected return on assets from equity plus the expected return on assets from deposits. The bank must pay the return on assets and deposits, therefore they contribute to the banks funding rate:
            \begin{align}
                wEr(S) + (1-w)Er(D) 
                    &= w \frac{\mathbb{E}\left[S\right]}{S_{0}} + (1-w)\frac{\mathbb{E}\left[D\right]}{D_{0}} \\ 
                    &= \frac{S_{0}}{A_{0}} \frac{\mathbb{E}\left[S\right]}{S_{0}} + \frac{D_{0}}{A_{0}} \frac{\mathbb{E}\left[D\right]}{D_{0}} \\ 
                    &= \frac{\mathbb{E}\left[S\right] + \mathbb{E}\left[D\right]}{A_{0}} \\
                    &= \frac{\mathbb{E}\left[A\right]}{A_{0}}
            \end{align}
        And likewise:
            \begin{align}
                w'Er(S') + (1-w')Er(D') = \frac{\mathbb{E}\left[A\right]}{A_{0}}
            \end{align}
        Note first that equity is riskier than debt, therefore equity should have a higher expected return than debt. Increasing the leverage by obtaining additional deposits will increase the returns required by the depositors, $\rho(D_{0})$, since higher leverage implies that debt is riskier. Likewise decreasing the leverage will decrease the returns required by the depositors, while simultaneously decreasing the required return on equity since the bank risk decreases with leverage. Therefore, it would seem that decreasing the leverage should reduce the total funding cost, since both the funding cost on debt and on equity decreases. However, decreasing the leverage requires an increase in the fraction of equity funding, which is a more expensive funding method. The increase in relative equity funding precisely offsets the decrease in funding rates from debt and equity stemming from the decrease in risk. 

        The conclusion is that the total funding rate is independent of the funding profile, i.e. the mix of debt and equity. 

    \subsection{Example}
        \textit{Prøv derefter at konstruere et projekt som er næsten risikofrit, men som har et lovet yield på 30bp over risikofri rente, når det er korrekt prisfastsat, men antag at det tilbydes til en lavere pris, som betyder at det har et (lovet) yield på 80 bp over risikofri rente. Lav også virksomheden således at dens gæld har lovet yield på 200 bp over risikofri rente.}

        \textit{Vis at det er en god ide for virksomheden at gå ind i projektet, idet summen af gæld og egenkapital stiger mere end udgiften til projektet, men overvej om det kan finansieres alene ved gældsudstedelse eller aktiefinansiering.} 

        Consider a bank considering an almost riskfree project with a time $0$ value of $A_{0}^{\ast}$ and a theoretical yield of $r_{f} + 30bp$. However the project trades at a discount, $A_{0}$, such that the yield is $r_{f} + 80bp$. The bank can obtain deposits at a funding rate of $\rho(D_{0}) = r_{f} + 200bp$.

        The bank must obtain the funding needed for the project through a mix of deposits and issuance of equity, i.e. the budget constraint takes the form:
            \begin{equation}
                A_{0} = S_{0} + D_{0}
            \end{equation}

        The bank must use some equity funding in order to fund the project, since the funding rate of debt is higher than the yield of the project. Funding completely from debt would therefore result in a negative return. 

        We are unable to argue against the firm being completely unlevered, i.e. funding the project entirely through equity. Equity funding should be more expensive than debt funding, so there should be a tradeoff, but we are not sure how this materializes. 
        
        \textit{Hvad skal man love nye aktionærer/gældshavere for at de vil finansiere? Problemet i disse sager er, at gevinsten ved projektet ikke altid tilfalder dem der finansierer. Free-rider problemer - ringer det en klokke? Ellers må vi snakke, men først skal I forstå vedhæftede.}

        The shareholders has a claim on a fraction $\frac{S_{0}}{A_{0}^{\ast}}$ of the bank. The shareholders paid a fraction $\frac{S_{0}}{A_{0}}$ of the project costs, but their share of the bank is diluted since the project traded at a discount. The funding rate of the project to the shareholders equals the fraction of the bank owned by the shareholders, $\frac{S_{0}}{A_{0}^{\ast}}$, multiplied by the expected return on equity, $Er(S)$. The funding rate of the project to the creditors equals the fraction of the price paid by the creditors, $\frac{D_{0}}{A_{0}}$, times the funding rate, $Er(D)$. The total funding rate is therefore:
            \begin{align}
                \frac{S_{0}}{A_{0}^{\ast}} Er(S)
                + \frac{D_{0}}{A_{0}} Er(D) 
                &= \frac{S_{0}}{A_{0}^{\ast}} 
                    \frac{\mathbb{E}\left[S\right]}{S_{0}}
                + \frac{D_{0}}{A_{0}} 
                    \frac{\mathbb{E}\left[D\right]}{D_{0}} \\
                &= \frac{\mathbb{E}\left[S\right]}{A_{0}^{\ast}}
                    + \frac{\mathbb{E}\left[D\right]}{A_{0}} 
                \intertext{The relation between the value and the price of the project, $A_{0}^{\ast} = A_{0}\frac{1.008}{1.003}$, can be substituted in:}
                &= \frac{\mathbb{E}\left[S\right]}{A_{0}\frac{1.008}{1.003}}
                    + \frac{\mathbb{E}\left[D\right]}{A_{0}} \\
                &= \frac{
                    \frac{1.003}{1.008}\mathbb{E}\left[S\right]
                    + \mathbb{E}\left[D\right]
                }{
                    A_{0}
                }
            \end{align}

        The funding rate if the project had not been trading at a discount:
        \begin{align}
            \frac{S_{0}}{A_{0}^{\ast}} Er(S)
            + \frac{D_{0}}{A_{0}^{\ast}} Er(D) 
            &= \frac{S_{0}}{A_{0}^{\ast}} 
                \frac{\mathbb{E}\left[S\right]}{S_{0}}
            + \frac{D_{0}}{A_{0}^{\ast}} 
                \frac{\mathbb{E}\left[D\right]}{D_{0}} \\
            &= \frac{
                \mathbb{E}\left[S\right] + \mathbb{E}\left[D\right]
            }{
                A_{0}^{\ast}
            }
        \end{align}

        The total funding costs for the project at a discount can be obtained by multiplying the price of the project by the funding rate:
            \begin{equation}
                A_{0}
                \frac{
                    \frac{1.003}{1.008}\mathbb{E}\left[S\right]
                    + \mathbb{E}\left[D\right]
                }{
                    A_{0}
                }
                = \frac{1.003}{1.008}\mathbb{E}\left[S\right]
                + \mathbb{E}\left[D\right]
            \end{equation}
        Total funding costs for the project at the theoretical price:
            \begin{equation}
                A_{0}^{\ast} 
                \frac{
                \mathbb{E}\left[S\right] + \mathbb{E}\left[D\right]
                }{
                    A_{0}^{\ast}  
                } 
                = \mathbb{E}\left[S\right] 
                    + \mathbb{E}\left[D\right]
            \end{equation}
        All else being equal, the funding cost paid to shareholders are less when the project trades at a discount, than if the project had been trading at the theoretical price. It would seem like the shareholder bears the full cost of the fact that the project can be funded at a discount. 
        \newpage
        \rule{\textwidth}{1px}

        The payoff to the depositors is:
            \begin{equation}
                D_{1} = \min\left(
                    D_{0}(1 + r_{f} + 200bp), \;
                    A_{0}(1 + r_{f} + 80bp)
                \right)
            \end{equation}
        The value of the bank equity is the following:
            \begin{equation}
                S_{1} = \left(
                    A_{0}(1 + r_{f} + 80bp) - D_{0}(1+r_{f} + 200bp)
                \right)^{+}
            \end{equation}
        When investing in the project the bank earns the discount cashing in a return of: 
        \begin{equation}
            A_{0}^{\ast} - A_{0} = A_{0}\frac{1+r_{f} + 80bp}{1+r_{f} + 30bp} - A_{0} = A_{0}\frac{50bp}{1+r_{f}+30bp}
        \end{equation}
        at time $0$. Therefore, the sum of time $0$ values of the debt- and equity are different from the time $0$ value of the assets:
            \begin{equation}
                \pi(S) + \pi(D) = A_{0}
            \end{equation}

        
        Consider a bank considering an almost riskfree project paying $100$ at time $1$ with a theoretical yield of $30bp$, i.e. a time $0$ value of $99.70$. However the project trades at a discount, $99.21$, such that the yield is $80bp$. The bank can obtain deposits at a funding rate of $\rho = 200bp$. 
        
        The bank must obtain the funding needed for the project through a mix of deposits and issuance of equity, i.e.:
            \begin{equation}
                99.21 = S_{0} + D_{0}
            \end{equation}

        The payoff to the depositors is:
            \begin{equation}
                D_{1} = \min\left(
                    D_{0}(1 + 200bp), \;
                    100
                \right)
            \end{equation}
        The value of the bank equity is the following:
            \begin{equation}
                S_{1} = \left(
                    100 - D_{0}(1 + 200bp)
                \right)^{+}
            \end{equation}
        Denoting the time $0$ value of the equity payoff and the debt payoff as respectively $\pi(S)$ and $\pi(D)$, we have that:
            \begin{gather}
                \pi(S) + \pi(D) = \pi(A) = 99.70 \\
                \Leftrightarrow \qquad \pi(S) = 99.70 - D_{0} > 99.21 - D_{0} = S_{0}
            \end{gather}
        The funding rate for the bank is:
            \begin{align}
                \frac{S_{0}}{99.21} \frac{\mathbb{E}\left[S_{1}\right]}{S_{0}} 
                + \frac{D_{0}}{99.21} \frac{\mathbb{E}\left[D_{1}\right]}{D_{0}} 
                &= \frac{\mathbb{E}\left[S_{1}\right]}{99.21} 
                + \frac{\mathbb{E}\left[D_{1}\right]}{99.21}
            \end{align}
    \subsection{Example Single Period Model}
        Consider the economy defined by a single period model with the states $\Omega = \{\omega_{1}, \dots, \omega_{5}\}$ and the following properties:
            \begin{table}[H]
                \centering
                \begin{tabular}{l|rr}
                    $i$ & $\mathbb{P}\left(\omega_{i}\right) $ & $\psi_{i}$ \\
                    \hline
                    $1$ & $0.10$ & $0.07$ \\
                    $2$ & $0.30$ & $0.25$ \\
                    $3$ & $0.30$ & $0.28$ \\
                    $4$ & $0.25$ & $0.27$ \\
                    $5$ & $0.05$ & $0.11$ \\
                \end{tabular}
            \end{table}

        Yielding a risk free rate of $r_{f} = 2.04\%$.

        A bank operates in this economy and invests in risky assets with payoff $A_{i}$ in state $\omega_{i}$. The bank is financed by equity and deposits. Depositors will be paid an amount of $D_{0}(1+\rho(D_{0}))$ if the assets can support it and otherwise take over the remaining assets of the bank. Shareholders will receive a payoff equal to the residual between the assets and the debt. Consider the bank has been funded by debt such that the face value of debt is $D_{l}^{FV} = D_{0}(1+\rho(D_{0})) = 80$. This gives rise to the following payoff structure:
            \begin{table}[H]
                \centering
                \begin{tabular}{l|rrrrr}
                    $i$ & 1 & 2 & 3 & 4 & 5 \\ 
                    \hline
                    $A_{i}$ & 120 & 110 & 100 & 95 & 60 \\
                    $D_{i}$ & 80 & 80 & 80 & 80 & 60 \\
                    $S_{i}$ & 40 & 30 & 20 & 15 & 0
                \end{tabular}
            \end{table}
        The associated present value of the payoffs are the sumproduct between the Arrow-Debreu prices and the payoffs, i.e.:
            \begin{equation}
                \pi(A) = \sum_{i=1}^{5} A_{i}\psi_{i} = 96.15 
                \qquad 
                \pi(D) = \sum_{i=1}^{5} D_{i}\psi_{i} = 76.20 
                \qquad 
                \pi(S) = \sum_{i=1}^{5} S_{i}\psi_{i} = 19.95 
            \end{equation}

    \subsubsection{Project 1}
            \newcommand{\legacydebt}{D_{i}^{l}}
            \newcommand{\newdebt}{\tilde{D}_{i}}
            \newcommand{\legacyfacevalue}{D_{FV}^{l}}
            \newcommand{\newfacevalue}{\tilde{D}_{FV}}
        Assume that the bank faces a new possible project. The project is sure to return a payoff of $1$, but still yields $80bp$ above the risk free rate such that the price of the project is $1 / (1 + r_{f} + 80bp) = 0.97$. The bank can finance the project by obtaining additional deposits from new creditors. The bank should issue bonds at a price of $0.97$ in order to fully finance the project. The bond will rank pari passu with the existing debt, such that the loss rate experienced in case of default is equal between the legacy and new creditors. Denoting the legacy debt by $\legacydebt$ and the new debt by $\newdebt$ Hence the loss rate should satisfy:
            \begin{equation}
                L = \frac{\legacydebt}{ \legacyfacevalue} = \frac{\newdebt}{\newfacevalue}
            \end{equation}

        The payoff recovered in default states is thus:
            \begin{equation}
                \legacydebt = A_{i} \frac{\legacyfacevalue}{\legacyfacevalue + \newfacevalue} 
                \qquad 
                \newdebt = A_{i} \frac{\newfacevalue}{\legacyfacevalue + \newfacevalue} 
                = A_{i} \left(1 - \frac{\legacyfacevalue}{\legacyfacevalue + \newfacevalue} \right)
            \end{equation}
        To maximize the value for the bank and at the same time ensuring that the trade is worthwhile for the new creditors the face value $\newfacevalue$ should be chosen in such a way that the present value of the future expected payoff $\pi(\tilde{D}) = \sum_{i=1}^{5} \newdebt \psi_{i}$ equals the initial investment made by the new creditors at time 0. To finance the new project the initial investment should be $0.97$. By inspection, the set $\{\omega_{1}, \dots, \omega_{4}\}$ contains the no-default states, while $\{\omega_{5}\}$ contains the default states. Thus the following equation must be satisfied:
            \begin{align}
                0.97 &= \sum_{i=1}^{4} \newdebt \psi_{i} + \tilde{D}_{5}\psi_{5} \\
                &= \sum_{i=1}^{4} \newfacevalue \psi_{i} + A_{5} \left(1 - \frac{\legacyfacevalue}{\legacyfacevalue + \newfacevalue} \right)\psi_{5} \\ 
                &= \sum_{i=1}^{4} \newfacevalue \psi_{i} + A_{i} \frac{\newfacevalue}{\legacyfacevalue + \newfacevalue} \psi_{5} \\ 
                &= \newfacevalue (0.07+0.25+0.28+0.27) + 61 * \frac{\newfacevalue}{80 + \newfacevalue} 0.11
            \end{align}
        Solving for the positive face value, $\newfacevalue$, yields the solution:
            \begin{equation}
                \newfacevalue = 1.02053
            \end{equation}

        The new project and its financing, the newly issued debt, has some implications on the firm value, the legacy creditors as well as the shareholders. Starting with the legacy creditors: In no-default states they are still paid in full the original face value of their debt, $\legacyfacevalue$. In default states the asset base has been increased by the new project, but the legacy creditors will have to share the remaining assets pari passu with the new creditors. In the single default state, $\omega_{5}$, the payoff to legacy creditors is:
            \begin{equation}
                D_{5}^{l} = A_{5} \frac{\legacyfacevalue}{\legacyfacevalue + \newfacevalue} = 61 * \frac{80}{80 + 1.02053} = 60.23
            \end{equation}
        which is higher than the pre-project payoff, i.e. the legacy creditors is better of. The new value of their debt is $\pi(D^{l}) = \sum_{i=1}^{5} D_{i}^{l} \psi_{i} = 76.23$, which is an increase of $0.0255$ from the original value.

        Turning now to the shareholders: In the default state, $\omega_{5}$, the shareholders are still not paid anything. In the no-default states the shareholders still receive the residual of the asset value after creditors have been paid. This payoff is reduced by the funding paid to the new creditors, $\newfacevalue - 1 = 0.0205$, such that the new present value of the shareholder payoff is $\pi(\tilde{D}) = 19.93$, a decrease of $0.0179$.

        Clearly the new project is a negative NPV investment for the shareholders and a positive NPV for the for the legacy debitors. The asset base of the bank increases by the theoretical fair value of the project, which is the payoff discounted by the Arrow Debreu prices, $1*\sum_{i=1}^{5} \psi_{i} = 0.98$. The aggregate wealth increases by an amount $0.0255 - 0.0179 = 0.0076$, compared to the case where the riskfree project had a yield spread of zero. The increase in aggregate wealth is exactly the discount on the project from the theoretical fair value, $0.98 - \frac{1}{1 + 80bp + r_{f}} = 0.0076$. A larger fraction of the discount has been distributed to the shareholders than to the legacy creditors, such that the loss experienced by the shareholders from entering into the project is less than before.

        The project's funding costs could be defined as the present value of the additional funding costs paid by the shareholders on the financing. That is the present value of the excess interest paid by the bank due to credit risk, i.e. the credit spread multiplied by the bond price and the discounting factor:
            \begin{align}
                \text{FVA} 
                    &= \left(\frac{\newfacevalue}{0.97} - r_{f} - 1\right) * 0.97 * \frac{1}{1 + r_{f}} \\
                    &= 0.0291 * 0.97 * 0.98 \\
                    &= 0.0277
            \end{align}
        When the bank does not default the funding costs are borne by the shareholders, why the present value of the shareholder payoffs, $\pi(S)$, are adjusted by the expected funding costs. Since the entire credit spread is compensation for default risk the FVA exactly equals the DVA which is defined as the present value of the new creditor's expected positive exposure, $\newfacevalue$, times the expected loss rate, $\sum_{i=1}^{5} \left(1 - \frac{\tilde{D}_{i}}{\newfacevalue}\right) * \psi_{i}$:
            \begin{align}
                \text{DVA} 
                    &= \frac{1}{1 + r_{f}} 
                        * \newfacevalue 
                        * \sum_{i=1}^{5} \left(
                            1 - \frac{\tilde{D}_{i}}{\newfacevalue}
                        \right) 
                        * \psi_{i} \\
                    &= 0.98 * 1.0205 * 0.0277 \\
                    &= 0.0277
            \end{align}

    \section{Difference between equity financing and debt financing}
        When raising money for business needs a company usually has two options to consider: Equity- and debt financing. Equity financing involves selling out some of its own shares in order to raise funds. This can be due to paying bills or buying market derivatives. When selling out own shares to investors, the company is not obliged to return the money to the investors, however the investors now own a percentage of the company, which means that they now take a share of the profits obtained, as well as take part in the decision making around the impacts of the company.

        The other way to potentially fund a market derivative is to lend money. Here the lender has no control over the business's operations, and once the loan has been paid back, the relationship ends. The cost of debt financing is the interest rates. These interest rates are tax deductible for a company, and the loan payments make forecasting for future expenses easier because the amount does not fluctuate. The debt financing can however place restrictions on a company's operations so that it might not have as much leverage to take advantage of opportunities outside of its core business.
        
        The \textbf{pecking order theory} explains the inverse relationship between profitability and debt ratios. A corporation can finance its different costs in one more way, which is the internal financing. The way source of financing is preferred over the other two, and only when the money from internal financing is depleted, the company then uses debt financing. And when it's no longer sensible to issue more debt, equity is used, and external ownership is then brought into the company. 

        
    \section{Funding costs causing asymmetric valuations}
        A firm will enter into a deal on a derivative transaction
        only if it considers the fair value to be greater than or equal to the price.
        On the other hand, the counterparty's valuation of the transaction is different,
        since the firm and the counterparty are in two completely different situations.
        This difference is partially caused by the funding costs,
        which for the firm is derived from the market costs of borrowing funds,
        as well as the costs from raising funds by the normal business activities.
        The funding costs of the counterparty are, on the contrary, derived from treasury activities such as wholesale funding.
        Despite the difference in valuation, the transaction can still occur,
        as long as they can agree on a price that does not neglect either of the fair values.

        The asymmetry between the valuation of a transaction between two counterparties with two different positions in the market can be explained by other components as well.
        The portfolio management is different in the sense,
        that for the counterparty the derivative is a part of a large derivative portfolio with many other counterparties,
        whereas the firm most likely uses the derivative to hedge balance sheet exposures,
        and provide better cash flow management.
        Also, the firm does not have to place any regulatory capital against the position,
        but the counterparty must on the contrary have the appropriate regulatory capital in place.

        (Even the transaction made between two banks can have asymmetric valuations.)

\end{document}
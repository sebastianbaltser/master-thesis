\documentclass[10pt,a4paper]{article}

\usepackage[utf8]{inputenc}
\usepackage[hidelinks]{hyperref}
\usepackage[english, danish]{babel}
\usepackage[left=3.00cm, right=3.00cm, top=3.00cm, bottom=3.00cm]{geometry}
\usepackage{parskip}

\usepackage{amsmath}
\usepackage{amsfonts}
\usepackage{amssymb}
\usepackage{bbm}

\begin{document}
    \section{First meeting}   
    \subsection{Questions about theory} 
        The following is an example taken from \textit{Hull and White: "Is FVA a Cost for Derivatives Desks":}

        \textit{Another argument against FVA is a well-established principle in corporate finance theory that pricing should be kept separate from funding. The discount rate used to value a project should depend on the risk of the project rather than the riskiness of the firm that undertakes it. For example, suppose a company that borrows at the risk-free rate plus 200bp has the opportunity to enter into a nearly risk-free project that returns the risk-free rate plus 80bp. Suppose the discount rate for the project, if considered as a stand- alone project, would be the risk-free rate plus 30bp. The project should be undertaken. Clearly, the project has a positive net present value when the appropriate discount rate (risk-free rate plus 30 bp) is used. The project will increase shareholder value because it reduces the risk of the company and, therefore, incrementally reduces its funding cost.}

        \textbf{Questions:}
        \begin{enumerate}
            \item What does it tell us that the discount rate is risk-free rate plus 30bp? Is that the return that would be expected if the firm did not undertake the project and instead put its resources to use elsewhere?
            \item Hull and White states that the project should be undertaken. However it seems wrong to ever undertake a project that returns less than what it costs to fund it (unless forced by some regulation or the like). We think the example sounds a lot like: "Would you borrow money from the bank at a rate of 200bp to invest the same money in bonds yielding 80bp." This sounds like a no-brainer, so we fail to see why Hull and White concludes otherwise.  
        \end{enumerate}

        \textit{Castagna: Yes, FVA is a Cost for Derivatives Desks} concludes that the example proposed by Hull and White does not work in reality:

        \textit{[...] the example proposed by Professors Hull and White, although not applicable to the derivatives replication, is anyway not really working in reality. In fact, it works only if the incremental cost for the capital needed to fund the (almost risk-free) project is actually the fair rate of return of the project (risk-free plus 30bps). If the bank raises funds at risk-free + 200bps, (not compensated by a decline of the return requested by shareholders, thus reducing the average cost of capital) the project is no more profitable so that it generates a loss that decreases, not increases, the value of the firm. The reduction of the riskiness of the total assets for the future is true, but unfortunately the loss generated by the project can more than counterbalance it. It should be noted that in reality it is quite unlikely to have an incremental cost of capital based on the return of last investment considered on a stand-alone basis. More frequently, the cost of capital will gradually update.}

        \textbf{Questions:}
        \begin{enumerate}
            \item Castagna argues that the example above only works when the funding rate of the company is the fair rate of return on the investment, risk-free plus 30bp. But isn't the investment profitable already when the funding rate is below risk-free plus 80bp? The project returns risk-free plus 80bp, so if it can be funded for less than this value it seems profitable to us.
        \end{enumerate}
    
    \subsection{Questions about thesis structure}
        We would like our thesis to follow the following general structure:
        \begin{enumerate}
            \item \textbf{Discuss how/if FVA should be used in derivatives pricing:} Present and summarize the discussions about FVA and reach a conclusion. The conclusion is necessary for the remaining analyses to be carried out correctly. The primary sources will be Hull and White, Castagna, and Ruiz.
            \item \textbf{Present a framework in which funding costs are present and formulate a method for calculating FVA:} Formulate a framework where firms faces funding costs e.g. as the one by Burgard and Kjaer. Do the necessary derivations to arrive at formulas for calculating CVA, DVA and FVA. 
            \item \textbf{Analyze FVA by examples:} Use the formulas derived in the previous part to carry out examples e.g. by simulations, to find interesting properties of FVA and study it further.
        \end{enumerate}

        \textbf{Questions}
        \begin{enumerate}
            \item Does this seem like a logical approach to the subject?
            \item How do we deal with reaching an answer to the question "Should an FVA be made in derivatives pricing"? The discussion seems unfinished, but it is quite unlikely that we ourselves can provide a final argument that closes it. Do we then just go with the conclusion that seems more appealing/justified to us or should we avoid concluding anything?
        \end{enumerate}

    \subsection{Question about academic writing}
        \begin{enumerate}
            \item Is it okay to combine the use of inline citations (Hull and White (2013)) and the use of footnote citations\footnote{Hull and White (2013)}? We find both useful in different scenarios so we would prefer being able to switch between them.
        \end{enumerate}

\end{document}